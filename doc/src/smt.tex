\documentclass{galahad}

% set the package name

\newcommand{\package}{smt}
\newcommand{\packagename}{SMT}
\newcommand{\fullpackagename}{\libraryname\_\packagename}

\begin{document}

\galheader

%%%%%%%%%%%%%%%%%%%%%% SUMMARY %%%%%%%%%%%%%%%%%%%%%%%%

\galsummary
This package defines a derived type capable of {\bf supporting 
a variety of sparse matrix storage schemes.} Its principal use is 
to allow exchange of data between \galahad\ subprograms  
and other codes. The derived type is structurally equivalent to 
the type {\tt ZD11\_type} available from the HSL package {\tt ZD11}.

%%%%%%%%%%%%%%%%%%%%%% attributes %%%%%%%%%%%%%%%%%%%%%%%%

\galattributes
\galversions{\tt  \fullpackagename\_single, \fullpackagename\_double}.
\galuses None.
\galdate March 1998.
\galorigin N. I. M. Gould and J. K. Reid, Rutherford Appleton Laboratory.
\gallanguage Fortran~95 + TR 15581 or Fortran~2003. 

%%%%%%%%%%%%%%%%%%%%%% HOW TO USE %%%%%%%%%%%%%%%%%%%%%%%%

\galhowto

%\subsection{Calling sequences}

Access to the package requires a {\tt USE} statement such as

\medskip\noindent{\em Single precision version}

\hspace{8mm} {\tt USE \fullpackagename\_single}

\medskip\noindent{\em Double precision version}

\hspace{8mm} {\tt USE  \fullpackagename\_double}

\medskip

\noindent
If it is required to use both modules at the same time, the derived type
{\tt \packagename\_type} 
(Section~\ref{galtype}),
the subroutine
{\tt \packagename\_put}, 
and the function
{\tt \packagename\_get} 
(Section~\ref{galarguments})
must be renamed on one of the {\tt USE} statements.

%%%%%%%%%%%%%%%%%%%%%% derived types %%%%%%%%%%%%%%%%%%%%%%%%

\galtype
A single derived data type, {\tt \packagename\_type},
is accessible from the package. It is intended that, for any particular
application, only those components which are needed will be set.
The components are:

\begin{description}
\ittf{id} is an allocatable array of rank one and type default \character\, that  
may be used to hold the name of the matrix. 
 
\ittf{type} is an allocatable array of rank one and type default \character\, that  
may be used to hold a key which indicates the type (or kind) of the matrix in 
question.  
 
\ittf{m} is a scalar component of type default \integer\, 
that may be used to hold the number of rows in the matrix. 
 
\ittf{n} is a scalar component of type default \integer\, 
that may be used to hold the number of columns in the matrix. 
 
\ittf{ne} is a scalar component of type default \integer\, 
that may be used to hold the number of entries in the matrix. 
 
\ittf{row} is an allocatable array of rank one and  
type default \integer\, that may be used to hold 
the row indices of the entries of the matrix. 
 
\ittf{col} is an allocatable array of rank one and  
type default \integer\, that may be used to hold 
the column indices of the entries of the matrix. 
 
\ittf{val} is an allocatable array of rank one and  
type default \realdp\, that may be used to hold 
the numerical values of the entries of the matrix. 
 
\ittf{ptr} is an allocatable array of rank one and  
type default \integer\, that may be used to hold 
the starting positions of each row in a row-wise storage scheme, 
or the starting positions of each column in a column-wise storage  
scheme. 
\end{description}

%%%%%%%%%%%%%%%%%%%%%% argument lists %%%%%%%%%%%%%%%%%%%%%%%%

\galarguments

To assist use of the character arrays in the components {\tt \%id} and
{\tt \%type}, the module provides two procedures:

\begin{enumerate}
\item The subroutine {\tt \packagename\_put}
allocates a character array and sets
its components from a character variable.

\item The function {\tt \packagename\_get} 
obtains the elements of a character array as a character variable.
\end{enumerate}
We use square brackets {\tt [ ]} to indicate \optional\ arguments.

\subsubsection{Allocate a character array and set its components}

To allocate a character array and set its components from a character variable,
\vspace*{-2mm}
{\tt 
\begin{verbatim}
        CALL SMT_put( array, string, stat )
\end{verbatim}
}
\vspace*{-4mm}
\begin{description}

\itt{array} is a rank one allocatable array of type default \character.
If {\tt string} is present, {\tt array} is allocated with size 
{\tt LEN\_TRIM(string)} and
its elements are given the values {\tt string(i:i), i} = 1, 2, \ldots ;
otherwise, {\tt array} is allocated to be of size zero.

\itt{string} is an \optional, \intentin\ argument of type \character
with any character length. 

\itt{stat} is an 
%\optional, 
\intentout\ argument of type default \integer.
%If present, 
An {\tt ALLOCATE} statement with this as its {\tt STAT=}
variable is executed and a successful allocation will be indicated
by the value zero. 
%If absent, an {\tt ALLOCATE} statement without a {\tt STAT=} 
%variable is executed.
\end{description}

\subsubsection{Obtain the elements of a character array as a character variable}

To obtain the elements of a character array as a character variable,
\vspace*{-2mm}
{\tt 
\begin{verbatim}
        string = SMT_get( array )
\end{verbatim}
}
\vspace*{-4mm}
\begin{description}

\itt{array} is an \intentin\ array of rank one and type default \character.
It is not altered.
\end{description}
The result is scalar and of type {\tt CHARACTER(LEN=SIZE(array))}.
{\tt SMT\_get(i:i)} is given the value {\tt array(i), i} = 1, 2, \ldots ,
{\tt SIZE(array)}.

%%%%%%%%%%%%%%%%%%%%%% GENERAL INFORMATION %%%%%%%%%%%%%%%%%%%%%%%%

\galgeneral

\galmodules None.
\galio None.
\galportability ISO Fortran~95 + TR 15581 or Fortran~2003. 
The package is thread-safe.
\end{description}

%%%%%%%%%%%%%%%%%%%%%% EXAMPLE %%%%%%%%%%%%%%%%%%%%%%%%

\galexample
Suppose that we wish to store the symmetric matrix  
\disp{\mat{ccc}{ 1.0 \\ & & 1.0 \\ & 1.0 },}
whose name is ``Sparse'', using a coordinate sparse matrix storage format.  
Then the following code is appropriate: 
{\tt \small
\begin{verbatim}
   PROGRAM GALAHAD_SMT_example
   USE GALAHAD_SMT_double 
   INTEGER :: i
   TYPE ( SMT_type ) :: A 
   A%n = 3 ; A%ne = 2 
   ALLOCATE( A%row( A%ne ), A%col( A%ne ), A%val( A%ne ) ) 
   CALL SMT_put( A%id, 'Sparse' )      ! Put name into A%id
   CALL SMT_put( A%type )              ! Allocate space for A%type
   A%row( 1 ) = 1 ; A%col( 1 ) = 1 ; A%val( 1 ) = 1.0 
   A%row( 2 ) = 2 ; A%col( 2 ) = 3 ; A%val( 2 ) = 1.0 
   WRITE( 6, "( 3A, I2, //, A )" ) ' Matrix ', SMT_get( A%id ), & 
          ' dimension', A%n, ' row col  value ' 
   DO i = 1, A%ne 
      WRITE( 6, "( I3, 1X, I3, ES9.1 )" ) A%row( i ), A%col( i ), A%val( i ) 
   END DO 
   DEALLOCATE( A%id, A%row, A%col, A%val )
   END PROGRAM GALAHAD_SMT_example
\end{verbatim}
}
\noindent
This produces the following output: 
{\tt \small
\begin{verbatim}
 Matrix Sparse dimension 3 
 
 row col  value  
  1   1  1.0E+00 
  2   3  1.0E+00 
\end{verbatim}
}
\noindent

For examples of how the derived data type
{\tt packagename\_problem\_type} may be used in conjunction with the
\galahad\ linear equation solver, see the specification sheet
for the package
{\tt \libraryname\_SILS}.
\end{document}
