\begin{description}

\itt{error} is a scalar variable of type \integer, that holds the
stream number for error messages. Printing of error messages in
{\tt \packagename\_solve} and {\tt \packagename\_terminate} is suppressed if
{\tt error} $\leq 0$.
The default is {\tt error = 6}.

\ittf{out} is a scalar variable of type \integer, that holds the
stream number for informational messages. Printing of informational messages in
{\tt \packagename\_solve} is suppressed if {\tt out} $< 0$.
The default is {\tt out = 6}.

\itt{print\_level} is a scalar variable of type \integer, that is used
to control the amount of informational output which is required. No
informational output will occur if {\tt print\_level} $\leq 0$. If
{\tt print\_level} $= 1$, a single line of output will be produced for each
iteration of the process. If {\tt print\_level} $\geq 2$, this output will be
increased to provide significant detail of each iteration.
The default is {\tt print\_level = 0}.

\itt{start\_print} is a scalar variable of type \integer, that specifies
the first iteration for which printing will occur in {\tt \packagename\_solve}.
If {\tt start\_print} is negative, printing will occur from the outset.
The default is {\tt start\_print = -1}.

\itt{stop\_print} is a scalar variable of type \integer, that specifies
the last iteration for which printing will occur in  {\tt \packagename\_solve}.
If {\tt stop\_print} is negative, printing will occur once it has been
started by {\tt start\_print}.
The default is {\tt stop\_print = -1}.

\itt{maxit} is a scalar variable of type \integer, that holds the
maximum number of iterations which will be allowed in {\tt \packagename\_solve}.
The default is {\tt maxit = 1000}.

\itt{max\_iterative\_refinements} is a scalar variable of type default
\integer, that holds
the maximum number of iterative refinements per linear system solved
that may be attempted.
The default is {\tt max\_iterative\_refinements = 0}.

\itt{min\_real\_factor\_size} is a scalar variable of type \integer,
that specifies the amount of real storage that will initially be
allocated for the factors and other data.
The default is {\tt min\_real\_factor\_size = 10000},
and this default is used if {\tt min\_real\_factor\_size < 1}.

\itt{min\_integer\_factor\_size} is a scalar variable of type \integer,
that specifies the amount of integer storage that will initially be
allocated for the factors and other data.
The default is {\tt min\_integer\_factor\_size = 20000},
and this default is used if {\tt min\_integer\_factor\_size < 1}.

\itt{random\_number\_seed} is a scalar variable of type \integer,
that holds the initial seed used by the random-number generator.
The default is {\tt random\_number\_seed = 0}.

\itt{infinity} is a scalar variable of type \realdp, that is used to
specify which constraint bounds are infinite.
Any bound larger than {\tt infinity} in modulus will be regarded as infinite.
The default is {\tt infinity =} $10^{19}$.

\itt{tol\_data} is a scalar variable of type \realdp, that hold the
maximum violation that a constraint is allowed and be still considered to be
feasible.
The default is {\tt tol\_data =} $u^{2/3}$,
where $u$ is {\tt EPSILON(1.0)} ({\tt EPSILON(1.0D0)} in
{\tt \fullpackagename\_double}).

\itt{feas\_tol} is a scalar variable of type \realdp, that hold the
maximum violation that a constraint is allowed and be still considered to be
feasible.
The default is {\tt feas\_tol =} $u^{2/3}$,
where $u$ is {\tt EPSILON(1.0)} ({\tt EPSILON(1.0D0)} in
{\tt \fullpackagename\_double}).

\itt{relative\_pivot\_tolerance} is a scalar variable of type \realdp,
that holds the relative pivot tolerance that is used to control the
stability of the factorizations of the basis matrices that arise as the
iteration proceeds.
The default is {\tt relative\_pivot\_tolerance = 0.1}.

\itt{growth\_limit} is a scalar variable of type \realdp, that
specifies the maximum growth in the entries of the factors of a basis
matrix that will be tolerated before a refactorization is required.
The default is {\tt growth\_limit = $1/u^{2/3}$},
where $u$ is {\tt EPSILON(1.0)} ({\tt EPSILON(1.0D0)} in
{\tt \fullpackagename\_double}).

\itt{zero\_tolerance} is a scalar variable of type \realdp, that
controls which small entries are to be ignored during the factorization
of a basis matrix. Any entry smaller in absolute value than
{\tt zero\_\-tolerance} will be treated as zero; as a consequence when
{\tt zero\_tolerance > 0}, the factors produced will be of a perturbation
of order {\tt zero\_tolerance}.
The default is {\tt zero\_tolerance = $u$},
where $u$ is {\tt EPSILON(1.0)} ({\tt EPSILON(1.0D0)} in
{\tt \fullpackagename\_double}).

\itt{change\_tolerance} is a scalar variable of type \realdp, that
provides a tolerance whose purpose is to assess when a
change to a solution component may be neglected. Specifically
any change that is smaller than this tolerence times the
largest change may be considered to be zero.
The default is {\tt change\_tolerance =} $u^{2/3}$,
where $u$ is {\tt EPSILON(1.0)} ({\tt EPSILON(1.0D0)} in
{\tt \fullpackagename\_double}).

\itt{identical\_bounds\_tol}
is a scalar variable of type \realdp.
Every pair of constraint bounds
$(c_{i}^{l}, c_{i}^{u})$ or $(x_{j}^{l}, x_{j}^{u})$
that is closer than {\tt identical\_bounds\_tol}
will be reset to the average of their values,
$\half (c_{i}^{l} + c_{i}^{u})$ or $\half (x_{j}^{l} + x_{j}^{u})$
respectively.
The default is {\tt identical\_bounds\_tol =} $u$,
where $u$ is {\tt EPSILON(1.0)} ({\tt EPSILON(1.0D0)} in
{\tt \fullpackagename\_double}).

\itt{cpu\_time\_limit} is a scalar variable of type \realdp,
that is used to specify the maximum permitted CPU time. Any negative
value indicates no limit will be imposed. The default is
{\tt cpu\_time\_limit = - 1.0}.

\itt{clock\_time\_limit} is a scalar variable of type \realdp,
that is used to specify the maximum permitted elapsed system clock time.
Any negative value indicates no limit will be imposed. The default is
{\tt clock\_time\_limit = - 1.0}.

\itt{scale} is a scalar variable of type
default \logical, that must be set \true\ if the problem data
is to be scaled prior to solution in an attempt to make the
process faster and more accurate.
The default is {\tt scale = .FALSE.}.

\itt{warm\_start} is a scalar variable of type default \logical,
that must be set \true\ if an initial guess to the optimal
status of the constraints and simple bounds is to be provided,
and  \false\ if the algorithm is to make its own initial choice.
The default is {\tt warm\_start = .FALSE.}.

\itt{dual} is a scalar variable of type default \logical,
that must be set \true\ if the simplex method is to be applied to
the dual of the linear program,
and  \false\ if the primal problem should be solved.
The default is {\tt dual = .FALSE.}.

\itt{steepest\_edge} is a scalar variable of type default \logical,
that must be set \true\ if steepest-edge weights are to be used to
try to improve the choice of variable that will leave the basis at
each iteration of the simplex method,
and  \false\ if weights of one are used.
The default is {\tt steepest\_edge = .TRUE.}.

\itt{space\_critical} is a scalar variable of type default \logical,
that must be set \true\ if space is critical when allocating arrays
and  \false\ otherwise. The package may run faster if
{\tt space\_critical} is \false\ but at the possible expense of a larger
storage requirement. The default is {\tt space\_critical = .FALSE.}.

\itt{deallocate\_error\_fatal} is a scalar variable of type default \logical,
that must be set \true\ if the user wishes to terminate execution if
a deallocation  fails, and \false\ if an attempt to continue
will be made. The default is {\tt deallocate\_error\_fatal = .FALSE.}.

\itt{prefix} is a scalar variable of type default \character\
and length 30, that may be used to provide a user-selected
character string to preface every line of printed output.
Specifically, each line of output will be prefaced by the string
{\tt prefix(2:LEN(TRIM(prefix))-1)},
thus ignoring the first and last non-null components of the
supplied string. If the user does not want to preface lines by such
a string, they may use the default {\tt prefix = ""}.

\end{description}
