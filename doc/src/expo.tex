\documentclass{galahad}

% set the package name

\newcommand{\package}{expo}
\newcommand{\packagename}{EXPO}
\newcommand{\fullpackagename}{\libraryname\_\packagename}
\newcommand{\solver}{{\tt \fullpackagename\_solve}}

% macros

\newcommand{\sL}{\s{L}}
\newcommand{\sU}{\s{U}}
\newcommand{\muL}{\mu\sL}
\newcommand{\muU}{\mu\sU}
\newcommand{\nuL}{\nu\sL}
\newcommand{\nuU}{\nu\sU}
\newcommand{\wL}{w\sL}
\newcommand{\wU}{w\sU}
\newcommand{\vL}{v\sL}
\newcommand{\vU}{v\sU}
\newcommand{\bmgL}{\bmg\sub{L}}
\newcommand{\bmHL}{\bmH\sub{L}}
%\renewcommand{\bmmu}{{\mathbf{\mu}}}
%\renewcommand{\bmnu}{{\mathbf{\nu}}}
%\renewcommand{\bmHL}{\mathbf{H\sub{L}}}

\begin{document}

\galheader

%%%%%%%%%%%%%%%%%%%%%% SUMMARY %%%%%%%%%%%%%%%%%%%%%%%%

\galsummary
This package uses an {\bf exponential-penalty function to find a (local)
minimizer of a differentiable objective function $f(\bmx)$
of $n$ variables $\bmx$, subject 
to $m$ general constraints $\bmc\sL \leq \bmc(\bmx) \leq \bmc\sU$
and simple bounds $\bmx\sL \leq \bmx \leq \bmx\sU$ on the variables.}
Here, any of the components of the vectors of bounds 
$\bmc\sL$, $\bmc\sU$, $\bmx\sL$ and $\bmx\sU$ may be infinite. 
The method offers the choice of direct
and iterative solution of the key trust-region subproblems, and
is most suitable for large problems. First derivatives are required,
and if second derivatives can be calculated, they will be exploited---if
the product of second derivatives with a vector may be found but
not the derivatives themselves, that may also be exploited.

%%%%%%%%%%%%%%%%%%%%%% attributes %%%%%%%%%%%%%%%%%%%%%%%%

\galattributes
\galversions{\tt  \fullpackagename\_single, \fullpackagename\_double}.
\galuses
{\tt GALAHAD\_CLOCK},
{\tt GALAHAD\_NLPT},
{\tt GALAHAD\_SY\-M\-BOLS},
{\tt GALAHAD\_USERDATA},
{\tt GALAHAD\_SPECFILE},
{\tt GALAHAD\_SMT},
{\tt GALAHAD\_BSC},
{\tt GALAHAD\_MOP},
{\tt GALAHAD\_\-SSLS},
{\tt GALAHAD\_TRU},
{\tt GALAHAD\_GLTR},
{\tt GALAHAD\_STRINGS},
{\tt GALAHAD\_SPACE},
{\tt GALAHAD\_\-NORMS},
{\tt GALAHAD\_BLAS\_interface},
and
{\tt GALAHAD\_LAPACK\_interface}.
\galdate may 2025.
\galorigin N. I. M. Gould, Rutherford Appleton Laboratory.
\gallanguage Fortran~95 + TR 15581 or Fortran~2003.

%%%%%%%%%%%%%%%%%%%%%% HOW TO USE %%%%%%%%%%%%%%%%%%%%%%%%

\galhowto

The package is available with single, double and (if available) 
quadruple precision reals, and either 32-bit or 64-bit integers. 
Access to the 32-bit integer,
single precision version requires the {\tt USE} statement
\medskip

\hspace{8mm} {\tt USE \fullpackagename\_single}

\medskip
\noindent
with the obvious substitution 
{\tt \fullpackagename\_double},
{\tt \fullpackagename\_quadruple},
{\tt \fullpackagename\_single\_64},
{\tt \fullpackagename\_double\_64} and
{\tt \fullpackagename\_quadruple\_64} for the other variants.


\noindent
If it is required to use more than one of the modules at the same time, 
the derived types
{\tt SMT\_type},
{\tt GALAHAD\_userdata\_\-type},
{\tt \packagename\_time\_\-type},
{\tt \packagename\_control\_type},
{\tt \packagename\_inform\_type},
{\tt \packagename\_data\_type}
and
{\tt NLPT\_problem\_type},
(Section~\ref{galtypes})
and the subroutines
{\tt \packagename\_initialize},
{\tt \packagename\_\-solve},
{\tt \packagename\_terminate},
(Section~\ref{galarguments})
and
{\tt \packagename\_read\_specfile}
(Section~\ref{galfeatures})
must be renamed on one of the {\tt USE} statements.

%%%%%%%%%%%%%%%%%%%%%% terminology %%%%%%%%%%%%%%%%%%%%%%%%

\galterminology

The exponential penalty function is defined to be
\eqn{gemf}{\arr{rl}{\psi(\bmx,\bmw,\bmmu,\bmv,\bmnu) \!\! & = f(x) 
 + \bigsum_{i} \muL_i \wL_i \exp[(c\sL_i-c_i(\bmx))/\muL_i]
 + \bigsum_{i} \muU_i \wU_i \exp[(c_i(\bmx) - c\sU_i)/\muU_i] \\
 & \;\;\;\;\;\;\;\;\;\;\;\;\; 
 + \bigsum_{j} \nuL_j \vL_j \exp[(x\sL_j-x_j)/\nuL_j]
 + \bigsum_{j} \nuU_j \vU_j \exp[(x_j - x\sU_j)/\nuU_j],}}
where $c\sL_i$, $c\sU_i$ and $c_i(\bmx)$ are the $i$-th components
of $\bmc\sL$, $\bmc\sU$ and $\bmc(\bmx)$, and 
$c\sL_j$, $c\sU_j$ and $x_j$ are the $j$-th components
of $\bmx\sL$, $\bmx\sU$ and $\bmx$, respectively. Here the components
of $\bmmu\sL$, $\bmmu\sU$, $\bmnu\sL$ and $\bmnu\sU$ 
are separate {\bf penalty parameters} for each lower and upper, 
general  and simple-bound constraint, respectively,
while those of $\bmw\sL$, $\bmw\sU$, $\bmv\sL$, $\bmv\sU$
are likewise separate {\bf weights} for the same. The algorithm iterates by
approximately minimizing $\psi(\bmx,\bmw,\bmmu,\bmv,\bmnu)$
for a fixed set of penalty parameters and weights, and then adjusting 
these parameters and weights. The adjustments are designed so the
sequence of approximate minimizers of $\psi$ converge
to that of the specified constrained optimization problem.

Key constructs are the {\bf gradient} of the objective function
\eqn{g}{\bmg(\bmx) \eqdef \nabla_x f(\bmx),}
the {\bf Jacobian} of the vector of constraints, 
\eqn{J}{\bmJ(\bmx) \eqdef \nabla_x c(\bmx),}
 and the {\bf gradient} and {\bf Hessian} of the Lagrangian function
\eqn{gHL}{\bmgL(\bmx,\bmy) \eqdef \bmg(\bmx) - \bmJ^T(\bmx)\bmy - \bmz 
 \;\;\mbox{and}\;\;
\bmHL(\bmx,\bmy) \eqdef 
\nabla_{xx}  \left[ f(\bmx) - \sum_{i} y_i \nabla^2 c_i(\bmx)\right]}
for given vectors $\bmy$ and $\bmz$. 

The required solution $\bmx$ necessarily satisfies
the primal optimality conditions
\eqn{po}{\bmc\sL \leq \bmc(\bmx) \leq \bmc\sU \;\;\mbox{and}\;\;
\bmx\sL \leq \bmx \leq \bmx\sU,}
the dual optimality conditions
\eqn{doa}{
 \bmg(\bmx) = \bmJ^T(\bmx) \bmy + \bmz \;\;}
where
\eqn{dob}{
 \bmy = \bmy\sL - \bmy\sU, \;\;
 \bmz = \bmz\sL - \bmz\sU, \,\, \mbox{and} \;\;
 (\bmy\sL, \bmy\sU, \bmz\sL,  \bmz\sU) \geq 0 ,}
and the complementary slackness conditions
\eqn{4.3}{
( \bmc(\bmx) - \bmc\sL )^{T} \bmy\sL = 0  ,\;\;
( \bmc(\bmx) - \bmc\sU )^{T} \bmy\sU = 0  ,\;\;
(\bmx -\bmx\sL )^{T} \bmz\sL = 0   \tim{and}
(\bmx -\bmx\sU )^{T} \bmz\sU = 0 ,}
where the vectors $\bmy$ and $\bmz$ are
known as the Lagrange multipliers for
the general linear constraints, and the dual variables for the bounds,
respectively, and where the vector inequalities hold component-wise.

%%%%%%%%%%%%%%%%%%%%%% matrix storage formats %%%%%%%%%%%%%%%%%%%%%%%%

\galmatrix The matrices $\bmJ(\bmx)$ and $\bmHL(\bmx,\bmy)$
(as required and when available) may be stored in a variety of input formats.

\subsubsection{Dense storage format}\label{dense}
The matrix $\bmJ$ is stored as a compact
dense matrix by rows, that is, the values of the entries of each row in turn are
stored in order within an appropriate real one-dimensional array.
Component $n \ast (i-1) + j$ of the storage array {\tt
  J\%val} will hold the value $\bmJ_{i,j}$ for $i = 1, \ldots , m$, $j =
1, \ldots , n$.  Since $\bmHL$ is symmetric, only the lower triangular
part (that is the part $\bmHL{}_{ij}$ for $1 \leq j \leq i \leq n$)
should be stored. In this case the lower triangle will be stored by rows,
that is component $i \ast (i-1)/2 + j$ of the storage array {\tt H\%val} will
hold the value $\bmHL{}_{ij}$ (and, by symmetry, $\bmHL{}_{ji}$) for 
$1 \leq j \leq i \leq n$. 

\subsubsection{Sparse co-ordinate storage format}\label{coordinate}
Only the nonzero entries of the matrices are stored. For the $l$-th
entry of $\bmJ$, its row index $i$, column index $j$ and value
$\bmJ_{ij}$ are stored in the $l$-th components of the integer arrays
{\tt J\%row}, {\tt J\%col} and real array {\tt J\%val}.
The order is unimportant, but the total number of entries {\tt J\%ne} is
required.  The same scheme is applicable to $\bmHL$
(thus requiring integer arrays {\tt H\%row}, {\tt H\%col}, a real
array {\tt H\%val}, and an integer value {\tt H\%ne}), except that only
the entries in the lower triangle should be stored.

%We recommend this storage format since it is the most efficient
%storage scheme for the underlying linear system solvers.  Any other
%storage formats are converted internally to sparse co-ordinate
%storage and then handled accordingly.

\subsubsection{Sparse row-wise storage format}\label{rowwise}
Again only the nonzero entries are stored, but this time they are
ordered so that those in row $i$ appear directly before those in row
$i+1$. For the $i$-th row of $\bmJ$, the $i$-th component of the integer
array {\tt J\%ptr} holds the position of the first entry in this row,
while {\tt J\%ptr} $(m+1)$ holds the total number of entries plus one.
The column indices $j$ and values $\bmJ_{ij}$ of the entries in the
$i$-th row are stored in components $l =$ {\tt J\%ptr}$(i)$, \ldots
,{\tt J\%ptr} $(i+1)-1$ of the integer array {\tt J\%col}, and real
array {\tt J\%val}, respectively.
The same scheme is applicable to
$\bmHL$ (thus requiring integer arrays {\tt H\%ptr}, {\tt H\%col}, and
a real array {\tt H\%val}),
except that only the entries in the lower triangle should be stored.

For sparse matrices, this scheme almost always requires less storage than
its predecessor.

\subsubsection{Diagonal storage format}\label{diagonal}
If $\bmHL$ is diagonal (i.e., $\bmHL{}_{ij} = 0$ for all 
$1 \leq i \neq j \leq n$)
only the diagonal entries $\bmHL{}_{ii}$ for $1 \leq i \leq n$ need be stored,
and the first $n$ components of the array {\tt H\%val} may be used for
the purpose. There is no sensible equivalent for the non-square matrix $\bmJ$.

%%%%%%%%%%%%%%%%%%%%%%%%%%% kinds %%%%%%%%%%%%%%%%%%%%%%%%%%%

%%%%%%%%%%%%%%%%%%%%%% kinds %%%%%%%%%%%%%%%%%%%%%%

\galkinds
We use the terms integer and real to refer to the fortran
keywords {\tt REAL(rp\_)} and {\tt INTEGER(ip\_)}, where 
{\tt rp\_} and {\tt ip\_} are the relevant kind values for the
real and integer types employed by the particular module in use. 
The former are equivalent to
default {\tt REAL} for the single precision versions,
{\tt DOUBLE PRECISION} for the double precision cases
and quadruple-precision if 128-bit reals are available, and 
correspond to {\tt rp\_ = real32}, {\tt rp\_ = real64} and
{\tt rp\_ = real128} respectively as defined by the 
fortran {\tt iso\_fortran\_env} module.
The latter are default (32-bit) and long (64-bit) integers, and
correspond to {\tt ip\_ = int32} and {\tt ip\_ = int64},
respectively, again from the {\tt iso\_fortran\_env} module.


%%%%%%%%%%%%%%%%%%%%%% derived types %%%%%%%%%%%%%%%%%%%%%%%%

\galtypes
Seven derived data types are accessible from the package.

%%%%%%%%%%% matrix data type %%%%%%%%%%%

\subsubsection{The derived data type for holding matrices}\label{typesmt}
The derived data type {\tt SMT\_TYPE} is used to hold the Jacobian matrix 
$\bmJ$ and Hessian matrix $\bmHL$
if these are available. The components of {\tt SMT\_TYPE} used here are:

\begin{description}

\ittf{m} is a scalar component of type \integer,
that holds the row dimension of the matrix.

\ittf{n} is a scalar component of type \integer,
that holds the column dimension of the matrix.

\ittf{ne} is a scalar variable of type \integer, that
holds the number of matrix entries.

\ittf{type} is a rank-one allocatable array of type default \character, that
is used to indicate the matrix storage scheme used. Its precise length and
content depends on the type of matrix to be stored (see \S\ref{typeprob}).

\ittf{val} is a rank-one allocatable array of type \realdp\,
and dimension at least {\tt ne}, that holds the values of the entries.
Each pair of off-diagonal entries $h_{ij} = h_{ji}$ of the {\em symmetric}
matrix $\bmHL$ is represented as a single entry
(see \S\ref{dense}--\ref{rowwise}).
Any duplicated entries that appear in the sparse
co-ordinate or row-wise schemes will be summed.

\ittf{row} is a rank-one allocatable array of type \integer,
and dimension at least {\tt ne}, that may hold the row indices of the entries.
(see \S\ref{coordinate}).

\ittf{col} is a rank-one allocatable array of type \integer,
and dimension at least {\tt ne}, that may hold the column indices of the entries
(see \S\ref{coordinate}--\ref{rowwise}).

\ittf{ptr} is a rank-one allocatable array of type \integer,
and dimension at least {\tt m + 1}, that may hold the pointers to
the first entry in each row (see \S\ref{rowwise}).

\end{description}

%%%%%%%%%%% problem type %%%%%%%%%%%

\subsubsection{The derived data type for holding the problem}\label{typeprob}
The derived data type {\tt NLPT\_problem\_type} is used to hold
the problem. The relevant components of
{\tt NLPT\_problem\_type}
are:

\begin{description}

\ittf{n} is a scalar variable of type \integer,
 that holds the number of optimization variables, $n$.

\ittf{m} is a scalar variable of type \integer,
 that holds the number of general constraints, $m$.

\ittf{H} is scalar variable of type {\tt SMT\_TYPE}
that holds the Hessian matrix of the Lagrangian, $\bmHL$. 
The following components are used here:

\begin{description}

\itt{H\%type} is an allocatable array of rank one and type default
\character, that
is used to indicate the storage scheme used. If the dense storage scheme
(see Section~\ref{dense}) is used,
the first five components of {\tt H\%type} must contain the
string {\tt DENSE}.
For the sparse co-ordinate scheme (see Section~\ref{coordinate}),
the first ten components of {\tt H\%type} must contain the
string {\tt COORDINATE},
for the sparse row-wise storage scheme (see Section~\ref{rowwise}),
the first fourteen components of {\tt H\%type} must contain the
string {\tt SPARSE\_BY\_ROWS},
and for the diagonal storage scheme (see Section~\ref{diagonal}),
the first eight components of {\tt H\%type} must contain the
string {\tt DIAGONAL}.

For convenience, the procedure {\tt SMT\_put}
may be used to allocate sufficient space and insert the required keyword
into {\tt H\%type}.
For example, if {\tt nlp} is of derived type {\tt \packagename\_problem\_type}
and involves a Hessian we wish to store using the co-ordinate scheme,
we may simply
%\vspace*{-2mm}
{\tt
\begin{verbatim}
        CALL SMT_put( nlp%H%type, 'COORDINATE' )
\end{verbatim}
}
%\vspace*{-4mm}
\noindent
See the documentation for the \galahad\ package {\tt SMT}
for further details on the use of {\tt SMT\_put}.

\itt{H\%ne} is a scalar variable of type \integer, that
holds the number of entries in the {\bf lower triangular} part of $\bmH$
in the sparse co-ordinate storage scheme (see Section~\ref{coordinate}).
It need not be set for any of the other three schemes.

\itt{H\%val} is a rank-one allocatable array of type \realdp, that holds
the values of the entries of the {\bf lower triangular} part
of the Hessian matrix $\bmH$ in any of the
storage schemes discussed in Section~\ref{galmatrix}.

\itt{H\%row} is a rank-one allocatable array of type \integer,
that holds the row indices of the {\bf lower triangular} part of $\bmH$
in the sparse co-ordinate storage
scheme (see Section~\ref{coordinate}).
It need not be allocated for any of the other three schemes.

\itt{H\%col} is a rank-one allocatable array variable of type \integer,
that holds the column indices of the {\bf lower triangular} part of
$\bmH$ in either the sparse co-ordinate
(see Section~\ref{coordinate}), or the sparse row-wise
(see Section~\ref{rowwise}) storage scheme.
It need not be allocated when the dense or diagonal storage schemes are used.

\itt{H\%ptr} is a rank-one allocatable array of dimension {\tt n+1} and type
\integer, that holds the starting position of
each row of the {\bf lower triangular} part of $\bmH$, as well
as the total number of entries plus one, in the sparse row-wise storage
scheme (see Section~\ref{rowwise}). It need not be allocated when the
other schemes are used.

\end{description}

\ittf{G} is a rank-one allocatable array of dimension {\tt n} and type
\realdp, that holds the gradient $\bmg$ of the objective function.
The $j$-th component of {\tt G}, $j = 1,  \ldots ,  n$, contains $\bmg_{j}$.

\ittf{f} is a scalar variable of type
\realdp, that holds the value of the objective function.

\ittf{J} is scalar variable of type {\tt SMT\_TYPE}
that holds the Jacobian matrix $\bmJ(\bmx)$ (if it is available).
The following components are used here:

\begin{description}

\itt{J\%type} is an allocatable array of rank one and type default
\character, that
is used to indicate the storage scheme used. If the dense storage scheme
(see Section~\ref{dense}) is used,
the first five components of {\tt J\%type} must contain the
string {\tt DENSE}.
For the sparse co-ordinate scheme (see Section~\ref{coordinate}),
the first ten components of {\tt J\%type} must contain the
string {\tt COORDINATE},
and for the sparse row-wise storage scheme (see Section~\ref{rowwise}),
the first fourteen components of {\tt J\%type} must contain the
string {\tt SPARSE\_BY\_ROWS}.

For convenience, the procedure {\tt SMT\_put}
may be used to allocate sufficient space and insert the required keyword
into {\tt J\%type}.
For example, if {\tt nlp} is of derived type {\tt \packagename\_problem\_type}
and involves a Jacobian we wish to store using the co-ordinate scheme,
we may simply
%\vspace*{-2mm}
{\tt
\begin{verbatim}
        CALL SMT_put( nlp%J%type, 'COORDINATE' )
\end{verbatim}
}
%\vspace*{-4mm}
\noindent
See the documentation for the \galahad\ package {\tt SMT}
for further details on the use of {\tt SMT\_put}.

\itt{J\%ne} is a scalar variable of type \integer, that
holds the number of entries in $\bmJ$
in the sparse co-ordinate storage scheme (see Section~\ref{coordinate}).
It need not be set for any of the other two schemes.

\itt{J\%val} is a rank-one allocatable array of type \realdp, that holds
the values of the entries of the Jacobian matrix $\bmJ$ in any of the
storage schemes discussed in Section~\ref{galmatrix}.

\itt{J\%row} is a rank-one allocatable array of type \integer,
that holds the row indices of $\bmJ$ in the sparse co-ordinate storage
scheme (see Section~\ref{coordinate}).
It need not be allocated for any of the other two schemes.

\itt{J\%col} is a rank-one allocatable array variable of type \integer,
that holds the column indices of $\bmJ$ in either the sparse co-ordinate
(see Section~\ref{coordinate}), or the sparse row-wise
(see Section~\ref{rowwise}) storage scheme.
It need not be allocated when the dense scheme is used.

\itt{J\%ptr} is a rank-one allocatable array of dimension {\tt m+1} and type
\integer, that holds the starting position of
each row of $\bmJ$, as well
as the total number of entries plus one, in the sparse row-wise storage
scheme (see Section~\ref{rowwise}). It need not be allocated when the
other schemes are used.

\end{description}

\ittf{C} is a rank-one allocatable array of dimension {\tt m} and type
\realdp, that holds the constraint function  $\bmc(\bmx)$.
The $i$-th component of
{\tt C}, $j = 1,  \ldots ,  m$, contains $\bmc_{j}(\bmx)$.

\ittf{C\_l} is a rank-one allocatable array of dimension {\tt m} and type
\realdp, that holds
the vector of lower bounds $\bmc\sL$ on the constraints.
The $i$-th component of {\tt C\_l}, $i = 1, \ldots , m$, contains $c_{i}\sL$.
Infinite bounds are allowed by setting the corresponding
components of {\tt C\_l} to any value smaller than {\tt -infinity},
where {\tt infinity} is a component of the control array {\tt control}
(see Section~\ref{typecontrol}).

\ittf{C\_u} is a rank-one allocatable array of dimension {\tt m} and type
\realdp, that holds
the vector of upper bounds $\bmc\sU$ on the constraints.
The $i$-th component of {\tt C\_u}, $i = 1, \ldots , m$, contains $c_{i}\sU$.
Infinite bounds are allowed by setting the corresponding
components of {\tt C\_u} to any value larger than that {\tt infinity},
where {\tt infinity} is a component of the control array {\tt control}
(see Section~\ref{typecontrol}).

\ittf{X\_l} is a rank-one allocatable array of dimension {\tt n} and type
\realdp, that holds
the vector of lower bounds $\bmx\sL$ on the the variables.
The $j$-th component of {\tt X\_l}, $j = 1, \ldots , n$,
contains $x_{j}\sL$.
Infinite bounds are allowed by setting the corresponding
components of {\tt X\_l} to any value smaller than {\tt -infinity},
where {\tt infinity} is a component of the control array {\tt control}
(see Section~\ref{typecontrol}).

\ittf{X\_u} is a rank-one allocatable array of dimension {\tt n} and type
\realdp, that holds
the vector of upper bounds $\bmx\sU$ on the variables.
The $j$-th component of {\tt X\_u}, $j = 1, \ldots , n$,
contains $x_{j}\sU$.
Infinite bounds are allowed by setting the corresponding
components of {\tt X\_u} to any value larger than that {\tt infinity},
where {\tt infinity} is a component of the control array {\tt control}
(see Section~\ref{typecontrol}).

\ittf{GL} is a rank-one allocatable array of dimension {\tt n} and type
\realdp, that holds the gradient $\bmgL$ of the Lagrangian function.
The $j$-th component of {\tt GL}, $j = 1,  \ldots ,  n$, contains $\bmgL{}_{j}$.

\ittf{X} is a rank-one allocatable array of dimension {\tt n} and type
\realdp, that holds the values $\bmx$ of the optimization variables.
The $j$-th component of {\tt X}, $j = 1,  \ldots , n$, contains $x_{j}$.

\ittf{X} is a rank-one allocatable array of dimension {\tt n} and type
\realdp, that holds the values $\bmx$ of the optimization variables.
The $j$-th component of {\tt X}, $j = 1,  \ldots , n$, contains $x_{j}$.

\ittf{Y} is a rank-one allocatable array of dimension {\tt m} and type
\realdp, that holds the values $\bmy$ of the Lagrange multipliers.
The $i$-th component of {\tt Y}, $i = 1,  \ldots , m$, contains $y_{i}$.

\ittf{Z} is a rank-one allocatable array of dimension {\tt n} and type
\realdp, that holds the values $\bmx$ of the dual variables.
The $j$-th component of {\tt Z}, $j = 1,  \ldots , n$, contains $z_{j}$.

\itt{pname} is a scalar variable of type
default \character\ and length 10, which contains the
``name'' of the problem for printing. The default ``empty'' string is
provided.

\itt{VNAMES} is a rank-one allocatable array of dimension {\tt n} and type
default \character\ and length 10, whose $j$-th entry contains the
``name'' of the $j$-th variable for printing. This is only used
if ``debug''printing {\tt control\%print\_level > 4}) is requested,
and will be ignored if the array is not allocated.

\itt{CNAMES} is a rank-one allocatable array of dimension {\tt m} and type
default \character\ and length 10, whose $i$-th entry contains the
``name'' of the $i$-th constraint for printing. This is only used
if ``debug''printing {\tt control\%print\_level > 4}) is requested,
and will be ignored if the array is not allocated.

\end{description}

%%%%%%%%%%% control type %%%%%%%%%%%

\subsubsection{The derived data type for holding control
 parameters}\label{typecontrol}
The derived data type
{\tt \packagename\_control\_type}
is used to hold controlling data. Default values may be obtained by calling
{\tt \packagename\_initialize}
(see Section~\ref{subinit}),
while components may also be changed by calling
{\tt \fullpackagename\_read\-\_spec}
(see Section~\ref{readspec}).
The components of
{\tt \packagename\_control\_type}
are:


\begin{description}

\itt{error} is a scalar variable of type \integer, that holds the
stream number for error messages. Printing of error messages in
{\tt \packagename\_solve} and {\tt \packagename\_terminate}
is suppressed if {\tt error} $\leq 0$.
The default is {\tt error = 6}.

\ittf{out} is a scalar variable of type \integer, that holds the
stream number for informational messages. Printing of informational messages in
{\tt \packagename\_solve} is suppressed if {\tt out} $< 0$.
The default is {\tt out = 6}.

\itt{print\_level} is a scalar variable of type \integer, that is used
to control the amount of informational output which is required. No
informational output will occur if {\tt print\_level} $\leq 0$. If
{\tt print\_level} $= 1$, a single line of output will be produced for each
iteration of the process. If {\tt print\_level} $\geq 2$, this output will be
increased to provide significant detail of each iteration.
The default is {\tt print\_level = 0}.

\itt{start\_print} is a scalar variable of type \integer, that specifies
the first iteration for which printing will occur in {\tt \packagename\_solve}.
If {\tt start\_print} is negative, printing will occur from the outset.
The default is {\tt start\_print = -1}.

\itt{stop\_print} is a scalar variable of type \integer, that specifies
the last iteration for which printing will occur in  {\tt \packagename\_solve}.
If {\tt stop\_print} is negative, printing will occur once it has been
started by {\tt start\_print}.
The default is {\tt stop\_print = -1}.

\itt{print\_gap} is a scalar variable of type \integer.
Once printing has been started, output will occur once every
{\tt print\_gap} iterations. If {\tt print\_gap} is no larger than 1,
printing will be permitted on every iteration.
The default is {\tt print\_gap = 1}.

\itt{max\_it} is a scalar variable of type \integer, that holds the
maximum number of iterations which that be allowed in {\tt \packagename\_solve}.
The default is {\tt max\_it = 100}.

\itt{max\_eval} is a scalar variable of type \integer, that holds the
maximum number of function evaluations that will be allowed in 
{\tt \packagename\_solve}.
The default is {\tt max\_eval = 10000}.

\itt{alive\_unit} is a scalar variable of type \integer.
If {\tt alive\_unit} $>$ 0, a temporary file named {\tt alive\_file} (see below)
will be created on stream number {\tt alive\_unit} on initial entry to
\solver, and execution of \solver\ will continue so
long as this file continues to exist. Thus, a user may terminate execution
simply by removing the temporary file from this unit.
If {\tt alive\_unit} $\leq$ 0, no temporary file will be created, and
execution cannot be terminated in this way.
The default is {\tt alive\_unit} $=$ 60.

\itt{update\_multipliers\_itmin} is a scalar variable of type \integer, 
that holds the smallest iteration number for which a 
Lagrange multipliers/dual variables update will be attempted.
Up until this value, only penalty parameter reductions will be allowed.
The default is {\tt update\_multipliers\_itmin = 0}.

\itt{update\_multipliers\_tol} is a scalar variable of type \realdp, that is 
used to specify the minimum value the dual infeasibility is allowed to be
before Lagrange multipliers/dual variables updates will be attempted.
The default is {\tt update\_multipliers\_tol =} $10^{19}$.

\itt{infinity} is a scalar variable of type \realdp, that is used to
specify which constraint bounds are infinite.
Any bound larger than {\tt infinity} in modulus will be regarded as infinite.
The default is {\tt infinity =} $10^{19}$.

\itt{stop\_abs\_p} and {\tt stop\_rel\_p}
are scalar variables of type \realdp, that hold the
required absolute and relative accuracy for the primal infeasibility
(see Section~\ref{galmethod}).
The absolute value of each component of the primal infeasibility
on exit is required to be smaller than the larger of {\tt stop\_abs\_p} and
{\tt stop\_rel\_p} times a ``typical value'' for this component.
The defaults are {\tt stop\_abs\_p =} {\tt stop\_rel\_p =} $u^{1/3}$,
where $u$ is {\tt EPSILON(1.0)} ({\tt EPSILON(1.0D0)} in
{\tt \fullpackagename\_double}).

\itt{stop\_abs\_d} and {\tt stop\_rel\_d}
are scalar variables of type \realdp, that hold the
required absolute and relative accuracy for the dual infeasibility
(see Section~\ref{galmethod}).
The absolute value of each component of the dual infeasibility
on exit is required to be smaller than the larger of {\tt stop\_abs\_p} and
{\tt stop\_rel\_p} times a ``typical value'' for this component.
The defaults are {\tt stop\_abs\_d =} {\tt stop\_rel\_d =} $u^{1/3}$,
where $u$ is {\tt EPSILON(1.0)} ({\tt EPSILON(1.0D0)} in
{\tt \fullpackagename\_double}).

\itt{stop\_abs\_c} and {\tt stop\_rel\_c}
are scalar variables of type \realdp, that hold the required absolute and 
relative accuracy for the violation of complementary slackness
(see Section~\ref{galmethod}).
The absolute value of each component of the complementary slackness
on exit is required to be smaller than the larger of {\tt stop\_abs\_p} and
{\tt stop\_rel\_p} times a ``typical value'' for this component.
The defaults are {\tt stop\_abs\_c =} {\tt stop\_rel\_c =} $u^{1/3}$,
where $u$ is {\tt EPSILON(1.0)} ({\tt EPSILON(1.0D0)} in
{\tt \fullpackagename\_double}).

\itt{stop\_s} is a scalar variable of type \realdp,
that is used to specify the minimum acceptable correction step $\bms$
relative to the current estimate of the solution $\bmx$
The algorithm will be deemed to have converged if $|s_i| \leq$
{\tt stop\_s} $\ast \max( 1, |x_i|)$ for all $i = 1, \ldots, n$.
The default is {\tt stop\_s =} $u$,
where $u$ is {\tt EPSILON(1.0)} ({\tt EPSILON(1.0D0)} in
{\tt \fullpackagename\_double}).

\itt{initial\_mu} is a scalar variable of type \realdp, that holds
the required initial value of the penalty parameter. If
{\tt initial\_radius} $\leq 0$, the initial penalty parameter will be 
chosen automatically by \solver\
%The default is {\tt initial\_radius = - 1.0}.
The default is {\tt initial\_radius = 0.1}.

\itt{mu\_reduce} is a scalar variable of type \realdp, that holds
the amount by which the penalty parameter is reduced at the end of 
an iteration.
The default is {\tt mu\_reduce = 0.5}.

\itt{obj\_unbounded} is a scalar variable of type default \realdp, that 
specifies smallest
value of the objective function that will be tolerated before the problem
is declared to be unbounded from below.
The default is {\tt potential\_u\-nbounded =} $-u^{-2}$,
where $u$ is {\tt EPSILON(1.0)} ({\tt EPSILON(1.0D0)} in
{\tt \fullpackagename\_double}).

\itt{try\_advanced\_start} is a scalar variable of type default \realdp, that 
specifies the largest value of the KKT residuals before an advanced start 
will be attempted.
The default is {\tt try\_advanced\_start =} $10^{-2}$.

\itt{try\_sqp\_start} is a scalar variable of type default \realdp, that 
specifies the largest value of the KKT residuals before an advanced SQP start 
will be attempted.
The default is {\tt try\_sqp\_start =} $10^{-4}$.

\itt{stop\_advance\_start} is a scalar variable of type default \realdp, that 
specifies the smallest value of the KKT residuals that an advanced start 
will be attempted.
The default is {\tt stop\_advance\_start =} $10^{-8}$.

\itt{cpu\_time\_limit} is a scalar variable of type \realdp,
that is used to specify the maximum permitted CPU time. Any negative
value indicates no limit will be imposed. The default is
{\tt cpu\_time\_limit = - 1.0}.

\itt{clock\_time\_limit} is a scalar variable of type \realdp,
that is used to specify the maximum permitted elapsed system clock time.
Any negative value indicates no limit will be imposed. The default is
{\tt clock\_time\_limit = - 1.0}.

\itt{hessian\_available}
is a scalar variable of type default \logical,
that should be set \true\ if the user will provide second derivatives
(either by providing an appropriate evaluation routine to the solver
or by reverse communication, see Section~\ref{reverse}), and
\false\ if the second derivatives are not explicitly available.
The default is {\tt hessian\_available = .TRUE.}.

\itt{subproblem\_direct}
is a scalar variable of type default \logical,
that should be set \true\ if a direct (factorization) method is
desired when solving for the step, and \false\ if an iterative
method suffices.
The default is {\tt subproblem\_direct = .FALSE.}.

\itt{space\_critical} is a scalar variable of type default \logical,
that must be set \true\ if space is critical when allocating arrays
and  \false\ otherwise. The package may run faster if
{\tt space\_critical} is \false\ but at the possible expense of a larger
storage requirement. The default is {\tt space\_critical = .FALSE.}.

\itt{deallocate\_error\_fatal} is a scalar variable of type default \logical,
that must be set \true\ if the user wishes to terminate execution if
a deallocation  fails, and \false\ if an attempt to continue
will be made. The default is {\tt deallocate\_error\_fatal = .FALSE.}.

\itt{alive\_file} is a scalar variable of type default \character\ and length
30, that gives the name of the temporary file whose removal from stream number
{\tt alive\_unit} terminates execution of \solver.
The default is {\tt alive\_unit} $=$ {\tt ALIVE.d}.

\itt{prefix} is a scalar variable of type default \character\
and length 30, that may be used to provide a user-selected
character string to preface every line of printed output.
Specifically, each line of output will be prefaced by the string
{\tt prefix(2:LEN(TRIM( prefix ))-1)},
thus ignoreing the first and last non-null components of the
supplied string. If the user does not want to preface lines by such
a string, they may use the default {\tt prefix = ""}.

\itt{BSC\_control} is a scalar variable of type
{\tt BSC\_control\_type}
whose components are used to control the formation of the Hessian matrix
of the penalty function, as performed by the package
{\tt \libraryname\_BSC}.
See the specification sheet for the package
{\tt \libraryname\_BSC}
for details, and appropriate default values.

\itt{SSLS\_control} is a scalar variable of type
{\tt SSLS\_control\_type}
whose components are used to control the linear solve
aspects of the calculation, as performed by the package
{\tt \libraryname\_SSLS}.
See the specification sheet for the package
{\tt \libraryname\_SSLS}
for details, and appropriate default values.

{\tt GLTR\_control\_type}
whose components are used to control the
iterative trust-region step calculation (if any),
performed by the package
{\tt \libraryname\_GLTR}.
See the specification sheet for the package
{\tt \libraryname\_GLTR}
for details, and appropriate default values
(but note that value of
{\tt GLTR\_control\%unitm}
may be changed by \solver).

\itt{TRU\_control} is a scalar variable of type
{\tt TRU\_control\_type}
whose components are used to control the
minimization of the penalty function,
performed by the package
{\tt \libraryname\_TRU}.
See the specification sheet for the package
{\tt \libraryname\_TRU}
for details, and appropriate default values
(but note that value for
{\tt TRU\_control\%hessian\_available},
will be overridden by \solver).

\end{description}

%%%%%%%%%%% time type %%%%%%%%%%%

\subsubsection{The derived data type for holding timing
 information}\label{typetime}
The derived data type
{\tt \packagename\_time\_type}
is used to hold elapsed CPU and system clock times for the various parts
of the calculation. The components of
{\tt \packagename\_time\_type}
are:
\begin{description}
\itt{total} is a scalar variable of type default \real, that gives
 the CPU total time spent in the package.

\itt{preprocess} is a scalar variable of type \realdp, that gives
 the CPU time spent reordering the problem to standard form prior to solution.

\itt{analyse} is a scalar variable of type \realdp, that gives
 the CPU time spent analysing required matrices prior to factorization.

\itt{factorize} is a scalar variable of type \realdp, that gives
 the CPU time spent factorizing the required matrices.

\itt{solve} is a scalar variable of type \realdp, that gives
 the CPU time spent using the factors to solve relevant linear equations.

\itt{clock\_total} is a scalar variable of type default \real, that gives
 the total elapsed system clock time spent in the package.

\itt{clock\_preprocess} is a scalar variable of type \realdp, that gives
 the elapsed system clock time spent reordering the problem to standard form
prior to solution.

\itt{clock\_analyse} is a scalar variable of type \realdp, that gives
 the  elapsed system clock time spent analysing required matrices prior to
factorization.

\itt{clock\_factorize} is a scalar variable of type \realdp, that gives
 the  elapsed system clock time spent factorizing the required matrices.

\itt{clock\_solve} is a scalar variable of type \realdp, that gives
 the  elapsed system clock time spent using the factors to solve relevant
linear equations.

\end{description}

%%%%%%%%%%% inform type %%%%%%%%%%%

\subsubsection{The derived data type for holding informational
 parameters}\label{typeinform}
The derived data type
{\tt \packagename\_inform\_type}
is used to hold parameters that give information about the progress and needs
of the algorithm. The components of
{\tt \packagename\_inform\_type}
are:

\begin{description}
\itt{status} is a scalar variable of type \integer, that gives the
exit status of the algorithm.
%See Sections~\ref{galerrors} and \ref{galinfo}
See Sections~\ref{reverse} and \ref{galerrors}
for details.

\itt{alloc\_status} is a scalar variable of type \integer, that gives
the status of the last attempted array allocation or deallocation.
This will be 0 if {\tt status = 0}.

\itt{bad\_alloc} is a scalar variable of type default \character\
and length 80, that  gives the name of the last internal array
for which there were allocation or deallocation errors.
This will be the null string if {\tt status = 0}.

\itt{n\_free} is a scalar variable of type \integer, that holds the
number of variables that are free from their bounds.

\ittf{iter} is a scalar variable of type \integer, that holds the
number of iterations performed.

\itt{factorization\_status} is a scalar variable of type \integer, that
gives the return status from the matrix factorization.

\itt{max\_entries\_factors} is a scalar variable of type \longinteger, that
gives the maximum number of entries in any of the matrix factorizations
performed during the calculation.

\itt{factorization\_max} is a scalar variable of type \integer, that
gives the largest number of factorizations required during a subproblem
solution.

\itt{factorization\_integer} is a scalar variable of type default
\integer, that gives the amount of integer storage used for the matrix
factorization.

\itt{factorization\_real} is a scalar variable of type \integer,
that gives the amount of real storage used for the matrix factorization.

\itt{f\_eval} is a scalar variable of type \integer, that gives the
total number of objective function evaluations performed.

\itt{g\_eval} is a scalar variable of type \integer, that gives the
total number of objective function gradient evaluations performed.

\itt{h\_eval} is a scalar variable of type \integer, that gives the
total number of objective function Hessian evaluations performed.

\ittf{obj} is a scalar variable of type \realdp, that holds the
value of the objective function at the best estimate of the solution found.

\itt{primal\_infeasibility} is a scalar variable of type \realdp,
that holds the norm of the violation of primal optimality
(see Section~\ref{typetime}) at the best estimate of the solution found.

\itt{dual\_infeasibility} is a scalar variable of type \realdp,
that holds the norm of the violation of dual optimality
(see Section~\ref{typetime}) at the best estimate of the solution found.

\itt{complementary\_slackness}
is a scalar variable of type \realdp,
that holds the norm of the violation of complementary slackness
(see Section~\ref{typetime}) at the best estimate of the solution found.

\itt{mu} is a scalar variable of type \realdp, that holds
the current value of the penalty parameter.

\itt{factorization\_average} is a scalar variable of type \realdp, that
gives the average number of factorizations per subproblem solved.

\ittf{time} is a scalar variable of type {\tt \packagename\_time\_type}
whose components are used to hold elapsed elapsed CPU and system clock
times for the various parts of the calculation (see Section~\ref{typetime}).

\itt{BSC\_inform} is a scalar variable of type
{\tt BSC\_inform\_type}
whose components give information about the formation of the Hessian matrix
of the penalty function, as performed by the package
{\tt \libraryname\_BSC}.
See the specification sheet for the package
{\tt \libraryname\_BSC}
for details.

\itt{SSLS\_inform} is a scalar variable of type
{\tt SSLS\_inform\_type}
whose components give information about the progress and needs
of the linear-solve stages of the algorithm performed by the package
{\tt \libraryname\_SSLS}.
See the specification sheet for the package
{\tt \libraryname\_SSLS}
for details.

\itt{GLTR\_inform} is a scalar variable of type
{\tt GLTR\_inform\_type}
whose components give information about the progress and needs
of the iterative solution stages of the algorithm performed by the package
{\tt \libraryname\_GLTR}.
See the specification sheet for the package
{\tt \libraryname\_GLTR}
for details.

\itt{TRU\_inform} is a scalar variable of type
{\tt TRU\_inform\_type}
whose components give information about the progress and needs
of the algorithm used to minimize the penalty function, as performed
by the package
{\tt \libraryname\_TRU}.
See the specification sheet for the package
{\tt \libraryname\_TRU}
for details.

\end{description}

%%%%%%%%%%% data type %%%%%%%%%%%

\subsubsection{The derived data type for holding problem data}\label{typedata}
The derived data type
{\tt \packagename\_data\_type}
is used to hold all the data for a particular problem,
or sequences of problems with the same structure, between calls of
{\tt \packagename} procedures.
This data should be preserved, untouched (except as directed on
return from \solver\ with positive values of {\tt inform\%status}, see
Section~\ref{reverse}),
from the initial call to
{\tt \packagename\_initialize}
to the final call to
{\tt \packagename\_terminate}.

%%%%%%%%%%% userdata type %%%%%%%%%%%

\subsubsection{The derived data type for holding user data}\label{typeuserdata}
The derived data type {\tt GALAHAD\_user\-data\_type}
is available from the package {\tt GALAHAD\_user\-data}
to allow the user to pass data to and from user-supplied
subroutines for function and derivative calculations (see Section~\ref{fdv}).
Components of variables of type {\tt GALAHAD\_user\-data\_type} may be 
allocated as necessary. The following components are available:

\begin{description}
\itt{integer} is a rank-one allocatable array of type default \integer.
\ittf{real} is a rank-one allocatable array of type default  \realdp
\itt{complex} is a rank-one allocatable array of type default \complexdp.
\itt{character} is a rank-one allocatable array of type default \character.
\itt{logical} is a rank-one allocatable array of type default \logical.
\itt{integer\_pointer} is a rank-one pointer array of type default \integer.
\itt{real\_pointer} is a rank-one pointer array of type default  \realdp
\itt{complex\_pointer} is a rank-one pointer array of type default \complexdp.
\itt{character\_pointer} is a rank-one pointer array of type default \character.
\itt{logical\_pointer} is a rank-one pointer array of type default \logical.
\end{description}


%%%%%%%%%%%%%%%%%%%%%% argument lists %%%%%%%%%%%%%%%%%%%%%%%%

\galarguments
There are three procedures for user calls
(see Section~\ref{galfeatures} for further features):

\begin{enumerate}
\item The subroutine
      {\tt \packagename\_initialize}
      is used to set default values, and initialize private data,
      before solving one or more problems with the
      same sparsity and bound structure.
\item The subroutine
      {\tt \packagename\_solve}
      is called to solve the problem.
\item The subroutine
      {\tt \packagename\_terminate}
      is provided to allow the user to automatically deallocate array
       components of the private data, allocated by
       {\tt \packagename\_solve},
       at the end of the solution process.
       It is important to do this if the data object is re-used for another
       problem {\bf with a different structure}
       since {\tt \packagename\_initialize} cannot test for this situation,
       and any existing associated targets will subsequently become unreachable.
\end{enumerate}
We use square brackets {\tt [ ]} to indicate \optional\ arguments.

%%%%%% initialization subroutine %%%%%%

\subsubsection{The initialization subroutine}\label{subinit}
 Default values are provided as follows:
\vspace*{1mm}

\hspace{8mm}
{\tt CALL \packagename\_initialize( data, control, inform )}

\vspace*{-2mm}
\begin{description}

\itt{data} is a scalar \intentinout\ argument of type
{\tt \packagename\_data\_type}
(see Section~\ref{typedata}). It is used to hold data about the problem being
solved.

\itt{control} is a scalar \intentout\ argument of type
{\tt \packagename\_control\_type}
(see Section~\ref{typecontrol}).
On exit, {\tt control} contains default values for the components as
described in Section~\ref{typecontrol}.
These values should only be changed after calling
{\tt \packagename\_initialize}.

\itt{inform} is a scalar \intentout\ argument of type
{\tt \packagename\_inform\_type}
(see Section~\ref{typeinform}). A successful call to
{\tt \packagename\_initialize}
is indicated when the  component {\tt status} has the value 0.
For other return values of {\tt status}, see Section~\ref{galerrors}.

\end{description}

%%%%%%%%% main solution subroutine %%%%%%

\subsubsection{The minimization subroutine}
The minimization algorithm is called as follows:
\vspace*{1mm}

\hspace{8mm}
{\tt CALL \packagename\_solve( nlp, control, inform, data, userdata[, eval\_FC,
eval\_GJ, \hspace{8mm}                  \&}
\vspace*{-1mm}

\hspace{37mm}
{\tt
eval\_HL,  eval\_HLPROD] )}

\vspace*{-2mm}
\begin{description}
\ittf{nlp} is a scalar \intentinout\ argument of type
{\tt NLPT\_problem\_type}
(see Section~\ref{typeprob}).
It is used to hold data about the problem being solved.
For a new problem, the user must allocate all the array components,
and set values for {\tt nlp\%n}
and the required integer components of {\tt nlp\%H} if second derivatives
will be used.
Users are free to choose whichever
of the matrix formats described in Section~\ref{galmatrix}
is appropriate for $\bmH$ for their application.

\noindent
The component {\tt nlp\%X} must be set to an initial estimate, $\bmx^{0}$,
of the minimization variables. A good choice will increase the speed
of the package, but the underlying method is designed to converge (at least
to a local solution) from an arbitrary initial guess.

\noindent
On exit, the component {\tt nlp\%X} will contain the best estimates of the
minimization variables $\bmx$, while {\tt nlp\%G} will contain the best 
estimates of the dual variables $\bmz$.

\noindent
\restrictions {\tt nlp\%n} $> 0$ and
{\tt nlp\%H\%type} $\in \{
  \mbox{\tt 'DENSE'}, \mbox{\tt 'COORDINATE'}, \mbox{\tt 'SPARSE\_BY\_ROWS'},
  \mbox{\tt 'DIAGONAL'} \}$.

\itt{control} is a scalar \intentin\ argument of type
{\tt \packagename\_control\_type}
(see Section~\ref{typecontrol}). Default values may be assigned by calling
{\tt \packagename\_initialize} prior to the first call to
{\tt \packagename\_solve}. 
Note that value for {\tt TRU\_control\%hessian\_available},
will be overridden by \solver.

\itt{inform} is a scalar \intentinout\ argument of type
{\tt \packagename\_inform\_type}
(see Section~\ref{typeinform}).
On initial entry, the  component {\tt status} must be set to the value 1.
Other entries need note be set.
A successful call to
{\tt \packagename\_solve}
is indicated when the  component {\tt status} has the value 0.
For other return values of {\tt status}, see Sections~\ref{reverse} and
\ref{galerrors}.

\itt{data} is a scalar \intentinout\ argument of type
{\tt \packagename\_data\_type}
(see Section~\ref{typedata}). It is used to hold data about the problem being
solved. With the possible exceptions of the components
{\tt eval\_status} and {\tt U} (see Section~\ref{reverse}),
it must not have been altered {\bf by the user} since the last call to
{\tt \packagename\_initialize}.

\itt{userdata} is a scalar \intentinout\ argument of type
{\tt GALAHAD\_userdata\_type} whose components may be used
to communicate user-supplied data to and from the
\optional\ subroutines
{\tt eval\_FC}, {\tt eval\_GJ},
{\tt eval\_HL} and {\tt eval\_HLPROD}
(see Section~\ref{typeuserdata}).

\itt{eval\_FC} is an \optional\
user-supplied subroutine whose purpose is to evaluate the value of the
objective function $f(\bmx)$ and constraints $\bmc(\bmx)$ at a given 
vector $\bmx$. See Section~\ref{fcfv} for details.
If {\tt eval\_FC} is present,
it must be declared {\tt EXTERNAL} in the calling program.
If {\tt eval\_FC} is absent, \solver\ will use reverse communication to
obtain function values (see Section~\ref{reverse}).

\itt{eval\_GJ} is an \optional\
user-supplied subroutine whose purpose is to evaluate the value of the
gradient of the objective function $\bmg(\bmx)$ in \req{g}
and the Jacobian of the constraints $\bmJ(\bmx)$ in \req{J}
at a given vector $\bmx$. See Section~\ref{gjfv} for details.
If {\tt eval\_GJ} is present,
it must be declared {\tt EXTERNAL} in the calling program.
If {\tt eval\_GJ} is absent, \solver\ will use reverse communication to
obtain gradient values (see Section~\ref{reverse}).

\itt{eval\_HL} is an \optional\
user-supplied subroutine whose purpose is to evaluate the value of the
Hessian of the Lagrangian function $\bmHL(\bmx,\bmy)$ in \req{gHL}
at a given vectors $\bmx$ and $\bmy$. See Section~\ref{hfv} for details.
If {\tt eval\_H} is present,
it must be declared {\tt EXTERNAL} in the calling program.
If {\tt eval\_H} is absent, \solver\ will use reverse communication to
obtain Hessian values (see Section~\ref{reverse}).

\itt{eval\_HLPROD} is an \optional\
user-supplied subroutine whose purpose is to evaluate the value of the
product $\bmHL(\bmx,\bmy) \bmv$ of the Hessian of the Lagrangian function
 with a given vector $\bmv$. See Section~\ref{hvfv} for details.
If {\tt eval\_HLPROD} is present,
it must be declared {\tt EXTERNAL} in the calling program.
If {\tt eval\_HLPROD} is absent, \solver\ will use reverse communication to
obtain Hessian-vector products (see Section~\ref{reverse}).

\end{description}

%%%%%%% termination subroutine %%%%%%

\subsubsection{The  termination subroutine}
All previously allocated arrays are deallocated as follows:
\vspace*{1mm}

\hspace{8mm}
{\tt CALL \packagename\_terminate( data, control, inform )}

\vspace*{-1mm}
\begin{description}

\itt{data} is a scalar \intentinout\ argument of type
{\tt \packagename\_data\_type}
exactly as for
{\tt \packagename\_solve},
which must not have been altered {\bf by the user} since the last call to
{\tt \packagename\_initialize}.
On exit, array components will have been deallocated.

\itt{control} is a scalar \intentin\ argument of type
{\tt \packagename\_control\_type}
exactly as for
{\tt \packagename\_solve}.

\itt{inform} is a scalar \intentout\ argument of type
{\tt \packagename\_inform\_type}
exactly as for
{\tt \packagename\_solve}.
Only the component {\tt status} will be set on exit, and a
successful call to
{\tt \packagename\_terminate}
is indicated when this  component {\tt status} has the value 0.
For other return values of {\tt status}, see Section~\ref{galerrors}.

\end{description}

%%%%%%%%%%%%%%%%%%%% Function and derivative values %%%%%%%%%%%%%%%%%%%%%

\subsection{Function and derivative values\label{fdv}}

%%%%%%%%%%%%%%%%%%%%%% Objective function value %%%%%%%%%%%%%%%%%%%%%%

\subsubsection{objective function and constraint values via internal evaluation\label{fcfv}}

If the argument {\tt eval\_FC} is present when calling \solver, the
user is expected to provide a subroutine of that name to evaluate the
values of the objective function $f(\bmx)$ and/or the constraints $\bmc(\bmx)$.

The routine must be specified as

\def\baselinestretch{0.8}
{\tt
\begin{verbatim}
      SUBROUTINE eval_FC( status, X, userdata, f, C )
\end{verbatim}
}
\def\baselinestretch{1.0}
\noindent whose arguments are as follows:

\begin{description}
\itt{status} is a scalar \intentout\ argument of type \integer,
that should be set to 0 if the routine has been able to evaluate
the objective function and constraints as required,
and to a non-zero value if the evaluation has not been possible.

\ittf{X} is a rank-one \intentin\ array argument of type \realdp\
whose components contain the vector $\bmx$.

\itt{userdata} is a scalar \intentinout\ argument of type
{\tt GALAHAD\_userdata\_type} whose components may be used
to communicate user-supplied data to and from the
subroutines {\tt eval\_FC}, {\tt eval\_GJ},
{\tt eval\_HL} and {\tt eval\_HLPROD}
(see Section~\ref{typeuserdata}).

\ittf{f} is an \optional\ scalar \intentout\ argument of type \realdp,
that should be set to the value of the objective function $f(\bmx)$
evaluated at the vector $\bmx$ input in {\tt X} if {\tt f} is present.

\itt{C} is an optional rank-one \intentout\ argument of type \realdp,
whose components should be set to the values of the constraints
$\bmc(\bmx)$ evaluated at the vector $\bmx$ input in {\tt X} 
if {\tt C} is present.

\end{description}

%%%%%%%%%%%%%%%%%%%%%% Gradient values %%%%%%%%%%%%%%%%%%%%%%

\subsubsection{Gradient and Jacobian values via internal evaluation\label{gjfv}}

If the argument {\tt eval\_GJ} is present when calling \solver, the
user is expected to provide a subroutine of that name to evaluate the
value of the gradient the objective function $\nabla_x f(\bmx)$.
The routine must be specified as

\def\baselinestretch{0.8}
{\tt
\begin{verbatim}
       SUBROUTINE eval_GJ( status, X, userdata, G, J\_val )
\end{verbatim} }
\def\baselinestretch{1.0}
\noindent whose arguments are as follows:

\begin{description}
\itt{status} is a scalar \intentout\ argument of type \integer,
that should be set to 0 if the routine has been able to evaluate
the gradient and Jacobian if required,
and to a non-zero value if the evaluation has not been possible.

\ittf{X} is a rank-one \intentin\ array argument of type \realdp\
whose components contain the vector $\bmx$.

\itt{userdata} is a scalar \intentinout\ argument of type
{\tt GALAHAD\_userdata\_type} whose components may be used
to communicate user-supplied data to and from the
subroutines {\tt eval\_FC}, {\tt eval\_GJ},
{\tt eval\_HL} and {\tt eval\_HLPROD}
(see Section~\ref{typeuserdata}).

\itt{G} is a rank-one \intentout\ argument of type \realdp,
whose components should be set to the values of the gradient
of the objective function $\nabla_x f(\bmx)$
evaluated at the vector $\bmx$ input in {\tt X} if {\tt G} is present.

\itt{J\_val} is a scalar \intentout\ argument of type \realdp,
whose components should be set to the values of the Jacobian $\bmJ(\bmx)$
evaluated at the vector $\bmx$ input in {\tt X} if {\tt J\_val} is present.
The values should be input in the same order as that in which the array 
indices were given in {\tt nlp\%J}.

\end{description}

%%%%%%%%%%%%%%%%%%%%%% Hessian values %%%%%%%%%%%%%%%%%%%%%%

\subsubsection{Hessian values via internal evaluation\label{hfv}}

If the argument {\tt eval\_HL} is present when calling \solver, the
user is expected to provide a subroutine of that name to evaluate the
values of the Hessian of the Lagrangian function $\bmHL(\bmx,\bmy)$.
The routine must be specified as

\def\baselinestretch{0.8}
{\tt
\begin{verbatim}
       SUBROUTINE eval_HL( status, X, Y, userdata, H\_val )
\end{verbatim} }
\def\baselinestretch{1.0}
\noindent whose arguments are as follows:

\begin{description}
\itt{status} is a scalar \intentout\ argument of type \integer,
that should be set to 0 if the routine has been able to evaluate
the Hessian of the Lagrangian function,
and to a non-zero value if the evaluation has not been possible.

\ittf{X} is a rank-one \intentin\ array argument of type \realdp\
whose components contain the vector $\bmx$.

\ittf{Y} is a \ rank-one \intentin\ array argument of type
\realdp\ whose components contain the vector $\bmy$.

\itt{userdata} is a scalar \intentinout\ argument of type
{\tt GALAHAD\_userdata\_type} whose components may be used
to communicate user-supplied data to and from the
subroutines {\tt eval\_FC}, {\tt eval\_GJ},
{\tt eval\_HL} and {\tt eval\_HLPROD}
(see Section~\ref{typeuserdata}).

\itt{H\_val} is a scalar \intentout\ argument of type \realdp,
whose components should be set to the values of the Hessian
of the Lagrangian function $\bmHL(\bmx,\bmy)$ in \req{gHL}
evaluated at the vectors $\bmx$ and $\bmy$  input in {\tt X} and {\tt Y}. 
The Hessian values should be input in the same order as that in which 
the array indices were given in {\tt nlp\%H}.

\end{description}

%%%%%%%%%%%%%%%%%%%%%% Hessian-vector products %%%%%%%%%%%%%%%%%%%%%%

\subsubsection{Hessian-vector products via internal evaluation\label{hvfv}}

If the argument {\tt eval\_HLPROD} is present when calling \solver, the
user is expected to provide a subroutine of that name to evaluate the
sum $\bmu + \bmHL(\bmx,\bmy) \bmv$ involving the
product of the Hessian of the Lagrangian function $\bmHL(\bmx,\bmy)$
with a given vector $\bmv$. The routine must be specified as

\def\baselinestretch{0.8}
{\tt
\begin{verbatim}
       SUBROUTINE eval_HLPROD( status, X, Y, userdata, U, V, got_h )
\end{verbatim} }
\def\baselinestretch{1.0}
\noindent whose arguments are as follows:

\begin{description}
\itt{status} is a scalar \intentout\ argument of type \integer,
that should be set to 0 if the routine has been able to evaluate the
sum $\bmu + \bmHL(\bmx,\bmy) \bmv$,
and to a non-zero value if the evaluation has not been possible.

\ittf{X} is a \ rank-one \intentin\ array argument of type
\realdp\ whose components contain the vector $\bmx$.

\ittf{Y} is a \ rank-one \intentin\ array argument of type
\realdp\ whose components contain the vector $\bmy$.

\itt{userdata} is a scalar \intentinout\ argument of type
{\tt GALAHAD\_userdata\_type} whose components may be used
to communicate user-supplied data to and from the
subroutines {\tt eval\_FC}, {\tt eval\_GJ},
{\tt eval\_HL} and {\tt eval\_HLPROD}
(see Section~\ref{typeuserdata}).

\ittf{U} is a rank-one \intentinout\ array argument of type \realdp\
whose components on input contain the vector $\bmu$ and on output the
sum $\bmu + \nabla_{xx} f(\bmx) \bmv$.

\ittf{V} is a rank-one \intentin\ array argument of type \realdp\
whose components contain the vector $\bmv$.

\ittf{got\_h} is an \optional\ scalar \intentin\ argument of type default
\logical. If the Hessian has already been evaluated at the current 
and $\bmy$, {\tt got\_h} will be \present\ and set \true; if this is the 
first time the Hessian is to be accessed at $\bmx$, either {\tt got\_h} 
will be absent
or \present\ and set \false. This gives the user the opportunity
to reuse ``start-up'' computations required for the first instance of
$\bmx$ to speed up subsequent products.

\end{description}


%%%%%%%%%%%%%%%%%%%%%% Reverse communication %%%%%%%%%%%%%%%%%%%%%%%%

\subsection{\label{reverse}Reverse Communication Information}

A positive value of {\tt inform\%status} on exit from
{\tt \packagename\_solve}
indicates that
\solver\ is seeking further information---this will happen
if the user has chosen not to evaluate function or
derivative values internally (see Section~\ref{fdv}).
The user should compute the required information and re-enter \solver\
with {\tt inform\%status} and all other arguments (except those specifically
mentioned below) unchanged.

Possible values of {\tt inform\%status} and the information required are
\begin{description}
\ittf{2.} The user should compute the objective function value $f(\bmx)$ 
     and the constraint values $\bmc(\bmx)$ at the point $\bmx$ indicated 
     in {\tt nlp\%X}.
     The required values should be set in {\tt nlp\%f} and {\tt nlp\%c}, and
     {\tt data\%eval\_status} should be set to 0. If the user is
     unable to evaluate $f(\bmx)$ or $\bmc(\bmx)$---for instance, if the 
     function is undefined at $\bmx$---the user need not set {\tt nlp\%f}
     or $\bmc(\bmx)$, but
     should then set {\tt data\%eval\_status} to a non-zero value.

\ittf{3.} The user should compute the gradient $\bmg(\bmx)$ of the 
     objective function and the Jacobian $\bmJ(\bmx)$ of the constraints
     at the point $\bmx$ indicated in {\tt nlp\%X}.
     The value of the $i$-th component of the gradient should be set
     in {\tt nlp\%G(i)}, for $i = 1, \ldots, n$ and those for the
     Jacobian should be set in  {\tt nlp\%J\%val}
     (in the same order as that in which the array indices were given 
     in {\tt nlp\%J},  and {\tt data\%eval\_status} should be set to 0. 
     If the user is unable to evaluate a component of $\bmg(\bmx)$ or 
     $\bmJ(\bmx)$---for  instance, if a component of the gradient is
     undefined at $\bmx$---the user need not set {\tt nlp\%G} and 
     {\tt nlp\%J\%val}, but  should then set {\tt data\%eval\_status} 
     to a non-zero value.

\ittf{4.} The user should compute the Hessian $\bmHL(\bmx,\bmy)$
     of the Lagrangian function at the point $\bmx$ and $\bmy$
     indicated in {\tt nlp\%X} and {\tt nlp\%Y}, respectively.
     The value $l$-th component of the Hessian stored according to the
     scheme input in the remainder of {\tt nlp\%H} (see Section~\ref{typeprob})
     should be set in {\tt nlp\%H\%val(l)}, for $l = 1, \ldots,$ 
     {\tt nlp\%H\%ne}, and {\tt data\%eval\_status} should be set to 0. 
     If the user is  unable to evaluate a component of $\bmHL(\bmx,\bmy)$---for
     instance, if a component of the Hessian is
     undefined at $\bmx$ and $\bmy$---the user need not set
     {\tt nlp\%H\%val}, but should then set {\tt data\%eval\_status} 
     to a non-zero value.

\ittf{5.} The user should compute the product $\bmHL(\bmx,\bmy) \bmv$ of the 
     Hessian of the Lagrangian function $\bmHL(\bmx,\bmy)$ at the point 
     $\bmx$ and $\bmy$ indicated in {\tt nlp\%X}  and {\tt nlp\%Y} with the 
     vector $\bmv$ and add the result to
     the vector $\bmu$. The vectors $\bmu$ and $\bmv$ are given in
     {\tt data\%U} and {\tt data\%V} respectively, the resulting
     vector $\bmu + \bmHL(\bmx,\bmy)\bmv$ should be set in {\tt data\%U}
     and  {\tt data\%eval\_status} should be set to 0. If the user is
     unable to evaluate the product---for instance, if a component of the 
     Hessian is undefined at $\bmx$ and $\bmy$---the user need not set 
    {\tt nlp\%H\%val}, but should then set {\tt data\%eval\_status} to a 
    non-zero value.
\end{description}


%%%%%%%%%%%%%%%%%%%%%% Warning and error messages %%%%%%%%%%%%%%%%%%%%%%%%

\galerrors
A negative value of {\tt inform\%status} on exit from
{\tt \packagename\_solve}
or
{\tt \packagename\_terminate}
indicates that an error has occurred. No further calls should be made
until the error has been corrected. Possible values are:

\begin{description}

\itt{\galerrallocate.} An allocation error occurred.
A message indicating the offending
array is written on unit {\tt control\%error}, and the returned allocation
status and a string containing the name of the offending array
are held in {\tt inform\%alloc\_\-status}
and {\tt inform\%bad\_alloc}, respectively.

\itt{\galerrdeallocate.} A deallocation error occurred.
A message indicating the offending
array is written on unit {\tt control\%error} and the returned allocation
status and a string containing the name of the offending array
are held in {\tt inform\%alloc\_\-status}
and {\tt inform\%bad\_alloc}, respectively.

\itt{\galerrrestrictions.}
  The restriction {\tt nlp\%n} $> 0$
  or requirement that {\tt nlp\%H\_type} contains its relevant string
  {\tt 'DENSE'}, {\tt 'COORDINATE'}, {\tt 'SPARSE\_BY\_ROWS'}
  or {\tt 'DIAGONAL'}
  has been violated.

%\itt{-3.} At least one of the arrays
% {\tt p\%A\_val}, {\tt p\%A\_row}, {\tt p\%A\_col},
% {\tt p\%H\_val}, {\tt p\%H\_row} or {\tt p\%H\_col},
% is not large enough to hold the original, or reordered, matrices $\bmA$
% or $\bmH$.

\itt{\galerrbadbounds.} The bound constraints are inconsistent.

\itt{\galerrunbounded.}  The objective function appears to be unbounded
 from below on the feasible set.

\itt{\galerranalysis.} The analysis phase of the factorization failed;
 the return status from the factorization
    package is given in the component {\tt inform\%fac\-t\-or\_status}.

\itt{\galerrfactorization.} The factorization failed;
 the return status from the factorization
    package is given in the component {\tt inform\%fac\-t\-or\_status}.

\itt{\galerrsolve.} The solution of a set of linear equations
 using factors from the factorization package failed;
 the return status from the factorization
    package is given in the component {\tt inform\%fac\-t\-or\_status}.

\itt{\galerrpreconditioner.} The preconditioner $\bmP(\bmx)$ appears not to
  be positive definite.

\itt{\galerrillconditioned.} The problem is so ill-conditioned that
  further progress is impossible.

\itt{\galerrtinystep.} The step is too small to make further impact.

\itt{\galerrmaxiterations.} Too many iterations have been performed.
  This may happen if
    {\tt control\%maxit} is too small, but may also be symptomatic of
    a badly scaled problem.

\itt{\galerrcpulimit.} The elapsed CPU or system clock time limit has been
    reached. This may happen if either {\tt control\%cpu\_time\_limit} or
    {\tt control\%clock\_time\_limit} is too small, but may also be symptomatic
    of a badly scaled problem.

\itt{\galerralive.} The user has forced termination of \solver\
     by removing the file named {\tt control\%a\-live\_file} from
     unit {\tt control\%alive\_unit}.

\end{description}

%%%%%%%%%%%%%%%%%%%%%% Further features %%%%%%%%%%%%%%%%%%%%%%%%

\galfeatures
\noindent In this section, we describe an alternative means of setting
control parameters, that is components of the variable {\tt control} of type
{\tt \packagename\_control\_type}
(see Section~\ref{typecontrol}),
by reading an appropriate data specification file using the
subroutine {\tt \packagename\_read\_specfile}. This facility
is useful as it allows a user to change  {\tt \packagename} control parameters
without editing and recompiling programs that call {\tt \packagename}.

A specification file, or specfile, is a data file containing a number of
"specification commands". Each command occurs on a separate line,
and comprises a "keyword",
which is a string (in a close-to-natural language) used to identify a
control parameter, and
an (optional) "value", which defines the value to be assigned to the given
control parameter. All keywords and values are case insensitive,
keywords may be preceded by one or more blanks but
values must not contain blanks, and
each value must be separated from its keyword by at least one blank.
Values must not contain more than 30 characters, and
each line of the specfile is limited to 80 characters,
including the blanks separating keyword and value.

The portion of the specification file used by
{\tt \packagename\_read\_specfile}
must start
with a "{\tt BEGIN \packagename}" command and end with an
"{\tt END}" command.  The syntax of the specfile is thus defined as follows:
\begin{verbatim}
  ( .. lines ignored by EXPO_read_specfile .. )
    BEGIN EXPO
       keyword    value
       .......    .....
       keyword    value
    END
  ( .. lines ignored by EXPO_read_specfile .. )
\end{verbatim}
where keyword and value are two strings separated by (at least) one blank.
The ``{\tt BEGIN \packagename}'' and ``{\tt END}'' delimiter command lines
may contain additional (trailing) strings so long as such strings are
separated by one or more blanks, so that lines such as
\begin{verbatim}
    BEGIN EXPO SPECIFICATION
\end{verbatim}
and
\begin{verbatim}
    END EXPO SPECIFICATION
\end{verbatim}
are acceptable. Furthermore,
between the
``{\tt BEGIN \packagename}'' and ``{\tt END}'' delimiters,
specification commands may occur in any order.  Blank lines and
lines whose first non-blank character is {\tt !} or {\tt *} are ignored.
The content
of a line after a {\tt !} or {\tt *} character is also
ignored (as is the {\tt !} or {\tt *}
character itself). This provides an easy manner to "comment out" some
specification commands, or to comment specific values
of certain control parameters.

The value of a control parameters may be of three different types, namely
integer, logical or real.
Integer and real values may be expressed in any relevant Fortran integer and
floating-point formats (respectively). Permitted values for logical
parameters are "{\tt ON}", "{\tt TRUE}", "{\tt .TRUE.}", "{\tt T}",
"{\tt YES}", "{\tt Y}", or "{\tt OFF}", "{\tt NO}",
"{\tt N}", "{\tt FALSE}", "{\tt .FALSE.}" and "{\tt F}".
Empty values are also allowed for
logical control parameters, and are interpreted as "{\tt TRUE}".

The specification file must be open for
input when {\tt \packagename\_read\_specfile}
is called, and the associated device number
passed to the routine in device (see below).
Note that the corresponding
file is {\tt REWIND}ed, which makes it possible to combine the specifications
for more than one program/routine.  For the same reason, the file is not
closed by {\tt \packagename\_read\_specfile}.

\subsubsection{To read control parameters from a specification file}
\label{readspec}

Control parameters may be read from a file as follows:
\hskip0.5in
\def\baselinestretch{0.8} {\tt
\begin{verbatim}
     CALL EXPO_read_specfile( control, device )
\end{verbatim}
}
\def\baselinestretch{1.0}

\begin{description}
\itt{control} is a scalar \intentinout argument of type
{\tt \packagename\_control\_type}
(see Section~\ref{typecontrol}).
Default values should have already been set, perhaps by calling
{\tt \packagename\_initialize}.
On exit, individual components of {\tt control} may have been changed
according to the commands found in the specfile. Specfile commands and
the component (see Section~\ref{typecontrol}) of {\tt control}
that each affects are given in Table~\vref{specfile}.

\bctable{|l|l|l|}
\hline
  command & component of {\tt control} & value type \\
\hline
  {\tt error-printout-device} & {\tt \%error} & integer \\
  {\tt printout-device} & {\tt \%out} & integer \\
  {\tt print-level} & {\tt \%print\_level} & integer \\
  {\tt start-print} & {\tt \%start\_print} & integer \\
  {\tt stop-print} & {\tt \%stop\_print} & integer \\
  {\tt iterations-between-printing} & {\tt \%print\_gap} & integer \\
  {\tt maximum-number-of-iterations} & {\tt \%max\_it} & integer \\
  {\tt maximum-number-of-evaluations} & {\tt \%max\_eval} & integer \\
  {\tt alive-device} & {\tt \%alive\_unit} & integer \\
  {\tt update-multipliers-from-iteration} & {\tt \%update\_multipliers\_itmin} & real \\
  {\tt update-multipliers-feasibility-tolerance} & {\tt \%update\_multipliers\_tol} & real \\
  {\tt infinity-value} & {\tt \%infinity} & real \\
  {\tt absolute-primal-accuracy} & {\tt \%stop\_abs\_p} & real \\
  {\tt relative-primal-accuracy} & {\tt \%stop\_rel\_p} & real \\
  {\tt absolute-dual-accuracy} & {\tt \%stop\_abs\_d} & real \\
  {\tt relative-dual-accuracy} & {\tt \%stop\_rel\_d} & real \\
  {\tt absolute-complementary-slackness-accuracy} & {\tt \%stop\_abs\_c} & real \\
  {\tt relative-complementary-slackness-accuracy} & {\tt \%stop\_rel\_c} & real \\
  {\tt minimum-relative-step-allowed} & {\tt \%stop\_s} & real \\
  {\tt initial-penalty-parameter} & {\tt     initial\_mu } & real \\
  {\tt penalty-parameter-reduction-factor} & {\tt mu\_reduce } & real \\
  {\tt minimum-objective-before-unbounded} & {\tt obj\_unbounded } & real \\
  {\tt try-advanced-start-tolerance} & {\tt     try\_advanced\_start } & real \\
  {\tt try-sqp-start-tolerance} & {\tt try\_sqp\_start } & real \\
  {\tt stop-advanced-start-tolerance} & {\tt stop\_advance\_start } & real \\
  {\tt maximum-cpu-time-limit} & {\tt \%cpu\_time\_limit} & real \\
  {\tt maximum-clock-time-limit} & {\tt \%clock\_time\_limit} & real \\
  {\tt hessian-available}  & {\tt \%hessian\_available} & logical \\
  {\tt sub-problem-direct}  & {\tt \%subproblem\_direct} & logical \\
  {\tt space-critical}   & {\tt \%space\_critical} & logical \\
  {\tt deallocate-error-fatal}   & {\tt \%deallocate\_error\_fatal} & logical \\
  {\tt alive-filename} & {\tt \%alive\_file} & character \\
\hline

\ectable{\label{specfile}Specfile commands and associated
components of {\tt control}.}

\itt{device} is a scalar \intentin argument of type \integer,
that must be set to the unit number on which the specfile
has been opened. If {\tt device} is not open, {\tt control} will
not be altered and execution will continue, but an error message
will be printed on unit {\tt control\%error}.

\end{description}

%%%%%%%%%%%%%%%%%%%%%% Information printed %%%%%%%%%%%%%%%%%%%%%%%%

\galinfo
If {\tt control\%print\_level} is positive, information about the progress
of the algorithm will be printed on unit {\tt control\-\%out}.
If {\tt control\%print\_level} $= 1$, a single line of output will be produced
for each iteration of the process.
This will include the values of the objective function and the norm of its
gradient, the ratio of actual to predicted decrease following the step, the
radius of the trust-region and the time taken so far. In addition, if
a direct solution of the subproblem has been attempted, the
Lagrange multiplier from the secular equation and the number of factorizations
used will be recorded, while if an iterative solution has been used, the
numbers of phase 1 and 2 iterations will be given.

If {\tt control\%print\_level} $\geq 2$ this
output will be increased to provide significant detail of each iteration.
This extra output includes residuals of the linear systems solved, and,
for larger values of {\tt control\%print\_level}, values of the variables
and gradients. Further details concerning the attempted solution of the models
may be obtained by increasing
{\tt control\%TRU\_control\%print\_level},
{\tt control\%SSLS\_control\%print\_level}
and
{\tt control\%GLTR\_control\%print\-\_level},
while details about factorizations are available
by increasing
{\tt control\%SSLS\_control\%print\_level}.
See the specification sheets for the packages
{\tt \libraryname\_GLTR},
{\tt \libraryname\_SSLS} and
{\tt \libraryname\_TRU}
for details.

%%%%%%%%%%%%%%%%%%%%%% GENERAL INFORMATION %%%%%%%%%%%%%%%%%%%%%%%%

\galgeneral

\galcommon None.
\galworkspace Provided automatically by the module.
\galroutines None.
\galmodules {\tt \packagename\_solve} calls the \galahad\ packages
{\tt GALAHAD\_CLOCK},
{\tt GALAHAD\_NLPT},
{\tt GALAHAD\_SY\-M\-BOLS}, \sloppy
{\tt GALAHAD\_USERDATA},
{\tt GALAHAD\_SPECFILE},
{\tt GALAHAD\_SMT},
{\tt GALAHAD\_BSC},
{\tt GALAHAD\_MOP},
{\tt GALAHAD\_SSLS},
{\tt GALAHAD\_TRU},
{\tt GALAHAD\_GLTR},
{\tt GALAHAD\_STRINGS},
{\tt GALAHAD\_SPACE},
{\tt GALAHAD\_\-NORMS},
{\tt GALAHAD\_BLAS\_interface},
and
{\tt GALAHAD\_LAPACK\_interface}.
\galio Output is under control of the arguments
 {\tt control\%error}, {\tt control\%out} and {\tt control\%print\_level}.
\galrestrictions {\tt nlp\%n} $> 0$ and
{\tt nlp\%H\_type} $\in \{${\tt 'DENSE'},
 {\tt 'COORDINATE'}, {\tt 'SPARSE\_BY\_ROWS'}, {\tt 'DIAGONAL'} $\}$.
\galportability ISO Fortran~95 + TR 15581 or Fortran~2003.
The package is thread-safe.

%%%%%%%%%%%%%%%%%%%%%% METHOD %%%%%%%%%%%%%%%%%%%%%%%%

\galmethod
To be updated \ldots
A trust-region method is used. In this, an improvement to a current
estimate of the required minimizer, $\bmx_k$ is sought by computing a
step $\bms_k$. The step is chosen to approximately minimize a model $m_k(\bms)$
of $f(\bmx_k + \bms)$ within the intersection of the bound constraints
$\bmx\sL \leq \bmx \leq \bmx\sU$ and a trust region $\|\bms_k\| \leq \Delta_k$
for some specified positive "radius" $\Delta_k$. The quality of the
resulting step $\bms_k$ is assessed by computing the "ratio"
$(f(\bmx_k) - f(\bmx_k + \bms_k))/ (m_k(\bmzero) - m_k(\bms_k))$.
The step is deemed to have succeeded if the ratio exceeds a given $\eta_s > 0$,
and in this case $\bmx_{k+1} = \bmx_k + \bms_k$. Otherwise
$\bmx_{k+1} = \bmx_k$, and the radius is reduced by powers of a given
reduction factor until it is smaller than $\|\bms_k\|$. If the ratio
is larger than  $\eta_v \geq \eta_d$, the radius will be increased so that
it exceeds $\|\bms_k\|$ by a given increase factor. The method will terminate
as soon as $\|\nabla_x f(\bmx_k)\|$ is smaller than a specified value.

Either linear or quadratic models $m_k(\bms)$ may be used. The former will be
taken as the first two terms $f(\bmx_k) + \bms^T \nabla_x f(\bmx_k)$
of a Taylor series about $\bmx_k$, while the latter uses an
approximation to the first three terms
$f(\bmx_k) + \bms^T \nabla_x f(\bmx_k) + \half
\bms^T \bmB_k \bms$,
for which $\bmB_k$ is a symmetric approximation to the Hessian
$\nabla_{xx} f(\bmx_k)$; possible approximations include the true Hessian,
limited-memory secant and sparsity approximations and a scaled identity matrix.
Normally a two-norm trust region will be used, but this may change
if preconditioning is employed.

The model minimization is carried out in two stages.
Firstly, the so-called generalized Cauchy point for the quadratic
subproblem is found---the purpose of this point is to ensure that the
algorithm converges and that the set of bounds which are satisfied as
equations at the solution is rapidly identified.  Thereafter an
improvement to the quadratic model on the face of variables predicted
to be active by the Cauchy point is sought using either a
direct approach involving factorization or an
iterative (conjugate-gradient/Lanczos) approach based on approximations
to the required solution from a so-called Krlov subspace. The direct
approach is based on the knowledge that the required solution
satisfies the linear system of equations $(\bmB_k + \lambda_k \bmI) \bms_k
= - \nabla_x f(\bmx_k)$, involving a scalar Lagrange multiplier $\lambda_k$,
on the space of inactive variables.
This multiplier is found by uni-variate root finding, using a safeguarded
Newton-like process, by {\tt \libraryname\_TRS} or {\tt \libraryname\_DPS}
(depending on the norm chosen). The iterative approach
uses {\tt \libraryname\_GLTR}, and is best accelerated by preconditioning
with good approximations to $\bmB_k$ using {\tt \libraryname\_PSLS}. The
iterative approach has the advantage that only matrix-vector products
$\bmB_k \bmv$ are required, and thus $\bmB_k$ is not required explicitly.
However when factorizations of $\bmB_k$ are possible, the direct approach
is often more efficient.
\vspace*{1mm}

The iteration is terminated as soon as the Euclidean norm of the 
projected gradient,
\disp{\|\min(\max( \bmx_k - \nabla_x f(\bmx_k), \bmx\sL), \bmx\sU) -\bmx_k\|_2,}
is sufficiently small. At such a point, $\nabla_x f(\bmx_k) = \bmz_k$, 
where the $i$-th dual variable $z_i$ is non-negative if $x_i$ is on its
lower bound $x\sL_i$, non-positive if $x_i$ is on its upper bound $x\sU_i$,
and zero if $x_i$ lies strictly between its bounds.

\galreferences
\vspace*{1mm}

\noindent
The generic bound-constrained trust-region method is described in detail in
\vspace*{1mm}

\noindent
A. R. Conn, N. I. M. Gould and Ph. L. Toint (2000).
Trust-region methods.
SIAM/MPS Series on Optimization.

%%%%%%%%%%%%%%%%%%%%%% EXAMPLE %%%%%%%%%%%%%%%%%%%%%%%%

\galexamples
Suppose we wish to minimize the objective function
$f(\bmx) = x_1^2 + x_2^2$ subject to the constraints
\disp{\arr{rl}{x_1 + x_2 \ngap & \geq 1 \\
x_1^2 + x_2^2 & \geq 1 \\
p x_1^2 + x_2^2 & \geq 9 \\
x_1^2 - x_2 & \geq 0 \\
x_2^2 - x_1 & \geq 0 \;\;\mbox{and} \\
-50 \leq x_1, x_2 & \leq 50
}}
when the parameter $p$ takes the value 9.
Starting from the initial guess $\bmx = (3,1)$, we may use the following code:

\renewcommand{\packagedir}{\galahaddir/src/forthcoming/\package}
{\tt \small
\VerbatimInput{\packageexample}
}
\noindent
Notice how the parameter $p$ is passed to the function evaluation
routines via the {\tt real} component of the derived type {\tt userdata}.
The code produces the following output:
{\tt \small
\VerbatimInput{\packageresults}
}
\noindent
\end{document}
If the Hessian is unavailable, but products of the form
$\bmu + \bmH \bmv$ are, the same problem may be solved as follows:

{\tt \small
%\VerbatimInput{\packageexampleb}
}
\noindent
Notice that storage for the Hessian is now not needed.
This produces the same output.
%following output:
%{\tt \small
%\VerbatimInput{\packageresultsb}
%}
%\noindent

If the user prefers to provide function and gradient information and
Hessian-vector products
without calls to specified routines, the following code is appropriate.
Note the product with the user-provided preconditioner
\disp{
\bmP(\bmx) = \mat{ccc}{\half & 0 & 0 \\ 0 & \half & 0 \\ 0 & 0 & \quarter }
}
which is a suitable approximation to the inverse of the Hessian:

{\tt \small
%\VerbatimInput{\packageexamplec}
}
\noindent
This produces the following output:
{\tt \small
%\VerbatimInput{\packageresultsc}
}
\noindent


\end{document}
