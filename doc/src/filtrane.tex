\documentclass{galahad}
%
%  FILTRANE documentation (Ph.L. Toint)
%  This version 09 VI 2003.

% set the package name

\newcommand{\package}{filtrane}
\newcommand{\packagename}{FILTRANE}
\newcommand{\fullpackagename}{\libraryname\_\packagename}
\newcommand{\sym}{\sf\small}
\newcommand{\fwbox}[1]{\fbox{\rule[-0.1cm]{0cm}{0.35cm}#1}}
\newcommand{\filtrane}{{\tt \packagename}}
%\newcommand{\galahad}{{\tt \libraryname}}
\renewcommand{\Re}{\hbox{I\hskip -1.5pt R}}
\begin{document}

\galheader

%%%%%%%%%%%%%%%%%%%%%% SUMMARY %%%%%%%%%%%%%%%%%%%%%%%%

\galsummary

\filtrane\ is a package for solving the general smooth feasibility
problem, that is the problem to find a ``feasible'' vector $x \in \Re^n$ such
that 
\[
\bmc^l \leq \bmc(\bmx) \leq \bmc^u,
\]
and
\[
\bmx^l \leq \bmx \leq \bmx^u,
\]
where $\bmc(x)$ is a smooth function from $\Re^n$ into
$\Re^m$ and where inequalities are understood componentwise.
The vectors $\bmc^l \leq \bmc^u$ and $\bmx^l \leq \bmx^u$ are $m$- and
$n$-dimensional, respectively, and may contain components equal to minus or
plus infinity. For a given $i$ between 1 and $m$, it is assumed that the
$i$-th component of either $\bmc^l$ or $\bmc^u$ is finite, while, for $j$
between 1 and $n$, the $j$-th components of both $\bmx^l$ and $\bmx^u$ are
allowed to be infinite.  In what follows, we will say that variable $j$ is
bounded if the $j$-th component of either $\bmx^l$ or $\bmx^u$ is
finite. Equalities may be specified by choosing identical lower and upper
bounds on either $\bmc$ or $\bmx$. The above framework therefore covers all
combinations of bounds with linear/nonlinear equalities and/or inequalities.

If a feasible point cannot be found, it is desired to find a local minimizer
of the Euclidean norm $\|\cdot\|$ of the constraints violations, that is to
find a local minimizer of the function
\begin{equation}\label{thetadef}
\min_x \half \| \theta(\bmx) \|^2,
\end{equation}
where we define
\[
\theta(x) \eqdef \left( \begin{array}{c}
                 \max( [\bmc^l - \bmc(\bmx)]_+, [ \bmc( \bmx ) - \bmc^u ]_+ ) \\
                 \max( [\bmx^l - \bmx]_+, [ \bmx - \bmx^u ]_+ )
                 \end{array} \right) \in \Re^p,
\]
where, for a vector $\bmy$, $[\bmy]_+ = \max[0,\bmy]$, and where all maxima
are taken componentwise.

\filtrane\ also allows the partitioning the problems's constraints and bounded
variables into $p$ (not necessarily disjoint) sets or ``groups''.
In this case, each $\theta(x)$ has $p$ components defined as
\[
\theta_{\ell}(x) \eqdef 
\left\| \begin{array}{c}
    \max( [ \bmc^l_{[\ell]} - \bmc_{[\ell]}( \bmx )]_+,
          [ \bmc_{[\ell]}( \bmx ) - \bmc^u_{[\ell]} ]_+ ) \\
    \max( [ \bmx^l_{[\ell]} - \bmx_{[\ell]}]_+,
          [ \bmx_{[\ell]} ]_+ ) - \bmx^u_{[\ell]} \\
        \end{array} \right\|
\tim{for} \ell = 1, \ldots, p, 
\]
where the subscript $[\ell]$ indicates that only components belonging to
the $\ell$-th group are considered.

%%%%%%%%%%%%%%%%%%%%%% attributes %%%%%%%%%%%%%%%%%%%%%%%%

\galattributes
\galversions{\tt  \fullpackagename\_single, \fullpackagename\_double}.

\galuses {\tt GALAHAD\_NLPT}, {\tt GALAHAD\_SPECFILE}, {\tt GALAHAD\_GLTR},
{\tt GALAHAD\_BAND}, {\tt GALAHAD\_SYMBOLS}, {\tt GALAHAD\_TOOLS}, {\tt
*NRM2}, {\tt *DOT}, {\tt *SWAP}.

\galdate May 2003.
\galorigin Ph. L. Toint, The University of Namur, Belgium.
\gallanguage Fortran~95 + TR 15581 or Fortran~2003. 

%%%%%%%%%%%%%%%%%%%%%% HOW TO USE %%%%%%%%%%%%%%%%%%%%%%%%

\galhowto

The package is available using both single and double precision reals, 
and either 32-bit or 64-bit integers. Access to the 32-bit integer,
single precision version requires the {\tt USE} statement
\medskip

\hspace{8mm} {\tt USE \fullpackagename\_single}

\medskip
\noindent
with the obvious substitution {\tt \fullpackagename\_double},
{\tt \fullpackagename\_single\_64} and 
{\tt \fullpackagename\_double\_64} for the other variants.

\noindent
If it is required to use more than one of the modules at the same time, 
the derived types
{\tt NLPT\_problem\_type}, 
{\tt \packagename\_control\_type}, 
{\tt \packagename\_inform\_type} 
and
{\tt \packagename\_data\_type}
(Section~\ref{galtypes})
and the four subroutines
{\tt \packagename\_initialize}, 
{\tt \packagename\_read\_specfile}, 
{\tt \packagename\_solve}, 
{\tt \packagename\_terminate}, 
(Section~\ref{galarguments})
must be renamed on one of the {\tt USE} statements.

%%%%%%%%%%%%%%%%%%%%%% matrix formats %%%%%%%%%%%%%%%%%%%%%%%%

\galmatrix

The constraint Jacobian matrix $\bmJ(\bmx)$ must be stored in sparse
co-ordinate format.  In this format, only the nonzero entries of the matrices
are stored. For the $l$-th entry of $\bmJ(\bmx)$, its row index $i$, column
index $j$ and value $\bmJ_{ij}$ are stored in the $l$-th components of the
integer arrays {\tt J\_row}, {\tt J\_col} and real array {\tt J\_val}.  The
order is unimportant, but the total number of entries {\tt J\_size} is also
required.

%%%%%%%%%%%%%%%%%%%%%%%%%%% kinds %%%%%%%%%%%%%%%%%%%%%%%%%%%

%%%%%%%%%%%%%%%%%%%%%% kinds %%%%%%%%%%%%%%%%%%%%%%

\galkinds
We use the terms integer and real to refer to the fortran
keywords {\tt REAL(rp\_)} and {\tt INTEGER(ip\_)}, where 
{\tt rp\_} and {\tt ip\_} are the relevant kind values for the
real and integer types employed by the particular module in use. 
The former are equivalent to
default {\tt REAL} for the single precision versions,
{\tt DOUBLE PRECISION} for the double precision cases
and quadruple-precision if 128-bit reals are available, and 
correspond to {\tt rp\_ = real32}, {\tt rp\_ = real64} and
{\tt rp\_ = real128} respectively as defined by the 
fortran {\tt iso\_fortran\_env} module.
The latter are default (32-bit) and long (64-bit) integers, and
correspond to {\tt ip\_ = int32} and {\tt ip\_ = int64},
respectively, again from the {\tt iso\_fortran\_env} module.


%%%%%%%%%%%%%%%%%%%%%% symbols %%%%%%%%%%%%%%%%%%%%%%%%

\galsymbols

The following description make use of ``symbols'' that are publicly available
in the GALAHAD\_SYMBOLS module. These symbols are conventional names given to
specific integer values, that allow a more natural specification of the
various options and parameters of the package.  Each symbol provided in the
SYMBOLS module is of the form {\tt GALAHAD\_NAME}, where {\tt NAME} is the name
of the symbol.  For clarify and conciseness, we will represent such a symbol
by {\sym GALAHAD\_NAME} (in sans-serif upper case font) in what follows. See
Section~\ref{galexample} to see how symbols may be used in the program unit
that calls the \packagename\ subroutines.

%%%%%%%%%%%%%%%%%%%%%% derived types %%%%%%%%%%%%%%%%%%%%%%%%

\galtypes

In addition to the problem data type, three derived data types are accessible
from the package.

\subsubsection{The problem type}\label{problem_type_s}

The derived data type {\tt NLPT\_problem\_type} is used to hold 
the problem: we refer the reader to the documentation of the {\tt
GALAHAD\_NLPT} module for a full description. We only consider here the
components that are of interest in conjunction with \filtrane.

%%%%%%%%%%% control type %%%%%%%%%%%
\noindent
We first review the components of the problem data type which must be set on
entry in {\tt \packagename\_solve}.
\begin{description}
\ittf{n} is a scalar variable of type \integer, 
 that holds the number of problem variables, $n$.  

\ittf{m} is a scalar variable of type \integer, 
 that holds the number of problem constraints, $m$.

\ittf{x} is a rank-one allocatable array of dimension {\tt n} and type 
\realdp, that holds the values $x$ of the problem variables at the
initial point.The $j$-th component of {\tt x}, $j = 1,  \ldots , n$, contains
$x_{j}$.

\ittf{x\_l} is a rank-one allocatable array of dimension {\tt n} and type 
\realdp, that holds
the vector of lower bounds $\bmx^l$ on the problem variables.
The $j$-th component of {\tt x\_l}, $j = 1, \ldots , n$, 
contains $\bmx^l_{j}$.
Infinite bounds are allowed by setting the corresponding 
components of {\tt x\_l} to any value smaller than that {\tt -infinity}.

\ittf{x\_u} is a rank-one allocatable array of dimension {\tt n} and type 
\realdp, that holds
the vector of upper bounds $\bmx^u$ on the problem variables.
The $j$-th component of {\tt x\_u}, $j = 1, \ldots , n$, 
contains $\bmx^u_{j}$.
Infinite bounds are allowed by setting the corresponding 
components of {\tt x\_u} to any value larger than that {\tt infinity}.

\ittf{c\_l} is a rank-one allocatable array of dimension {\tt m} and type 
\realdp, that holds the vector of lower bounds $\bmc^{l}$ 
on the general constraints. The $i$-th component of 
{\tt c\_l}, $i = 1, \ldots , m$, contains $\bmc_{i}^{l}$.
Infinite bounds are allowed by setting the corresponding 
components of {\tt c\_l} to any value smaller than {\tt -infinity}. 

\ittf{c\_u} is a rank-one allocatable array of dimension {\tt m} and type 
\realdp, that holds the vector of upper bounds $\bmc^{l}$ 
on the general constraints. The $i$-th component of 
{\tt c\_u}, $i = 1, \ldots , m$, contains $\bmc_{i}^{u}$.
Infinite bounds are allowed by setting the corresponding 
components of {\tt c\_u} to any value larger than {\tt infinity}. 

\itt{J\_size} is a scalar variable of type \integer, 
which holds the number of nonzero entries of the constraints Jacobian matrix.

\itt{J\_type} is a scalar variable of type \integer, 
which specifies the format in which the constraints Jacobian is stored.
Only the value {\sym GALAHAD\_COORDINATE} is allowed as this is the only
storage scheme provided within \filtrane.

\itt{infinity} is a scalar variable of type \realdp, which
holds the value such that real numbers with absolute value less than
{\tt infinity} are finite, all others being considered infinite.
\end{description}

\noindent
In addition, the following components must be allocated:
\begin{description}
\itt{x\_status} is a rank-one allocatable array of dimension {\tt n} and type 
\integer,

\ittf{c} is a rank-one allocatable array of dimension {\tt m} and type 
\realdp,

\ittf{y} is a rank-one allocatable array of dimension {\tt m} and type 
\realdp,

\ittf{g} is a rank-one allocatable array of dimension {\tt n} and type 
\realdp,

\itt{equation} is a rank-one allocatable array of dimension {\tt n} and type 
default \logical,

\itt{J\_val} is a rank-one allocatable array of dimension {\tt J\_size + n} and
type \realdp,

\itt{J\_row} is a rank-one allocatable array of dimension {\tt J\_size + n} and
type \integer,

\itt{J\_col} is a rank-one allocatable array of dimension {\tt J\_size + n} and
type \integer.
\end{description}
However, they need not be assigned a value.

\noindent
Additionally, the arrays {\tt x\_l}, {\tt x\_u} and {\tt x\_status} need not
be allocated if there is no bound on the problem's variables.  Similarly, the
arrays {\tt c}, {\tt c\_l}, {\tt c\_u}, {\tt y}, {\tt J\_val}, {\tt J\_row}
and {\tt J\_col} need not be allocated if there are no constraints (i.e.\ {\tt
m} $=0$), in which case the values of {\tt J\_ne} and {\tt J\_type} are also
irrelevant. Furthermore, {\tt J\_val}, {\tt J\_row} and {\tt J\_col} need not
be allocated if {\tt m} $>0$ and external Jacobian products are requested (see
Section~\ref{typecontrol}), in which case the values of {\tt J\_ne} and {\tt
J\_type} are again irrelevant.

%%%%%%%%%%% control type %%%%%%%%%%%

\subsubsection{The derived data type for holding control 
 parameters}\label{typecontrol}
The derived data type 
{\tt \packagename\_control\_type} 
is used to hold controlling data. Default values may be obtained by calling 
{\tt \packagename\_initialize} (see Section~\ref{subinit}),
while individual components may also be changed by calling 
{\tt \packagename\_read\-\_specfile}
(see Section~\ref{readspec}). 
The components of 
{\tt \packagename\_control\_type} 
are:

\begin{description}
\itt{c\_accuracy} is a scalar variable of type \realdp, that
specifies an accuracy threshold such that the \filtrane\ iteration is
successfully terminated if each constraint violation is under the threshold.
The default is {\tt c\_accuracy} = $10^{-4}$ in single precision, and
{\tt c\_accuracy} = $10^{-6}$ in double precision.

\itt{g\_accuracy} is a scalar variable of type \realdp, that
specifies an accuracy threshold such that the \filtrane\ iteration is
successfully terminated if the (possibly preconditioned) Euclidean norm
of $\nabla_x \theta$ is under the threshold.
The default is {\tt g\_accuracy} = $10^{-4}$ in single precision, and
{\tt c\_accuracy} = $10^{-6}$ in double precision.

\itt{stop\_on\_prec\_g} is a scalar variable of type default \logical, that
has the value {\tt .TRUE.} iff the preconditioned gradient must be used 
in the gradient termination test, and has the value {\tt .FALSE.} iff the
unpreconditioned gradient must be used instead.
The default is {\tt stop\_on\_prec\_g} = {\tt .TRUE.}.

\itt{stop\_on\_g\_max} is a scalar variable of type default \logical, that
has the value {\tt .TRUE.} iff the maximum norm of the gradient must be used 
in the gradient termination test, and has the value {\tt .FALSE.} iff the
Euclidean norm must be used instead. Requiring the use of the maximum norm
is only possible for {prec\_used} = {\sym GALAHAD\_NONE} or for
{prec\_used} = {\sym GALAHAD\_BANDED} with {\tt semi\_bandwdith} $= 0$.  It is
reset to .FALSE. in all other cases.
The default is {\tt stop\_on\_g\_max} = {\tt .FALSE.}.

\itt{max\_iterations} is a scalar variable of type \integer, that
determines the maximum number of iterations of the filter-trust-region
algorithm during a call to {\tt \packagename\_solve}. If negative, no upper
limit is imposed on the number of iterations.
The default is {\tt max\_iterations} = 1000.

\itt{max\_cg\_iterations} is a scalar variable of type \integer, such
that {\tt max\_cg\_iterations} times the number of problem's variables
is the maximum number of Lanczos-conjugate-gradients iterations of the 
Generalized Lanczos Trust-Region subproblems solver at each iteration of
filter-trust-region algorithm during a call to {\tt \packagename\_solve}.
The default is {\tt max\_cg\_iterations} = 15.

\itt{grouping} is a scalar variable of type \integer, that holds the
type of equations/inequalities/bounds groups required for the filter tests in
the \filtrane\ algorithm. Valid values are:
\begin{description}
\item[\sym GALAHAD\_NONE:] 
each equation is considered individually in the filter,
\item[\sym GALAHAD\_AUTOMATIC:] the automatic constraints grouping strategy 
provided by the package is to be used,
\item[\sym GALAHAD\_USER\_DEFINED:] the constraints groups are defined by 
the user and specified in {\tt group}.
\end{description}
The default is {\tt grouping} = {\sym GALAHAD\_NONE}.

\itt{nbr\_groups} is a scalar variable of type \integer, that
specifies the number of constraints groups. 
If {\tt grouping} is set to {\tt GALAHAD\_AUTOMATIC}, and {\tt
nbr\_groups} is positive, the smallest between this value and the
problem's number of variables is used as the number of groups.
If {\tt grouping} is set to {\sym GALAHAD\_AUTOMATIC}, and {\tt
nbr\_groups} is negative, the number of constraints groups is set to 
the number of constraints plus the number of bounded variables
divided by the absolute value of the {\tt nbr\_groups}.
If {\tt grouping} is set to {\sym GALAHAD\_USER\_DEFINED}, {\tt
nbr\_groups} is the number of user-defined constraints groups specified in {\tt
group}. The parameter {\tt nbr\_groups} is not referenced if {\tt
grouping} is {\sym GALAHAD\_NONE}.

\itt{group} is a pointer to a vector of type \integer\ of dimension
equal to $m$ plus the number of bounded variables. It is only referenced if
{\tt grouping} is set to {\sym GALAHAD\_USER\_DEFINED}. In this case,
it must be allocated and {\tt group(i)}, its $i$-th component, for $i$ between
1 and $m$, specifies the group to which the $i$-th constraint belongs. Its
$(m+\ell)$-th component specifies the group to which the $\ell$-th bounded
variable belong (bounded variables are numbered consecutively by increasing
index, skipping unbounded variables). All its components must be
lie between 1 and {\tt nbr\_groups}.  It is nullified by \filtrane\ if
{\tt grouping} is not set to {\sym GALAHAD\_USER\_DEFINED}.

\itt{balance\_group\_values} is a scalar variable of type default \logical, 
that has the value {\tt .TRUE.} iff the constraints values (at the initial
point) must be sorted before they are distributed into groups.
This has the effect of approximately balancing the 
constraint violations (at the initial point) between the groups. It is
relevant only if {\tt grouping} is set to {\sym GALAHAD\_AUTOMATIC}.
The default is {\tt balance\_group\_values} = {\tt .FALSE.}.

\itt{prec\_used} is a scalar variable of type \integer, that indicates
which preconditioning strategy must be used. Valid values are:
\begin{description}
\item[\sym GALAHAD\_NONE:] no preconditioning,
\item[\sym GALAHAD\_BANDED:] a band preconditioner must be used, where the
band has semi-bandwidth {\tt semi\_bandwidth} and is extracted from
the Hessian of $\theta(x)$ (and possibly modified to ensure
positive-definiteness),
\item[\sym USER\_DEFINED:] a user-defined preconditioner for the model's
Hessian must be used, which is applied outside \filtrane\ via the reverse
communication interface (see Section~\ref{galreverse}).
\end{description}
The default is {\tt prec\_used} = {\sym GALAHAD\_NONE}.

\itt{semi\_bandwidth} is a scalar variable of type \integer, that 
specifies the semi-bandwidth of the banded preconditioner, in the case where
{\tt prec\_used} is set to {\sym GALAHAD\_BANDED}. It must lie between
0 (diagonal precondtioning) and $n$ (full preconditioning).
The default is {\tt semi\_bandwidth} = 5.

\itt{external\_J\_products} is a scalar variable of type default \logical, that 
specifies whether ({\tt external\_J\_products} = {\tt .FALSE.}) the Jacobian
is passed as a matrix to \filtrane, which then computes the products of this
matrix or its transpose times any vector internally, or if these products are
to be computed outside the package via the reverse communication interface ({\tt
external\_J\_products} = {\tt .TRUE.}, see Section~\ref{galreverse}), in which
case the Jacobian is not used at all inside \filtrane.

\ittf{out} is a scalar variable of type \integer, that holds the
unit number associated with the device used for normal output.
The default is {\tt out} = 6.

\itt{errout} is a scalar variable of type \integer, that holds the
unit number associated with the device used for error ouput.
The default is {\tt errout} = 6.

\itt{print\_level} is a scalar variable of type \integer, that holds the
level of printout requested by the user. See Section~\ref{galinfo}.
The default is {\tt print\_level} = {\sym GALAHAD\_SILENT}.

\itt{start\_print} is a scalar variable of type \integer, that holds
the index of the first \filtrane\ iteration at which printing must occur.
The default is {\tt start\_print} = 0 (print from initialization on).

\itt{stop\_print} is a scalar variable of type \integer, that holds
the index of the last \filtrane\ iteration at which printing must occur. If
negative, printing does not stop once started.
The default is {\tt stop\_print} = -1 (always print once started).

\itt{model\_type} is a scalar variable of type \integer, that holds
the type of model to be used by \filtrane\ for the objective function.  Valid
values are:
\begin{description}
\item[\sym GALAHAD\_GAUSS\_NEWTON:] the Gauss-Newton model must be used,
\item[\sym GALAHAD\_NEWTON:] the full Newton model (including constraints
curvatures) must be used,
\item[\sym GALAHAD\_AUTOMATIC:] an adaptive choice is to be made by 
\filtrane\ between the Gauss-newton and Newton models, based on their
respective past performance in terms or prediction or decrease.
\end{description}
The default is {\tt model\_type} = {\sym GALAHAD\_AUTOMATIC}.

\itt{model\_inertia} is a scalar variable of type \integer, that holds
the number of past iterations to consider for determining the actual model
used. It must be at least 1 and is only relevant if {\tt model\_type}
is set to {\sym GALAHAD\_AUTOMATIC}.
The default is {\tt model\_inertia} = 5.

\itt{model\_criterion} is a scalar variable of type \integer, that
specifies the criterion to apply for the automatic model selection. Valid
values are:
\begin{description}
\item[\sym GALAHAD\_BEST\_FIT:] the model is preferred whose prediction of the
objective function value is most accurate,
\item[\sym GALAHAD\_BEST\_REDUCTION:] the model is preferred whose use leads
to a larger objective function reduction.
\end{description}
The default is {\tt model\_criterion} = {\sym GALAHAD\_BEST\_FIT}.

\itt{inequality\_penalty\_type} is a scalar variable of type \integer,
that specifies the type of penalty function used to measure constraint
violations.  Valid values are:
\begin{description}
\item[2:] the $\ell_2$ penalty function is used,
\item[3:] the $\ell_3$ penalty function is used,
\item[4:] the $\ell_4$ penalty function is used.
\end{description}
The default is {\tt inequality\_penalty\_type} = 2, which corresponds to the
use of the Euclidean norm in (\ref{thetadef}).

\itt{subproblem\_accuracy} is a scalar variable of type \integer,
that specifies the type of accuracy requirement for stopping the trust-region
subproblem solution. Valid values are
\begin{description}
\item[\sym GALAHAD\_ADAPTIVE:] the (possibly preconditioned) norm of the
residual for the model must be at most 
\begin{equation}\label{gltr_stop}
\min\left[\epsilon_1,\| r_0 \|^{\epsilon_2}\right] . \|r_0\|
\end{equation}
where $\|r_0\|$ is the (possibly preconditioned) norm of the residual at the
current iterate; 
\item[\sym GALAHAD\_FULL:] the (possibly preconditioned) norm of the residual
for the model must be at most square root of the machine precision times
the (possibly preconditioned) norm of the residual at the current iterate.
\end{description}
The default is {\tt subproblem\_accuracy} = {\sym GALAHAD\_ADAPTIVE}.

\itt{min\_gltr\_accuracy} is a scalar variable of type \realdp, that
holds the minimum relative accuracy $\epsilon_1$ in the accuracy
requirement (\ref{gltr_stop}) for the subproblem solution.
It is only relevant if {\tt subproblem\_accuracy} is set
to {\sym GALAHAD\_ADAPTIVE}, in which case it must be strictly between zero
and one. 
The default is {\tt min\_gltr\_accuracy} = 0.01.

\itt{gltr\_accuracy\_power} is a scalar variable of type \realdp, that
holds the power $\epsilon_2$ at which the current residual norm is raised in
the accuracy requirement (\ref{gltr_stop}) for the subproblem solution. 
It is only relevant if {\tt subproblem\_accuracy} is set
to {\sym GALAHAD\_ADAPTIVE}, in which case it must be positive.
The default is {\tt gltr\_accuracy\_power} = 1.0.

\itt{use\_filter} is a scalar variable of type \integer, that
specifies when the filter criterion must be used to accept new trial points.
Valid values are:
\begin{description}
\item[\sym GALAHAD\_NEVER:] the filter must not be used (resulting in a pure
trust-region method),
\item[\sym GALAHAD\_INITIAL:] the filter is used as long as trial points are
accepted, but is no longer used after a first trial point has been rejected.
\item[\sym GALAHAD\_ALWAYS:] the filter must be used at every iteration.
\end{description}
The default is {\tt use\_filter} = {\sym GALAHAD\_ALWAYS}.

\itt{filter\_sign\_restriction} is a scalar variable of type default \logical,
whose value is {\tt .TRUE.} iff the filter must be constructed by considering
the absolute value of the constraints/bounds violations.
The default is {\tt filter\_sign\_restriction} = {\tt .FALSE.}.

\itt{maximal\_filter\_size} is a scalar variable of type default \logical, that
holds the maximum number of points that the filter can hold. Once this maximum
is attained, no further point can be acceptable for the filter and the
algorithm reduces to a pure trust-region scheme.  If set to a negative value,
no upper limit is set on the number of filter entries.
The default is {\tt maximal\_filter\_size} = -1.

\itt{filter\_size\_increment} is a scalar variable of type \integer,
that holds the initial filter size (if used), and is also used as an increment
for the case where the filter capacity (in memory) must be extended.
The default is {\tt filter\_size\_increment} = 50.

\itt{remove\_dominated} is a scalar variable of type default \logical, whose
vale is {\tt .TRUE.} iff \filtrane\ is to remove the dominated filter entries.
Setting this parameter to {\tt .FALSE.} marginally speeds up inclusion of new
filter points, at the expense of increased memory requirements and a
(marginally) slower acceptance test. 
The default is {\tt remove\_dominated} = {\tt .TRUE.}.

\itt{margin\_type} is a scalar variable of type \integer, that
specifies the quantity that is used to determine the width of the filter
margin. Valid values are:
\begin{description}
\item[\sym GALAHAD\_CURRENT:] the norm of the violations at the current
iterate is used,
\item[\sym GALAHAD\_FIXED:] the norm of the violations at the filter point
itself is used,
\item[\sym GALAHAD\_SMALLEST:] the smallest of these two norms is used.
\end{description}
The default is {\tt margin\_type} = {\sym GALAHAD\_CURRENT}.

\itt{gamma\_f} is a scalar variable of type \realdp, that holds the
value of the constant defining the filter margin.
The default is {\tt gamma\_f} = 0.001.

\itt{itr\_relax} is a scalar variable of type \realdp, that holds the
value of the initial trust-region relaxation factor, that is the factor by
which the trust-region constraint is relaxed during the initial sequence of
unrestricted step.
The default is {\tt itr\_relax} = $10^{20}$.

\itt{str\_relax} is a scalar variable of type \realdp, that holds the
value of the secondary trust-region relaxation factor, that is the factor by
which the trust-region constraint is relaxed after a first restricted step has
been encountered.
The default is {\tt str\_relax} = 1000.

\itt{weak\_accept\_power} is a scalar variable of type \realdp, 
that holds the power $\alpha_2$ in the ``weak acceptance criterion'' that
accepts a trial point $x_k^+$ if
\begin{equation}\label{wa}
\theta(x_k)-\theta(x_k^+) \geq 
\alpha_1 \min\left[ 1, \theta(x_k)^{\alpha_2} \right]. 
\end{equation}
This test weakens the filter/trust-region criteria 
by also accepting steps that produces sufficient descent.  If {\tt
weak\_accept\_power} $<0$, this test is not used.
The default is {\tt weak\_accept\_power} = 2.0.

\itt{min\_weak\_accept\_factor} is a scalar variable of type \realdp, 
that holds the parameter $\alpha_1$ in (\ref{wa}).  It is only relevant
if {\tt weak\_accept\_power} $\geq 0$, in which case it must be strictly
positive.
The default is {\tt min\_weak\_accept\_factor} = 0.1.

\itt{initial\_radius} is a scalar variable of type \realdp, that
holds the initial trust-region radius. It si only relevant if
{\tt use\_filter} is different from {\sym GALAHAD\_NEVER}.
The default is {\tt initial\_radius} = 1.0.

\itt{eta\_1} is a scalar variable of type \realdp, that holds the
minimum ratio of achieved to predicted reduction
for declaring a \filtrane\ iteration successful.
The default is {\tt eta\_1} = 0.01.

\itt{eta\_2} is a scalar variable of type \realdp, that holds the
minimum ratio of achieved to predicted reduction
for declaring a \filtrane\ iteration very successful.
The default is {\tt eta\_2} = 0.9.

\itt{gamma\_0} is a scalar variable of type \realdp, that holds the
strongest factor by which the trust-region radius is decreased when the ratio
of achieved to predicted reduction is negative.
The default is {\tt gamma\_0} = 0.0625.

\itt{gamma\_1} is a scalar variable of type \realdp, that holds the
factor by which the trust-region radius is decreased when the \filtrane\
iteration is unsuccessful.
The default is {\tt gamma\_1} = 0.25.

\itt{gamma\_2} is a scalar variable of type \realdp, that holds the
factor by which the trust-region radius is increased when the \filtrane\
iteration is very successful.
The default is {\tt gamma\_2} = 2.0.

\itt{save\_best\_point} is a scalar variable of type default \logical, 
whose value is {\tt .TRUE.} iff the best point found so far must be saved to be
returned as the final iterate.  This is only relevant when
{\tt use\_filter} is different from {\sym GALAHAD\_NEVER}
and requires the storage of an additional vector of dimension $n$.
The default is {\tt save\_best\_point} = {\tt .FALSE.}.

\itt{checkpoint\_freq} is a scalar variable of type \integer, that
holds the frequency (expressed in number of iterations) at which
the current values of the problem's variables and the trust-region radius are
saved on a checkpointing file for a possible package restart. It must be
non-negative. 
The default is {\tt checkpoint\_freq} = 0 (no checkpointing).

\itt{checkpoint\_file} is a scalar variable of type default \character of
length 30, that holds the name of the file use for storing checkpointing 
information on disk.
The default is {\tt checkpoint\_file} = FILTRANE.sav.

\itt{checkpoint\_dev} is a scalar variable of type \integer, that
holds the number of the device that must be used for input/output of
checkpointing operations.
The default is {\tt checkpoint\_dev} = 55.

\itt{restart\_from\_checkpoint} is a scalar variable of type default \logical,
whose value is {\tt .TRUE.} iff the initial point and constraints values must
be read from the checkpointing file checkpoint\_file,
overriding the input value of {\tt problem\%x}.
The default is {\tt restart\_from\_checkpoint} = {\tt .FALSE.}.

\end{description}

%%%%%%%%%%% info type %%%%%%%%%%%

\subsubsection{The derived data type for holding informational
parameters}\label{typeinform}
The derived data type 
{\tt \packagename\_inform\_type} 
is used to hold parameters that give information about the progress and needs 
of the algorithm. The components of 
{\tt \packagename\_inform\_type} 
are:

\begin{description}

\itt{status} is a scalar variable of type \integer, that gives the
exit status of the algorithm. See Sections~\ref{galerrors} and \ref{galinfo}
for details.

\itt{message} is a character array of 3 lines of 80 characters each, 
containing a description of the exit condition
on exit, typically including more information
than contained in {\tt status}.
It is printed out on device {\tt errout} at the end of execution
unless {\tt print\_level} is {\sym GALAHAD\_SILENT}.

\itt{nbr\_iterations} is a scalar variable of type \integer, that
gives the the final number of \filtrane\ iterations.

\itt{nbr\_cg\_iterations} is a scalar variable of type \integer, that
gives the the final number of iterations performed by {\tt GALAHAD\_GLTR} in
solving the subproblems on all successive \filtrane\ iterations.

\itt{nbr\_c\_evaluations} is a scalar variable of type \integer, that
gives the final number of evaluations of the constraints values.

\itt{nbr\_J\_evaluations} is a scalar variable of type \integer, that
gives the final number of Jacobian evaluations.
\end{description}

%%%%%%%%%%% data type %%%%%%%%%%%

\subsubsection{The derived data type for holding problem data}\label{typedata}
The derived data type 
{\tt \packagename\_data\_type} 
is used to hold all the data for a the current problem between calls of 
{\tt \packagename} procedures. 
This data should be preserved, untouched, from the initial call to 
{\tt \packagename\_initialize}
to the final call to
{\tt \packagename\_terminate}, except for components that have to be set in
the reverse communication interface (see Section~\ref{galreverse}).

%%%%%%%%%%%%%%%%%%%%%% argument lists %%%%%%%%%%%%%%%%%%%%%%%%

\galarguments
There are four procedures for user calls
(see Section~\ref{galfeatures} for further features): 

\begin{enumerate}
\item The subroutine {\tt \packagename\_initialize} 
      is used to set default values, and initialize private data. 
\item The routine {\tt \packagename\_read\_specfile}
      is used to read the {\tt \packagename} specfile in order to
      possibly modify the algoritmic default parameters (see
      Section~\ref{readspec}). 
\item The subroutine {\tt \packagename\_solve} 
      is called to solve the problem by applying the \filtrane\ algorithm.
\item The subroutine {\tt \packagename\_terminate} 
      is provided to allow the user to automatically deallocate array 
      components of the private data, allocated by 
      {\tt \packagename}, at the end of the solution process. 
      It is important to do this if the data object is re-used for another 
      problem {\bf with a different structure}
      since {\tt \packagename\_initialize} cannot test for this situation, 
      and any existing associated targets will subsequently become unreachable.
\end{enumerate}

%%%%%% initialization subroutine %%%%%%

\subsubsection{The initialization subroutine}\label{subinit}

Default values for the control parameters are provided as follows:
\vspace*{1mm}

\hspace{8mm}
{\tt CALL \packagename\_initialize( control, inform, data )}

\begin{description}
\itt{control} is a scalar \intentout\ argument of type 
{\tt \packagename\_control\_type}
(see Section~\ref{typecontrol}). 
On exit, {\tt control} contains default values for the components as
described in Section~\ref{typecontrol}.
These values should only be changed after calling 
{\tt \packagename\_initialize}.

\itt{inform} is a scalar \intentout\ argument of type 
{\tt \packagename\_inform\_type} (see Section~\ref{typeinform}).  
A successful call to the routine {\tt \packagename\_initialize}
is indicated when the  component {\tt status} has the value 0. 
For other return values of {\tt status}, see Sections~\ref{galreverse} and
\ref{galerrors}.

\itt{data} is a scalar \intentout\ argument of type 
{\tt \packagename\_data\_type}
(see Section~\ref{typedata}). It is used to hold data about the problem being 
solved. It should never be altered by the user, except for returning values to
\packagename\_solve via the reverse communication interface (see
Section~\ref{galreverse}).
\end{description}

%%%%%%%%% main solver %%%%%%

\subsubsection{The subroutine that applies the \filtrane\ algorithm to the
problem}
The \filtrane\ solver is called as follows:
\vspace*{1mm}

\hspace{8mm}
{\tt CALL \packagename\_solve( problem, control, inform, data )}

\noindent
Such a call must always be preceded by a call to {\tt \packagename\_initialize}.

\begin{description}
\itt{problem} is a scalar \intentout\ argument of type 
{\tt NLPT\_problem\_type} that contains the problem statement. 

\noindent
On input, its
{\tt n}, {\tt m}, {\tt x}, {\tt x\_l}, {\tt x\_u}, {\tt x\_status}, {\tt c\_l},
{\tt c\_u}, {\tt J\_size}, {\tt J\_type} and {\tt infinity} components must be
set. In addition, its {\tt c}, {\tt y}, {\tt g}, {\tt equation}, {\tt J\_val},
{\tt J\_row} and {\tt J\_col} components must be allocated (see 
Section~\ref{problem_type_s}).

\noindent
On successful output, the following components of the problem data type are of
interest:
\begin{description}
\ittf{x} now contains the values of the variables at the point where
\filtrane\ was terminated,


\ittf{f} contains the value of the objective function $\theta$ at the point
{\tt x},

\ittf{g} contains the gradient of $\theta$ at the point {\tt x},
\end{description}
If {\tt problem\%m} $>0$, then
\begin{description}
\ittf{c} contains the values of the constraints at the point {\tt x},
\end{description}
If, additionally, {\tt control\%external\_Jacobian\_products} is {\tt
.FALSE.},
then
\begin{description}
\itt{J\_val} contains the values of the nonzero entries of the constraints
Jacobian at {\tt x},

\itt{J\_row} contains the row indices of the the nonzero entries of the
constraints Jacobian at {\tt x},

\itt{J\_col} contains the column indices of the the nonzero entries of the
constraints Jacobian at {\tt x}.
\end{description}
 
\itt{control} is a scalar \intentin\ argument of type 
{\tt \packagename\_control\_type}
(see Section~\ref{typecontrol}). Default values may be assigned by calling 
{\tt \packagename\_initialize} prior to the first call to 
{\tt \packagename\_solve}.

\itt{info} is a scalar \intentout\ argument of type {\tt
\packagename\_inform\_type} 
(see Section~\ref{typeinform}). A successful call to the routine
{\tt \packagename\_solve}
is indicated when the  component {\tt status} has the value 0. 
For other return values of {\tt status}, see Section~\ref{galerrors}.

\itt{data} is a scalar \intentinout\ argument of type 
{\tt \packagename\_data\_type}
(see Section~\ref{typedata}). It is used to hold data about the problem being 
solved. It must never be  altered by the user since the last call to 
any of the {\tt \packagename} routines, except for returning values to
\packagename\_solve via the reverse communication interface (see
Section~\ref{galreverse}).

\end{description}

%%%%%%% termination subroutine %%%%%%

\subsubsection{The  termination subroutine}
All previously allocated workspace arrays for \packagename\ are deallocated as
follows: 
\vspace*{1mm}

\hspace{8mm}
{\tt CALL \packagename\_terminate( control, info, data )}

\begin{description}
\itt{control} is a scalar \intentin\ argument of type 
{\tt \packagename\_control\_type}
exactly as for
{\tt \packagename\_initialize}.

\itt{info} is a scalar \intentout\ argument of type
{\tt \packagename\_inform\_type}
exactly as for
{\tt \packagename\_initialize}.
A successful call to {\tt \packagename\_terminate}
is indicated when the component {\tt status} has the value 0. 
For other return values of {\tt status}, see Section~\ref{galerrors}.

\itt{data} is a scalar \intentinout\ argument of type 
{\tt \packagename\_data\_type} 
exactly as for
{\tt \packagename\_solve},
which must not have been altered by the user since the last call to 
{\tt \packagename\_initialize}, except for returning values to
\packagename\_solve via the reverse communication interface (see
Section~\ref{galreverse}).
On exit, array components will have been deallocated.
\end{description}

\noindent
Note that a call to this routine is mandatory before {\tt \packagename\_solve}
is called for a new problem whose structure differs from the current
one.
 
%%%%%%%%%%%%%%%%%%%%%% Reverse communications %%%%%%%%%%%%%%%%%%%%%%%%

\galreverse

A positive value of {\tt info\%status} on exit from  
{\tt \packagename\_solve} indicates that the user needs to take appropriate 
action before re-entering the subroutine. Possible values are: 
 
\begin{description} 
 
\itt{1.} The user must compute, at the point given in the {\tt problem\%x},
\begin{itemize}
\item the values of the constraints, \\ 
      and place the result in {\tt problem\%c(1:problem\%m)},
\item the values of the nonzero entries of the constraints Jacobian, \\
      and place the result in {\tt problem\%J\_val(1:problem\%J\_size)},
\item the row indices of the nonzero entries of the constraints Jacobian, \\
      and place the result in {\tt problem\%J\_row(1:problem\%J\_size)},
\item the column indices of the nonzero entries of the constraints Jacobian,\\
      and place the result in {\tt problem\%J\_col(1:problem\%J\_size)}.
\end{itemize}
No other argument of \packagename\_solve may be modified 
before \packagename\_solve is called again.
This is only used at the initial (starting) point.
 
\itt{2.} The user must compute, at the point given in the {\tt problem\%x},
\begin{itemize}
\item the values of the constraints, \\ 
      and place the result in {\tt problem\%c(1:problem\%m)},
\item the values of the nonzero entries of the constraints Jacobian, \\
      and place the result in {\tt problem\%J\_val(1:problem\%J\_size)}.
\end{itemize}
No other argument of \packagename\_solve may be modified 
before \packagename\_solve is called again.
This is only possibly used at the exit of \filtrane\ when {\tt
control\%save\_best\_point} is {\tt .TRUE.}.


\itt{3, 4 and 5.} The user must compute the values of the constraints at the
point given in the {\tt problem\%x}, and place the result
in {\tt problem\%c(1:problem\%m)}.
No other argument of \packagename\_solve may be modified 
before \packagename\_solve is called again.
 
\itt{6.} The user must compute the values of the nonzero entries of the
constraints Jacobian at the point given in the {\tt problem\%x}, and place 
the result in {\tt problem\%J\_val(1:problem\%J\_size)}.
No other argument of \packagename\_solve may be modified 
before \packagename\_solve is called again.

\itt{7.} The user must compute the product $\bmJ(x) \bmv$, where
$\bmv$ is given by {\tt data\%RC\_v(1:problem\%n)} and place the result in
the vector {\tt data\%RC\_Mv(1:problem\%m)}. No other argument of 
\packagename\_solve may be modified before \packagename\_solve is called
again. 
This only occurs if {\tt control\%external\_J\_products} is {\tt .TRUE.} 
(see Section~\ref{typecontrol}).

\itt{8, 9, 10 and 11.} The user must compute the product $\bmJ(x)^T \bmv$, where
$\bmv$ is given by {\tt data\%RC\_v(1:problem\%m)} and place the result in
the vector {\tt data\%RC\_Mv(1:problem\%n)}. No other argument of 
\packagename\_solve may be modified before \packagename\_solve is called
again. 
This only occurs if {\tt control\%external\_J\_products} is {\tt .TRUE.} 
(see Section~\ref{typecontrol}).

\itt{12, 13 and 14.} The user must apply a
preconditioner for the model's Hessian to {\tt data\%RC\_Pv(1:problem\%n)} and
place the result in the same vector.  The model's Hessian is given by
\[
\frac{\alpha(\alpha - 1)}{2} \bigg\{
\bmJ(\bmx) {\rm diag}\left([\theta_c(\bmx)]_i^{\alpha-2}\right) \bmJ(\bmx)^T 
+ {\rm diag}\left( [\theta_x(\bmx)]_i^{\alpha-2} \right)
\bigg\}
+ \frac{\alpha}{2} \sum_{i=1}^m [\theta_c(\bmx)]_i^{\alpha-1}
  \nabla_{xx}^2 \bmc_i(\bmx), 
\]
where $\theta_c(x)$ contains the first $m$ components of $\theta(x)$ (the
constraints's violations) and $\theta_x(x)$ the last $n$ (the bounds'
violations), where $\alpha$ is the exponent of the inequality penalty
function used (by default: $\alpha = 2$ for the Euclidean norm, see the
information on {\tt control\%inequality\_penalty\_type} in
Section~\ref{typecontrol}), and where the last term is only present if the full
Newton model is used at the current iteration.  This last condition can be
checked by verifying that the value of {\tt data\%model\_used} is equal to
{\sym NEWTON}. If {\tt data\%model\_used} is equal to {\sym GAUSS\_NEWTON}
instead, then the last term does not appear in the expression of the Hessian.
No other argument of \packagename\_solve than {\tt data\%RC\_Pv} may be
modified  before \packagename\_solve is called again.
This only occurs if {\tt control\%prec\_used} is {\sym USER\_DEFINED} 
(see Section~\ref{typecontrol}).

\itt{15 and 16.} The user must compute the product
\[
     \sum_{i=1}^m \bmy_i \nabla_{xx} \bmc_i( \bmx ) \bmv,
\]
where 
\begin{itemize}
\item $\bmy$ is given by {\tt problem\%y(1:problem\%m)},
\item $\bmx$ is given by {\tt problem\%x(1:problem\%n)},
\item $\bmv$ is given by {\tt data\%RC\_v(1:problem\%n)},
\end{itemize}
and place the result in the vector {\tt data\%RC\_Mv(1:problem\%n)}. No other
argument of \packagename\_solve may be modified before \packagename\_solve 
is called again. The user may use the fact that {\tt data\%RC\_newx} is {\tt
.TRUE.} iff the current product request is the first that involves the same
vector {\tt problem\%x}. This only occurs if {\tt control\%model\_type} is
{\sym NEWTON} or {\sym ADAPTIVE}
(see Section~\ref{typecontrol}).
\end{description} 

%%%%%%%%%%%%%%%%%%%%%% Warning and error messages %%%%%%%%%%%%%%%%%%%%%%%%

\galerrors
A negative value of {\tt info\%status} on exit from 
{\tt \packagename\_initialize},
{\tt \packagename\_read\_specfile},
{\tt \packagename\_apply},
{\tt \packagename\_restore},
or 
{\tt \packagename\_terminate}
indicates that an error has occurred. No further calls should be made
to the four three of these routines until the error has been
corrected. Possible values are:

\begin{description}

\itt{~-1.} The memory allocation failed.

\itt{~-2.} A file intended for saving checkpointing information could not be
opened.

\itt{~-3.} An IO error occurred while saving checkpointing information on 
the relevant disk file.

\itt{~-5.} Further progress of the algorithms appears to be impossible,
although successful termination is not recognized. This may happen if the
problem is extremely ill-conditioned, if the current preconditioner is
inefficient, or if there are errors in the calculation of the constraints
Jacobian.  In the first case, it may happen that the current iterate gives a
reasonable approximation of the solution (the components of the problem data
type can then be interpreted as for successful termination).

\itt{~-6.} The maximum number of iterations has been reached and computation
terminated.

\itt{~-8.} The number of variables is non-positive.

\itt{~-9.} The number of constraints is negative.

\itt{-21.} The information contained in the checkpointing file could not
be read or does not correspond to the problem being solved.

\itt{-22.} The step could not be computed by the {\tt GALAHAD\_GLTR} procedure.

\itt{-23.} The dimension of the gradient {\tt problem\%G} is not equal to
the number of variables in the problem {\tt problem\%n}.

\itt{-24.} One of the vectors {\tt problem\%x}, {\tt problem\%x\_l}, {\tt
problem\%x\_u}, {\tt problem\%c}, {\tt problem\%c\_l}, {\tt problem\%c\_u}, 
{\tt problem\%y}, {\tt problem\%g}, {\tt problem\%J\_val}, {\tt
problem\%J\_col}, {\tt problem\%J\_row}, {\tt
problem\%equation} or {\tt problem\%x\_status} is not allocated on input,
although it should be.

\itt{-25.} The user-supplied number of groups is either negative
or exceeds the number of constraints plus the number of bounded variables.

\itt{-26.} The vector control\%group is not associated although user-defined
groups are requested.

\itt{-27.} The dimension of the vector {\tt control\%group} is different
from the sum of the number of constraints and the number of bounded
variables.

\itt{-28.} The user-supplied group index (for a constraint or a bound) is
either negative, or exceeds  {\tt control\%nbr\_groups}.

\itt{-29.} \filtrane\ was re-entered (in the reverse communication
protocol) with an invalid value for {\tt inform\%status}.

\itt{-100.} This should not happen! (If it does anyway, please
report (with problem data and specfile) to Ph. Toint. Thanks in advance.)
\end{description}

%%%%%%%%%%%%%%%%%%%%%% Further features %%%%%%%%%%%%%%%%%%%%%%%%

\galfeatures
\noindent In this section, we describe an alternative means of setting 
control parameters, that is components of the variable {\tt control} of type
{\tt \packagename\_control\_type}
(see Section~\ref{typecontrol}), 
by reading an appropriate data specification file using the
subroutine {\tt \packagename\_read\_specfile}. This facility
is useful as it allows a user to change  {\tt \packagename} control parameters 
without editing and recompiling programs that call {\tt \packagename}.

A specification file, or specfile, is a data file containing a number of 
"specification commands". Each command occurs on a separate line, 
and comprises a "keyword", 
which is a string (in a close-to-natural language) used to identify a 
control parameter, and 
an (optional) "value", which defines the value to be assigned to the given
control parameter. All keywords and values are case insensitive, 
keywords may be preceded by one or more blanks but
values must not contain blanks, and
each value must be separated from its keyword by at least one blank.
Values must not contain more than 30 characters, and 
each line of the specfile is limited to 80 characters,
including the blanks separating keyword and value.

The portion of the specification file used by {\tt \packagename\_read\_specfile}
must start with a "{\tt BEGIN \packagename}" command and end with an 
"{\tt END}" command.  The syntax of the specfile is thus defined as follows:
\begin{verbatim}
  ( .. lines ignored by FILTRANE_read_specfile .. )
    BEGIN FILTRANE
       keyword    value
       .......    .....
       keyword    value
    END 
  ( .. lines ignored by FILTRANE_read_specfile .. )
\end{verbatim}
where {\tt keyword} and {\tt value} are two strings separated by (at least) one
blank. The ``{\tt BEGIN \packagename}'' and ``{\tt END}'' delimiter command
lines  may contain additional (trailing) strings so long as such strings are 
separated by one or more blanks, so that lines such as
\begin{verbatim}
    BEGIN FILTRANE SPECIFICATION
\end{verbatim}
and
\begin{verbatim}
    END FILTRANE SPECIFICATION
\end{verbatim}
are acceptable. Furthermore, between the
``{\tt BEGIN \packagename}'' and ``{\tt END}'' delimiters,
specification commands may occur in any order.  Blank lines and
lines whose first non-blank character is {\tt !} or {\tt *} are ignored. 
The content of a line after a {\tt !} or {\tt *} character is also 
ignored (as is the {\tt !} or {\tt *}
character itself). This provides an easy manner to "comment off" some 
specification commands, or to comment specific values 
of certain control parameters.  

The value of a control parameters may be of five different types, namely
integer, logical, real, string or symbol.
Integer and real values may be expressed in any relevant Fortran integer and
floating-point formats (respectively). Permitted values for logical
parameters are "{\tt ON}", "{\tt TRUE}", "{\tt .TRUE.}", "{\tt T}", 
"{\tt YES}", "{\tt Y}", or "{\tt OFF}", "{\tt NO}",
"{\tt N}", "{\tt FALSE}", "{\tt .FALSE.}" and "{\tt F}". 
Empty values are also allowed for logical control parameters, and are
interpreted as "{\tt TRUE}". String are specified as a sequence of characters.  
A symbolic value is a special string obtained from one of the predefined
symbols of the SYMBOLS module by deleting the leading {\tt GALAHAD\_}
characters in its name. Thus, the specification command
\begin{verbatim}
   print-level SILENT
\end{verbatim}
implies that the value {\sym GALAHAD\_SILENT} is assigned to {\tt
control\%print\_level}.  This technique is intended to help expressing an
(integer) control parameter for an algorithm  in a "language" that is close to
natural (see Section~\ref{galsymbols}). 

The specification file must be open for input when {\tt
\packagename\_read\_specfile} is called, and the associated device number
passed to the routine in device (see below).  Note that the corresponding  
file is {\tt REWIND}ed, which makes it possible to combine the specifications 
for more than one program/routine.  For the same reason, the file is not
closed by {\tt \packagename\_read\_specfile}.

\subsubsection{To read control parameters from a specification file}
\label{readspec}

Control parameters may be read from a file as follows:
\hskip0.5in 
\def\baselinestretch{0.8} {\tt \begin{verbatim}
     CALL FILTRANE_read_specfile( device, control, inform )
\end{verbatim}}
\def\baselinestretch{1.0}

\begin{description}
\itt{device} is a scalar \intentin argument of type \integer,
that must be set to the unit number on which the specfile
has been opened. If {\tt device} is not open, {\tt control} will
not be altered and execution will continue, but an error message
will be printed on unit {\tt control\%error}.

\itt{control} is a scalar \intentinout argument of type 
{\tt \packagename\_control\_type} (see Section~\ref{typecontrol}). 
Default values should have already been set, perhaps by calling 
{\tt \packagename\_initialize}. On exit, individual components of {\tt
control} may have been changed according to the commands found in the
specfile. Specfile commands and  the component (see Section~\ref{typecontrol})
of {\tt control}  that each affects are given in
Tables~\ref{specfile_1}--\ref{specfile_2}.

\itt{inform} is a scalar \intentout\ argument of type 
{\tt \packagename\_inform\_type} (see Section~\ref{typeinform}).
\end{description}

\bctable{|l|l|l|} 
\hline
  command & component of {\tt control} & value type/    \\
          &                            & symbolic value \\
\hline
{\tt printout-device} & {\tt \%out} & integer \\
{\tt error-printout-device} & {\tt \%errout} & integer \\
{\tt print-level} & 
     {\tt \%print\_level} & {\tt SILENT},  \\
                  &       & {\tt TRACE},   \\
                  &       & {\tt ACTION},  \\
                  &       & {\tt DETAILS}, \\
                  &       & {\tt DEBUG},   \\
                  &       & {\tt CRAZY}    \\
{\tt start-printing-at-iteration} &  {\tt \%start\_print} & integer \\
{\tt stop-printing-at-iteration}  &  {\tt \%stop\_print} & integer \\
{\tt residual-accuracy} & {\tt \%c\_accuracy} & real \\
{\tt gradient-accuracy} & {\tt \%g\_accuracy} & real \\
{\tt stop-on-preconditioned-gradient-norm} & {\tt \%stop\_on\_prec\_g} & logical \\
{\tt stop-on-maximum-gradient-norm} & {\tt \%stop\_on\_g\_max} & logical \\
{\tt maximum-number-of-iterations} & {\tt \%max\_iterations} & logical \\
\hline
\ectable{\label{specfile_1}Specfile commands and associated 
components of {\tt control} (part 1).}
\pctable{|l|l|l|} 
\hline
  command & component of {\tt control} & value type/    \\ 
          &                            & symbolic value \\
\hline
{\tt model-type} & {\tt \%model\_type} & 
                                 {\tt GAUSS\_NEWTON},  \\
                             & & {\tt NEWTON}, \\
                             & & {\tt ADAPTIVE}, \\
{\tt automatic-model-inertia} & {\tt \%model\_inertia} & integer \\
{\tt automatic-model-criterion} & {\tt \%model\_criterion} &
                                 {\tt BEST\_FIT} \\
                             & & {\tt BEST\_REDUCTION} \\
{\tt maximum-number-of-cg-iterations} & {\tt \%max\_cg\_iterations} & logical \\
{\tt subproblem-accuracy} & {\tt \%subproblem\_accuracy} &
                                 {\tt ADAPTIVE} \\
                             & & {\tt FULL} \\
{\tt relative-subproblem-accuracy-power} & {\tt \%gltr\_accuracy\_power} &real\\
{\tt minimum-relative-subproblem-accuracy} & {\tt \%min\_gltr\_accuracy} &real\\
{\tt preconditioner-used} & {\tt \%prec\_used} &
                                 {\tt NONE} \\
                             & & {\tt BANDED} \\
{\tt semi-bandwidth-for-band-preconditioner} &{\tt \%semi\_bandwidth} &integer\\
{\tt external\_Jacobian\_products} & {\tt external\_J\_products} & logical \\
{\tt equations-grouping} & {\tt \%grouping} &
                                 {\tt NONE} \\
                             & & {\tt AUTOMATIC}\\
                             & & {\tt USER\_DEFINED} \\
{\tt number-of-groups} & {\tt \%nbr\_groups} & integer \\
{\tt balance-initial-group-values} & {\tt \%balance\_group\_values} & logical \\
{\tt use-filter} & {\tt \%use\_filter} & 
                                 {\tt NEVER} \\
                             & & {\tt INITIAL} \\
                             & & {\tt ALWAYS} \\
{\tt maximum-filter-size} & {\tt \%maximal\_filter\_size} & integer \\
{\tt filter-size-increment} & {\tt \%filter\_size\_increment} & integer \\
{\tt filter-margin-type}  & {\tt \%filter\_margin\_type} &
                                 {\tt CURRENT} \\
                             & & {\tt FIXED} \\
                             & & {\tt SMALLEST} \\
{\tt filter-margin-factor} & {\tt \%gamma\_f} & real \\
{\tt remove-dominated-entries} & {\tt \%remove\_dominated} & logical \\
{\tt weak-acceptance-power} & {\tt \%weak\_accept\_power} & real \\
{\tt minimum-weak-acceptance-factor}&{\tt \%min\_weak\_accept\_factor} & real \\
{\tt initial-radius} & {\tt \%initial\_radius} & real \\
{\tt initial-TR-relaxation-factor} & {\tt \%itr\_relax} & real \\
{\tt secondary-TR-relaxation-factor} & {\tt \%str\_relax} & real \\
{\tt minimum-rho-for-successful-iteration} & {\tt \%eta\_1} & real \\
{\tt minimum-rho-for-very-successful-iteration} & {\tt \%eta\_2} & real \\
{\tt radius-increase-factor} & {\tt \%gamma\_2} & real \\
{\tt radius-reduction-factor} & {\tt \%gamma\_1} & real \\
{\tt worst-case-radius-reduction-factor} & {\tt \%gamma\_0} & real \\
{\tt save-best-point} & {\tt \%save\_best\_point} & logical \\
{\tt checkpointing-frequency} & {\tt \%checkpoint\_freq} & integer \\
{\tt checkpointing-device} & {\tt \%checkpoint\_dev} & integer \\
{\tt checkpointing-file} & {\tt \%checkpoint\_file} & character( 30 ) \\
{\tt restart-from-checkpoint} & {\tt \%restart\_from\_checkpoint} & logical \\
\hline
\ectable{\label{specfile_2}Specfile commands and associated 
components of {\tt control} (part 2).}

%%%%%%%%%%%%%%%%%%%%%% Information printed %%%%%%%%%%%%%%%%%%%%%%%%

\galinfo
The meaning of the various {\tt control\%print\_level} values is defined as
follows:
\begin{description}
\item[\sym GALAHAD\_SILENT:] no printout is produced,
\item[\sym GALAHAD\_TRACE:] only reports a one line summary of each iteration.
     This summary includes the current values of the objective function,
the (possibly preconditioned) norm of its gradient, the ratio $\rho$ of
achieved to predicted reduction, the norm of the step and the trust-region
radius. It also reports the cumulative number of GLT iterations, the iteration
type and the number of entries currently in the filter.  

\noindent
The iteration type is a four character string whose interpretation requires
some detailed knowledge of the algorithm (see Section~\ref{galmethod} and the
references therein). 
The first character describes the model type used at the current
iteration:
\begin{description}
\itt{G :} the Gauss-Newton model was used, 
\itt{N :} the full Newton model was used.
\end{description}
The second character describes the type of step that was allowed:
\begin{description}
\itt{R :} the current step was restricted to lie in the trust region,
\itt{U :} the current step was not restricted to lie in the trust region.
\end{description}
The third character described the manner in which the {\tt GALAHAD\_GLTR}
procedure has been terminated for the step computation:
\begin{description}
\itt{I :} the stopping criterion was met for a step internal to the
trust-region,
\itt{B :} the stopping criterion was met for a step lying on the trust-region
boundary,
\itt{E :} the stopping criterion was met for a step exterior to the trust
region,
\itt{M :} the maximum number of iterations allowed for {\tt GALAHAD\_GLTR} has
been reached. 
\end{description}
The fourth character describes the result of applying the various acceptance
tests to the trial point:
\begin{description}
\itt{W :} the trial point was acceptable for the weak acceptance test
(\ref{wa}),
\itt{F :} the trial point was acceptable for the filter, but its violation
was not included in the filter,
\itt{f :} the trial point was acceptable for the filter, and its violation
was included in the filter,
\itt{S :} the trial point was accepted as very successful by the trust-region
tests,
\itt{s :} the trial point was accepted as successful by the trust-region
tests,
\itt{u :} the trial point was rejected as unsuccessful by the trust-region
tests,
\itt{U :} the trial point was rejected as very unsuccessful by the
trust-region tests.
\end{description}
\item[\sym GALAHAD\_ACTION:] additionally reports the mains steps of each
iteration,
\item[\sym GALAHAD\_DETAILS:] additionally reports the values of variables,
constraints,  Jacobian entries, gradient and step components at each
iteration,
\item[\sym GALAHAD\_DEBUG:] additionally reports LOTS of information,
including details of subprocesses within each iteration,
\item[\sym GALAHAD\_CRAZY:] reports a completely silly amount of information.
\end{description}

%%%%%%%%%%%%%%%%%%%%%% GENERAL INFORMATION %%%%%%%%%%%%%%%%%%%%%%%%

\galgeneral

\galcommon None.
\galworkspace Provided automatically by the module.
\galroutines {\tt \packagename\_solve} calls the BLAS functions {\tt *NRM2},
{\tt *DOT} and {\tt *SWAP}, where {\tt *} is {\tt S} for 
the default real version and {\tt D} for the double precision version.
\galmodules {\tt \packagename} calls the \galahad\ modules
{\tt NLPT}, {\tt GLTR}, {\tt BAND}, {\tt SYMBOLS}, {\tt SORT}, {\tt TOOLS} and
{\tt SPECFILE}. 
\galio Output is under control of the arguments
 {\tt control\%errout}, {\tt control\%out} and {\tt control\%print\_level}.
\galrestrictions {\tt problem\%n} $> 0$, 
{\tt problem\%m} $\geq  0$.
\galportability ISO Fortran~95 + TR 15581 or Fortran~2003. 
The package is thread-safe.

%%%%%%%%%%%%%%%%%%%%%% METHOD %%%%%%%%%%%%%%%%%%%%%%%%

\galmethod

The method used by \filtrane\ is iterative and combines multidimensional
filter and trust-region techniques.  Trust-region methods 
build a local (in the case of \filtrane , quadratic) approximation of
$\theta(\bmx)$ and then compute a step that decreases the value of that model
in a ``trust region'', where the model is deemed to approximate the true
function well enough. The new trial point is then compared to previous
iterates. If it provides reduction in the violation of at least one of the
(groups of) constraints, it is accepted as the new iterate, according to the
multidimensional filter acceptance criterion, or if it provides sufficient
reduction in the objective function. If the trial point is rejected,
the trust-region radius is reduced and another step computed in the smaller
region, until a trial point can be accepted.

The calculation of the step is performed using the Generalized Lanczos Trust
Region Method of Gould, Lucidi and Toint, as implemented in the \galahad\ {\tt
GLTR} module.

The package has a full reverse-communication interface. 

The use of \filtrane\ for the solution of a
problem follows one on the two sequences:

\vspace*{0.2cm}

\noindent
\fwbox{initialize} 
$\rightarrow$ \bigg[ \fwbox{solve}
$\rightarrow$ \fwbox{terminate} \bigg]
\vspace*{0.2cm}

\noindent or

\vspace*{0.2cm}
\noindent
\fwbox{initialize} 
$\rightarrow$ \bigg[ \fwbox{read specfile} 
$\rightarrow$ \bigg[ \fwbox{solve}
$\rightarrow$ \fwbox{terminate} \bigg] \bigg]
\vspace*{0.2cm}

\noindent
where the procedure's control parameter may be modified by reading the
specfile (see Section~\ref{galfeatures}). Each of the ``boxed'' steps in these
sequences corresponds to calling a specific routine of the package (see
Section~\ref{galarguments}). In the above diagrams, brackated subsequence of
steps means that they can be repeated.

\galreferences
\vspace*{1mm}

\noindent
The algorithm is described in more detail in
\vspace*{1mm}

N. Gould, S. Leyffer and Ph. L. Toint (2003),
A Multidimensional Filter Algorithm for Nonlinear Equations and
Nonlinear Least-Squares,
Technical report 03/01, The University of Namur, Belgium.

\noindent
N. Gould and Ph. L. Toint (2003).
{\sf FILTRANE}, a Fortran~95 + TR 15581 or Fortran~2003 filter-trust-region package for 
solving nonlinear feasibility problems,
Technical report 03/??, The University of Namur, Belgium.
\vspace*{1mm}

%%%%%%%%%%%%%%%%%%%%%% EXAMPLE %%%%%%%%%%%%%%%%%%%%%%%%

\galexample
Suppose that we wish to apply \filtrane\ to solve the nonlinear feasibility
problem for the constraints
\[
3\bmx_1^2 + 2 \bmx_2^3 + \bmx_1 \bmx_2 = 0,
\]
\[
\bmx_1 + \bmx_2 = 0,
\]
\[
-2 \leq \bmx_1 \leq 2 \;\;\mbox{and}\;\; -2 \leq \bmx_2 \leq 2,
\]
starting from the initial point $\bmx^T = ( 1 1 )$.
We thus have that $n=2$, $m=2$.  Computing the constraints Jacobian, we verify
that it is given by
\[
\bmJ(\bmx) =  \left( \begin{array}{cc}
              6\bmx_1 + \bmx_2   & \bmx_1+ 6\bmx_2^2  \\
              1                  & 1                  \\ 
              \end{array} \right).
\]
We may then use the following code:
{\tt \small
\VerbatimInput{\packageexample}
}

\noindent
This produces the following output:
{\tt \small
\VerbatimInput{\packageresults}
}
\noindent
We could also make use of the {\tt \packagename\_read\_specfile} routine to
set the printing level, in which case the lines
{\small
\begin{verbatim}
! Nevertheless... ask for some output:
  FILTRANE_control%print_level = GALAHAD_TRACE
\end{verbatim}
}
\noindent
are replaced by
{\small
\begin{verbatim}
! open specfile
   OPEN( 57, FILE = FILTRANE.SPC', STATUS = 'OLD' )
! read its content (asking for some output)
   CALL FILTRANE_read_specfile( 57, control, inform )
! close it
   CLOSE( 57 )
\end{verbatim}
}
\noindent
where we assume that the file {\tt FILTRANE.SPC} exists in the current
directory and contains the lines
{\small
\begin{verbatim}
BEGIN FILTRANE SPECIFICATION
   print-level           TRACE
END FILTRANE SPECIFICATION
\end{verbatim}
}
\end{document}

