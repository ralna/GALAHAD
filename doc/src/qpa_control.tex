\begin{description}

\itt{error} is a scalar variable of type default \integer, that holds the
stream number for error messages. Printing of error messages in
{\tt QPA\_solve} and {\tt QPA\_terminate}
is suppressed if {\tt error} $\leq 0$.
The default is {\tt error = 6}.

\ittf{out} is a scalar variable of type default \integer, that holds the
stream number for informational messages. Printing of informational messages in
{\tt QPA\_solve} is suppressed if {\tt out} $< 0$.
The default is {\tt out = 6}.

\itt{print\_level} is a scalar variable of type default \integer, that is used
to control the amount of informational output which is required. No
informational output will occur if {\tt print\_level} $\leq 0$. If
{\tt print\_level} $= 1$, a single line of output will be produced for each
iteration of the process. If {\tt print\_level} $\geq 2$, this output will be
increased to provide significant detail of each iteration.
The default is {\tt print\_level = 0}.

\itt{maxit} is a scalar variable of type default \integer, that holds the
maximum number of iterations which will be allowed in {\tt QPA\_solve}.
The default is {\tt maxit = 1000}.

\itt{start\_print} is a scalar variable of type default \integer, that specifies
the first iteration for which printing will occur in {\tt QPA\_solve}.
If {\tt start\_print} is negative, printing will occur from the outset.
The default is {\tt start\_print = -1}.

\itt{stop\_print} is a scalar variable of type default \integer, that specifies
the last iteration for which printing will occur in  {\tt QPA\_solve}.
If {\tt stop\_print} is negative, printing will occur once it has been
started by {\tt start\_print}.
The default is {\tt stop\_print = -1}.

\itt{factor} is a scalar variable of type default \integer, that indicates
the type of factorization of the preconditioner to be used.
Possible values are:

\begin{description}
\itt{0} the type is chosen automatically on the basis of which option looks
        likely to be the most efficient.
\itt{1} a Schur-complement factorization will be used.
\itt{2} an augmented-system factorization will be used.
\end{description}
The default is {\tt factor = 0}.

\itt{max\_col} is a scalar variable of type default \integer, that specifies
the maximum number of nonzeros in a column of $\bmA$ which is permitted
by the Schur-complement factorization.
The default is {\tt max\_col = 35}.

\itt{max\_sc} is a scalar variable of type default \integer, that specifies
the maximum number of columns permitted in the Schur complement of the
reference matrix (see Section~\ref{galmethod})
before a refactorization is triggered.
The default is {\tt max\_sc = 75}.

%\itt{indmin} is a scalar variable of type default \integer, that specifies
%an initial estimate as to the amount of integer workspace required by
%the factorization package {\tt SILS.
%The} default is {\tt indmin = 1000}.

%\itt{valmin} is a scalar variable of type default \integer, that specifies
%an initial estimate as to the amount of real workspace required by
%the factorization package {\tt SILS.
%The} default is {\tt valmin = 1000}.

\itt{itref\_max} is a scalar variable of type default \integer, that specifies
the maximum number of iterative refinements allowed with each application
of the preconditioner.
The default is {\tt itref\_max = 1}.

\itt{cg\_maxit} is a scalar variable of type default \integer, that holds the
maximum number of conjugate-gradient inner iterations that may be performed
during the computation of each search direction in {\tt QPA\_solve}.
If {\tt cg\_itmax} is set to a negative number, it will be reset by
{\tt QPA\_solve} to the dimension of the relevant linear system $+ 1$.
The default is {\tt cg\_itmax = -1}.

\itt{precon} is a scalar variable of type default \integer, that specifies
which preconditioner to be used to accelerate the conjugate-gradient
inner iteration.  Possible values are:

\begin{description}
\itt{0} the type is chosen automatically on the basis of which option looks
        likely to be the most efficient at any given stage of the solution
        process. Different preconditioners may be used at different stages.
\itt{1}  a full factorization using the Hessian, which is equivalent to
        replacing the conjugate gradient inner iteration by a direct method.
        The Hessian may be perturbed to ensure that the resultant matrix
        is a preconditioner.
\itt{2} the Hessian matrix is replaced by the identity matrix.
\itt{3} the Hessian matrix is replaced by a band of given semi-bandwidth
        (see {\tt nsemib} below).
\itt{4}  the Hessian matrix terms in the current reference matrix
        (see Section~\ref{galmethod}) are replaced by the identity matrix.
\itt{5}  the Hessian matrix terms outside a band of given semi-bandwidth
        in the current reference matrix are replaced by zeros
        (see {\tt nsemib} below).
\end{description}
The default is {\tt precon = 0}.

\itt{nsemib} is a scalar variable of type default \integer, that specifies
the semi-bandwidth of the band preconditioner when {\tt precon = 3},
if appropriate.
The default is {\tt nsemib = 5}.

\itt{full\_max\_fill} is a scalar variable of type default \integer.
If the ratio of the number of nonzeros in the factors
of the reference matrix (see Section~\ref{galmethod}) to the number of
nonzeros in the matrix itself exceeds {\tt full\_max\_fil},
and the preconditioner is being selected
automatically ({\tt precon = 0}), a banded approximation (see {\tt precon = 3})
will be used instead.
The default is {\tt full\_max\_fill = 10}.

\itt{deletion\_strategy} is a scalar variable of type default \integer, that
specifies the rules used to determine which constraint to remove
from the working set (see Section~\ref{galmethod}) when necessary to ensure
further progress towards the solution. Possible values are:
\begin{description}
\itt{0} the constraint whose Lagrange multiplier most violates its
required optimality bound will be removed.
\itt{1}the most-recently added constraint whose Lagrange multiplier violates its
required optimality bound will be removed.
\itt{k $>$ 1} among the $k$ most-recently added constraints whose
Lagrange multipliers violates their required optimality bounds,
the one which most violates its bound will be removed.
\end{description}
The default is {\tt deletion\_strategy = 0}.

\itt{restore\_problem} is a scalar variable of type default \integer, that
specifies how much of the input problem is to be retored on output.
Possible values are:
\begin{description}
\itt{0} nothing is restored.
\itt{1} the vector data $\bmg$,
   $\bmc^{l}$, $\bmc^{u}$, $\bmx^{l}$, and $\bmx^{u}$
   will be restored to their input values.
\itt{2} the entire problem, that is the above vector data along with
the Hessian matrix $\bmH$ and the Jacobian matrix $\bmA$, will be restored.
\end{description}
The default is {\tt restore\_problem = 2}.

\itt{monitor\_residuals} is a scalar variable of type default \integer, that
specifies the frequency at which working constraint residuals will be monitored
to ensure feasibility. The  residuals will be monitored every
{\tt monitor\_residu\-al} iterations.
The default is {\tt monitor\_residuals = 1}.

\itt{cold\_start} is a scalar variable of type default \integer, that controls
the initial working set (see Section~\ref{galmethod})). Possible values are:
\begin{description}
\itt{0} a "warm start" will be performed. The values set in
     {\tt C\_stat} and {\tt B\_stat} indicate which
     constraints will be included in the initial working set.
     (see {\tt C\_stat} and {\tt B\_stat} in Section~\ref{qps}).
\itt{1} the constraints "active" at {\tt p\%X} (see Section~\ref{qps})
     will determine the initial working set.
\itt{2} the initial working set will be empty.
\itt{3} the initial working set will only contain equality constraints.
\itt{4} the initial working set will contain as many active constraints as
     possible, chosen (in order) from equality constraints, simple bounds, and
     finally general linear constraints.
\end{description}
The default is {\tt cold\_start = 3}.

\itt{infeas\_check\_interval} is a scalar variable of type default \integer,
that gives the number of iterations that are permitted before the
infeasibility is checked for improvement.
The default is {\tt infeas\_check\_interval = 100}.

\itt{infinity} is a scalar variable of type default \realdp, that is used to
specify which constraint bounds are infinite.
Any bound larger than {\tt infinity} in modulus will be regarded as infinite.
The default is {\tt infinity =} $10^{19}$.

\itt{feas\_tol} is a scalar variable of type default \realdp, that specifies the
maximum amount by which a constraint may be violated and yet still be
considered to be satisfied.
The default is {\tt feas\_tol =} $u^{3/4}$,
where $u$ is {\tt EPSILON(1.0)} ({\tt EPSILON(1.0D0)} in
{\tt \fullpackagename\_dou\-ble}).

\itt{obj\_unbounded}  is a scalar variable of type default
\realdp, that specifies smallest
value of the objective function that will be tolerated before the problem
is declared to be unbounded from below.
The default is {\tt potential\_u\-nbounded =} $-u^{-2}$,
where $u$ is {\tt EPSILON(1.0)} ({\tt EPSILON(1.0D0)} in
{\tt \fullpackagename\_double}).

\itt{increase\_rho\_g\_factor} is a scalar variable of type default \realdp,
that gives the factor by which the current penalty parameter $\rho_g$
for the general constraints may be increased when solving quadratic programs.
The default is {\tt increase\_rho\_g\_factor = 2}.

\itt{increase\_rho\_b\_factor} is a scalar variable of type default \realdp,
that gives the factor by which the current penalty parameter $\rho_b$ for
the simple bound constraints may be increased when solving quadratic programs.
The default is {\tt increase\_rho\_b\_factor = 2}.

\itt{infeas\_g\_improved\_by\_factor} is a scalar variable of type default
\realdp, that specifies the relative improvement in the infeasibility
that must be achieved when solving quadratic programs if the current
value of $\rho_g$ is to be maintained.
Specifically if the infeasibility of the general constraints
has not fallen by at least a factor {\tt infeas\_g\_improved\_by\_factor}
during the previous {\tt infeas\_check\_interval} iterations, the
penalty parameter will be increased by a factor
{\tt increase\_rho\_g\_factor}.
The default is {\tt infeas\_im\-proved\_g\_by\_factor = 0.75}.

\itt{infeas\_b\_improved\_by\_factor} is a scalar variable of type default
\realdp, that specifies the relative improvement in the infeasibility
that must be achieved when solving quadratic programs if the current
value of $\rho_b$ is to be maintained.
Specifically if the infeasibility of the simple bound constraints
has not fallen by at least a factor {\tt infeas\_b\_improved\_by\_factor}
during the previous {\tt infeas\_check\_in\-terval} iterations, the
penalty parameter will be increased by a factor
{\tt increase\_rho\_b\_factor}.
The default is {\tt infeas\_im\-proved\_b\_by\_factor = 0.75}.

%\itt{pivot\_tol}  is a scalar variable of type default
%\realdp, that holds the
%threshold pivot tolerance used by the matrix factorization.  See
%the documentation for the package {\tt SILS} for details.
%The default is {\tt pivot\_tol =} $0.1 \sqrt{u}$,
%where $u$ is {\tt EPSILON(1.0)} ({\tt EPSILON(1.0D0)} in
%{\tt \fullpackagename\_double}).
%%The default is {\tt pivot\_tol = 0.01}.

\itt{pivot\_tol\_for\_dependencies} is a scalar variable of type default
\realdp, that holds the relative
threshold pivot  tolerance used by the matrix factorization when
attempting to detect linearly dependent constraints. A value larger
than  {\tt pivot\_tol} is appropriate. See
the documentation for the package {\tt SLS} for details.
The default is {\tt pivot\_tol\_for\_dependencies = 0.5}.

%\itt{zero\_pivot} is a scalar variable of type default \realdp.
%Any pivots smaller than  {\tt zero\_pivot} in absolute value will be regarded
%to be zero when attempting to detect linearly dependent constraints.
%%The default is {\tt zero\_pivot =} $u^{3/4}$,
%where $u$ is {\tt EPSILON(1.0)} ({\tt EPSILON(1.0D0)} in
%{\tt \fullpackagename\_dou\-ble}).

\itt{multiplier\_tol} is a scalar variable of type default \realdp.
Any dual variable or Lagrange multiplier which is less than
{\tt multiplier\_tol} outside its optimal interval will be regarded
as being acceptable when checking for optimality.
The default is {\tt zero\_pivot =} $\sqrt{u}$,
where $u$ is {\tt EPSILON(1.0)} ({\tt EPSILON(1.0D0)} in
{\tt \fullpackagename\_dou\-ble}).

\itt{inner\_stop\_relative} and {\tt inner\_stop\_absolute}
are scalar variables of type default \realdp,
that hold the relative and absolute convergence tolerances for the
inner iteration (search direction) problem.
and correspond to the values {\tt control\%stop\_relative} and
{\tt control\%stop\_absolute} in the \galahad\ package
{\tt \libraryname\_GLTR}.
The defaults are {\tt inner\_stop\_relative = 0.0}
and \sloppy {\tt inner\_stop\_absolute =} $\sqrt{u}$,
where $u$ is {\tt EPSILON(1.0)} ({\tt EPSILON(1.0D0)} in
{\tt \fullpackagename\_double}).

\itt{cpu\_time\_limit} is a scalar variable of type default \realdp,
that is used to specify the maximum permitted CPU time. Any negative
value indicates no limit will be imposed. The default is
{\tt cpu\_time\_limit = - 1.0}.

\itt{clock\_time\_limit} is a scalar variable of type default \realdp,
that is used to specify the maximum permitted elapsed system clock time.
Any negative value indicates no limit will be imposed. The default is
{\tt clock\_time\_limit = - 1.0}.

\itt{treat\_zero\_bounds\_as\_general} is a scalar variable of type
default \logical.
If it is set to \false, variables which
are only bounded on one side, and whose bound is zero,
will be recognised as non-negativities/non-positivities rather than simply as
lower- or upper-bounded variables.
If it is set to \true, any variable bound
$x_{j}^{l}$ or $x_{j}^{u}$ which has the value 0.0 will be
treated as if it had a general value.
Setting {\tt treat\_zero\_bounds\_as\_general} to \true\ has the advantage
that if a sequence of problems are reordered, then bounds which are
``accidentally'' zero will be considered to have the same structure as
those which are nonzero. However, {\tt \fullpackagename} is
able to take special advantage of non-negativities/non-positivities, so
if a single problem, or if a sequence of problems whose
bound structure is known not to change, is/are to be solved,
it will pay to set the variable to \false.
The default is {\tt treat\_zero\_bounds\_as\_general = .FALSE.}.

\ifthenelse{\equal{\package}{qpa}}{
\itt{solve\_qp} is a scalar variable of type default \logical, that
must be set \true\ if the algorithm will aim to solve the quadratic
programming problem \req{1.2}--\req{1.4}
(by adjusting $\rho_g$ and $\rho_b$ as necessary), and
and \false\ if the solution of the $l_1$-quadratic program
for the specified values of $\rho_g$ and $\rho_b$ is required.
The default is {\tt solve\_qp = .FALSE.}.

\itt{solve\_within\_bounds} is a scalar variable of type default \logical, that
must be set \true\ if the algorithm will aim to solve
the bound-constrained $\mathbf{\ell_1}$ quadratic programming problem
\req{1.4}--\req{1.5} (by adjusting $\rho_b$ as necessary), and
and \false\ otherwise.
If {\tt solve\_qp} is \true, the value of {\tt solve\_within\_bounds}
will be ignored. The default is {\tt solve\_within\_bounds = .FALSE.}.
}{}

\itt{ramdomize} is a scalar variable of type default \logical, that
must be set \true\ if the user wishes to perturb the constraint bounds
by small random quantities during the first stage of the solution process,
and \false\ otherwise.
Any randomization will ultimately be removed. Randomization
helps when solving degenerate problems and is usually to be recommended.
The default is {\tt randomize = .TRUE.}.

%\itt{array\_syntax\_worse\_than\_do\_loop} is a scalar variable of type
%default \logical, that should be set \true\ if the compiler is
%better able to optimize Fortran 77-style do-loops than to exploit
%Fortran 95-style array syntax when performing vector operations,
%and \false\ otherwise.
%The default is {\tt array\_syntax\_worse\_than\_do\_loop = .FALSE.}.

\itt{symmetric\_linear\_solver} is a scalar variable of type default \character\
and length 30, that specifies the external package to be used to
solve any symmetric linear system that might arise. Current possible
choices are {\tt 'sils'}, {\tt 'ma27'}, {\tt 'ma57'}, {\tt 'ma77'},
{\tt 'ma86'}, {\tt 'ma97'}, {\tt ssids}, {\tt 'pardiso'}
and {\tt 'wsmp'},
although only {\tt 'sils'} and, for OMP 4.0-compliant compilers,
{\tt 'ssids'} are installed by default.
See the documentation for the \galahad\ package {\tt SLS} for further details.
The default is {\tt symmetric\_linear\_solver = 'sils'}.

\itt{definite\_linear\_solver} is a scalar variable of type default \character\
and length 30, that specifies the external package to be used to
solve any symmetric positive-definite linear system that might arise.
Current possible
choices are {\tt 'sils'}, {\tt 'ma27'}, {\tt 'ma57'}, {\tt 'ma77'},
{\tt 'ma86'}, {\tt 'ma87'}, {\tt 'ma97'}, {\tt ssids}, {\tt 'pardiso'}
and {\tt 'wsmp'},
although only {\tt 'sils'} and, for OMP 4.0-compliant compilers,
{\tt 'ssids'} are installed by default.
See the documentation for the \galahad\ package {\tt SLS} for further details.
The default is {\tt definite\_linear\_solver = 'sils'}.

\itt{prefix} is a scalar variable of type default \character\
and length 30, that may be used to provide a user-selected
character string to preface every line of printed output.
Specifically, each line of output will be prefaced by the string
{\tt prefix(2:LEN(TRIM(prefix))-1)},
thus ignoring the first and last non-null components of the
supplied string. If the user does not want to preface lines by such
a string, they may use the default {\tt prefix = ""}.

\itt{SLS\_control} is a scalar variable argument of type
{\tt SLS\_control\_type} that is used to pass control
options to external packages used to
factorize relevant symmetric matrices that arise.
See the documentation for the \galahad\ package {\tt SLS} for further details.
In particular, default values are as for {\tt SLS}, except that
{\tt SLS\_control\%rela\-tive\_pivot\_tolerance} is reset to
{\tt pivot\_tol}.

\end{description}
