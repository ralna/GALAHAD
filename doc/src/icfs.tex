\documentclass{galahad}

% set the package name

\newcommand{\package}{icfs}
\newcommand{\packagename}{ICFS}
\newcommand{\fullpackagename}{\libraryname\_\packagename}
\newcommand{\solver}{{\tt \fullpackagename\_analyse}}

\begin{document}

\galheader

%%%%%%%%%%%%%%%%%%%%%% SUMMARY %%%%%%%%%%%%%%%%%%%%%%%%

\galsummary
Given a symmetric matrix $\bmA$, this package
{\bf computes a symmetric, positive-definite approximation
$\bmL \bmL^T$ using an incomplete Cholesky factorization}; the
resulting matrix $\bmL$ is lower triangular.
Subsequently, the solution $\bmx$ to the either of the linear systems
$\bmL \bmx = \bmb$ and $\bmL^T \bmx = \bmb$
may be found for a given vector $\bmb$.

%%%%%%%%%%%%%%%%%%%%%% attributes %%%%%%%%%%%%%%%%%%%%%%%%

\galattributes
\galversions{\tt  \fullpackagename\_single, \fullpackagename\_double}.
\galuses
{\tt GALAHAD\_SY\-M\-BOLS},
{\tt GALAH\-AD\-\_\-SP\-ECFILE} and
{\tt GALAHAD\_SPACE}.
\galdate May 1998/December 2022.
\galorigin C.-J, Lin and J. J. Mor\'{e}, Argonne National Laboratory,
enhanced for modern fortran by N. I. M. Gould, Rutherford Appleton Laboratory.
\gallanguage Fortran~2003.

%%%%%%%%%%%%%%%%%%%%%% HOW TO USE %%%%%%%%%%%%%%%%%%%%%%%%

\galhowto

%\subsection{Calling sequences}

Access to the package requires a {\tt USE} statement such as

\medskip\noindent{\em Single precision version}

\hspace{8mm} {\tt USE \fullpackagename\_single}

\medskip\noindent{\em Double precision version}

\hspace{8mm} {\tt USE  \fullpackagename\_double}

\medskip

\noindent
The user is {\bf strongly advised} to use the double
precision version unless single precision corresponds to 8-byte arithmetic.

If it is required to use both modules at the same time, the derived types
{\tt \packagename\_control\_type},
{\tt \packagename\_inform\_type},
{\tt \packagename\_data\_type}
and
{\tt NLPT\_userdata\_type},
(Section~\ref{galtypes})
and the subroutines
{\tt \packagename\_initialize},
{\tt \packagename\_\-factorize},
{\tt \packagename\_\-solve},
{\tt \packagename\_terminate},
(Section~\ref{galarguments})
and
{\tt \packagename\_read\_specfile}
(Section~\ref{galfeatures})
must be renamed on one of the {\tt USE} statements.

%%%%%%%%%%%%%%%%%%%%%% derived types %%%%%%%%%%%%%%%%%%%%%%%%

\galtypes
Four derived data types are accessible from the package.

%%%%%%%%%%% control type %%%%%%%%%%%

\subsubsection{The derived data type for holding control
 parameters}\label{typecontrol}
The derived data type
{\tt \packagename\_control\_type}
is used to hold controlling data. Default values may be obtained by calling
{\tt \packagename\_initialize}
(see Section~\ref{subinit}),
while components may also be changed by calling
{\tt \fullpackagename\_read\-\_spec}
(see Section~\ref{readspec}).
The components of
{\tt \packagename\_control\_type}
are:


\begin{description}

\itt{error} is a scalar variable of type default \integer, that holds the
stream number for error messages. Printing of error messages in
{\tt \packagename\_analyse},
{\tt \packagename\_estimate}
and {\tt \packagename\_terminate}
is suppressed if {\tt error} $\leq 0$.
The default is {\tt error = 6}.

\ittf{out} is a scalar variable of type default \integer, that holds the
stream number for informational messages. Printing of informational messages in
{\tt \packagename\_analyse} and {\tt \packagename\_estimate}
is suppressed if {\tt out} $< 0$.
The default is {\tt out = 6}.

\itt{print\_level} is a scalar variable of type default \integer, that is used
to control the amount of informational output which is required. No
informational output will occur if {\tt print\_level} $\leq 0$. If
{\tt print\_level} $> 01$, details of any data errors encountered
will be reported.
The default is {\tt print\_level = 0}.

\itt{icfs\_vectors} is a scalar variable of type default \integer, that holds
the number of extra vectors of length $n$ required by the
incomplete Cholesky preconditioner.
Usually, the larger the number, the
better the preconditioner, but the more space and effort required to
use it. Any negative value will be regarded as {\tt 0}.
The default is {\tt icfs\_vectors = 10}.

\itt{shift} is a scalar variable of type default \realdp, that holds
an initial estimate of the shift $\alpha$ used so that the incomplete
factorization of $\bmA + \bmD$ is positive definite, where $\bmD$ is
a diagonal matrix whose entries are no larger than $\alpha$.
This value $\alpha$ may subsequently be increased, as necessary, by the
package, see {\tt inform\%shift}.
The default is {\tt shift - 0.0}.

\itt{space\_critical} is a scalar variable of type default \logical,
that must be set \true\ if space is critical when allocating arrays
and  \false\ otherwise. The package may run faster if
{\tt space\_critical} is \false\ but at the possible expense of a larger
storage requirement. The default is {\tt space\_critical = .FALSE.}.

\itt{deallocate\_error\_fatal} is a scalar variable of type default \logical,
that must be set \true\ if the user wishes to terminate execution if
a deallocation  fails, and \false\ if an attempt to continue
will be made. The default is {\tt deallocate\_error\_fatal = .FALSE.}.

\itt{prefix} is a scalar variable of type default \character\
and length 30, that may be used to provide a user-selected
character string to preface every line of printed output.
Specifically, each line of output will be prefaced by the string
{\tt prefix(2:LEN(TRIM( prefix ))-1)},
thus ignoreing the first and last non-null components of the
supplied string. If the user does not want to preface lines by such
a string, they may use the default {\tt prefix = ""}.

\end{description}

%%%%%%%%%%% time type %%%%%%%%%%%

\subsubsection{The derived data type for holding timing
 information}\label{typetime}
The derived data type
{\tt \packagename\_time\_type}
is used to hold elapsed CPU and system clock times for the various parts of
the calculation. The components of
{\tt \packagename\_time\_type}
are:
\begin{description}
\itt{total} is a scalar variable of type default \realdp, that gives
 the total CPU time spent in the package.

\itt{factorize} is a scalar variable of type default \realdp, that gives
 the CPU time spent factorizing the required matrices.

\itt{solve} is a scalar variable of type default \realdp, that gives
 the CPU time spent solving the resulting triangular systems.

\itt{clock\_total} is a scalar variable of type default \realdp, that gives
 the total elapsed system clock time spent in the package.

\itt{clock\_factorize} is a scalar variable of type default \realdp, that gives
 the elapsed system clock time spent factorizing the required matrices.

\itt{clock\_solve} is a scalar variable of type default \realdp, that gives
 the elapsed system clock time spent solving the resulting triangular systems.

\end{description}

%%%%%%%%%%% inform type %%%%%%%%%%%

\subsubsection{The derived data type for holding informational
 parameters}\label{typeinform}
The derived data type
{\tt \packagename\_inform\_type}
is used to hold parameters that give information about the progress and needs
of the algorithm. The components of
{\tt \packagename\_inform\_type}
are:

\begin{description}
\itt{status} is a scalar variable of type default \integer, that gives the
exit status of the algorithm.
See Section~\ref{galerrors} for details.

\itt{alloc\_status} is a scalar variable of type default \integer, that gives
the status of the last attempted array allocation or deallocation.
This will be 0 if {\tt status = 0}.

\itt{bad\_alloc} is a scalar variable of type default \character\
and length 80, that  gives the name of the last internal array
for which there were allocation or deallocation errors.
This will be the null string if {\tt status = 0}.

\itt{shift} is a scalar variable of type default \realdp, that holds
the final value of the shift $\alpha$ used so that the incomplete
factorization of $\bmA + \bmD$ is positive definite, where $\bmD$ is
a diagonal matrix whose entries are no larger than $\alpha$.

\ittf{time} is a scalar variable of type {\tt \packagename\_time\_type}
whose components are used to hold elapsed CPU and system clock times for the
various parts of the calculation (see Section~\ref{typetime}).

\end{description}

%%%%%%%%%%% data type %%%%%%%%%%%

\subsubsection{The derived data type for holding problem data}\label{typedata}
The derived data type
{\tt \packagename\_data\_type}
is used to hold all the data for a particular problem,
or sequences of problems with the same structure, between calls of
{\tt \packagename} procedures.
This data should be preserved, untouched,
from the initial call to
{\tt \packagename\_initialize}
to the final call to
{\tt \packagename\_terminate}.

%%%%%%%%%%%%%%%%%%%%%% argument lists %%%%%%%%%%%%%%%%%%%%%%%%

\galarguments
There are four procedures for user calls
(see Section~\ref{galfeatures} for further features):

\begin{enumerate}
\item The subroutine
      {\tt \packagename\_initialize}
      is used to set default values, and initialize private data.
\item The subroutine
      {\tt \packagename\_factorize}
      is called to form the incomplete Cholesky factor $\bmL$ from $\bmA$.
\item The subroutine
      {\tt \packagename\_triangular\_solve}
      is called to solve either of the triangular systems
      $\bmL \bmx = \bmb$ or $\bmL^T \bmx = \bmb$
      for given vectors $\bmb$.
\item The subroutine
      {\tt \packagename\_terminate}
      is provided to allow the user to automatically deallocate array
       components of the private data, allocated by
       {\tt \packagename\_factorize},
       at the end of the solution process.
       It is important to do this if the data object is re-used for another
       matrix {\bf with a different structure}
       since {\tt \packagename\_initialize} cannot test for this situation,
       and any existing associated targets will subsequently become unreachable.
\end{enumerate}

\noindent
We note that in addition to the above calls, the user may also call the
original fortran 77 subroutines {\tt DICFS} and {\tt DSTRSOL} directly.
For details of the necessary argument lists, see Section~\ref{galappendix}.

%We use square brackets {\tt [ ]} to indicate \optional\ arguments.

%%%%%% initialization subroutine %%%%%%

\subsubsection{The initialization subroutine}\label{subinit}
 Default values are provided as follows:
\vspace*{1mm}

\hspace{8mm}
{\tt CALL \packagename\_initialize( data, control, inform )}

\vspace*{-3mm}
\begin{description}

\itt{data} is a scalar \intentinout\ argument of type
{\tt \packagename\_data\_type}
(see Section~\ref{typedata}). It is used to hold data about the
matrix and its factors.

\itt{control} is a scalar \intentout\ argument of type
{\tt \packagename\_control\_type}
(see Section~\ref{typecontrol}).
On exit, {\tt control} contains default values for the components as
described in Section~\ref{typecontrol}.
These values should only be changed after calling
{\tt \packagename\_initialize}.

\itt{inform} is a scalar \intentout\ argument of type
{\tt \packagename\_inform\_type}
(see Section~\ref{typeinform}). A successful call to
{\tt \packagename\_initialize}
is indicated when the  component {\tt status} has the value 0.
For other return values of {\tt status}, see Section~\ref{galerrors}.

\end{description}

%%%%%%%%% factorize subroutine %%%%%%

\subsubsection{The incomplete factorization subroutine}
The factorization phase, in which incomplete Cholesky factors $\bmL$ of
$\bmA$ are determined, is performed as follows:

\vspace*{1mm}

\hspace{8mm}
{\tt CALL \packagename\_factorize( n, PTR, ROW, DIAG, VAL, data, control, inform )}

\vspace*{-2mm}
\begin{description}
\ittf{n}
is a scalar \intentin\ scalar argument of type default \integer, that must be
set to $n$ the dimension of the matrix $\bmA$.
\restrictions {\tt n} $> 0$.

\ittf{PTR} is a scalar \intentin\ rank-one array argument of type
default \integer\ and dimension {\tt n + 1}, whose {\tt j}-th component
gives the starting address for list of nonzero values and their
corresponding row  indices in column {\tt j} of the
{\bf strict lower triangular part} of $\bmA$ (The entry $a_{i,j}$ is in
the strict lower triangular part of $\bmA$ if $i > j$).
That is, the nonzeros in column {\tt j} of the strict lower triangle of
$\bmA$ must be in positions {\tt PTR(j)} \ldots, {\tt PTR(j+1) - 1} for
{\tt j = 1,} \ldots {\tt n}.
Note that {\tt PTR(n+1)} points to the first position beyond that needed
to store $\bmA$.

\ittf{ROW} is a scalar \intentin\ rank-one array argument of type
default \integer\ and dimension at least {\tt  PTR(n+1)-1},
that contains the row indices of the strict lower triangular part of $\bmA$
in the compressed column storage format described above.

\itt{DIAG} is a scalar \intentin\ rank-one array argument of type
default \realdp\ and dimension {\tt n}, whose {\tt j}-th component
contains the value of the {\tt j}-th diagonal of $\bmA$.

\ittf{VAL} is a scalar \intentin\ rank-one array argument of type
default \realdp\ and dimension at least {\tt  PTR(n+1)-1},
that contains the values of the entries strict lower triangular part of $\bmA$
that correspond to the row indices described above.

\itt{data} is a scalar \intentinout\ argument of type
{\tt \packagename\_data\_type}
(see Section~\ref{typedata}). It is used to hold data about the problem being
solved. It must not have been altered {\bf by the user} since the last call to
{\tt \packagename\_initialize}.

\itt{control} is a scalar \intentin\ argument of type
{\tt \packagename\_control\_type}
(see Section~\ref{typecontrol}). Default values may be assigned by calling
{\tt \packagename\_initialize} prior to the first call to
{\tt \packagename\_analyse}.

\itt{inform} is a scalar \intentinout\ argument of type
{\tt \packagename\_inform\_type}
(see Section~\ref{typeinform}).
A successful call to
{\tt \packagename\_analyse}
is indicated when the  component {\tt status} has the value 0.
For other return values of {\tt status}, see Section~\ref{galerrors}.

\end{description}

%%%%%%%%% estimation subroutine %%%%%%

\subsubsection{The triangular solution subroutine}

The solution phase, in which one of the triangular systems
$\bmL \bmx = \bmb$ or $\bmL^T \bmx = \bmb$ for given vectors $\bmb$,
is performed as follows:
\vspace*{1mm}

\hspace{8mm}
{\tt CALL \packagename\_triangular\_solve( n, X, transpose, data, control, inform )}
\vspace*{-1mm}

%\vspace*{-2mm}
\begin{description}
\itt{n,} {\tt data}, {\tt control} and {\tt inform} are
exactly as described and input to {\tt \packagename\_factorize},
and must not have been changed in the interim.

\ittf{X} is a scalar \intentinout\ rank-one array argument of type
default \realdp, and dimension {\tt n}, that  must be set on input so that
{\tt X(}$i${\tt)} contains the component $b_i$, $i = 1, \ldots,$ {\tt n}
of the vector $\bmb$. On output, this will have been overwritten by the
desired vector $\bmx$.

\itt{transpose} is a scalar \intentin\ argument of type
default \logical, that must be set \false\ if the user wishes to solve
$\bmL \bmx = \bmb$ and \true\ if instead the solution to $\bmL^T \bmx = \bmb$
is sought.

\end{description}

%%%%%%% termination subroutine %%%%%%

\subsubsection{The  termination subroutine}
All previously allocated arrays are deallocated as follows:
\vspace*{1mm}

\hspace{8mm}
{\tt CALL \packagename\_terminate( data, control, inform )}

\vspace*{-1mm}
\begin{description}

\itt{data} is a scalar \intentinout\ argument of type
{\tt \packagename\_data\_type}
exactly as for
{\tt \packagename\_factorize},
which must not have been altered {\bf by the user} since the last call to
{\tt \packagename\_initialize}.
On exit, array components will have been deallocated.

\itt{control} is a scalar \intentin\ argument of type
{\tt \packagename\_control\_type}
exactly as for
{\tt \packagename\_factorize}.

\itt{inform} is a scalar \intentout\ argument of type
{\tt \packagename\_inform\_type}
exactly as for
{\tt \packagename\_factorize}.
Only the component {\tt status} will be set on exit, and a
successful call to
{\tt \packagename\_terminate}
is indicated when this  component {\tt status} has the value 0.
For other return values of {\tt status}, see Section~\ref{galerrors}.

\end{description}

%%%%%%%%%%%%%%%%%%%%%% Warning and error messages %%%%%%%%%%%%%%%%%%%%%%%%

\galerrors
A negative value of {\tt inform\%status} on exit from
{\tt \packagename\_factorize},
{\tt \packagename\_triangular\_solve}
or
{\tt \packagename\_terminate}
indicates that an error has occurred. No further calls should be made
until the error has been corrected. Possible values are:

\begin{description}

\itt{\galerrallocate.} An allocation error occurred.
A message indicating the offending
array is written on unit {\tt control\%error}, and the returned allocation
status and a string containing the name of the offending array
are held in {\tt inform\%alloc\_\-status}
and {\tt inform\%bad\_alloc}, respectively.

\itt{\galerrdeallocate.} A deallocation error occurred.
A message indicating the offending
array is written on unit {\tt control\%error} and the returned allocation
status and a string containing the name of the offending array
are held in {\tt inform\%alloc\_\-status}
and {\tt inform\%bad\_alloc}, respectively.

\itt{\galerrrestrictions.}
The restriction
$0 <$ {\tt n}
 has been violated.

\end{description}

%%%%%%%%%%%%%%%%%%%%%% Further features %%%%%%%%%%%%%%%%%%%%%%%%

\galfeatures
\noindent In this section, we describe an alternative means of setting
control parameters, that is components of the variable {\tt control} of type
{\tt \packagename\_control\_type}
(see Section~\ref{typecontrol}),
by reading an appropriate data specification file using the
subroutine {\tt \packagename\_read\_specfile}. This facility
is useful as it allows a user to change  {\tt \packagename} control parameters
without editing and recompiling programs that call {\tt \packagename}.

A specification file, or specfile, is a data file containing a number of
"specification commands". Each command occurs on a separate line,
and comprises a "keyword",
which is a string (in a close-to-natural language) used to identify a
control parameter, and
an (optional) "value", which defines the value to be assigned to the given
control parameter. All keywords and values are case insensitive,
keywords may be preceded by one or more blanks but
values must not contain blanks, and
each value must be separated from its keyword by at least one blank.
Values must not contain more than 30 characters, and
each line of the specfile is limited to 80 characters,
including the blanks separating keyword and value.

The portion of the specification file used by
{\tt \packagename\_read\_specfile}
must start
with a "{\tt BEGIN \packagename}" command and end with an
"{\tt END}" command.  The syntax of the specfile is thus defined as follows:
\begin{verbatim}
  ( .. lines ignored by ICFS_read_specfile .. )
    BEGIN ICFS
       keyword    value
       .......    .....
       keyword    value
    END
  ( .. lines ignored by ICFS_read_specfile .. )
\end{verbatim}
where keyword and value are two strings separated by (at least) one blank.
The ``{\tt BEGIN \packagename}'' and ``{\tt END}'' delimiter command lines
may contain additional (trailing) strings so long as such strings are
separated by one or more blanks, so that lines such as
\begin{verbatim}
    BEGIN ICFS SPECIFICATION
\end{verbatim}
and
\begin{verbatim}
    END ICFS SPECIFICATION
\end{verbatim}
are acceptable. Furthermore,
between the
``{\tt BEGIN \packagename}'' and ``{\tt END}'' delimiters,
specification commands may occur in any order.  Blank lines and
lines whose first non-blank character is {\tt !} or {\tt *} are ignored.
The content
of a line after a {\tt !} or {\tt *} character is also
ignored (as is the {\tt !} or {\tt *}
character itself). This provides an easy manner to "comment out" some
specification commands, or to comment specific values
of certain control parameters.

The value of a control parameters may be of three different types, namely
integer, logical or real.
Integer and real values may be expressed in any relevant Fortran integer and
floating-point formats (respectively). Permitted values for logical
parameters are "{\tt ON}", "{\tt TRUE}", "{\tt .TRUE.}", "{\tt T}",
"{\tt YES}", "{\tt Y}", or "{\tt OFF}", "{\tt NO}",
"{\tt N}", "{\tt FALSE}", "{\tt .FALSE.}" and "{\tt F}".
Empty values are also allowed for
logical control parameters, and are interpreted as "{\tt TRUE}".

The specification file must be open for
input when {\tt \packagename\_read\_specfile}
is called, and the associated device number
passed to the routine in device (see below).
Note that the corresponding
file is {\tt REWIND}ed, which makes it possible to combine the specifications
for more than one program/routine.  For the same reason, the file is not
closed by {\tt \packagename\_read\_specfile}.

\subsubsection{To read control parameters from a specification file}
\label{readspec}

Control parameters may be read from a file as follows:
\hskip0.5in
\def\baselinestretch{0.8} {\tt \begin{verbatim}
     CALL ICFS_read_specfile( control, device )
\end{verbatim}
}
\def\baselinestretch{1.0}

\begin{description}
\itt{control} is a scalar \intentinout argument of type
{\tt \packagename\_control\_type}
(see Section~\ref{typecontrol}).
Default values should have already been set, perhaps by calling
{\tt \packagename\_initialize}.
On exit, individual components of {\tt control} may have been changed
according to the commands found in the specfile. Specfile commands and
the component (see Section~\ref{typecontrol}) of {\tt control}
that each affects are given in Table~\ref{specfile}.

\bctable{|l|l|l|}
\hline
  command & component of {\tt control} & value type \\
\hline
  {\tt error-printout-device} & {\tt \%error} & integer \\
  {\tt printout-device} & {\tt \%out} & integer \\
  {\tt print-level} & {\tt \%print\_level} & integer \\
  {\tt number-of-icfs-vectors} & {icfs\_vectors} & integer \\
  {\tt initial-shift} & {\tt shift} & real \\
  {\tt space-critical}   & {\tt \%space\_critical} & logical \\
  {\tt deallocate-error-fatal}   & {\tt \%deallocate\_error\_fatal} & logical \\
  {\tt output-line-prefix} & {\tt \%prefix} & character \\
\hline

\ectable{\label{specfile}Specfile commands and associated
components of {\tt control}.}

\itt{device} is a scalar \intentin argument of type default \integer,
that must be set to the unit number on which the specfile
has been opened. If {\tt device} is not open, {\tt control} will
not be altered and execution will continue, but an error message
will be printed on unit {\tt control\%error}.

\end{description}

%%%%%%%%%%%%%%%%%%%%%% Information printed %%%%%%%%%%%%%%%%%%%%%%%%

\galinfo
If {\tt control\%print\_level} is positive, information about
errors encountered will be printed on unit {\tt control\-\%out}.

%%%%%%%%%%%%%%%%%%%%%% GENERAL INFORMATION %%%%%%%%%%%%%%%%%%%%%%%%

\galgeneral

\galcommon None.
\galworkspace Provided automatically by the module.
\galroutines None.
\galmodules The module uses the \galahad\ packages
{\tt GALAHAD\_SY\-M\-BOLS},
{\tt GALAHAD\_SPECFILE} and
{\tt GALAHAD\_SPACE}.
\galio Output is under control of the arguments
 {\tt control\%error}, {\tt control\%out} and {\tt control\%print\_level}.
\galrestrictions {\tt n} $> 0$ .
\galportability ISO Fortran~2003.
The package is thread-safe.

%%%%%%%%%%%%%%%%%%%%%% METHOD %%%%%%%%%%%%%%%%%%%%%%%%

\galmethod
The package computes incomplete Cholesky factors $\bmL$ of $\bmA$ so that
\disp{\bmA + \bmD = \bmL \bmL + \bmE.}
The pattern of entries in $\bmL$ match those in $\bmA$ with additional
``fill-ins'' controlled by the extra storage provided. The shifted diagonal
$\bmD$ whose values do not exceed a scalar shift $\alpha$ allows for
indefinite $\bmA$, and also guarantees stable factors $\bmL$. Increasing the
extra storage provided generally decreases the size of the error matrix
$\bmE$, but increases the cost of the algorithm.

\vspace*{1mm}

\galreference
\vspace*{1mm}

\noindent
The method is described in detail in
\vspace*{1mm}

\noindent
C.\-J.\ Lin and J.\ J.\ More\'{e}. ``Incomplete Cholesky factorizations
with limited memory''.  SIAM J. Sci. Computing. {\bf 21} (1999) 24--45.

%%%%%%%%%%%%%%%%%%%%%% EXAMPLE %%%%%%%%%%%%%%%%%%%%%%%%

\galexamples
Suppose that
\disp{\bmA = \mat{ccccc}{
2 & 1 &   &   &   \\
1 & 5 & 1 &   & 1 \\
  & 1 & 1 & 1 &   \\
  &   & 1 & 7 &   \\
  & 1 &   &   & 2 } \;\; \mbox{and} \;\;
\bmb = \vect{ 3 \\ 8 \\ 3 \\ 8 \\ 3},}
where the missing entries in $\bmA$ are structural zeros. Then we may find
suitable incomplete factors $\bmL$, and subsequently solve
$\bmL \bmy = \bmb$ and $\bmL^T \bmx = \bmy$ using the following code;
notice that we initialize $\bmb$ in $\bmx$, overwrite this with $\bmy$ in the
first triangular solve, and finally recover $\bmx$ from $\bmy$
in the second solve:
{\tt \small
\VerbatimInput{\packageexample}
}
\noindent
The code produces the following output:
{\tt \small
\VerbatimInput{\packageresults}
}
\noindent

%%%%%%%%%%%%%%%%%%%%%% APPENDIX %%%%%%%%%%%%%%%%%%%%%%%%

\galappendix
The original subroutine calls {\tt DICFS} and {\tt DSTRSOL},
as encoded in {\tt \packagename\_factorize} and
{\tt \packagename\_triangular\_solve},
may also be called as part of the package. To quote from the package source,

\begin{verbatim}
!     Subroutine dicfs
!
!     Given a symmetric matrix A in compressed column storage, this
!     subroutine computes an incomplete Cholesky factor of A + alpha*D,
!     where alpha is a shift and D is the diagonal matrix with entries
!     set to the l2 norms of the columns of A.
!
!     The subroutine statement is
!
!       subroutine dicfs(n,nnz,a,adiag,acol_ptr,arow_ind,l,ldiag,lcol_ptr,
!                        lrow_ind,p,alpha,iwa,wa1,wa2)
!
!     where
!
!       n is an integer variable.
!         On entry n is the order of A.
!         On exit n is unchanged.
!
!       nnz is an integer variable.
!         On entry nnz is the number of nonzeros in the strict lower
!            triangular part of A.
!         On exit nnz is unchanged.
!
!       a is a real array of dimension nnz.
!         On entry a must contain the strict lower triangular part
!            of A in compressed column storage.
!         On exit a is unchanged.
!
!       adiag is a real array of dimension n.
!         On entry adiag must contain the diagonal elements of A.
!         On exit adiag is unchanged.
!
!       acol_ptr is an integer array of dimension n + 1.
!         On entry acol_ptr must contain pointers to the columns of A.
!            The nonzeros in column j of A must be in positions
!            acol_ptr(j), ... , acol_ptr(j+1) - 1.
!         On exit acol_ptr is unchanged.
!
!       arow_ind is an integer array of dimension nnz.
!         On entry arow_ind must contain row indices for the strict
!            lower triangular part of A in compressed column storage.
!         On exit arow_ind is unchanged.
!
!       l is a real array of dimension nnz+n*p.
!         On entry l need not be specified.
!         On exit l contains the strict lower triangular part
!            of L in compressed column storage.
!
!       ldiag is a real array of dimension n.
!         On entry ldiag need not be specified.
!         On exit ldiag contains the diagonal elements of L.
!
!       lcol_ptr is an integer array of dimension n + 1.
!         On entry lcol_ptr need not be specified.
!         On exit lcol_ptr contains pointers to the columns of L.
!            The nonzeros in column j of L are in the
!            lcol_ptr(j), ... , lcol_ptr(j+1) - 1 positions of l.
!
!       lrow_ind is an integer array of dimension nnz+n*p.
!         On entry lrow_ind need not be specified.
!         On exit lrow_ind contains row indices for the strict lower
!            triangular part of L in compressed column storage.
!
!       p is an integer variable.
!         On entry p specifes the amount of memory available for the
!            incomplete Cholesky factorization.
!         On exit p is unchanged.
!
!       alpha is a real variable.
!         On entry alpha is the initial guess of the shift.
!         On exit alpha is final shift
!
!       iwa is an integer work array of dimension 3*n.
!
!       wa1 is a real work array of dimension n.
!
!       wa2 is a real work array of dimension n.
!
!     Subroutine dstrsol
!
!     This subroutine solves the triangular systems L*x = r or L'*x = r.
!
!     The subroutine statement is
!
!       subroutine dstrsol(n,l,ldiag,jptr,indr,r,task)
!
!     where
!
!       n is an integer variable.
!         On entry n is the order of L.
!         On exit n is unchanged.
!
!       l is a real array of dimension *.
!         On entry l must contain the nonzeros in the strict lower
!            triangular part of L in compressed column storage.
!         On exit l is unchanged.
!
!       ldiag is a real array of dimension n.
!         On entry ldiag must contain the diagonal elements of L.
!         On exit ldiag is unchanged.
!
!       jptr is an integer array of dimension n + 1.
!         On entry jptr must contain pointers to the columns of A.
!            The nonzeros in column j of A must be in positions
!            jptr(j), ... , jptr(j+1) - 1.
!         On exit jptr is unchanged.
!
!       indr is an integer array of dimension *.
!         On entry indr must contain row indices for the strict
!            lower triangular part of L in compressed column storage.
!         On exit indr is unchanged.
!
!       r is a real array of dimension n.
!         On entry r must contain the vector r.
!         On exit r contains the solution vector x.
!
!       task is a character variable of length 60.
!         On entry
!            task(1:1) = 'N' if we need to solve L*x = r
!            task(1:1) = 'T' if we need to solve L'*x = r
!         On exit task is unchanged.
!
!     MINPACK-2 Project. October 1998.
!     Argonne National Laboratory.
!     Chih-Jen Lin and Jorge J. More'.
\end{verbatim}

\end{document}
