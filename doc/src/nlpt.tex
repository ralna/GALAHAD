\documentclass{galahad}

% set the package name

\newcommand{\package}{nlpt}
\newcommand{\packagename}{NLPT}
\newcommand{\fullpackagename}{\libraryname\_\packagename}
\newcommand{\sym}{\sf\small}


\begin{document}

\galheader

%%%%%%%%%%%%%%%%%%%%%% SUMMARY %%%%%%%%%%%%%%%%%%%%%%%%

\galsummary
This package defines a derived type capable of supporting the storage of
a variety of smooth {\bf nonlinear programming problems} of the form
\[
\min \, \bmf(\bmx)
\]
subject to the general constraints
\[
\bmc^l \leq \bmc(\bmx) \leq \bmc^u,
\]
and
\[
\bmx^l \leq \bmx \leq \bmx^u,
\]
where $\bmf$ is a smooth ``objective function'', where $\bmc(x)$ is a smooth
function from $\Re^n$ into $\Re^m$ and where inequalities are understood
componentwise. The vectors $\bmc^l \leq \bmc^u$ and $\bmx^l \leq \bmx^u$ are
$m$- and $n$-dimensional, respectively, and may contain components equal to
minus or plus infinity. An important function associated with the problem
is its Lagrangian
\[
\bmL( \bmx,\bmy, \bmz) = \bmf(\bmx) - \bmy^T\bmc(\bmx) - \bmz^T\bmx
\]
where $\bmy$ belongs to $\Re^m$ and $\bmz$ belongs to $\Re^n$.
The solution of such problem may require the storage of the objective
function's gradient 
\[
\bmg(\bmx) = \nabla_{\bmx} \bmf(\bmx),
\]
the $n \times n$ symmetric objective function's Hessian
\[
\bmH_f(\bmx) = \nabla_{\bmx\bmx} \bmf(\bmx)
\]
the $m \times n$ constraints' Jacobian whose $i$-th row is the 
gradient of the $i$-th constraint:
\[
\bme_i^T \bmJ(\bmx) = [ \nabla_{\bmx} \bmc_i(\bmx) ]^T,
\]
the gradient of the Lagrangian with respect to $\bmx$,
\[
\bmg_L(\bmx,\bmy,\bmz) = \nabla_{\bmx} \bmL( \bmx,\bmy, \bmz)
\]
and of the Lagrangian's Hessian with respect to $\bmx$
\[
\bmH_L(\bmx,\bmy,\bmz) = \nabla_{\bmx\bmx} \bmL( \bmx,\bmy, \bmz).
\]
Note that this last matrix is equal to the Hessian of the objective function
when the problem is unconstrained ($m=0$), which autorizes us
to use the same symbol $\bmH$ for both cases.

\noindent
Full advantage can be taken of any zero coefficients in the matrices $\bmH$ 
or $\bmJ$.

\noindent
The module also contains subroutines that are designed for printing parts 
of the problem data, and for matrix storage scheme conversions. 

%%%%%%%%%%%%%%%%%%%%%% attributes %%%%%%%%%%%%%%%%%%%%%%%%

\galattributes
\galversions{\tt  \fullpackagename\_single, \fullpackagename\_double},
\galcalls {\tt GALAHAD\_TOOLS}.
\galdate May 2003.
\galorigin N. I. M. Gould, Rutherford Appleton Laboratory, and
Ph. L. Toint, University of Namur, Belgium.
\gallanguage Fortran~95 + TR 15581 or Fortran~2003. 

%%%%%%%%%%%%%%%%%%%%%% HOW TO USE %%%%%%%%%%%%%%%%%%%%%%%%

\galhowto

%\subsection{Calling sequences}

Access to the package requires a {\tt USE} statement such as

\medskip\noindent{\em Single precision version}

\hspace{8mm} {\tt USE \fullpackagename\_single}

\medskip\noindent{\em Double precision version}

\hspace{8mm} {\tt USE \fullpackagename\_double}

\medskip

\noindent
If it is required to use both modules at the same time, the derived type
{\tt NLPT\_problem\_type}, 
(Section~\ref{galtype})
must be renamed on one of the {\tt USE} statements.

%%%%%%%%%%%%%%%%%%%%%% matrix formats %%%%%%%%%%%%%%%%%%%%%%%%

\galmatrix
Both the Hessian matrix $\bmH$ and the Jacobian $\bmJ$
may be stored in one of three input formats (the format for the two matrices
being possibly different).

\subsubsection{Dense storage format}\label{dense}

The matrix $\bmJ$ is stored as a compact 
dense matrix by rows, that is, the values of the entries of each row in turn are
stored in order within an appropriate real one-dimensional array.
Component $n \ast (i-1) + j$ of the storage array {\tt J\_val} will hold the 
value $J_{ij}$ for $i = 1, \ldots , m$, $j = 1, \ldots , n$.
Since $\bmH$ is symmetric, only the lower triangular part (that is the part 
$h_{ij}$ for $1 \leq j \leq i \leq n$) need be held. In this case
the lower triangle will be stored by rows, that is 
component $i \ast (i-1)/2 + j$ of the storage array {\tt H\_val}  
will hold the value $h_{ij}$ (and, by symmetry, $h_{ji}$)
for $1 \leq j \leq i \leq n$.

\noindent
If this storage scheme is used, {\tt J\_type} and/or {\tt H\_type} must be set
the value of the symbol {\sym GALAHAD\_DENSE}.

\subsubsection{Sparse co-ordinate storage format}\label{coordinate}

Only the nonzero entries of the matrices are stored. For the 
$l$-th entry of $\bmJ$, its row index $i$, column index $j$ 
and value $J_{ij}$
are stored in the $l$-th components of the integer arrays {\tt J\_row}, 
{\tt J\_col} and real array {\tt J\_val}. 
The order is unimportant, but the total
number of entries {\tt J\_ne} is also required. 
The same scheme is applicable to
$\bmH$ (thus requiring integer arrays {\tt H\_row}, {\tt H\_col}, a real array 
{\tt H\_val} and an integer value {\tt H\_ne}),
except that only the entries in the lower triangle need be stored.

\noindent
If this storage scheme is used, {\tt J\_type} and/or {\tt H\_type} must be set
the value of the symbol {\sym GALAHAD\_COORDINATE}.

\subsubsection{Sparse row-wise storage format}\label{rowwise}

Again only the nonzero entries are stored, but this time
they are ordered so that those in row $i$ appear directly before those
in row $i+1$. For the $i$-th row of $\bmJ$, the $i$-th component of a 
integer array {\tt J\_ptr} holds the position of the first entry in this row,
while {\tt J\_ptr} $(m+1)$ holds the total number of entries plus one.
The column indices $j$ and values $J_{ij}$ of the entries in the $i$-th row 
are stored in components 
$l =$ {\tt J\_ptr}$(i)$, \ldots ,{\tt J\_ptr} $(i+1)-1$ of the 
integer array {\tt J\_col}, and real array {\tt J\_val}, respectively. 
The same scheme is applicable to
$\bmH$ (thus requiring integer arrays {\tt H\_ptr}, {\tt H\_col}, and 
a real array {\tt H\_val}),
except that only the entries in the lower triangle need be stored.
The values of {\tt J\_ne} and {\tt H\_ne} are not mandatory, since they can be
recovered from
\[
{\tt J\_ne} = {\tt J\_ptr}({\tt n}+1) - 1
\;\;\;\; \mbox{and} \;\;\;\;
{\tt H\_ne} = {\tt H\_ptr}({\tt n}+1) - 1
\]
For sparse matrices, this scheme almost always requires less storage than 
its predecessor.

\noindent
If this storage scheme is used, {\tt J\_type} and/or {\tt H\_type} must be set
the value of the symbol {\sym GALAHAD\_SPARSE\_BY\_ROWS}.

%%%%%%%%%%%%%%%%%%% optimality conditions %%%%%%%%%%%%%%%%%%%

\subsection{Optimality conditions\label{galopt}}

The solution $\bmx$ necessarily satisfies 
the primal first-order optimality conditions
\[
\bmc^l \leq \bmc(\bmx) \leq \bmc^u,
\;\;\;\; \mbox{and} \;\;\;\;
\bmx^l \leq \bmx \leq \bmx^u,
\]
the dual first-order optimality conditions
\[
\bmg( \bmx ) = \bmJ(\bmx)^{T} \bmy + \bmz \;\;
\]
where
\[
\bmy = \bmy^{l} + \bmy^{u}, \;\;
 \bmz = \bmz^{l} + \bmz^{u}\,\,
 \bmy^{l} \geq 0 , \;\;
 \bmy^{u} \leq 0 , \;\;
 \bmz^{l} \geq 0 \tim{and}
 \bmz^{u} \leq 0 ,
\]
and the complementary slackness conditions 
\disp{
 (\bmc(\bmx) -\bmc^l)^{T} \bmy^{l} = 0  ,\;\;
 ( \bmc(\bmx) - \bmc^{u} )^{T} \bmy^{u} = 0  ,\;\;
 (\bmx -\bmx^{l} )^{T} \bmz^{l} = 0   \tim{and}
 (\bmx -\bmx^{u} )^{T} \bmz^{u} = 0 ,}
where the vectors $\bmy$ and $\bmz$ are 
known as the Lagrange multipliers for
the general constraints, and the dual variables for the bounds,
respectively, and where the vector inequalities hold componentwise.
The dual first-order optimality condition is equivalent to
the condition that $\bmg_L( \bmx,\bmy, \bmz) = 0$.

%%%%%%%%%%%%%%%%%%%%%% derived types %%%%%%%%%%%%%%%%%%%%%%%%

\galtype
A single derived data type, {\tt \packagename\_problem\_type},
is accessible from the package. It is intended that, for any particular
application, only those components which are needed will be set.
The components are:

\begin{description}
\itt{pname} is a scalar variable of type default {\tt CHARACTER( LEN = 10 )},
that holds the problem's name.

\ittf{n} is a scalar variable of type default \integer, 
 that holds the number of optimization variables, $n$.  

\itt{vnames} is a rank-one allocatable array of dimension {\tt n} and type 
{\tt CHARACTER( LEN = 10 )} that holds the names of the problem's
variables. The $j$-th component of {\tt vnames}, $j = 1,  \ldots , n$, 
contains the name of $x_{j}$.  
              
\ittf{x} is a rank-one allocatable array of dimension {\tt n} and type 
default \realdp, 
that holds the values $\bmx$ of the optimization variables.
The $j$-th component of {\tt x}, $j = 1,  \ldots , n$, contains $x_{j}$.  

\ittf{x\_l} is a rank-one allocatable array of dimension {\tt n} and type 
default \realdp, that holds
the vector of lower bounds $\bmx^{l}$ on the variables.
The $j$-th component of {\tt x\_l}, $j = 1, \ldots , n$, 
contains $\bmx_{j}^{l}$.
Infinite bounds are allowed by setting the corresponding 
components of {\tt x\_l} to any value smaller than {\tt -infinity}.

\ittf{x\_u} is a rank-one allocatable array of dimension {\tt n} and type 
default \realdp, that holds
the vector of upper bounds $\bmx^{u}$ on the variables.
The $j$-th component of {\tt x\_u}, $j = 1, \ldots , n$, 
contains $\bmx_{j}^{u}$.
Infinite bounds are allowed by setting the corresponding 
components of {\tt X\_u} to any value larger than that {\tt infinity}. 

\ittf{z} is a rank-one allocatable array of dimension {\tt n} and type default 
\realdp, that holds
the values $\bmz$ of estimates  of the dual variables 
corresponding to the simple bound constraints (see Section~\ref{galopt}).
The $j$-th component of {\tt z}, $j = 1,  \ldots ,  n$, contains $z_{j}$.  

\itt{x\_status} is  a rank-one allocatable array of dimension {\tt n} and type 
default \integer, that holds the status of the problem's variables
corresponding to the presence of their bounds. The $j$-th component of 
{\tt x\_status}, $j = 1,  \ldots ,  n$, contains the status of $\bmx_j$.
Typical values are
{\sym GALAHAD\_FREE},  {\sym GALAHAD\_LOWER}, {\sym GALAHAD\_UPPER},
{\sym GALAHAD\_RANGE}, {\sym GALAHAD\_FIXED}, {\sym GALAHAD\_STRUCTURAL},
{\sym GALAHAD\_ELIMINATED}, {\sym GALAHAD\_ACTIVE}, 
{\sym GALAHAD\_INACTIVE} or {\sym GALAHAD\_UNDEFINED}.
        
\ittf{f} is a scalar variable of type default \realdp,
that holds the current value of the objective function.

\ittf{g} is a rank-one allocatable array of dimension {\tt n} and type 
default \realdp, that holds the gradient $\bmg$ 
of of the objective function.
The $j$-th component of 
{\tt g}, $j = 1,  \ldots ,  n$, contains $\bmg_{j}$.

\itt{H\_type} is a scalar variable of type default \integer, 
that specifies the type of storage used for the lower triangle of the 
objective function's or Lagrangian's Hessian $\bmH$
Possible values are {\sym GALAHAD\_DENSE}, {\sym GALAHAD\_COORDINATE} 
or {\sym GALAHAD\_SPARSE\_BY\_ROWS}.

\itt{H\_ne} is a scalar variable of type default \integer, 
that holds the number of non-zero entries in the lower triangle of the
objective function's or Lagrangian's Hessian $\bmH$.

\itt{H\_val} is a rank-one allocatable array of type default \realdp, that holds
the values of the entries of the {\bf lower triangular} part
of the Hessian matrix $\bmH$ in any of the 
storage schemes discussed in Section~\ref{galmatrix}.

\itt{H\_row} is a rank-one allocatable array of type default \integer,
that holds the row indices of the {\bf lower triangular} part of $\bmH$ 
in the sparse co-ordinate storage
scheme (see Section~\ref{coordinate}). 
It need not be allocated for either of the other two schemes.

\itt{H\_col} is a rank-one allocatable array variable of type default \integer,
that holds the column indices of the {\bf lower triangular} part of 
$\bmH$ in either the sparse co-ordinate 
(see Section~\ref{coordinate}), or the sparse row-wise 
(see Section~\ref{rowwise}) storage scheme.
It need not be allocated when the dense storage scheme is used.

\itt{H\_ptr} is a rank-one allocatable array of dimension {\tt n+1} and type 
default \integer, that holds the starting position of 
each row of the {\bf lower triangular} part of $\bmH$, as well
as the total number of entries plus one, in the sparse row-wise storage
scheme (see Section~\ref{rowwise}). It need not be allocated when the
other schemes are used.

\ittf{m} is a scalar variable of type default \integer, 
 that holds the number of general linear constraints, $m$.

\ittf{c} is a rank-one allocatable array of dimension {\tt m} and type default 
\realdp, that holds
the values $\bmc(\bmx )$ of the constraints.
The $i$-th component of {\tt c}, $i = 1,  \ldots ,  m$, contains 
$\bmc_{i}(\bmx)$.  

\ittf{c\_l} is a rank-one allocatable array of dimension {\tt m} and type 
default \realdp, that holds the vector of lower bounds $\bmc^{l}$ 
on the general constraints. The $i$-th component of 
{\tt c\_l}, $i = 1, \ldots , m$, contains $\bmc_{i}^{l}$.
Infinite bounds are allowed by setting the corresponding 
components of {\tt c\_l} to any value smaller than {\tt -infinity}. 

\ittf{c\_u} is a rank-one allocatable array of dimension {\tt m} and type 
default \realdp, that holds the vector of upper bounds $\bmc^{u}$ 
on the general constraints. The $i$-th component of 
{\tt c\_u}, $i = 1,  \ldots ,  m$, contains $\bmc_{i}^{u}$.
Infinite bounds are allowed by setting the corresponding 
components of {\tt c\_u} to any value larger than {\tt infinity}.

\itt{equation} is a rank-one allocatable array of dimension {\tt m} and
type default \logical, that specifies if each constraint is an equality
or an inequality. The $i$-th component of {\tt equation} is {\tt .TRUE.}
iff the $i$-th constraint is an equality, i.e.\ iff $\bmc^l_i = \bmc^u_i$.

\itt{linear} is a rank-one allocatable array of dimension {\tt m} and
type default \logical, that specifies if each constraint is linear.
The $i$-th component of {\tt linear} is {\tt .TRUE.}
iff the $i$-th constraint is linear.

\ittf{y} is a rank-one allocatable array of dimension {\tt m} and type 
default \realdp, that holds
the values $\bmy$ of estimates  of the Lagrange multipliers
corresponding to the general constraints (see Section~\ref{galopt}).
The $i$-th component of {\tt y}, $i = 1,  \ldots ,  m$, contains $\bmy_{i}$.  

\itt{c\_status} is  a rank-one allocatable array of dimension {\tt m} and type 
default \integer, that holds the status of the problem's constraints
corresponding to the presence of their bounds. The $i$-th component of 
{\tt c\_status}, $j = 1,  \ldots ,  m$, contains the status of $\bmc_i$.
Typical values are
{\sym GALAHAD\_FREE},  {\sym GALAHAD\_LOWER}, {\sym GALAHAD\_UPPER},
{\sym GALAHAD\_RANGE}, {\sym GALAHAD\_FIXED}, {\sym GALAHAD\_STRUCTURAL},
{\sym GALAHAD\_ELIMINATED}, {\sym GALAHAD\_ACTIVE}, 
{\sym GALAHAD\_INACTIVE} or {\sym GALAHAD\_UNDEFINED}.
        
\itt{J\_type} is a scalar variable of type default \integer, 
that specifies the type of storage used for the 
constraints' Jacobian $\bmJ$.
Possible values are {\sym GALAHAD\_DENSE}, {\sym GALAHAD\_COORDINATE} 
or {\sym GALAHAD\_SPARSE\_BY\_ROWS}.
              
\itt{J\_ne} is a scalar variable of type default \integer, 
that holds the number of non-zero entries in the 
constraints' Jacobian $\bmJ$.

\itt{J\_val} is a rank-one allocatable array of type default \realdp, that holds
the values of the entries of the Jacobian matrix $\bmJ$ in any of the 
storage schemes discussed in Section~\ref{galmatrix}.

\itt{J\_row} is a rank-one allocatable array of type default \integer,
that holds the row indices of $\bmJ$ in the sparse co-ordinate storage
scheme (see Section~\ref{coordinate}). 
It need not be allocated for either of the other two schemes.

\itt{J\_col} is a rank-one allocatable array variable of type default \integer,
that holds the column indices of $\bmJ$ in either the sparse co-ordinate 
(see Section~\ref{coordinate}), or the sparse row-wise 
(see Section~\ref{rowwise}) storage scheme.
It need not be allocated when the dense storage scheme is used.

\itt{J\_ptr} is a rank-one allocatable array of dimension {\tt m+1} and type 
default \integer, that holds the 
starting position of each row of $\bmJ$, as well
as the total number of entries plus one, in the sparse row-wise storage
scheme (see Section~\ref{rowwise}). It need not be allocated when the
other schemes are used.

\ittf{gL} is a rank-one allocatable array of dimension {\tt n} and type 
default \realdp, that holds the gradient $\bmg_L$ 
of of the problem's Lagrangian $\bmL$ with respect to $\bmx$.
The $j$-th component of 
{\tt gL}, $j = 1,  \ldots ,  n$, contains $[\bmg_L]_{j}$.

\end{description}

\noindent
Note that not every component of this data type is used by every package.

%%%%%%%%%%%%%%%%%%%%%% argument lists %%%%%%%%%%%%%%%%%%%%%%%%

\galarguments
There are seven procedures for user calls:

\begin{enumerate}
\item The subroutine {\tt \packagename\_write\_stats} 
      is used to write general information on the problem such
      as the number of variables and constraints of different types.
\item The subroutine {\tt \packagename\_write\_variables}
      is used to write the current values of the problem's variables, bounds
      and of their associated duals.
\item The subroutine {\tt \packagename\_write\_constraints}
      is used to write the current values of the problem's constraints, bounds
      and of their associated multipliers.
\item The subroutine {\tt \packagename\_write\_problem}
      is used to write the problem's number of variables and
      constraints per type, as well as current values of the problem's 
      variables and constraints. This broadly corresponds to successively 
      calling the three subroutines mentioned above. The subroutine
      additionally (optionally) writes the values of the Lagrangian's Hessian
      $\bmH$ and constraints Jacobian $\bmJ$.
\item The subroutine {\tt \packagename\_J\_from\_C\_to\_Srow}
      builds the permutation that transforms the Jacobian from
      coordinate storage to sparse-by-row storage, as well as the
      {\tt J\_ptr} and {\tt J\_col} vectors.
\item The subroutine {\tt \packagename\_J\_from\_C\_to\_Scol}
      builds the permutation that transforms the Jacobian from
      coordinate storage to sparse-by-column storage, as well as the
      {\tt J\_ptr} and {\tt J\_row} vectors.
\item The subroutine {\tt \packagename\_cleanup} is used to deallocate
      the memory space used by a problem data structure.
\end{enumerate}

\subsubsection{Writing the problem's statistics}\label{w_stats}

The number of variables and constraints for each type of bounds (free,
lower/upper bounded, range bounded, linear, equalities/fixed) is output
by using the call
\vspace*{1mm}

\hspace{8mm}
{\tt CALL \packagename\_write\_stats( problem, out )}

\noindent where
\begin{description}
\itt{problem} is a scalar \intentin\ argument of type {\tt NLPT\_problem\_type},
that holds the problem for which statistics must be written.

\ittf{out} is a scalar \intentin\ argument of type default \integer, that
holds the device number on which problem statistics should be written.
\end{description}

\noindent
Note that {\tt problem\%pname} is assumed to be defined and that both
{\tt problem\%c\_l} and {\tt problem\%c\_u} are assumed to be associated
whenever {\tt problem\%m} $> 0$.

\subsubsection{Writing the problem's variables, bounds and duals}
\label{w_vars}


The values of the variables and associated bounds and duals is output
by using the call
\vspace*{1mm}

\hspace{8mm}
{\tt CALL \packagename\_write\_variables( problem, out )}

\noindent where
\begin{description}
\itt{problem} is a scalar \intentin\ argument of type {\tt NLPT\_problem\_type},
that holds the problem for which variables values, bounds and duals must be
written.

\ittf{out} is a scalar \intentin\ argument of type default \integer, that
holds the device number on which problem variables values, bounds and duals
should be written.
\end{description}

\noindent
This routine assumes that {\tt problem\%pname} and {\tt problem\%x} are
associated. The bounds are printed whenever {\tt problem\%x\_l} and 
{\tt problem\%x\_u} are associated. Moreover, it is also
assumed in this case that {\tt problem\%g} is associated when {\tt problem\%m}
$= 0$, and that {\tt problem\%z}  is associated when {\tt problem\%m} $> 0$. 
The variables' names are used whenever {\tt problem\%vnames} is associated, but
this is not mandatory.

\subsubsection{Writing the problem's constraints, bounds and multipliers}
\label{w_cons}

The values of the constraints and associated bounds and multipliers is output
by using the call
\vspace*{1mm}

\hspace{8mm}
{\tt CALL \packagename\_write\_constraints( problem, out )}

\noindent where
\begin{description}
\itt{problem} is a scalar \intentin\ argument of type {\tt NLPT\_problem\_type},
that holds the problem for which constraints values, bounds and multipliers
must be written.

\ittf{out} is a scalar \intentin\ argument of type default \integer, that
holds the device number on which problem constraints values, bounds and
multipliers should be written.
\end{description}

\noindent
This routine assumes that {\tt problem\%pname}, {\tt problem\%c} {\tt
problem\%c\_l}, {\tt problem\%c\_u} and {\tt problem\%y} are associated.  The
types of constraints are used whenever {\tt problem\%equation} and/or {\tt
problem\%linear} are associated, but this is not mandatory.  The constraints'
names are used whenever {\tt problem\%cnames} is associated, but this is not
mandatory.

\subsubsection{Writing the entire problem}
\label{w_prob}

The most important data of a problem can be output by the call
\vspace*{1mm}

\hspace{8mm}
{\tt CALL \packagename\_write\_problem( problem, out, print\_level )}

\noindent where
\begin{description}
\itt{problem} is a scalar \intentin\ argument of type {\tt NLPT\_problem\_type},
that holds the problem whose data must be written.

\ittf{out} is a scalar \intentin\ argument of type default \integer, that
holds the device number on which the problem data should be written.

\itt{print\_level} is a scalar \intentin\ argument of type default \integer,
that holds the level of details required for output. Possible values are:
\begin{description}
\item[\sym GALAHAD\_SILENT:] no output is produced;
\item[\sym GALAHAD\_TRACE:] the problem's statistics are output, plus the
norms of the current vector of variables, the objective function's value and
the norm of its gradient, and the maximal bound and constraint violations.
\item[\sym GALAHAD\_ACTION:] the problem's statistics are output, plus the
values of the variables, bounds and associated duals, the value of the
objective function, the value of the objective function's gradient, the values
of the constraints and associated bounds and multipliers.
\item[\sym GALAHAD\_DETAILS:] as for {\sym GALAHAD\_ACTION}, plus the values
of the Lagrangian's Hessian and of the constraints' Jacobian.
\end{description}

\noindent
This routine assumes that {\tt problem\%pname} and {\tt problem\%x} are
associated. The bounds on the variables are printed whenever {\tt
problem\%x\_l} and {\tt problem\%x\_u} are associated. Moreover, it is also
assumed in this case that {\tt problem\%g} is associated when {\tt problem\%m}
$= 0$, and that {\tt problem\%z} is associated when {\tt problem\%m} $> 0$.
The variables' names are used whenever {\tt problem\%vnames} is associated,
but this is not mandatory. In the case where {\tt problem\%m} $>0$, it is
furthermore assumed that {\tt problem\%c} {\tt problem\%c\_l}, {\tt
problem\%c\_u} and {\tt problem\%y} are associated.  The types of constraints
are used whenever {\tt problem\%equation} and/or {\tt problem\%linear} are
associated, but this is not mandatory.  The constraints' names are used
whenever {\tt problem\%cnames} is associated, but this is not

\end{description}

\subsubsection{Problem cleanup}
\label{cleanup}

The memory space allocated to allocatable in the problem data structure is
deallocated by the call
\vspace*{1mm}

\hspace{8mm}
{\tt CALL \packagename\_cleanup( problem )}

\noindent where
\begin{description}
\itt{problem} is a scalar \intentin\ argument of type {\tt NLPT\_problem\_type},
that holds the problem whose memory space must be deallocated.
\end{description}

\subsubsection{Transforming the Jacobian from co-ordinate storage to
sparse-by-rows}
\label{toSrow}

The permutation that transforms the Jacobian from co-ordinate 
storage to sparse-by-rows, as well as the associated {\tt ptr} and {\tt col}
vectors can be obatined by the call
\vspace*{1mm}

\hspace{8mm}
{\tt CALL \packagename\_J\_perm\_from\_C\_to\_Srow( problem, perm, col, ptr )}

\noindent where
\begin{description}
\itt{problem} is a scalar \intentin\ argument of type {\tt
NLPT\_problem\_type}, that holds the Jacobian matrix to transform. Note that
we must have {problem\%J\_type} $=$ {\sym GALAHAD\_COORDINATE}.

\itt{perm} is an allocatable to a vector \intentout\ of type default \integer\ and
dimension equal to {\tt problem\%J\_nnz}, that returns the permutation of the
elements of {\tt problem\%J\_val} that must be applied to transform the
Jacobian from co-ordinate storage to sparse-by-rows.

\ittf{col} is an allocatable to a vector \intentout\ of type default \integer\ and
dimension {\tt problem\%J\_ne} whose $k$-th component is the column index of 
the $k$-th element of {\tt problem\%J\_val} after permutation by {\tt perm}.

\ittf{ptr} is an allocatable to a vector \intentout\ of type default \integer\ and
dimension {\tt problem\%m} $+1$ whose $i$-the component is the index in {\tt
problem\%J\_val} (after permutation by {\tt perm}) of the first entry of row
$i$. Moreover,
\[
{\tt ptr}({\tt problem\%m}+1 ) = {\tt problem\%J\_ne} + 1.
\]
\end{description}

\subsubsection{Transforming the Jacobian from co-ordinate storage to
sparse-by-columns}
\label{toScol}

The permutation that transforms the Jacobian from co-ordinate 
storage to sparse-by-columns, as well as the associated {\tt ptr} and {\tt row}
vectors can be obtained by the call
\vspace*{1mm}

\hspace{8mm}
{\tt CALL \packagename\_J\_perm\_from\_C\_to\_Scol( problem, perm, row, ptr )}

\noindent where
\begin{description}
\itt{problem} is a scalar \intentin\ argument of type {\tt
NLPT\_problem\_type}, that holds the Jacobian matrix to transform. Note that
we must have {problem\%J\_type} $=$ {\sym GALAHAD\_COORDINATE}.

\itt{perm} is an allocatable to a vector \intentout\ of type default \integer\ and
dimension equal to {\tt problem\%J\_nnz}, that returns the permutation of the
elements of {\tt problem\%J\_val} that must be applied to transform the
Jacobian from co-ordinate storage to sparse-by-columns.

\ittf{col} is an allocatable to a vector \intentout\ of type default \integer\ and
dimension {\tt problem\%J\_ne} whose $k$-th component is the row index of 
the $k$-th element of {\tt problem\%J\_val} after permutation by {\tt perm}.

\ittf{ptr} is an allocatable to a vector \intentout\ of type default \integer\ and
dimension {\tt problem\%m} $+1$ whose $i$-the component is the index in {\tt
problem\%J\_val} (after permutation by {\tt perm}) of the first entry of column
$i$. Moreover,
\[
{\tt ptr}({\tt problem\%m}+1 ) = {\tt problem\%J\_ne} + 1.
\]
\end{description}

%%%%%%%%%%%%%%%%%%%%%% GENERAL INFORMATION %%%%%%%%%%%%%%%%%%%%%%%%

\galgeneral

\galmodules None.
\galroutines {\tt \packagename\_solve} calls the BLAS functions {\tt *NRM2},
where {\tt *} is {\tt S} for the default real version and {\tt D} for the
double precision version.
\galmodules {\tt \packagename} calls the {\tt TOOLS} \galahad\ module.
\galio Output is under the control of the {\tt print\_level} argument for
the {\tt \packagename\_write\_problem} subroutine.
\galrestrictions {\tt problem\%n} $> 0$, 
{\tt problem\%m} $\geq  0$. Additionally, the subroutines
{\tt \packagename\_write\_*} require that {\tt problem\%n} $<10^{14}$ and
{\tt problem\%m} $< 10^{14}$.
\galportability ISO Fortran~95 + TR 15581 or Fortran~2003. 
The package is thread-safe.
\end{description}

%%%%%%%%%%%%%%%%%%%%%% EXAMPLE %%%%%%%%%%%%%%%%%%%%%%%%

\galexample
Suppose we wish to present the data for the problem of minimizing
the objective function
$(\bmx_1 - 2 ) \bmx_2$
subject to the constraints
$\bmx_1^2 +\bmx_2^2 \leq 1$,
$0 \leq -\bmx_1 + \bmx_2 $, and the simple bound
$0 \leq  \bmx_1$, where the values are computed at the point
$\bmx^T = (0,1)$, which, together with the values $z_1 = 1$
and $\bmy^T = ( -1, 0 )$ defines a first-order critical point for the problem.
Assume that we wish to store the Lagrangian's Hessian and the Jacobian in
co-ordinate format. Assume also that we wish to write this data.  We may
accomplish these objectives by using the code:
{\tt \small
\begin{verbatim}
PROGRAM GALAHAD_NLPT_EXAMPLE
  USE GALAHAD_NLPT_double      ! the problem type
  USE GALAHAD_SYMBOLS
  IMPLICIT NONE
  INTEGER, PARAMETER                :: wp = KIND( 1.0D+0 )
  INTEGER,           PARAMETER      :: iout = 6        ! stdout and stderr
  REAL( KIND = wp ), PARAMETER      :: INFINITY = (10.0_wp)**19
  TYPE( NLPT_problem_type     )     :: problem 
! Set the problem up.
  problem%pname    = 'NLPT-TEST'
  problem%infinity = INFINITY
  problem%n        = 2
  ALLOCATE( problem%vnames( problem%n ), problem%x( problem%n )  ,             &
            problem%x_l( problem%n )   , problem%x_u( problem%n ),             &
            problem%g( problem%n )     , problem%z( problem%n )  )
  problem%m        = 2
  ALLOCATE( problem%equation( problem%m ), problem%linear( problem%m ),        &
            problem%c( problem%m ) , problem%c_l( problem%m ),                 &
            problem%c_u( problem%m), problem%y( problem%m ),                   &
            problem%cnames( problem%m ) ) 
  problem%J_ne     = 4
  ALLOCATE( problem%J_val( problem%J_ne ), problem%J_row( problem%J_ne ),      &
            problem%J_col( problem%J_ne ) )
  problem%H_ne     = 3
  ALLOCATE( problem%H_val( problem%H_ne ), problem%H_row( problem%H_ne ),      &
            problem%H_col( problem%H_ne ) )
  problem%H_type   = GALAHAD_COORDINATE
  problem%J_type   = GALAHAD_COORDINATE
  problem%vnames   = (/    'X1'  ,    'X2'   /)
  problem%x        = (/   0.0D0  ,   1.0D0   /)
  problem%x_l      = (/   0.0D0  , -INFINITY /)
  problem%x_u      = (/  INFINITY,  INFINITY /)
  problem%cnames   = (/    'C1'  ,    'C2'   /)
  problem%c        = (/   0.0D0  ,   1.0D0   /)
  problem%c_l      = (/ -INFINITY,   0.0D0   /)
  problem%c_u      = (/   1.0D0  ,  INFINITY /)
  problem%y        = (/  -1.0D0  ,   0.0D0   /)
  problem%equation = (/  .FALSE. ,  .FALSE.  /)
  problem%linear   = (/  .FALSE. ,   .TRUE.  /)
  problem%z        = (/   1.0D0  ,   0.0D0   /)
  problem%f        = -2.0_wp
  problem%g        = (/   1.0D0  ,  -1.0D0   /)
  problem%J_row    = (/     1    ,     1     ,     2     ,     2     /)
  problem%J_col    = (/     1    ,     2     ,     1     ,     2     /)
  problem%J_val    = (/   0.0D0  ,   2.0D0   ,  -1.0D0   ,   1.0D0   /)
  problem%H_row    = (/     1    ,     2     ,     2     /)
  problem%H_col    = (/     1    ,     1     ,     2     /)
  problem%H_val    = (/   2.0D0  ,   1.0D0   ,   2.0D0   /)
  NULLIFY( problem%x_status, problem%H_ptr, problem%J_ptr, problem%gL )
  CALL NLPT_write_problem( problem, iout, GALAHAD_DETAILS )
! Cleanup the problem.
  CALL NLPT_cleanup( problem )
  STOP
END PROGRAM GALAHAD_NLPT_EXAMPLE
\end{verbatim}
}
which gives the following output:
{\small
\begin{verbatim}

           +--------------------------------------------------------+
           |                  Problem : NLPT-TEST                   |
           +--------------------------------------------------------+

                 Free    Lower    Upper    Range     Fixed/   Linear  Total
                       bounded  bounded  bounded  equalities
 Variables          1        1        0        0        0                 2
 Constraints                 1        1        0        0        1        2


           +--------------------------------------------------------+
           |                  Problem : NLPT-TEST                   |
           +--------------------------------------------------------+

     j Name           Lower         Value        Upper     Dual value

     1 X1           0.0000E+00   0.0000E+00                1.0000E+00
     2 X2                        1.0000E+00


 OBJECTIVE FUNCTION value     = -2.0000000E+00

 GRADIENT of the objective function:
  
    1   1.000000E+00  -1.000000E+00
  

 Lower triangle of the HESSIAN of the Lagrangian:
 
    i    j       value          i    j       value          i    j       value  
  
    1    1    2.0000E+00        2    1    1.0000E+00        2    2    2.0000E+00
 

           +--------------------------------------------------------+
           |                  Problem : NLPT-TEST                   |
           +--------------------------------------------------------+

     i Name           Lower         Value        Upper     Dual value

     1 C1                        0.0000E+00   1.0000E+00  -1.0000E+00           
     2 C2           0.0000E+00   1.0000E+00                0.0000E+00     linear



 JACOBIAN matrix:
 
    i    j       value          i    j       value          i    j       value  
  
    1    1    0.0000E+00        1    2    2.0000E+00        2    1   -1.0000E+00
    2    2    1.0000E+00
 

           -------------------- END OF PROBLEM ----------------------

\end{verbatim}
}
\noindent
We could choose to hold the lower triangle of $\bmH$ is sparse-by-rows format
by replacing the lines
{\tt \small
\begin{verbatim}
  ALLOCATE( problem%H_val( problem%H_ne ), problem%H_row( problem%H_ne ),      &
            problem%H_col( problem%H_ne ) )
  problem%H_type   = GALAHAD_COORDINATE
\end{verbatim}
}
\noindent
and
{\tt \small
\begin{verbatim}
  problem%H_row    = (/     1    ,     2     ,     2     /)
  problem%H_col    = (/     1    ,     1     ,     2     /)
  problem%H_val    = (/   2.0D0  ,   1.0D0   ,   2.0D0   /)
  NULLIFY( problem%x_status, problem%H_ptr, problem%J_ptr, problem%gL )
\end{verbatim}
}
\noindent
by
{\tt \small
\begin{verbatim}
  ALLOCATE( problem%H_val( problem%H_ne ), problem%H_col( problem%H_ne ),      &
            problem%H_ptr( problem%n + 1 ) )
  problem%H_type   = GALAHAD_SPARSE_BY_ROWS
\end{verbatim}
}
\noindent
and
{\tt \small
\begin{verbatim}
  problem%H_ptr    = (/     1    ,     2     ,     4     /)
  problem%H_col    = (/     1    ,     1     ,     2     /)
  problem%H_val    = (/   2.0D0  ,   1.0D0   ,   2.0D0   /)
  NULLIFY( problem%x_status, problem%H_row, problem%J_ptr, problem%gL )
\end{verbatim}
}
\noindent
or using a dense storage format with the replacement lines
{\tt \small
\begin{verbatim}
  ALLOCATE( problem%H_val( ( ( problem%n + 1 ) * problem%n ) / 2 ) )
  problem%H_type   = GALAHAD_DENSE
\end{verbatim}
}
\noindent
and 
{\tt \small
\begin{verbatim}
  problem%H_val    = (/   2.0D0  ,   1.0D0   ,   2.0D0   /)
  NULLIFY( problem%x_status, problem%H_row, problem%H_col, problem%H_ptr,      &
           problem%J_ptr, problem%gL )
\end{verbatim}
}
\noindent
respectively.

For examples of how the derived data type
{\tt \packagename\_problem\_type} may be used in conjunction with the
\galahad\ nonlinear feasibility code, see the specification sheets 
for the {\tt \libraryname\_FILTRANE} package.
\end{document}


