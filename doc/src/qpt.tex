\documentclass{galahad}

% set the package name

\newcommand{\package}{qpt}
\newcommand{\packagename}{QPT}
\newcommand{\fullpackagename}{\libraryname\_\packagename}

\begin{document}

\galheader

%%%%%%%%%%%%%%%%%%%%%% SUMMARY %%%%%%%%%%%%%%%%%%%%%%%%

\galsummary
This package defines a derived type capable of supporting
a variety of {\bf quadratic programming problem storage schemes.}
Quadratic programming aims to minimize or maximize either
a general objective function
\eqn{qp}{\half \bmx^T \bmH \bmx + \bmg^T \bmx + f,}
or sometimes a (shifted) squared-least-distance objective function,
\eqn{lsqp}{\half \sum_{j=1}^n w_j^2 ( x_j^{ } - x_j^0 )^2 + \bmg^T \bmx + f,}
subject to the general linear constraints
\disp{c_{i}^{l}  \leq  \bma_{i}^{T}\bmx  \leq  c_{i}^{u}, \;\;\;
 i = 1, \ldots , m,}
and the simple bound constraints
\disp{x_{j}^{l} \leq x_{j}^{ } \leq x_{j}^{u} , \;\;\; j = 1, \ldots , n,}
where the $n$ by $n$ symmetric matrix $\bmH$,
the vectors $\bmg$, $\bmw$, $\bmx^{0}$,
$\bma_{i}$, $\bmc^{l}$, $\bmc^{u}$, $\bmx^{l}$,
and $\bmx^{u}$, and the scalar $f$ are given.
Full advantage is taken of any zero coefficients in the matrix $\bmH$ or the
vectors $\bma_{i}$.
Any of the constraint bounds $c_{i}^{l}$, $c_{i}^{u}$,
$x_{j}^{l}$ and $x_{j}^{u}$ may be infinite.

The derived type is also capable of supporting {\em parametric}
quadratic programming problems, in which an additional objective
term $\theta \delta \bmg^T \bmx$ is included, and the trajectory of
solution are required for all $0 \leq \theta \leq \theta_{\max}$
for which
\disp{
c_{i}^{l}  + \theta \delta c_{i}^{l}
\leq  \bma_{i}^{T}\bmx  \leq  c_{i}^{u} + \theta \delta c_{i}^{u}, \;\;\;
 i = 1, \ldots , m,}
and
\disp{x_{j}^{l} + \theta x_{j}^{l} \leq x_{j}^{ } \leq
x_{j}^{u} + \delta x_{j}^{u} , \;\;\; j = 1, \ldots , n.}

The principal use of the package is to allow exchange of data between
\galahad\ subprograms and other codes.

%%%%%%%%%%%%%%%%%%%%%% attributes %%%%%%%%%%%%%%%%%%%%%%%%

\galattributes
\galversions{\tt  \fullpackagename\_single, \fullpackagename\_double}.
\galuses {\tt \libraryname\_SMT}.
\galdate April 2001.
\galorigin N. I. M. Gould, Rutherford Appleton Laboratory, and
Ph. L. Toint, University of Namur, Belgium.
\gallanguage Fortran~95 + TR 15581 or Fortran~2003.

%%%%%%%%%%%%%%%%%%%%%% HOW TO USE %%%%%%%%%%%%%%%%%%%%%%%%

\galhowto

%\subsection{Calling sequences}

Access to the package requires a {\tt USE} statement such as

\medskip\noindent{\em Single precision version}

\hspace{8mm} {\tt USE \fullpackagename\_single}

\medskip\noindent{\em Double precision version}

\hspace{8mm} {\tt USE  \fullpackagename\_double}

\medskip

\noindent
If it is required to use both modules at the same time, the derived types
{\tt SMT\_TYPE}
and
{\tt QPT\_problem\_type},
(Section~\ref{galtype})
must be renamed on one of the {\tt USE} statements.

%%%%%%%%%%%%%%%%%%%%%% matrix formats %%%%%%%%%%%%%%%%%%%%%%%%

\galmatrix
Both the Hessian matrix $\bmH$ and
the constraint Jacobian $\bmA$, the matrix
whose rows are the vectors $\bma_{i}^{T}$, $i = 1, \ldots , m$,
may be stored in a variety of input formats.

\subsubsection{Dense storage format}\label{dense}
The matrix $\bmA$ is stored as a compact
dense matrix by rows, that is, the values of the entries of each row in turn are
stored in order within an appropriate real one-dimensional array.
Component $n \ast (i-1) + j$ of the storage array {\tt A\%val} will hold the
value $a_{ij}$ for $i = 1, \ldots , m$, $j = 1, \ldots , n$.
Since $\bmH$ is symmetric, only the lower triangular part (that is the part
$h_{ij}$ for $1 \leq j \leq i \leq n$) need be held. In this case
the lower triangle will be stored by rows, that is
component $i \ast (i-1)/2 + j$ of the storage array {\tt H\%val}
will hold the value $h_{ij}$ (and, by symmetry, $h_{ji}$)
for $1 \leq j \leq i \leq n$.

\subsubsection{Sparse co-ordinate storage format}\label{coordinate}
Only the nonzero entries of the matrices are stored. For the
$l$-th entry of $\bmA$, its row index $i$, column index $j$
and value $a_{ij}$
are stored in the $l$-th components of the integer arrays {\tt A\%row},
{\tt A\%col} and real array {\tt A\%val}, respectively.
The order is unimportant, but the total
number of entries {\tt A\%ne} is also required.
The same scheme is applicable to
$\bmH$ (thus requiring integer arrays {\tt H\%row}, {\tt H\%col}, a real array
{\tt H\%val} and an integer value {\tt H\%ne}),
except that only the entries in the lower triangle need be stored.

\subsubsection{Sparse row-wise storage format}\label{rowwise}
Again only the nonzero entries are stored, but this time
they are ordered so that those in row $i$ appear directly before those
in row $i+1$. For the $i$-th row of $\bmA$, the $i$-th component of a
integer array {\tt A\%ptr} holds the position of the first entry in this row,
while {\tt A\%ptr} $(m+1)$ holds the total number of entries plus one.
The column indices $j$ and values $a_{ij}$ of the entries in the $i$-th row
are stored in components
$l =$ {\tt A\%ptr}$(i)$, \ldots ,{\tt A\%ptr} $(i+1)-1$ of the
integer array {\tt A\%col}, and real array {\tt A\%val}, respectively.
The same scheme is applicable to
$\bmH$ (thus requiring integer arrays {\tt H\%ptr}, {\tt H\%col}, and
a real array {\tt H\%val}),
except that only the entries in the lower triangle need be stored.

For sparse matrices, this scheme almost always requires less storage than
its predecessor.

\subsubsection{Diagonal storage format}\label{diagonal}
If $\bmH$ is diagonal (i.e., $h_{ij} = 0$ for all $1 \leq i \neq j \leq n$)
only the diagonals entries $h_{ii}$, $1 \leq i \leq n$,  need be stored,
and the first $n$ components of the array {\tt H\%val} may be used for
the purpose. There is no sensible equivalent for the non-square $\bmA$.

\subsubsection{Scaled-identity-matrix storage format}\label{scaled-identity}
If $\bmH$ is a scalar multiple of the identity matrix
(i.e., $h_{ii} = h_{11}$  and $h_{ij} = 0$ for all $1 \leq i \neq j \leq n$)
only the first diagonal entry $h_{11}$ needs be stored,
and the first component of the array {\tt H\%val} may be used for
the purpose. Again, there is no sensible equivalent for the non-square $\bmA$.

\subsubsection{Identity-matrix storage format}\label{identity}
If $\bmH$ is the identity matrix
(i.e., $h_{ii} = 1$ and $h_{ij} = 0$ for all $1 \leq i \neq j \leq n$),
no explicit entries needs be stored.

\subsubsection{Zero-matrix storage format}\label{zero}
If $\bmH = \bmzero$ (i.e., $h_{ij} = 0$ for all $1 \leq i, j \leq n$),
no explicit entries needs be stored.

%%%%%%%%%%%%%%%%%%% optimality conditions %%%%%%%%%%%%%%%%%%%

\subsection{Optimality conditions\label{galopt}}

The required solution $\bmx$ necessarily satisfies
the primal optimality conditions
\disp{\bmA \bmx = \bmc, \;\;
 \bmc^{l} \leq \bmc \leq \bmc^{u}, \tim{and}
 \bmx^{l} \leq\bmx \leq\bmx^{u},}
the dual optimality conditions
\disp{
 \bmH \bmx + \bmg = \bmA^{T} \bmy + \bmz \;\;
 (\mbox{or} \;\; \bmW^{2} (\bmx -\bmx^{0}) + \bmg =
 \bmA^{T} \bmy + \bmz \;\; \mbox{for the least-distance type objective})}
where
\disp{\bmy = \bmy^{l} + \bmy^{u}, \;\;
 \bmz = \bmz^{l} + \bmz^{u}\,\,
 \bmy^{l} \geq 0 , \;\;
 \bmy^{u} \leq 0 , \;\;
 \bmz^{l} \geq 0 \tim{and}
 \bmz^{u} \leq 0 ,}
and the complementary slackness conditions
\disp{
 ( \bmA \bmx - \bmc^{l} )^{T} \bmy^{l} = 0  ,\;\;
 ( \bmA \bmx - \bmc^{u} )^{T} \bmy^{u} = 0  ,\;\;
 (\bmx -\bmx^{l} )^{T} \bmz^{l} = 0   \tim{and}
 (\bmx -\bmx^{u} )^{T} \bmz^{u} = 0 ,}
where the diagonal matrix $\bmW^{2}$ has diagonal entries $w_{j}^{2}$,
$j = 1, \ldots , n$, where the vectors $\bmy$ and $\bmz$ are
known as the Lagrange multipliers for
the general linear constraints, and the dual variables for the bounds,
respectively, and where the vector inequalities hold componentwise.

%%%%%%%%%%%%%%%%%%%%%% derived types %%%%%%%%%%%%%%%%%%%%%%%%

\galtype
Two derived data types,
{\tt SMT\_TYPE} and
{\tt \packagename\_problem\_type},
are accessible from the package. It is intended that, for any particular
application, only those components which are needed will be set.

%%%%%%%%%%% matrix data type %%%%%%%%%%%

\subsubsection{The derived data type for holding matrices}\label{typesmt}
The derived data type {\tt SMT\_TYPE} is used to hold the matrices $\bmA$
and $\bmH$. The components of {\tt SMT\_TYPE} used here are:

\begin{description}

\ittf{type} is an allocatable array of rank one and type default \character, that
holds a string which indicates the storage scheme used.

\ittf{m} is a scalar component of type default \integer,
that holds the number of rows in the matrix.

\ittf{n} is a scalar component of type default \integer,
that holds the number of columns in the matrix.

\ittf{ne} is a scalar variable of type default \integer, that may
hold the number of matrix entries (see \S\ref{coordinate}).

\ittf{val} is a rank-one allocatable array of type default \realdp\,
and dimension at least {\tt ne}, that holds the values of the entries.
Each pair of off-diagonal entries $h_{ij} = h_{ji}$ of a {\em symmetric}
matrix $\bmH$ is represented as a single entry
(see \S\ref{dense}--\ref{rowwise}).
Any duplicated entries that appear in the sparse
co-ordinate or row-wise schemes will be summed.

\ittf{row} is a rank-one allocatable array of type default \integer,
and dimension at least {\tt ne}, that may hold the row indices of the entries
(see \S\ref{coordinate}).

\ittf{col} is a rank-one allocatable array of type default \integer,
and dimension at least {\tt ne}, that may hold the column indices of the entries
(see \S\ref{coordinate}--\ref{rowwise}).

\ittf{ptr} is a rank-one allocatable array of type default \integer,
and dimension at least {\tt m + 1}, that may hold the pointers to
the first entry in each row (see \S\ref{rowwise}).

\end{description}

%%%%%%%%%%% problem type %%%%%%%%%%%

\subsubsection{The derived data type for holding quadratic programs}
\label{typeprob}

The derived data type
{\tt \packagename\_problem\_type}
is used to hold the problem.
The components of
{\tt \packagename\_problem\_type}
are:

\begin{description}

\ittf{name} is a rank-one allocatable array of type default \character\, that
may be used to hold the name of the problem.

\itt{new\_problem\_structure} is a scalar variable of type default \logical,
 that is \true\ if this is the first (or only) problem in a sequence of
 problems with identical "structure" to be attempted, and \false\ if
 a previous problem with the same "structure" (but different
 numerical data) has been solved. Here, the term "structure" refers both to
 the sparsity patterns of the Jacobian matrices $\bmA$ involved
 (but not their numerical values), to the zero/nonzero/infinity patterns
 (a bound is either zero, $\pm$ infinity, or a finite but arbitrary
 nonzero) of each of the constraint bounds, and to the variables and constraints
 that are fixed (both bounds are the same) or free (the lower and upper
 bounds are $\pm$ infinity, respectively).

\ittf{n} is a scalar variable of type default \integer,
 that holds the number of optimization variables, $n$.

\ittf{m} is a scalar variable of type default \integer,
 that holds the number of general linear constraints, $m$.

\itt{Hessian\_kind} is a scalar variable of type default \integer,
that is used to indicate what type of Hessian the problem involves.
Possible values for {\tt Hessian\_kind} are:

\begin{description}
\itt{<0}  In this case, a general quadratic program of the form
\req{qp} is given. The Hessian matrix $\bmH$ will be provided in the
component {\tt H} (see below).

\itt{0}  In this case, a linear program, that is a problem of the form
\req{lsqp} with weights $\bmw = 0$, is given.

\itt{1} In this case, a least-distance problem of the form \req{lsqp}
with weights $w_{j} = 1$ for $j = 1, \ldots , n$ is given.

\itt{>1} In this case, a weighted least-distance problem of the form \req{lsqp}
with general weights $\bmw$ is given. The weights will be
provided in the component {\tt WEIGHT} (see below).
\end{description}

\ittf{H} is scalar variable of type {\tt SMT\_TYPE}
that contains the Hessian matrix $\bmH$ whenever {\tt Hessian\_kind} $<0$.
The following components are used:

\begin{description}

\itt{H\%type} is an allocatable array of rank one and type default \character, that
is used to indicate the storage scheme used. If the dense storage scheme
(see Section~\ref{dense}) is used,
the first five components of {\tt H\%type} must contain the
string {\tt DENSE}.
For the sparse co-ordinate scheme (see Section~\ref{coordinate}),
the first ten components of {\tt H\%type} must contain the
string {\tt COORDINATE},
for the sparse row-wise storage scheme (see Section~\ref{rowwise}),
the first fourteen components of {\tt H\%type} must contain the
string {\tt SPARSE\_BY\_ROWS},
for the diagonal storage scheme (see Section~\ref{diagonal}),
the first eight components of {\tt H\%type} must contain the
string {\tt DIAGONAL},
for the scaled-identity matrix storage scheme
(see Section~\ref{scaled-identity}),
the first fifteen components of {\tt H\%type} must contain the
string {\tt SCALED\_IDENTITY},
for the identity matrix storage scheme
(see Section~\ref{identity}),
the first eight components of {\tt H\%type} must contain the
string {\tt IDENTITY}, and
for the zero matrix storage scheme
(see Section~\ref{zero}),
the first four components of {\tt H\%type} must contain the
string {\tt ZERO}.

For convenience, the procedure {\tt SMT\_put}
may be used to allocate sufficient space and insert the required keyword
into {\tt H\%type}.
For example, if {\tt prob} is of derived type {\tt \packagename\_problem\_type}
and involves a Hessian we wish to store using the co-ordinate scheme,
we may simply
%\vspace*{-2mm}
{\tt
\begin{verbatim}
        CALL SMT_put( prob%H%type, 'COORDINATE' )
\end{verbatim}
}
%\vspace*{-4mm}
\noindent
See the documentation for the \galahad\ package {\tt SMT}
for further details on the use of {\tt SMT\_put}.

\itt{H\%ne} is a scalar variable of type default \integer, that
holds the number of entries in the {\bf lower triangular} part of $\bmH$
in the sparse co-ordinate storage scheme (see Section~\ref{coordinate}).
It need not be set for any of the other schemes.

\itt{H\%val} is a rank-one allocatable array of type default \realdp, that holds
the values of the entries of the {\bf lower triangular} part
of the Hessian matrix $\bmH$ in any of non-trivial storage schemes
mentioned in Sections~\ref{coordinate}--\ref{diagonal}.
For the scaled-identity scheme (see Section~\ref{scaled-identity}),
the first component, {\tt H\%val(1)}, holds the scale factor $h_{11}$.
It need not be allocated for any of the remaining schemes.

\itt{H\%row} is a rank-one allocatable array of type default \integer,
that holds the row indices of the {\bf lower triangular} part of $\bmH$
in the sparse co-ordinate storage scheme (see Section~\ref{coordinate}).
It need not be allocated for any of the other schemes.

\itt{H\%col} is a rank-one allocatable array variable of type default \integer,
that holds the column indices of the {\bf lower triangular} part of
$\bmH$ in either the sparse co-ordinate
(see Section~\ref{coordinate}), or the sparse row-wise
(see Section~\ref{rowwise}) storage scheme.
It need not be allocated when any of the other storage schemes are used.

\itt{H\%ptr} is a rank-one allocatable array of dimension {\tt n+1} and type
default \integer, that holds the starting position of
each row of the {\bf lower triangular} part of $\bmH$, as well
as the total number of entries plus one, in the sparse row-wise storage
scheme (see Section~\ref{rowwise}). It need not be allocated when the
other schemes are used.
\end{description}
If {\tt Hessian\_kind} $\geq 0$, the components of {\tt H} need not be set.

\itt{WEIGHT} is a rank-one allocatable array type default \realdp, that
should be allocated to have length {\tt n}, and its $j$-th component
filled with the value $w_{j}$ for $j = 1, \ldots , n$,
whenever {\tt Hessian\_kind} $>1$.
If {\tt Hessian\_kind} $\leq 1$, {\tt WEIGHT} need not be allocated.

\itt{target\_kind} is a scalar variable of type default \integer,
that is used to indicate whether the components of the targets $\bmx^0$
(if they are used) have special or general values. Possible values for
{\tt target\_kind} are:
\begin{description}
\itt{0}  In this case, $\bmx^0 = 0$.

\itt{1} In this case, $x^0_{j} = 1$ for $j = 1, \ldots , n$.

\itt{$\neq$ 0,1} In this case, general values of $\bmx^0$ will be used,
     and will be provided in the component {\tt X0} (see below).
\end{description}

\ittf{X0} is a rank-one allocatable array type default \realdp, that
should be allocated to have length {\tt n}, and its $j$-th component
filled with the value $x_{j}^0$ for $j = 1, \ldots , n$,
whenever {\tt Hessian\_kind} $>0$ and {\tt target\_kind} $\neq 0,1$.
If {\tt Hessian\_kind} $\leq 0$ or {\tt target\_kind} $= 0,1$,
{\tt X0} need not be allocated.

\itt{gradient\_kind} is a scalar variable of type default \integer,
that is used to indicate whether the components of the gradient $\bmg$
have special or general values. Possible values for {\tt gradient\_kind} are:
\begin{description}
\itt{0}  In this case, $\bmg = 0$.

\itt{1} In this case, $g_{j} = 1$ for $j = 1, \ldots , n$.

\itt{$\neq$ 0,1} In this case, general values of $\bmg$ will be used,
     and will be provided in the component {\tt G} (see below).
\end{description}

\ittf{G} is a rank-one allocatable array type default \realdp, that
should be allocated to have length {\tt n}, and its $j$-th component
filled with the value $g_{j}$ for $j = 1, \ldots , n$,
whenever {\tt gradient\_kind} $\neq$ 0,1.
If {\tt gradient\_kind} {= 0, 1}, {\tt G} need not be allocated.

\ittf{DG} is a rank-one allocatable array of dimension {\tt n} and type
default \realdp, that may hold the gradient $\delta \bmg$
of the parametric linear term of the quadratic objective function.
The $j$-th component of
{\tt DG}, $j = 1,  \ldots ,  n$, contains $\delta g_{j}$.

\ittf{f} is a scalar variable of type
default \realdp, that holds
the constant term, $f$, in the objective function.

\ittf{A} is scalar variable of type {\tt SMT\_TYPE}
that holds the Jacobian matrix $\bmA$. The following components are used:

\begin{description}

\itt{A\%type} is an allocatable array of rank one and type default
\character, that
is used to indicate the storage scheme used. If the dense storage scheme
(see Section~\ref{dense}) is used,
the first five components of {\tt A\%type} must contain the
string {\tt DENSE}.
For the sparse co-ordinate scheme (see Section~\ref{coordinate}),
the first ten components of {\tt A\%type} must contain the
string {\tt COORDINATE}, while
for the sparse row-wise storage scheme (see Section~\ref{rowwise}),
the first fourteen components of {\tt A\%type} must contain the
string {\tt SPARSE\_BY\_ROWS}.

Just as for {\tt H\%type} above, the procedure {\tt SMT\_put}
may be used to allocate sufficient space and insert the required keyword
into {\tt A\%type}.
Once again, if {\tt prob} is of derived type {\tt \packagename\_problem\_type}
and involves a Jacobian we wish to store using the sparse row-wise
storage scheme, we may simply
%\vspace*{-2mm}
{\tt
\begin{verbatim}
        CALL SMT_put( prob%A%type, 'SPARSE_BY_ROWS' )
\end{verbatim}
}
%\vspace*{-4mm}
\noindent

\itt{A\%ne} is a scalar variable of type default \integer, that
holds the number of entries in $\bmA$
in the sparse co-ordinate storage scheme (see Section~\ref{coordinate}).
It need not be set for either of the other two appropriate schemes.

\itt{A\%val} is a rank-one allocatable array of type default \realdp, that holds
the values of the entries of the Jacobian matrix $\bmA$ in any of the
appropriate storage schemes discussed in Section~\ref{galmatrix}.

\itt{A\%row} is a rank-one allocatable array of type default \integer,
that holds the row indices of $\bmA$ in the sparse co-ordinate storage
scheme (see Section~\ref{coordinate}).
It need not be allocated for either of the other two appropriate schemes.

\itt{A\%col} is a rank-one allocatable array variable of type default \integer,
that holds the column indices of $\bmA$ in either the sparse co-ordinate
(see Section~\ref{coordinate}), or the sparse row-wise
(see Section~\ref{rowwise}) storage scheme.
It need not be allocated when the dense storage scheme is used.

\itt{A\%ptr} is a rank-one allocatable array of dimension {\tt m+1} and type
default \integer, that holds the
starting position of each row of $\bmA$, as well
as the total number of entries plus one, in the sparse row-wise storage
scheme (see Section~\ref{rowwise}). It need not be allocated when the
other appropriate schemes are used.

\end{description}

\ittf{C\_l} is a rank-one allocatable array of dimension {\tt m} and type
default \realdp, that holds the vector of lower bounds $\bmc^{l}$
on the general constraints. The $i$-th component of
{\tt C\_l}, $i = 1, \ldots , m$, contains $c_{i}^{l}$.
Infinite bounds are allowed by setting the corresponding
components of {\tt C\_l} to any value smaller than {\tt -infinity},
where {\tt infinity} is a  solver-dependent value that will be recognised as
infinity.

\ittf{C\_u} is a rank-one allocatable array of dimension {\tt m} and type
default \realdp, that holds the vector of upper bounds $\bmc^{u}$
on the general constraints. The $i$-th component of
{\tt C\_u}, $i = 1,  \ldots ,  m$, contains $c_{i}^{u}$.
Infinite bounds are allowed by setting the corresponding
components of {\tt C\_u} to any value larger than {\tt infinity},
where {\tt infinity} is a  solver-dependent value that will be recognised as
infinity.

\ittf{DC\_l} is a rank-one allocatable array of dimension {\tt m} and type
default \realdp, that may hold the vector of parametric lower bounds
$\delta \bmc^{l}$ on the general constraints. The $i$-th component of
{\tt DC\_l}, $i = 1, \ldots , m$, contains $\delta c_{i}^{l}$.
Only components corresponding to finite lower bounds $c_{i}^{l}$
need be set.

\ittf{DC\_u} is a rank-one allocatable array of dimension {\tt m} and type
default \realdp, that may hold the vector of parametric upper bounds
$\delta \bmc^{u}$  on the general constraints. The $i$-th component of
{\tt DC\_u}, $i = 1,  \ldots ,  m$, contains $\delta c_{i}^{u}$.
Only components corresponding to finite upper bounds $c_{i}^{u}$
need be set.

\ittf{X\_l} is a rank-one allocatable array of dimension {\tt n} and type
default \realdp, that holds
the vector of lower bounds $\bmx^{l}$ on the variables.
The $j$-th component of {\tt X\_l}, $j = 1, \ldots , n$,
contains $x_{j}^{l}$.
Infinite bounds are allowed by setting the corresponding
components of {\tt X\_l} to any value smaller than {\tt -infinity},
where {\tt infinity} is a  solver-dependent value that will be recognised as
infinity.

\ittf{X\_u} is a rank-one allocatable array of dimension {\tt n} and type
default \realdp, that holds
the vector of upper bounds $\bmx^{u}$ on the variables.
The $j$-th component of {\tt X\_u}, $j = 1, \ldots , n$,
contains $x_{j}^{u}$.
Infinite bounds are allowed by setting the corresponding
components of {\tt X\_u} to any value larger than that {\tt infinity},
where {\tt infinity} is a  solver-dependent value that will be recognised as
infinity.

\ittf{DX\_l} is a rank-one allocatable array of dimension {\tt n} and type
default \realdp, that may hold the vector of parametric lower bounds
$\delta \bmx^{l}$ on the variables. The $j$-th component of
{\tt DX\_l}, $j = 1, \ldots , n$, contains $\delta x_{j}^{l}$.
Only components corresponding to finite lower bounds $x_{j}^{l}$
need be set.

\ittf{DX\_u} is a rank-one allocatable array of dimension {\tt n} and type
default \realdp, that may hold the vector of parametric upper bounds
$\delta \bmx^{u}$  on the variables. The $j$-th component of
{\tt DX\_u}, $j = 1,  \ldots ,  n$, contains $\delta x_{j}^{u}$.
Only components corresponding to finite upper bounds $x_{j}^{u}$
need be set.

\ittf{X} is a rank-one allocatable array of dimension {\tt n} and type
default \realdp,
that holds the values $\bmx$ of the optimization variables.
The $j$-th component of {\tt X}, $j = 1,  \ldots , n$, contains $x_{j}$.

\itt{X\_status} is a rank-one allocatable array of dimension {\tt m} and type
default \integer, that holds the status of the problem variables (active or
inactive). Variable $j$ is said to be inactive if its value is fixed to the
current value of {\tt X(j)}, in which case it can be interpreted as a
parameter of the problem.

\ittf{Z} is a rank-one allocatable array of dimension {\tt n} and type default
\realdp, that holds
the values $\bmz$ of estimates  of the dual variables
corresponding to the simple bound constraints (see Section~\ref{galopt}).
The $j$-th component of {\tt Z}, $j = 1,  \ldots ,  n$, contains $z_{j}$.

\ittf{Z\_l} is a rank-one allocatable array of dimension {\tt n} and type
default \realdp, that may be used to hold
a vector of lower bounds $\bmz^{l}$ on the dual variables.
The $j$-th component of {\tt Z\_l}, $j = 1, \ldots , n$,
contains $z_{j}^{l}$.
Infinite bounds are allowed by setting the corresponding
components of {\tt Z\_l} to any value smaller than {\tt -infinity},
where {\tt infinity} is a  solver-dependent value that will be recognised as
infinity.

\ittf{Z\_u} is a rank-one allocatable array of dimension {\tt n} and type
default \realdp, that may be used to hold
a vector of upper bounds $\bmz^{u}$ on the dual variables.
The $j$-th component of {\tt Z\_u}, $j = 1, \ldots , n$,
contains $z_{j}^{u}$.
Infinite bounds are allowed by setting the corresponding
components of {\tt Z\_u} to any value larger than that {\tt infinity},
where {\tt infinity} is a  solver-dependent value that will be recognised as
infinity.

\ittf{C} is a rank-one allocatable array of dimension {\tt m} and type default
\realdp, that holds
the values $\bmA \bmx$ of the constraints.
The $i$-th component of {\tt C}, $i = 1,  \ldots ,  m$, contains
$\bma_{i}^{T}\bmx \equiv (\bmA \bmx)_{i}$.

\itt{C\_status} is a rank-one allocatable array of dimension {\tt m} and type
default \integer, that holds the status of the problem constraints (active or
inactive). A constraint is said to be inactive if it is not included in the
formulation of the problem under consideration.

\ittf{Y} is a rank-one allocatable array of dimension {\tt m} and type
default \realdp, that holds
the values $\bmy$ of estimates  of the Lagrange multipliers
corresponding to the general linear constraints (see Section~\ref{galopt}).
The $i$-th component of {\tt Y}, $i = 1,  \ldots ,  m$, contains $y_{i}$.

\ittf{Y\_l} is a rank-one allocatable array of dimension {\tt n} and type
default \realdp, that may be used to hold
a vector of lower bounds $\bmy^{l}$ on the Lagrange multipliers
The $i$-th component of {\tt Y\_l}, $i = 1, \ldots , m$,
contains $y_{i}^{l}$.
Infinite bounds are allowed by setting the corresponding
components of {\tt Y\_l} to any value smaller than {\tt -infinity},
where {\tt infinity} is a  solver-dependent value that will be recognised as
infinity.

\ittf{Y\_u} is a rank-one allocatable array of dimension {\tt n} and type
default \realdp, that may be used to hold
a vector of upper bounds $\bmy^{u}$ on the Lagrange multipliers
The $i$-th component of {\tt Y\_u}, $i = 1, \ldots , m$,
contains $y_{i}^{u}$.
Infinite bounds are allowed by setting the corresponding
components of {\tt Y\_u} to any value larger than that {\tt infinity},
where {\tt infinity} is a  solver-dependent value that will be recognised as
infinity.

\end{description}

%%%%%%%%%%%%%%%%%%%%%% GENERAL INFORMATION %%%%%%%%%%%%%%%%%%%%%%%%

\galgeneral

\galmodules {\tt GALAHAD\_SMT}.
\galio None.
\galportability ISO Fortran~95 + TR 15581 or Fortran~2003.
The package is thread-safe.
\end{description}

%%%%%%%%%%%%%%%%%%%%%% EXAMPLE %%%%%%%%%%%%%%%%%%%%%%%%

\galexample
Suppose we wish to present the data for the problem, ``{\tt QPprob}'',
of minimizing the objective function
$\half x_1^2 + x_2^2 + \threehalves x_3^2 + 4 x_1 x_3 + 2 x_2 + 1$
subject to the general linear constraints
$1 \leq  2 x_{1}  +  x_{2}  \leq  2$,
$x_{2}  +  x_{3}  =  2$, and simple bounds
$-1  \leq  x_{1}  \leq  1$ and $x_{3}  \leq  2$
to a minimizer in sparse co-ordinate format.
Then, on writing the data for this problem as
\disp{\bmH = \mat{ccc}{1 & & 4 \\ & 2 & \\ 4 &  & 3}, \;\;
 \bmg = \vect{ 0 \\ 2 \\ 2 }, \;\;
\bmx^{l} = \vect{ -1 \\ - \infty \\ - \infty } \tim{and}
\bmx^{u} = \vect{ 1 \\ \infty \\ 2 },}
and
\disp{
 \bmA = \mat{ccc}{ 2 & 1 & \\ & 1 & 1},\;\;
 \bmc^{l} = \vect{ 1 \\ 2 }, \tim{and}
 \bmc^{u} = \vect{ 2 \\ 2 }}
we may use the following code segment:

{\tt \small
\begin{verbatim}
   PROGRAM GALAHAD_QPT_EXAMPLE
   USE GALAHAD_QPT_double                       ! double precision version
   IMPLICIT NONE
   INTEGER, PARAMETER :: wp = KIND( 1.0D+0 ) ! set precision
   REAL ( KIND = wp ), PARAMETER :: infinity = 10.0_wp ** 20 ! solver-dependent
   TYPE ( QPT_problem_type ) :: p
   INTEGER, PARAMETER :: n = 3, m = 2, h_ne = 4, a_ne = 4
! start problem data
   ALLOCATE( p%name( 6 ) )
   ALLOCATE( p%G( n ), p%X_l( n ), p%X_u( n ) )
   ALLOCATE( p%C( m ), p%C_l( m ), p%C_u( m ) )
   ALLOCATE( p%X( n ), p%Y( m ), p%Z( n ) )
   p%name = TRANSFER( 'QPprob', p%name )      ! name
   p%new_problem_structure = .TRUE.           ! new structure
   p%Hessian_kind = - 1 ; p%gradient_kind = - 1 ! generic quadratic program
   p%n = n ; p%m = m ; p%f = 1.0_wp           ! dimensions & objective constant
   p%G = (/ 0.0_wp, 2.0_wp, 0.0_wp /)         ! objective gradient
   p%C_l = (/ 1.0_wp, 2.0_wp /)               ! constraint lower bound
   p%C_u = (/ 2.0_wp, 2.0_wp /)               ! constraint upper bound
   p%X_l = (/ - 1.0_wp, - infinity, - infinity /) ! variable lower bound
   p%X_u = (/ 1.0_wp, infinity, 2.0_wp /)     ! variable upper bound
   p%X = 0.0_wp ; p%Y = 0.0_wp ; p%Z = 0.0_wp ! start from zero
!  sparse co-ordinate storage format
   CALL SMT_put( p%H%type, 'COORDINATE' )  ! Specify co-ordinate
   CALL SMT_put( p%A%type, 'COORDINATE' )  ! storage for H and A
   ALLOCATE( p%H%val( h_ne ), p%H%row( h_ne ), p%H%col( h_ne ) )
   ALLOCATE( p%A%val( a_ne ), p%A%row( a_ne ), p%A%col( a_ne ) )
   p%H%val = (/ 1.0_wp, 2.0_wp, 3.0_wp, 4.0_wp /) ! Hessian H
   p%H%row = (/ 1, 2, 3, 3 /)                     ! NB lower triangle
   p%H%col = (/ 1, 2, 3, 1 /) ; p%H%ne = h_ne
   p%A%val = (/ 2.0_wp, 1.0_wp, 1.0_wp, 1.0_wp /) ! Jacobian A
   p%A%row = (/ 1, 1, 2, 2 /)
   p%A%col = (/ 1, 2, 2, 3 /) ; p%A%ne = a_ne
! problem data complete
! now call minimizer ....
! ...
! ... minimization call completed. Deallocate arrays
   DEALLOCATE( p%name, p%G, p%X_l, p%X_u, p%C, p%C_l, p%C_u, p%X, p%Y, p%Z )
   DEALLOCATE( p%H%val, p%H%row, p%H%col, p%A%val, p%A%row, p%A%col )
   END PROGRAM GALAHAD_QPT_EXAMPLE
\end{verbatim}
}
\noindent
The same problem may be handled holding the data in
a sparse row-wise storage format by replacing the lines
{\tt \small
\begin{verbatim}
!  sparse co-ordinate storage format
...
! problem data complete
\end{verbatim}
}
\noindent
and
{\tt \small
\begin{verbatim}
  DEALLOCATE( p%H%val, p%H%row, p%H%col, p%A%val, p%A%row, p%A%col )
\end{verbatim}
}
\noindent
by
{\tt \small
\begin{verbatim}
!  sparse row-wise storage format
   CALL SMT_put( p%H%type, 'SPARSE_BY_ROWS' )  ! Specify sparse row-wise
   CALL SMT_put( p%A%type, 'SPARSE_BY_ROWS' )  ! storage for H and A
   ALLOCATE( p%H%val( h_ne ), p%H%col( h_ne ), p%H%ptr( n + 1 ) )
   ALLOCATE( p%A%val( a_ne ), p%A%col( a_ne ), p%A%ptr( m + 1 ) )
   p%H%val = (/ 1.0_wp, 2.0_wp, 3.0_wp, 4.0_wp /) ! Hessian H
   p%H%col = (/ 1, 2, 3, 1 /)
   p%H%ptr = (/ 1, 2, 3, 5 /)                     ! Set row pointers
   p%A%val = (/ 2.0_wp, 1.0_wp, 1.0_wp, 1.0_wp /) ! Jacobian A
   p%A%col = (/ 1, 2, 2, 3 /)
   p%A%ptr = (/ 1, 3, 5 /)                        ! Set row pointers
! problem data complete
\end{verbatim}
}
\noindent
and
{\tt \small
\begin{verbatim}
   DEALLOCATE( p%H%val, p%H%col, p%H%ptr, p%A%val, p%A%col, p%A%ptr )
\end{verbatim}
}
\noindent
or using a dense storage format with the replacement lines
{\tt \small
\begin{verbatim}
!  dense storage format
   CALL SMT_put( p%H%type, 'DENSE' )  ! Specify dense
   CALL SMT_put( p%A%type, 'DENSE' )  ! storage for H and A
   ALLOCATE( p%H%val( n*(n+1)/2 ), p%A%val( n*m ) )
   p%H%val = (/ 1.0_wp, 0.0_wp, 2.0_wp, 4.0_wp, 0.0_wp, 3.0_wp /) ! Hessian
   p%A%val = (/ 2.0_wp, 1.0_wp, 0.0_wp, 0.0_wp, 1.0_wp, 1.0_wp /) ! Jacobian
! problem data complete
\end{verbatim}
}
\noindent
and
{\tt \small
\begin{verbatim}
! dense storage format: real components
   DEALLOCATE( p%H%val, p%A%val )
! real components complete
\end{verbatim}
}
\noindent
respectively.

If instead $\bmH$ had been the diagonal matrix
\disp{\bmH = \mat{ccc}{1 & &   \\ & 0 & \\  &  & 3}}
but the other data is as before, the diagonal storage scheme
might be used for $\bmH$, and in this case we would instead
{\tt \small
\begin{verbatim}
   CALL SMT_put( p%H%type, 'DIAGONAL' )  ! Specify dense storage for H
   ALLOCATE( p%H%val( n ) )
   p%H%val = (/ 1.0_wp, 0.0_wp, 3.0_wp /) ! Hessian values
\end{verbatim}
}
\noindent
Notice here that zero diagonal entries are stored.

For examples of how the derived data type
{\tt packagename\_problem\_type} may be used in conjunction with
\galahad\ quadratic and least-distance programming codes,
see the specification sheets
for the packages
{\tt \libraryname\_QPA},
{\tt \libraryname\_QPB},
{\tt \libraryname\_LSQP}
and
{\tt \libraryname\_PRESOLVE}.
\end{document}


