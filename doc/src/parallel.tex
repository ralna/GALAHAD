%%%%%%%%%%%%%%%%%%%%%% parallel usage %%%%%%%%%%%%%%%%%%%%%%%%

\subsection{Parallel usage}
OpenMP may be used by the {\tt \fullpackagename} package to provide
parallelism for some solvers in shared memory environments.
See the documentation for the \galahad\ package {\tt SLS} for more details.
To run in parallel, OpenMP
must be enabled at compilation time by using the correct compiler flag
(usually some variant of {\tt -openmp}).
The number of threads may be controlled at runtime
by setting the environment variable {\tt OMP\_NUM\_THREADS}.

\noindent
MPI may also be used by the package to provide
parallelism for some solvers in a distributed memory environment.
To use this form of parallelism, MPI must be enabled at runtime 
by using the correct compiler flag
(usually some variant of {\tt -lmpi}).
Although the MPI process will be started automatically when required,
it should be stopped by the calling program once no further use of
this form of parallelism is needed. Typically, this will be via
statements of the form
\begin{verbatim}
      CALL MPI_INITIALIZED( flag, ierr )
      IF ( flag ) CALL MPI_FINALIZE( ierr )
\end{verbatim}

\noindent
The code may be compiled and run in serial mode.
