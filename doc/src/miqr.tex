\documentclass{galahad}

% set the release and package names

\newcommand{\package}{miqr}
\newcommand{\packagename}{MIQR}
\newcommand{\fullpackagename}{\libraryname\_\packagename}

\begin{document}

\galheader

%%%%%%%%%%%%%%%%%%%%%% SUMMARY %%%%%%%%%%%%%%%%%%%%%%%%

\galsummary
Given a real matrix $m$ by $n$ matrix $\bmA$, form a multilevel 
incomplete QR factorization $\bmQ \bmR$ so that $\bmA \approx \bmQ\bmR$ or
optionally $\bmA^T \approx \bmQ\bmR$.
Only the $n$ by $n$ triangular factor $\bmR$ is retained, and facilities
are provided to allow the solutions of the systems $\bmR \bmu = \bmv$ and 
$\bmR^T \bmx = \bmz$
for given vectors $\bmv$ and $\bmz$. The matrix $\bmR$ is particularly helpful
as a preconditioner when applying an iterative method to solve the
least-squares problems $\min \|\bmA\bmx-\bmb\|_2$ or 
$\min \|\bmA^T\bmy-\bmc\|_2$.
Full advantage is taken of any zero coefficients in the matrices $\bmA$.

%%%%%%%%%%%%%%%%%%%%%% attributes %%%%%%%%%%%%%%%%%%%%%%%%

\galattributes
\galversions{\tt  \fullpackagename\_single, \fullpackagename\_double}.
\galuses {\tt GALAHAD\_CLOCK},
{\tt GALAHAD\_\-SY\-M\-BOLS}, 
{\tt GALAHAD\-\_SPACE}, 
{\tt GALAHAD\_SMT},
{\tt GALAHAD\_NORMS},
{\tt GALAHAD\_CONVERT},
{\tt GALAHAD\_SPECFILE},
\galdate June 2014.
\galorigin N. I. M. Gould,
Rutherford Appleton Laboratory.
\gallanguage Fortran~95 + TR 15581 or Fortran~2003. 
%\galparallelism Some options may use OpenMP and its runtime library.

%%%%%%%%%%%%%%%%%%%%%% HOW TO USE %%%%%%%%%%%%%%%%%%%%%%%%

\galhowto

The package is available with single, double and (if available) 
quadruple precision reals, and either 32-bit or 64-bit integers. 
Access to the 32-bit integer,
single precision version requires the {\tt USE} statement
\medskip

\hspace{8mm} {\tt USE \fullpackagename\_single}

\medskip
\noindent
with the obvious substitution 
{\tt \fullpackagename\_double},
{\tt \fullpackagename\_quadruple},
{\tt \fullpackagename\_single\_64},
{\tt \fullpackagename\_double\_64} and
{\tt \fullpackagename\_quadruple\_64} for the other variants.


\noindent
If it is required to use more than one of the modules at the same time, 
the derived types
{\tt SMT\_type}, 
{\tt \packagename\_time\_type}, 
{\tt \packagename\_control\_type}, 
{\tt \packagename\_inform\_type} 
and
{\tt \packagename\_data\_type}
(\S\ref{galtypes})
and the subroutines
{\tt \packagename\_initialize}, 
{\tt \packagename\_\-form},
{\tt \packagename\_\-apply},
{\tt \packagename\_terminate},
(\S\ref{galarguments})
and 
{\tt \packagename\_read\_specfile}
(\S\ref{galfeatures})
must be renamed on one of the {\tt USE} statements.

%%%%%%%%%%%%%%%%%%%%%% matrix formats %%%%%%%%%%%%%%%%%%%%%%%%

\galmatrix
The input matrix $\bmA$ may be stored in a variety of input formats.

\subsubsection{Dense storage format}\label{dense}
The matrix $\bmA$ is stored as a compact 
dense matrix by rows, that is, the values of the entries of each row in turn are
stored in order within an appropriate real one-dimensional array.
Component $n \ast (i-1) + j$ of the storage array {\tt A\%val} will hold the 
value $a_{ij}$ for $i = 1, \ldots , m$, $j = 1, \ldots , n$.

\subsubsection{Sparse co-ordinate storage format}\label{coordinate}
Only the nonzero entries of the matrices are stored. For the 
$l$-th entry of $\bmA$, its row index $i$, column index $j$ 
and value $a_{ij}$
are stored in the $l$-th components of the integer arrays {\tt A\%row}, 
{\tt A\%col} and real array {\tt A\%val}, respectively.
The order is unimportant, but the total
number of entries {\tt A\%ne} is also required. 

\subsubsection{Sparse row-wise storage format}\label{rowwise}
Again only the nonzero entries are stored, but this time
they are ordered so that those in row $i$ appear directly before those
in row $i+1$. For the $i$-th row of $\bmA$, the $i$-th component of a 
integer array {\tt A\%ptr} holds the position of the first entry in this row,
while {\tt A\%ptr} $(m+1)$ holds the total number of entries plus one.
The column indices $j$ and values $a_{ij}$ of the entries in the $i$-th row 
are stored in components 
$l =$ {\tt A\%ptr}$(i)$, \ldots , {\tt A\%ptr} $(i+1)-1$ of the 
integer array {\tt A\%col}, and real array {\tt A\%val}, respectively. 

\subsubsection{Sparse column-wise storage format}\label{colwise}
Again only the nonzero entries are stored, but this time
they are ordered so that those in column $j$ appear directly before those
in column $j+1$. For the $j$-th column of $\bmA$, the $j$-th component of a 
integer array {\tt A\%ptr} holds the position of the first entry in this column,
while {\tt A\%ptr} $(n+1)$ holds the total number of entries plus one.
The row indices $i$ and values $a_{ij}$ of the entries in the $j$-th row 
are stored in components 
$l =$ {\tt A\%ptr}$(j)$, \ldots , {\tt A\%ptr} $(j+1)-1$ of the 
integer array {\tt A\%row}, and real array {\tt A\%val}, respectively. 

For sparse matrices, the row and column schemes almost always requires 
less storage than the coordinate and dense schemes.

%%%%%%%%%%%%%%%%%%%%%%%%%%% kinds %%%%%%%%%%%%%%%%%%%%%%%%%%%

%%%%%%%%%%%%%%%%%%%%%% kinds %%%%%%%%%%%%%%%%%%%%%%

\galkinds
We use the terms integer and real to refer to the fortran
keywords {\tt REAL(rp\_)} and {\tt INTEGER(ip\_)}, where 
{\tt rp\_} and {\tt ip\_} are the relevant kind values for the
real and integer types employed by the particular module in use. 
The former are equivalent to
default {\tt REAL} for the single precision versions,
{\tt DOUBLE PRECISION} for the double precision cases
and quadruple-precision if 128-bit reals are available, and 
correspond to {\tt rp\_ = real32}, {\tt rp\_ = real64} and
{\tt rp\_ = real128} respectively as defined by the 
fortran {\tt iso\_fortran\_env} module.
The latter are default (32-bit) and long (64-bit) integers, and
correspond to {\tt ip\_ = int32} and {\tt ip\_ = int64},
respectively, again from the {\tt iso\_fortran\_env} module.


%%%%%%%%%%%%%%%%%%%%%% parallel usage %%%%%%%%%%%%%%%%%%%%%%%%

%%%%%%%%%%%%%%%%%%%%%%% parallel usage %%%%%%%%%%%%%%%%%%%%%%%%

\subsection{Parallel usage}
OpenMP may be used by the {\tt \fullpackagename} package to provide
parallelism for some solvers in shared memory environments.
See the documentation for the \galahad\ package {\tt SLS} for more details.
To run in parallel, OpenMP
must be enabled at compilation time by using the correct compiler flag
(usually some variant of {\tt -openmp}).
The number of threads may be controlled at runtime
by setting the environment variable {\tt OMP\_NUM\_THREADS}.

\noindent
MPI may also be used by the package to provide
parallelism for some solvers in a distributed memory environment.
To use this form of parallelism, MPI must be enabled at runtime
by using the correct compiler flag
(usually some variant of {\tt -lmpi}).
Although the MPI process will be started automatically when required,
it should be stopped by the calling program once no further use of
this form of parallelism is needed. Typically, this will be via
statements of the form
\begin{verbatim}
      CALL MPI_INITIALIZED( flag, ierr )
      IF ( flag ) CALL MPI_FINALIZE( ierr )
\end{verbatim}

\noindent
The code may be compiled and run in serial mode.


%%%%%%%%%%%%%%%%%%%%%% derived types %%%%%%%%%%%%%%%%%%%%%%%%

\galtypes
Five derived data types are accessible from the package.

%%%%%%%%%%% matrix data type %%%%%%%%%%%

\subsubsection{The derived data type for holding matrices}\label{typesmt}
The derived data type {\tt SMT\_TYPE} is used to hold the matrix $\bmA$.
The components of {\tt SMT\_TYPE} used here are:

\begin{description}

\ittf{m} is a scalar component of type \integer, 
that holds the number of rows in the matrix. 
 
\ittf{n} is a scalar component of type \integer, 
that holds the number of columns in the matrix. 
 
\ittf{type} is a rank-one allocatable array of type default \character, that
is used to indicate the storage scheme used. If the dense storage scheme 
(see \S\ref{dense}), is used, 
the first five components of {\tt type} must contain the
string {\tt DENSE}.
For the sparse co-ordinate scheme (see \S\ref{coordinate}), 
the first ten components of {\tt type} must contain the
string {\tt COORDINATE}, 
for the sparse row-wise storage scheme (see \S\ref{rowwise}),
the first fourteen components of {\tt type} must contain the
string {\tt SPARSE\_BY\_ROWS}, and
for the sparse column-wise storage scheme (see \S\ref{colwise}),
the first seventeen components of {\tt type} must contain the
string {\tt SPARSE\_BY\_COLUMNS}.

For convenience, the procedure {\tt SMT\_put} 
may be used to allocate sufficient space and insert the required keyword
into {\tt type}.
For example, if {\tt A} is of derived type {\tt SMT\_type}
and we wish to use the co-ordinate storage scheme, we may simply
%\vspace*{-2mm}
{\tt 
\begin{verbatim}
        CALL SMT_put( A%type, 'COORDINATE', istat )
\end{verbatim}
}
%\vspace*{-4mm}
\noindent
See the documentation for the \galahad\ package {\tt SMT} 
for further details on the use of {\tt SMT\_put}.

\ittf{ne} is a scalar variable of type \integer, that
holds the number of matrix entries.

\ittf{val} is a rank-one allocatable array of type \realdp\, 
and dimension at least {\tt ne}, that holds the values of the entries. 
Any duplicated entries that appear in the sparse 
co-ordinate or row-wise schemes will be summed. 

\ittf{row} is a rank-one allocatable array of type \integer, 
and dimension at least {\tt ne}, that may hold the row indices of the entries. 
(see \S\ref{coordinate} and \S\ref{colwise}).

\ittf{col} is a rank-one allocatable array of type \integer, 
and dimension at least {\tt ne}, that may hold the column indices of the entries
(see \S\ref{coordinate}--\ref{rowwise}).

\ittf{ptr} is a rank-one allocatable array of type \integer, 
and dimension at least {\tt m + 1}, that may hold the pointers to
the first entry in each row (see \S\ref{rowwise}) or of dimension
at least {\tt n + 1}, that may hold the pointers to
the first entry in each column (see \S\ref{colwise}).

\end{description}


%%%%%%%%%%% control type %%%%%%%%%%%

\subsubsection{The derived data type for holding control 
 parameters}\label{typecontrol}
The derived data type 
{\tt \packagename\_control\_type} 
is used to hold controlling data. Default values may be obtained by calling 
{\tt \packagename\_initialize}
(see \S\ref{subinit}),
while components may also be changed by calling 
{\tt \fullpackagename\_read\-\_spec}
(see \S\ref{readspec}). 
The components of 
{\tt \packagename\_control\_type} 
are:

\begin{description}

\itt{error} is a scalar variable of type \integer, that holds the
stream number for error messages. Printing of error messages in 
{\tt \packagename\_form}, {\tt \packagename\_apply} 
and {\tt \packagename\_terminate} 
is suppressed if {\tt error} $\leq 0$.
The default is {\tt error = 6}.

\ittf{out} is a scalar variable of type \integer, that holds the
stream number for informational messages. Printing of informational messages in 
{\tt \packagename\_form}, {\tt \packagename\_apply} 
is suppressed if {\tt out} $< 0$.
The default is {\tt out = 6}.

\itt{print\_level} is a scalar variable of type \integer, that is used
to control the amount of informational output which is required. No 
informational output will occur if {\tt print\_level} $\leq 0$. If 
{\tt print\_level} $= 1$, a single line of output will be produced for each
level of the process. If {\tt print\_level} $\geq 2$, this output will be
increased to provide significant detail of the factorization.
The default is {\tt print\_level = 0}.

\itt{max\_level} is a scalar variable of type \integer, that is used
to specify the maximum level allowed when a multi-level factorization (MIQR)
is attempted (see {\tt multi\_level} below).
The default is {\tt max\_level = 4}.

\itt{max\_order} is a scalar variable of type \integer, that is used
to specify the maximum number of columns that will be processed per level 
when a multi-level factorization (MIQR) is attempted 
(see {\tt multi\_level} below).
Any non-positive value will be interpreted as {\tt n}.
The default is {\tt max\_order = -1}.

\itt{max\_fill} is a scalar variable of type \integer, that is used
is used to control the incomplete factorization. In particular
the maximum number of elements allowed in each column of $\bmR$ will not exceed
{\tt max\_fill}.
Any negative value will be interpreted as {\tt n}.
The default is {\tt max\_fill = 100}.

\itt{max\_fill\_q} is a scalar variable of type \integer, that is used
is used to control the incomplete factorization. In particular
the maximum number of elements allowed in each column of $\bmQ$ will not exceed
{\tt max\_fill\_q}.
Any negative value will be interpreted as {\tt m}.
The default is {\tt max\_fill\_q = 100}.

\itt{increase\_size}  is a scalar variable of type \integer, that is 
used increase array sizes in chunks of this when needed.
The default is {\tt increase\_size = 100}.

\itt{buffer} is a scalar variable of type \integer, that is used
to specify the unit for any out-of-core writing when expanding arrays
needed to store $\bmR$ and other intermediary data. 
The default is {\tt buffer = 70}.

\itt{smallest\_diag} is a scalar variable of type \realdp, that is used
to identify those diagonal entries of $\bmR$ that are considered to be zero
(and thus indicate rank deficiency). Any diagonal entry in the $\bmR$
factor that is smaller than {\tt smallest\_dia} will be judged to be zero, 
and modified accordingly.
The default is {\tt smallest\_diag = }$10^{-10}$.

\itt{tol\_level} is a scalar variable of type \realdp, that is used
as a tolerance for stopping multi-level phase. In particular, the
multi-level phase ends if the dimension of the reduced problem 
is no smaller that {\tt tol\_level} times that of the previous
reduced problem.
The default is {\tt tol\_level = 0.3}.

\itt{tol\_orthogonal} is a scalar variable of type \realdp, 
that is used to judge if two vectors are roughly orthogonal; specifically
vectors $u$ and $v$ are orthogonal if $|u^T v| \leq$ {\tt tol\_orthogonal} 
$\ast \|u\| \|v\|$.
The default is {\tt tol\_orthogonal = 0.0}.

\itt{tol\_orthogonal\_increase} is a scalar variable of type \realdp, 
that is used to indicate the increase in the orthogonality tolerance 
that will be applied at each successive level.
The default is {\tt tol\_orthogonal\_increase = 0.01}.

\itt{average\_max\_fill} is a scalar variable of type \realdp, that 
is used to control the incomplete factorization. In particular
 the maximum number of elements allowed in each column of $\bmR$ will not exceed
{\tt average\_max\_fill} {\tt * ne / n}.
The default is {\tt average\_max\_fill = 6.0}.

\itt{average\_max\_fill\_q} is a scalar variable of type \realdp, that 
is used to control the incomplete factorization. In particular
 the maximum number of elements allowed in each column of $\bmQ$ will not exceed
{\tt average\_max\_fill\_q} {\tt * ne / m}.
The default is {\tt average\_max\_fill\_q = 24.0}.

\itt{tol\_drop} is a scalar variable of type \realdp, that is used
as a dropping tolerance for small generated entries. Any entry smaller
than  {\tt tol\_drop} will be excluded from the factorization.
The default is {\tt tol\_drop = 0.01}.

\itt{transpose} is a scalar variable of type default \logical, that is used
to indicate whether the factorization of $\bmA^T$ should be found rather
than that of $\bmA$. 
The default is {\tt transpose = .FALSE.}.

\itt{multi\_level} is a scalar variable of type default \logical, that is used
to specify whether a multi-level incomplete factorization (MIQR) will be 
attempted or whether an incomplete QR factorization (IQR) suffices.
The default is {\tt multi\_level = .TRUE.}.

\itt{sort} is a scalar variable of type default \logical, that is used
to specify whether the nodes of the graph of $\bmA^T \bmA$ should be sorted
in order of increasing degree. This often improves the quality of the
multilevel factorization. 
The default is {\tt sort = .TRUE.}.

\itt{deallocate\_after\_factorization} is a scalar variable of type default 
\logical, that is used to specify whether temporary workspace 
should be deallocated after every factorization. This may save space
at the expense of multiple allocations if many factorizations are required.
The default is {\tt deallocate\_after\_factorization = .FALSE.}.

\itt{space\_critical} is a scalar variable of type default \logical, 
that must be set \true\ if space is critical when allocating arrays
and  \false\ otherwise. The package may run faster if 
{\tt space\_critical} is \false\ but at the possible expense of a larger
storage requirement. The default is {\tt space\_critical = .FALSE.}.

\itt{deallocate\_error\_fatal} is a scalar variable of type default \logical, 
that must be set \true\ if the user wishes to terminate execution if
a deallocation  fails, and \false\ if an attempt to continue
will be made. The default is {\tt deallocate\_error\_fatal = .FALSE.}.

\itt{prefix} is a scalar variable of type default \character\
and length 30, that may be used to provide a user-selected 
character string to preface every line of printed output. 
Specifically, each line of output will be prefaced by the string 
{\tt prefix(2:LEN(TRIM(prefix))-1)},
thus ignoring the first and last non-null components of the
supplied string. If the user does not want to preface lines by such
a string, they may use the default {\tt prefix = ""}.

\itt{CONVERT\_control} is a scalar variable of type 
{\tt CONVERT\_control\_type}
whose components are used to control the conversion of the
input matrix type into the column-wise scheme used internally
by  {\tt \fullpackagename}, as performed by the package 
{\tt \libraryname\_CONVERT}. 
See the specification sheet for the package 
{\tt \libraryname\_CONVERT} 
for details, and appropriate default values.

\end{description}

%%%%%%%%%%% time type %%%%%%%%%%%

\subsubsection{The derived data type for holding timing 
 information}\label{typetime}
The derived data type 
{\tt \packagename\_time\_type} 
is used to hold elapsed CPU and system clock times for the various parts of 
the calculation. The components of 
{\tt \packagename\_time\_type} 
are:
\begin{description}
\itt{total} is a scalar variable of type \realdp, that gives
 the total CPU time spent in the package.

\itt{form} is a scalar variable of type \realdp, that gives
 the CPU time spent computing the multi-level incomplete factorization.

\itt{levels} is a scalar variable of type \realdp, that gives
 the CPU time spent in the multi-level phase of the factorization.

\itt{iqr} is a scalar variable of type \realdp, that gives
 the CPU time spent in the incomplete QR phase of the factorization.

\itt{apply} is a scalar variable of type \realdp, that gives
 the CPU time spent  solving systems involving the computed factor $\bmR$.

\itt{clock\_total} is a scalar variable of type \realdp, that gives
 the total elapsed system clock time spent in the package.

\itt{clock\_form} is a scalar variable of type \realdp, that gives
 the elapsed system clock time spent computing the multi-level incomplete 
factorization.

\itt{clock\_levels} is a scalar variable of type \realdp, that gives
 the elapsed system clock time spent in the multi-level phase of the 
factorization.

\itt{clock\_iqr} is a scalar variable of type \realdp, that gives
the elapsed system clock time spent in the incomplete QR phase of the 
factorization.

\itt{clock\_apply} is a scalar variable of type \realdp, that gives
 the elapsed system clock time spent solving systems involving the computed 
factor $\bmR$.

\end{description}


%%%%%%%%%%% inform type %%%%%%%%%%%

\subsubsection{The derived data type for holding informational
 parameters}\label{typeinform}
The derived data type 
{\tt \packagename\_inform\_type} 
is used to hold parameters that give information about the progress and needs 
of the algorithm. The components of 
{\tt \packagename\_inform\_type} 
are:

\begin{description}

\itt{status} is a scalar variable of type \integer, that gives the
exit status of the algorithm. 
%See Sections~\ref{galerrors} and \ref{galinfo}
See \S\ref{galerrors} 
for details.

\itt{alloc\_status} is a scalar variable of type \integer, that gives
the status of the last attempted array allocation or deallocation.
This will be 0 if {\tt status = 0}.

\itt{bad\_alloc} is a scalar variable of type default \character\
and length 80, that  gives the name of the last internal array 
for which there were allocation or deallocation errors.
This will be the null string if {\tt status = 0}. 

\itt{entries\_in\_factors} is a scalar variable of type \integer, that 
gives the number of nonzeros in the incomplete factor $\bmR$.

\itt{drop} is a scalar variable of type \integer, that 
gives the number of entries that were dropped during the incomplete 
factorization.

\itt{zero\_diagonals} is a scalar variable of type \integer, that 
gives the number of diagonal entries of $\bmR$ that were judged to be zero
during the incomplete factorization.

\ittf{time} is a scalar variable of type {\tt \packagename\_time\_type} 
whose components are used to hold elapsed CPU and system clock times for 
the various parts of the calculation (see Section~\ref{typetime}).

\itt{CONVERT\_inform} is a scalar variable of type 
{\tt CONVERT\_inform\_type}
whose components are used to provide information concerning
the conversion of the input matrix type into the column-wise scheme 
used internally by  {\tt \fullpackagename}, as performed by the package 
{\tt \libraryname\_CONVERT}. 
See the specification sheet for the package 
{\tt \libraryname\_CONVERT} 
for details, and appropriate default values.

\end{description}

%%%%%%%%%%% data type %%%%%%%%%%%

\subsubsection{The derived data type for holding problem data}\label{typedata}
The derived data type
{\tt \packagename\_data\_type} 
is used to hold all the data for the problem and the workspace arrays 
used to construct the multi-level incomplete factorization between calls of 
{\tt \packagename} procedures. 
This data should be preserved, untouched, from the initial call to 
{\tt \packagename\_initialize}
to the final call to
{\tt \packagename\_terminate}.

%%%%%%%%%%%%%%%%%%%%%% argument lists %%%%%%%%%%%%%%%%%%%%%%%%

\galarguments
There are four procedures for user calls
(see \S\ref{galfeatures} for further features): 

\begin{enumerate}
\item The subroutine 
      {\tt \packagename\_initialize} 
      is used to set default values, and initialize private data, 
      before solving one or more problems with the
      same sparsity and bound structure.
\item The subroutine 
      {\tt \packagename\_form} 
      is called to form the  multi-level incomplete factorization.
\item The subroutine 
      {\tt \packagename\_apply} 
      is called to apply the computed factor $\bmR$ to solve systems 
      $\bmR \bmx = \bmb$ or $\bmR^T \bmx = \bmb$ 
      for a given vector $\bmb$.
\item The subroutine 
      {\tt \packagename\_terminate} 
      is provided to allow the user to automatically deallocate array 
       components of the private data, allocated by 
       {\tt \packagename\_form} 
       at the end of the solution process. 
\end{enumerate}
%We use square brackets {\tt [ ]} to indicate \optional arguments.

%%%%%% initialization subroutine %%%%%%

\subsubsection{The initialization subroutine}\label{subinit}
 Default values are provided as follows:
\vspace*{1mm}

\hspace{8mm}
{\tt CALL \packagename\_initialize( data, control, inform )}

\vspace*{-3mm}
\begin{description}

\itt{data} is a scalar \intentinout\ argument of type 
{\tt \packagename\_data\_type}
(see \S\ref{typedata}). It is used to hold data about the problem being 
solved. 

\itt{control} is a scalar \intentout\ argument of type 
{\tt \packagename\_control\_type}
(see \S\ref{typecontrol}). 
On exit, {\tt control} contains default values for the components as
described in \S\ref{typecontrol}.
These values should only be changed after calling 
{\tt \packagename\_initialize}.

\itt{inform} is a scalar \intentout\ argument of type 
{\tt \packagename\_inform\_type}
(see Section~\ref{typeinform}). A successful call to
{\tt \packagename\_initialize}
is indicated when the  component {\tt status} has the value 0. 
For other return values of {\tt status}, see Section~\ref{galerrors}.

\end{description}

%%%%%%%%% factorization subroutine %%%%%%

\subsubsection{The subroutine for forming the multi-level incomplete factorization}
The multi-level incomplete QR factorization $\bmA \approx \bmQ \bmR$ or
$\bmA^T \approx \bmQ \bmR$ or is formed as follows:
\vspace*{1mm}

\hspace{8mm}
{\tt CALL \packagename\_form(  A, data, control, inform )}

\vspace*{-3mm}
\begin{description}
\itt{A} is a scalar \intentin\ argument of type {\tt SMT\_type} whose
components must be set to specify the data defining the matrix $\bmA$ 
(see \S\ref{typesmt}).

\itt{data} is a scalar \intentinout\ argument of type 
{\tt \packagename\_data\_type}
(see \S\ref{typedata}). It is used to hold data about the factors obtained.
It must not have been altered {\bf by the user} since the last call to 
{\tt \packagename\_initialize}.

\itt{control} is a scalar \intentin\ argument of type 
{\tt \packagename\_control\_type}
(see \S\ref{typecontrol}). Default values may be assigned by calling 
{\tt \packagename\_initialize} prior to the first call to 
{\tt \packagename\_form}.

\itt{inform} is a scalar \intentout\ argument of type 
{\tt \packagename\_inform\_type}
(see \S\ref{typeinform}). A successful call to
{\tt \packagename\_form}
is indicated when the  component {\tt status} has the value 0. 
For other return values of {\tt status}, see \S\ref{galerrors}.

\end{description}

%%%%%%%%% application subroutine %%%%%%

\subsubsection{The subroutine for solving systems involving the incomplete
factors}
Given the right-hand side $\bmb$, one or other of the systems 
$\bmR \bmx = \bmb$ or $\bmR^T \bmx = \bmb$ may be solved as follows:
\vspace*{1mm}

\hspace{8mm}
{\tt CALL \packagename\_apply(  SOL, transpose, data, inform )}

\vspace*{-3mm}
\begin{description}
\itt{SOL} is a rank-one  \intentinout\ array of type default \real\
that must be set on entry to hold the components of the vector $\bmy$.
On successful exit, the components of {\tt SOL} will contain the solution 
$\bmx$.

\itt{transpose} is a scalar \intentin\ argument of type default \logical,
that should be set {\tt .TRUE.} if the user wishes to solve
$\bmR^T \bmx = \bmb$ and {\tt .FALSE.} if the solution to 
$\bmR \bmx = \bmb$ is required.

\itt{data} is a scalar \intentinout\ argument of type 
{\tt \packagename\_data\_type}
(see \S\ref{typedata}). It is used to hold data about the factors obtained.
It must not have been altered {\bf by the user} since the last call to 
{\tt \packagename\_initialize}.

\itt{inform} is a scalar \intentout\ argument of type 
{\tt \packagename\_inform\_type}
(see \S\ref{typeinform}). A successful call to {\tt \packagename\_apply} 
is indicated when the  component {\tt status} has the value 0. 
For other return values of {\tt status}, see \S\ref{galerrors}.

\end{description}

%%%%%%% termination subroutine %%%%%%

\subsubsection{The  termination subroutine}
All previously allocated arrays are deallocated as follows:
\vspace*{1mm}

\hspace{8mm}
{\tt CALL \packagename\_terminate( data, control, inform )}

%\vspace*{-3mm}
\begin{description}

\itt{data} is a scalar \intentinout\ argument of type 
{\tt \packagename\_data\_type} 
exactly as for
{\tt \packagename\_form},
which must not have been altered {\bf by the user} since the last call to 
{\tt \packagename\_initialize}.
On exit, array components will have been deallocated.

\itt{control} is a scalar \intentin\ argument of type 
{\tt \packagename\_control\_type}
exactly as for
{\tt \packagename\_form}.

\itt{inform} is a scalar \intentout\ argument of type
{\tt \packagename\_inform\_type}
exactly as for
{\tt \packagename\_form}.
Only the component {\tt status} will be set on exit, and a 
successful call to 
{\tt \packagename\_terminate}
is indicated when this  component {\tt status} has the value 0. 
For other return values of {\tt status}, see \S\ref{galerrors}.

\end{description}

%%%%%%%%%%%%%%%%%%%%%% Warning and error messages %%%%%%%%%%%%%%%%%%%%%%%%

\galerrors
A negative value of {\tt inform\%status} on exit from 
{\tt \packagename\_form}, 
{\tt \packagename\_apply}
or 
{\tt \packagename\_terminate}
indicates that an error has occurred. No further calls should be made
until the error has been corrected. Possible values are:

\begin{description}

\itt{\galerrallocate.} An allocation error occurred. 
A message indicating the offending 
array is written on unit {\tt control\%error}, and the returned allocation 
status and a string containing the name of the offending array
are held in {\tt inform\%alloc\_\-status}
and {\tt inform\%bad\_alloc} respectively.

\itt{\galerrdeallocate.} A deallocation error occurred. 
A message indicating the offending 
array is written on unit {\tt control\%error} and the returned allocation 
status and a string containing the name of the offending array
are held in {\tt inform\%alloc\_\-status}
and {\tt inform\%bad\_alloc} respectively.

\itt{\galerrrestrictions.} One of the restrictions 
   {\tt A\%n} $> 0$ or {\tt a\%m} $> 0$
    or requirements that {\tt prob\%A\_type}
    contain the string
    {\tt 'DENSE'}, {\tt 'COORDINATE'}, {\tt 'SPARSE\_BY\_ROWS'}
    or {\tt 'SPARSE\_BY\_COLUMNS'}
    has been violated.

\end{description}

%%%%%%%%%%%%%%%%%%%%%% Further features %%%%%%%%%%%%%%%%%%%%%%%%

\galfeatures
\noindent In this section, we describe an alternative means of setting 
control parameters, that is components of the variable {\tt control} of type
{\tt \packagename\_control\_type}
(see \S\ref{typecontrol}), 
by reading an appropriate data specification file using the
subroutine {\tt \packagename\_read\_specfile}. This facility
is useful as it allows a user to change  {\tt \packagename} control parameters 
without editing and recompiling programs that call {\tt \packagename}.

A specification file, or specfile, is a data file containing a number of 
"specification commands". Each command occurs on a separate line, 
and comprises a "keyword", 
which is a string (in a close-to-natural language) used to identify a 
control parameter, and 
an (optional) "value", which defines the value to be assigned to the given
control parameter. All keywords and values are case insensitive, 
keywords may be preceded by one or more blanks but
values must not contain blanks, and
each value must be separated from its keyword by at least one blank.
Values must not contain more than 30 characters, and 
each line of the specfile is limited to 80 characters,
including the blanks separating keyword and value.



The portion of the specification file used by 
{\tt \packagename\_read\_specfile}
must start
with a "{\tt BEGIN \packagename}" command and end with an 
"{\tt END}" command.  The syntax of the specfile is thus defined as follows:
\begin{verbatim}
  ( .. lines ignored by MIQR_read_specfile .. )
    BEGIN MIQR
       keyword    value
       .......    .....
       keyword    value
    END 
  ( .. lines ignored by MIQR_read_specfile .. )
\end{verbatim}
where keyword and value are two strings separated by (at least) one blank.
The ``{\tt BEGIN \packagename}'' and ``{\tt END}'' delimiter command lines 
may contain additional (trailing) strings so long as such strings are 
separated by one or more blanks, so that lines such as
\begin{verbatim}
    BEGIN MIQR SPECIFICATION
\end{verbatim}
and
\begin{verbatim}
    END MIQR SPECIFICATION
\end{verbatim}
are acceptable. Furthermore, 
between the
``{\tt BEGIN \packagename}'' and ``{\tt END}'' delimiters,
specification commands may occur in any order.  Blank lines and
lines whose first non-blank character is {\tt !} or {\tt *} are ignored. 
The content 
of a line after a {\tt !} or {\tt *} character is also 
ignored (as is the {\tt !} or {\tt *}
character itself). This provides an easy manner to "comment out" some 
specification commands, or to comment specific values 
of certain control parameters.  

The value of a control parameters may be of three different types, namely
integer, logical or real.
Integer and real values may be expressed in any relevant Fortran integer and
floating-point formats (respectively). Permitted values for logical
parameters are "{\tt ON}", "{\tt TRUE}", "{\tt .TRUE.}", "{\tt T}", 
"{\tt YES}", "{\tt Y}", or "{\tt OFF}", "{\tt NO}",
"{\tt N}", "{\tt FALSE}", "{\tt .FALSE.}" and "{\tt F}". 
Empty values are also allowed for 
logical control parameters, and are interpreted as "{\tt TRUE}".  

The specification file must be open for 
input when {\tt \packagename\_read\_specfile}
is called, and the associated device number 
passed to the routine in device (see below). 
Note that the corresponding 
file is {\tt REWIND}ed, which makes it possible to combine the specifications 
for more than one program/routine.  For the same reason, the file is not
closed by {\tt \packagename\_read\_specfile}.

\subsubsection{To read control parameters from a specification file}
\label{readspec}

Control parameters may be read from a file as follows:
\hskip0.5in 

\def\baselinestretch{0.8}
{\tt 
\begin{verbatim}
     CALL MIQR_read_specfile( control, device )
\end{verbatim}
}
\def\baselinestretch{1.0}

\begin{description}
\itt{control} is a scalar \intentinout argument of type 
{\tt \packagename\_control\_type}
(see \S\ref{typecontrol}). 
Default values should have already been set, perhaps by calling 
{\tt \packagename\_initialize}.
On exit, individual components of {\tt control} may have been changed
according to the commands found in the specfile. Specfile commands and 
the component (see \S\ref{typecontrol}) of {\tt control} 
that each affects are given in Table~\ref{specfile}.
\bctable{|l|l|l|} 
\hline
  command & component of {\tt control} & value type \\ 
\hline
  {\tt error-printout-device} & {\tt \%error} & integer \\
  {\tt printout-device} & {\tt \%out} & integer \\
  {\tt print-level} & {\tt \%print\_level} & integer \\
  {\tt max-level-allowed} & {\tt \%max\_level} & integer \\
  {\tt max-order-allowed-per-level} & {\tt \%max\_order} & integer \\
  {\tt out-of-core-buffer} & {\tt \%buffer} & integer \\
  {\tt increase-array-size-by} & {\tt \%increase\_size} & real  \\
  {\tt max-entries-per-column} & {\tt \%max\_fill} & real  \\
  {\tt max-entries-per-column-of-q} & {\tt \%max\_fill\_q} & real  \\
  {\tt smallest-diagonal-factor-allowed} & {\tt \%smallest\_diag} & real  \\
  {\tt level-stop-tolerance} & {\tt \%tol\_level} & real  \\
  {\tt orthogonality-tolerance} & {\tt \%tol\_orthogonal} & real  \\
  {\tt orthogonality-tolerance-increase} & {\tt \%tol\_orthogonal\_increase} & real  \\
  {\tt dropping-tolerance} & {\tt \%tol\_drop} & real  \\
  {\tt proportion-max-entries-per-column} & {\tt \%average\_max\_fill} & real \\
  {\tt proportion-max-entries-per-column-of-q} & {\tt \%average\_max\_fill\_q} & real \\
  {\tt factorize-transpose} & {\tt \%transpose} & logical \\
  {\tt use-multi-level} & {\tt \%multi\_level} & logical \\
  {\tt sort-vertices} & {\tt \%sort} & logical \\
  {\tt deallocate-workspace-after-factorization} & {\tt \%deallocate\_after\_factorization} & logical \\
  {\tt space-critical}   & {\tt \%space\_critical} & logical \\
  {\tt deallocate-error-fatal}   & {\tt \%deallocate\_error\_fatal} & logical \\
  {\tt output-line-prefix} & {\tt \%prefix} & character \\
\hline

\ectable{\label{specfile}Specfile commands and associated 
components of {\tt control}.}

%Individual components of the components {\tt control\%SLS\_control} 
%and {\tt control\%ULS\_control} 
%may be changed by specifying separate specfiles of the form
%\begin{verbatim}
%    BEGIN SLS
%     .......
%    END 
%\end{verbatim}
%and
%\begin{verbatim}
%    BEGIN ULS
%     .......
%%    END 
%\end{verbatim}
%just as described in the documentation for the \galahad\ packages 
%{\tt SLS} and {\tt ULS}.

\itt{device} is a scalar \intentin argument of type \integer,
that must be set to the unit number on which the specfile
has been opened. If {\tt device} is not open, {\tt control} will
not be altered and execution will continue, but an error message
will be printed on unit {\tt control\%error}.

\end{description}

%%%%%%%%%%%%%%%%%%%%%% Information printed %%%%%%%%%%%%%%%%%%%%%%%%

\galinfo
If {\tt control\%print\_level} is positive, information about the progress 
of the algorithm will be printed on unit {\tt control\-\%out}.
If {\tt control\%print\_level} $\geq 1$, statistics concerning the 
formation of $\bmR$
as well as warning and error messages will be reported. 

%%%%%%%%%%%%%%%%%%%%%% GENERAL INFORMATION %%%%%%%%%%%%%%%%%%%%%%%%

\galgeneral

\galcommon None.
\galworkspace Provided automatically by the module.
\galroutines None. 
\galmodules {\tt \packagename\_form} calls the \galahad\ packages
{\tt GALAHAD\_CLOCK},
{\tt GALAHAD\_SY\-M\-BOLS}, \\
{\tt GALAHAD\-\_SPACE},
{\tt GALAHAD\_SMT},
{\tt GALAHAD\_NORMS},
{\tt GALAHAD\_CONVERT} and
{\tt GALAHAD\_SPECFILE},
\galio Output is under control of the arguments
 {\tt control\%error}, {\tt control\%out} and {\tt control\%print\_level}.
\galrestrictions {\tt A\%n} $> 0$, {\tt a\%m} $>  0$, 
{\tt A\_type} $\in \{${\tt 'DENSE'},  {\tt 'COORDINATE'}, 
{\tt 'SPARSE\_BY\_ROWS'}, {\tt 'SPARSE\_BY\_COLUMNS'}$\}$. 
\galportability ISO Fortran~95 + TR 15581 or Fortran~2003. 
The package is thread-safe.

%\end{description}

%%%%%%%%%%%%%%%%%%%%%% METHOD %%%%%%%%%%%%%%%%%%%%%%%%

\galmethod

Given the matrix $\bmA$, a decomposition
\disp{ \bmA \approx \bmQ_0 \mat{cc}{ \bmR_0 & \bmS_0 \\ & \bmA_1}}
is found. Here $\bmR_0$ is upper triangular and 
$\bmQ_0$ is constructed as normalized,
structurally-orthogonal columns of $\bmA$, or columns that are approximately
so. The same approach is then applied
recursively to $\bmA_1$ to obtain $\bmR_1$ and $\bmA_2$, etc. The
recursion ends either before or at a prescribed level $k$, and thereafter
an incomplete QR factorization of the remaining block $\bmA_k$ is computed.

\noindent
The basic algorithm is a slight generalisation of that given by
\vspace*{1mm}

\noindent
Na Li and Yousef Saad (2006).
MIQR: A Multilevel Incomplete QR preconditioner for large sparse 
least-squares problems.
SIAM. J. Matrix Anal. \& Appl., {\bf 28}(2) 524--550,
\vspace*{1mm}

\noindent
and follows in many aspects the design of the C package

%\noindent
\url{http://www-users.cs.umn.edu/~saad/software/MIQR.tar.gz}

\noindent
The principal use of $\bmR$ is as a preconditioner when solving
linear least-squares problems via an iterative method. In particular,
the minimizer of $\|\bmA\bmx-\bmb\|_2$ satisfies the normal equations
$\bmA^T \bmA \bmx = \bmA^T \bmb$, and if $\bmA = \bmQ \bmR$
with orthogobal $\bmQ$, it follows
that we may find $\bmx$ by forward and back substitution from
$\bmR^T \bmR \bmx = \bmA^T \bmb$. Moreover 
$\bmR^{-T} \bmA^T \bmA \bmR^{-1} = \bmI$, 
the $n$ by $n$ identity matrix. Since the matrix $\bmR$
computed by {\tt \packagename} is incomplete, we expect instead that
$\bmR^{-T} \bmA^T \bmA \bmR^{-1} \approx \bmI$, and this may be used
to advantage by iterative methods like CGNE and LSQR. See \$2.5 of
the specification sheets for the packages
{\tt \libraryname\_LSTR},
{\tt \libraryname\_LSRT} 
and 
{\tt \libraryname\_L2RT} for uses within \galahad.

%%%%%%%%%%%%%%%%%%%%%% EXAMPLE %%%%%%%%%%%%%%%%%%%%%%%%

\galexample
Suppose 
\disp{ \bmA = \mat{ccc}{ 1 & 2 & \\ 3 & & \\ & 4 & \\ & & 5 },}
that we wish to form a multi-level incomplete factorization of $\bmA$,
and then to solve the resulting systems 
\disp{ \bmR^T \bmz = \bmb \tim{and} \bmR \bmx = \bmz, \tim{where}
\bmb = \vect{14 \\ 42 \\ 75}.}
Then storing the matrices in sparse row format,
we may use the following code:

{\tt \small
\VerbatimInput{\packageexample}
}
\noindent
This produces the following output:
{\tt \small
\VerbatimInput{\packageresults}
}
\noindent
The same problem may be solved holding the data in 
a sparse column-wise storage format by replacing the lines
{\tt \small
\begin{verbatim}
! sparse row-wise storage format
...
! problem data complete   
\end{verbatim}
}
\noindent
by
{\tt \small
\begin{verbatim}
! sparse column-wise storage format
   CALL SMT_put( A%type, 'SPARSE_BY_COLUMNS', i ) ! storage for A
   ALLOCATE( A%val( A%ne ), A%row( A%ne ), A%ptr( A%n + 1 ) )
   A%val = (/ 1.0_wp, 3.0_wp, 2.0_wp, 4.0_wp, 5.0_wp /) ! matrix A
   A%row = (/ 1, 2, 1, 3, 4 /)
   A%ptr = (/ 1, 3, 5, 6 /)                      ! set column pointers  
! problem data complete   
\end{verbatim}
}
\noindent
or using a sparse co-ordinate storage format with the replacement lines
{\tt \small
\begin{verbatim}
!  sparse co-ordinate storage format
   CALL SMT_put( A%type, 'COORDINATE', i ) ! storage for A
   ALLOCATE( A%val( A%ne ), A%row( A%ne ), A%col( A%ne ) )
   A%val = (/ 1.0_wp, 2.0_wp, 3.0_wp, 4.0_wp, 5 /) ! matrix A
   A%row = (/ 1, 1, 2, 3, 4 /)
   A%col = (/ 1, 2, 1, 2, 3 /)
! problem data complete   
\end{verbatim}
}
\noindent
or using a dense storage format with the replacement lines
{\tt \small
\begin{verbatim}
! dense storage format
   CALL SMT_put( A%type, 'DENSE', i )  ! storage for A
   ALLOCATE( A%val( A%n * A%m ) )
   A%val = (/ 1.0_wp, 2.0_wp, 0.0_wp, 3.0_wp, 0.0_wp, 0.0_wp, 0.0_wp, 4.0_wp, &
              0.0_wp, 0.0_wp, 0.0_wp, 5.0_wp /) ! matrix A, dense by rows
! problem data complete   
\end{verbatim}
}
\noindent
respectively.

\end{document}

