\documentclass{galahad}

% set the package name

\newcommand{\package}{check}
\newcommand{\packagename}{CHECK}
\newcommand{\fullpackagename}{\libraryname\_\packagename}
\newcommand{\solver}{{\tt \fullpackagename\_verify}}

\begin{document}

\galheader

%%%%%%%%%%%%%%%%%%%%%% SUMMARY %%%%%%%%%%%%%%%%%%%%%%%%

\galsummary
This package uses finite difference approximations to {\bf check the gradient of an objective
  function $f(\bmx)$, the Jacobian matrix of a constraint function $\bmc(\bmx)$, and the
  second derivative Hessian matrix of the Lagrangian function $L(\bmx,\bmy) = f(\bmx) -
  \bmc(\bmx)^T \bmy$}.  These quantities are typically associated with a nonlinear
optimization problem
\disp{\mbox{minimize}\;\; f(\bmx)}
subject to the general linear constraints
\disp{a_{i}^{l}  \leq  \bma_{i}^{T}\bmx  \leq  a_{i}^{u}, \;\;\; i = 1, \ldots , m_a,}
general nonlinear constraints
\disp{c_{i}^{l}  \leq  c_{i}(\bmx)  \leq  c_{i}^{u}, \;\;\; i = 1, \ldots , m,}
and simple bound constraints
\disp{x_{j}^{l}  \leq  x_{j}^{ } \leq  x_{j}^{u} , \;\;\; j = 1, \ldots , n,}
where the vectors $\bma_{i}$, $\bma^{l}$, $\bma^{u}$, $\bmc^{l}$,
$\bmc^{u}$, $\bmx^{l}$, and $\bmx^{u}$ are given, and the vectors
$\bmx\in\Re^n$ and $\bmy\in\Re^m$ are known as the primal and dual (Lagrange
multiplier) vectors, respectively.  The user may choose to perform a ``cheap''
verification of the requested derivatives, or a more detailed
and ``expensive'' check.  Function values can be supplied via internal
subroutine evaluation or reverse communication.

%%%%%%%%%%%%%%%%%%%%%% attributes %%%%%%%%%%%%%%%%%%%%%%%%

\galattributes
\galversions{\tt  \fullpackagename\_single} and {\tt \fullpackagename\_double}.
\galuses 
{\tt GALAHAD\_SY\-M\-BOLS},
{\tt GAL\-AHAD\_SPECFILE},
{\tt GALAHAD\_SPACE},
{\tt GALAHAD\_MOP},
{\tt GALAHAD\_SMT}, and
{\tt GALAHAD\_NLPT}.
\galdate September 2010.
\galorigin D. P. Robinson, University of Oxford, UK, and
N. I. M. Gould, Rutherford Appleton Laboratory.
\gallanguage Fortran~95 + TR 15581 or Fortran~2003. 

%%%%%%%%%%%%%%%%%%%%%% HOW TO USE %%%%%%%%%%%%%%%%%%%%%%%%

\galhowto

%\subsection{Calling sequences}

Access to the package requires a {\tt USE} statement such as

\medskip\noindent{\em Single precision version}

\hspace{8mm} {\tt USE \fullpackagename\_single}

\medskip\noindent{\em Double precision version}

\hspace{8mm} {\tt USE  \fullpackagename\_double}

\medskip

\noindent
If it is required to use both modules at the same time, the derived types 
{\tt SMT\_type}, 
{\tt NLPT\_problem\_type}, 
{\tt \packagename\_\-control\_type}, 
{\tt \packagename\_inform\_type},
{\tt \packagename\_data\_type},
{\tt \packagename\_reverse\_communication\_type} and
{\tt GALAHAD\_userdata\_\-type}
(Section~\ref{galtypes}),
and the subroutines
{\tt \packagename\_initialize}, 
{\tt \packagename\_verify},
{\tt \packagename\_terminate}
(Section~\ref{galarguments}),
and 
{\tt \packagename\_read\_specfile}
(Section~\ref{galfeatures})
must be renamed on one of the {\tt USE} statements.

%%%%%%%%%%%%%%%%%%%%%% matrix formats %%%%%%%%%%%%%%%%%%%%%%%%

\galmatrix
The Jacobian matrix $\bmJ = \nabla_x\bmc(\bmx)$ and the Hessian matrix $\bmH = \nabla_{xx} L(\bmx,\bmy)$
may be stored in a variety of input formats.

\subsubsection{Dense storage format}\label{dense}
The matrix $\bmJ$ is stored as a compact 
dense matrix by rows, that is, the values of the entries of each row in turn are
stored in order within an appropriate real one-dimensional array.
Component $n \ast (i-1) + j$ of the storage array {\tt J\%val} will hold the 
value $j_{ij}$ for $i = 1, \ldots , m$ and $j = 1, \ldots , n$.
Since $\bmH$ is symmetric, only the lower triangular part (that is the part 
$h_{ij}$ for $1 \leq j \leq i \leq n$) should be stored.  In this case
the lower triangle will be stored by rows, that is 
component $i \ast (i-1)/2 + j$ of the storage array {\tt H\%val}  
will hold the value $h_{ij}$ (and, by symmetry, $h_{ji}$)
for $1 \leq j \leq i \leq n$.

\subsubsection{Sparse co-ordinate storage format}\label{coordinate}
Only the nonzero entries of the matrices are stored. For the $l$-th
entry of $\bmJ$, its row index $i$, column index $j$ and value
$j_{ij}$ are stored in the $l$-th components of the integer arrays
{\tt J\%row}, {\tt J\%col} and real array {\tt J\%val}.  The order is
unimportant, but the total number of entries {\tt J\%ne} is also
required.  Since $\bmH$ is symmetric, the same scheme is applicable,
except that only the entries in the lower triangle should be stored.

\subsubsection{Sparse row-wise storage format}\label{rowwise}
Again only the nonzero entries are stored, but this time they are
ordered so that those in row $i$ appear directly before those in row
$i+1$. For the $i$-th row of $\bmJ$, the $i$-th component of a integer
array {\tt J\%ptr} holds the position of the first entry in this row,
while {\tt J\%ptr} $(m+1)$ holds the total number of entries plus one.
The column indices $j$ and values $j_{ij}$ of the entries in the
$i$-th row are stored in components $l =$ {\tt J\%ptr}$(i)$, \ldots
,{\tt J\%ptr} $(i+1)-1$ of the integer array {\tt J\%col}, and real
array {\tt J\%val}, respectively.  Since $\bmH$ is symmetric, the same
scheme is applicable, except that only the entries in the lower
triangle should be stored.

For sparse matrices, this scheme almost always requires less storage than 
its predecessor.

\subsubsection{Sparse column-wise storage format}\label{columnwise}
Again only the nonzero entries are stored, but this time they are
ordered so that those in column $j$ appear directly before those in column
$j+1$. For the $j$-th column of $\bmJ$, the $j$-th component of the integer
array {\tt J\%ptr} holds the position of the first entry in this column,
while {\tt J\%ptr} $(n+1)$ holds the total number of entries plus one.
The row indices $i$ and values $j_{ij}$ of the entries in the
$j$-th column are stored in components $l =$ {\tt J\%ptr}$(j)$, \ldots
,{\tt J\%ptr} $(j+1)-1$ of the integer array {\tt J\%row}, and real
array {\tt J\%val}, respectively.  Since $\bmH$ is symmetric, the same
scheme is applicable, except that only the entries in the lower
triangle should be stored.

\subsubsection{Diagonal storage format}\label{diagonal}
If $\bmJ$ is diagonal (i.e., $j_{ij} = 0$ for all $1 \leq i \neq j
\leq n$) only the diagonal entries $j_{ii}$ for $1 \leq i \leq n$ should
be stored, and the first $n$ components of the array {\tt J\%val}
should be used for this purpose.  The same holds for $\bmH$.

%%%%%%%%%%%%%%%%%%%%%% derived types %%%%%%%%%%%%%%%%%%%%%%%%

\galtypes
Seven derived data types are accessible from the package.

%%%%%%%%%%% matrix data type %%%%%%%%%%%

\subsubsection{The derived data type for holding matrices}\label{typesmt}
The derived data type {\tt SMT\_TYPE} is used to hold the Jacobian
matrix $\bmJ$ and the Hessian matrix $\bmH$. The components of {\tt
  SMT\_TYPE} used here are:

\begin{description}

\ittf{m} is a scalar component of type default \integer\ 
that holds the number of rows of the matrix. 

\ittf{n} is a scalar component of type default \integer\ 
that holds the number of columns of the matrix. 
 
\ittf{ne} is a scalar variable of type default \integer\ that
holds the number of matrix entries.

\ittf{type} is a rank-one allocatable array of type default \character\ that
is used to indicate the matrix storage scheme used. Its precise length and
content depends on the type of matrix to be stored (see \S\ref{typeprob}).

\ittf{val} is a rank-one allocatable array of type default \realdp\, 
and dimension at least {\tt ne}, that holds the values of the entries. 
Each pair of off-diagonal entries $h_{ij} = h_{ji}$ for the {\em symmetric}
matrix $\bmH$ is represented as a single entry 
(see \S\ref{dense}--\ref{rowwise}).
Any duplicated entries that appear in the sparse 
co-ordinate, row-wise, or column-wise schemes will be summed. 

\ittf{row} is a rank-one allocatable array of type default \integer, 
and dimension at least {\tt ne}, that may hold the row indices of the entries 
(see \S\ref{coordinate}).

\ittf{col} is a rank-one allocatable array of type default \integer, 
and dimension at least {\tt ne}, that may hold the column indices of the entries
(see \S\ref{coordinate}--\ref{rowwise}).

\ittf{ptr} is a rank-one allocatable array of type default \integer.
If sparse row-wise storage is used (see \S\ref{rowwise}), then it
must have dimension at least {\tt m + 1} and hold the pointers to
the first entry in each row.  If sparse column-wise storage is used (see \S\ref{columnwise}), then it
must have dimension at least {\tt n + 1} and hold the pointers to
the first entry in each column.   

\end{description}

%%%%%%%%%%% problem type %%%%%%%%%%%

\subsubsection{The derived data type for holding the problem}\label{typeprob}
The derived data type {\tt NLPT\_problem\_type} holds 
the problem. The relevant components of 
{\tt NLPT\_problem\_type} 
are:

\begin{description}

\ittf{m} is a scalar variable of type default \integer\ 
 that holds the number of nonlinear constraints $m$.  

\ittf{n} is a scalar variable of type default \integer\ 
 that holds the number of optimization variables $n$.  
              
\ittf{H} is scalar variable of type {\tt SMT\_TYPE} 
that holds the Hessian matrix $\bmH=\nabla_{xx}f(\bmx)$. The following components
are used here:

\begin{description}

\itt{H\%type} is an allocatable array of rank one and type default 
\character, that
is used to indicate the storage scheme used. If the dense storage scheme 
(see Section~\ref{dense}) is used, 
the first five components of {\tt H\%type} must contain the
string {\tt DENSE}.
For the sparse co-ordinate scheme (see Section~\ref{coordinate}), 
the first ten components of {\tt H\%type} must contain the
string {\tt COORDINATE},  
for the sparse row-wise storage scheme (see Section~\ref{rowwise}),
the first fourteen components of {\tt H\%type} must contain the
string {\tt SPARSE\_BY\_ROWS},
for the sparse column-wise storage scheme (see Section~\ref{columnwise}),
the first seventeen components of {\tt H\%type} must contain the
string {\tt SPARSE\_BY\_COLUMNS},
and for the diagonal storage scheme (see Section~\ref{diagonal}),
the first eight components of {\tt H\%type} must contain the
string {\tt DIAGONAL}.

For convenience, the procedure {\tt SMT\_put} 
may be used to allocate sufficient space and insert the required keyword
into {\tt H\%type}.
For example, if {\tt nlp} is of derived type {\tt \packagename\_problem\_type}
and involves a Hessian we wish to store using the co-ordinate scheme,
we may simply
%\vspace*{-2mm}
{\tt 
\begin{verbatim}
        CALL SMT_put( nlp%H%type, 'COORDINATE' )
\end{verbatim}
}
%\vspace*{-4mm}
\noindent
See the documentation for the \galahad\ package {\tt SMT} 
for further details on the use of {\tt SMT\_put}.

\itt{H\%ne} is a scalar variable of type default \integer, that 
holds the number of entries in the {\bf lower triangular} part of $\bmH$
in the sparse co-ordinate storage scheme (see Section~\ref{coordinate}). 
It need not be set for any of the other four schemes.

\itt{H\%val} is a rank-one allocatable array of type default \realdp, that holds
the values of the entries of the {\bf lower triangular} part
of the Hessian matrix $\bmH$ in any of the 
storage schemes discussed in Section~\ref{galmatrix}.

\itt{H\%row} is a rank-one allocatable array of type default \integer,
that holds the row indices of the {\bf lower triangular} part of $\bmH$ 
in the sparse co-ordinate storage
scheme (see Section~\ref{coordinate}) and the sparse column-wise
scheme (see Section~\ref{columnwise}). 
It need not be allocated for any of the other three schemes.

\itt{H\%col} is a rank-one allocatable array variable of type default \integer,
that holds the column indices of the {\bf lower triangular} part of 
$\bmH$ in either the sparse co-ordinate 
(see Section~\ref{coordinate}), or the sparse row-wise 
(see Section~\ref{rowwise}) storage scheme.
It need not be allocated for any of the other three schemes.

\itt{H\%ptr} is a rank-one allocatable array of dimension {\tt n+1} and type 
default \integer\ that holds the starting position of 
each row of the {\bf lower triangular} part of $\bmH$, as well
as the total number of entries plus one, in the sparse row-wise storage
scheme (see Section~\ref{rowwise}), or the starting position of 
each column of the {\bf lower triangular} part of $\bmH$, as well
as the total number of entries plus one, in the sparse column-wise storage
scheme (see Section~\ref{columnwise}).
It need not be allocated when the other schemes are used.

\end{description}

\ittf{J} is scalar variable of type {\tt SMT\_TYPE} 
that holds the Jacobian matrix $\bmJ=\nabla_x\bmc(\bmx)$. The following components
are used here:

\begin{description}

\itt{J\%type} is an allocatable array of rank one and type default 
\character, that
is used to indicate the storage scheme used. If the dense storage scheme 
(see Section~\ref{dense}) is used, 
the first five components of {\tt J\%type} must contain the
string {\tt DENSE}.
For the sparse co-ordinate scheme (see Section~\ref{coordinate}), 
the first ten components of {\tt J\%type} must contain the
string {\tt COORDINATE},  
for the sparse row-wise storage scheme (see Section~\ref{rowwise}),
the first fourteen components of {\tt J\%type} must contain the
string {\tt SPARSE\_BY\_ROWS},
for the sparse column-wise storage scheme (see Section~\ref{columnwise}),
the first seventeen components of {\tt J\%type} must contain the
string {\tt SPARSE\_BY\_COLUMNS},
and for the diagonal storage scheme (see Section~\ref{diagonal}),
the first eight components of {\tt J\%type} must contain the
string {\tt DIAGONAL}.

For convenience, the procedure {\tt SMT\_put} 
may be used to allocate sufficient space and insert the required keyword
into {\tt J\%type}.
For example, if {\tt nlp} is of derived type {\tt \packagename\_problem\_type}
and involves a Jacobian we wish to store using the co-ordinate scheme,
we may simply
%\vspace*{-2mm}
{\tt 
\begin{verbatim}
        CALL SMT_put( nlp%J%type, 'COORDINATE' )
\end{verbatim}
}
%\vspace*{-4mm}
\noindent
See the documentation for the \galahad\ package {\tt SMT} 
for further details on the use of {\tt SMT\_put}.

\itt{J\%ne} is a scalar variable of type default \integer\ that 
holds the number of entries in the sparse co-ordinate storage scheme
(see Section~\ref{coordinate}). 
It need not be set for any of the other schemes.

\itt{J\%val} is a rank-one allocatable array of type default \realdp\ that holds
the values of the entries of the Jacobian matrix $\bmJ$ in any of the 
storage schemes discussed in Section~\ref{galmatrix}.

\itt{J\%row} is a rank-one allocatable array of type default \integer\
that holds the row indices of $\bmJ$ 
in the sparse co-ordinate storage
scheme discussed in Section~\ref{coordinate} and the sparse column-wise storage
scheme discussed in Section~\ref{columnwise}.
It need not be allocated for any of the other three schemes.

\itt{J\%col} is a rank-one allocatable array variable of type default \integer\
that holds the column indices of $\bmJ$ in either the sparse co-ordinate 
scheme discussed in Section~\ref{coordinate} or the sparse row-wise 
scheme discussed in Section~\ref{rowwise}.
It need not be allocated for any of the other three schemes.

\itt{J\%ptr} for the sparse row-wise storage
scheme discussed in Section~\ref{rowwise}, it is a rank-one allocatable array of dimension {\tt m+1} and type 
default \integer\ that holds the starting position of 
each row of $\bmJ$ as well
as the total number of entries plus one. For the sparse column-wise storage
scheme discussed in Section~\ref{columnwise}, it is a rank-one allocatable array of dimension {\tt n+1} and type 
default \integer\ that holds the starting position of 
each column of $\bmJ$ as well
as the total number of entries plus one. It need not be
allocated for any of the other schemes.

\end{description}

\ittf{G} is a rank-one allocatable array of dimension {\tt n} and type 
default \realdp, that holds the gradient $\bmg$ 
of the objective function.
The $j$-th component of 
{\tt G}, $j = 1,  \ldots ,  n$, contains $\bmg_{j}$.

\ittf{C} is a rank-one allocatable array of dimension {\tt m} and type 
default \realdp\ that holds the value of the constraint function.
The $j$-th component of {\tt C}, $j = 1, \ldots , m$, contains $\bmc_{j}$.

\ittf{f} is a scalar variable of type 
default \realdp\ that holds the value of the objective function.

\ittf{X} is a rank-one allocatable array of dimension {\tt n} and type 
default \realdp, 
that holds the values $\bmx$ of the optimization variables.
The $j$-th component of {\tt X}, $j = 1,  \ldots , n$, contains
$x_{j}$.

\ittf{X\_l} is a rank-one allocatable array of dimension {\tt n} and type 
default \realdp\ 
that holds the lower bounds on the optimization variables $\bmx$.

\ittf{X\_u} is a rank-one allocatable array of dimension {\tt n} and type 
default \realdp\ 
that holds upper  bounds on the optimization variables $\bmx$.

\ittf{Y} is a rank-one allocatable array of dimension {\tt m} and type 
default \realdp\ 
that holds the value $\bmy$ of the Lagrange multiplier estimate.
The $j$-th component of {\tt Y}, $j = 1,  \ldots , m$, contains
$y_{j}$.  

\end{description}

%%%%%%%%%%% control type %%%%%%%%%%%

\subsubsection{The derived data type for holding control 
 parameters}\label{typecontrol}
The derived data type 
{\tt \packagename\_control\_type} 
is used to hold controlling data. Default values may be obtained by calling 
{\tt \packagename\_initialize}
(see Section~\ref{subinit}),
while components may also be changed by calling 
{\tt \fullpackagename\_read\-\_spec}
(see Section~\ref{readspec}). 
The components of 
{\tt \packagename\_control\_type} 
are:

\begin{description}

\itt{error} is a scalar variable of type default \integer, that holds the
stream number for error messages. Printing of error messages in 
{\tt \packagename\_verify} and {\tt \packagename\_terminate} 
is suppressed if {\tt error} $\leq 0$.
The default is {\tt error = 6}.

\ittf{out} is a scalar variable of type default \integer, that holds the
stream number for informational messages. Printing of informational messages in 
{\tt \packagename\_verify} is suppressed if {\tt out} $\leq 0$.
The default is {\tt out = 6}.

\itt{print\_level} is a scalar variable of type default \integer, that is used
to control the amount of informational output which is required. No 
informational output will occur if {\tt print\_level} $\leq 0$. If 
{\tt print\_level} $= 1$, a brief summary of the derivative
verification is provided.  If {\tt print\_level} $\geq 2$, this output will be
increased to provide significant detail of each iteration (see
Section~\ref{galinfo} for more details).
The default is {\tt print\_level = 0}.

\itt{verify\_level} is a scalar variable of type default \integer\
that determines the detail of verification performed.  A ``cheap''
check will be performed if {\tt verify\_level} $= 1$. If 
{\tt verify\_level} $\geq 2$, an ``expensive''---but more
detailed---verification of the derivatives is done.  No
checking is performed if {\tt verify\_level} $\leq 0$.
The default is {\tt verify\_level = 2}.

\itt{f\_availability} is a scalar variable of type default \integer\
that controls how the user is expected to supply objective function
values, when required.  The user should set {\tt f\_availability} $=1$
if an appropriate evaluation routine is supplied (see
Section~\ref{ffv}), and {\tt f\_availability} $=2$ if reverse
communication will be used (see Section~\ref{reverse}).

\itt{c\_availability} is a scalar variable of type default \integer\
that controls how the user is expected to supply constraint function
values, when required.  The user should set {\tt c\_availability} $=1$
if an appropriate evaluation routine is supplied (see
Section~\ref{cfv}), and {\tt c\_availability} $=2$ if reverse
communication will be used (see Section~\ref{reverse}).  Any other
value will result in an error message.

\itt{g\_availability} is a scalar variable of type default \integer\
that controls how the user is expected to supply the gradient of the
objective function, when required.  The user should set {\tt g\_availability} $=1$
if an appropriate evaluation routine is supplied (see
Section~\ref{gfv}), and {\tt g\_availability} $=2$ if reverse
communication will be used (see Section~\ref{reverse}).  Any other
value will result in an error message.

\itt{J\_availability} is a scalar variable of type default \integer\
that controls how the user is expected to supply the Jacobian of the
constraint function, when required.  The user should set {\tt J\_availability} $=1$
if an appropriate evaluation routine is supplied (see
Section~\ref{jfv}), {\tt J\_availability} $=2$ if reverse
communication will be used to obtain Jacobian values (see
Section~\ref{reverse}), {\tt J\_availability} $=3$ if an appropriate
Jacobian-vector product routine is supplied (see Section~\ref{jvfv}), and
{\tt J\_availability} $=4$ if reverse communication will be used to
get Jacobian-vector products (see Section~\ref{reverse}).   Any other
value will result in an error message.

\itt{H\_availability} is a scalar variable of type default \integer\
that controls how the user is expected to supply the Hessian of the
Lagrangian function, when required.  The user should set {\tt H\_availability} $=1$
if an appropriate evaluation routine is supplied (see
Section~\ref{hfv}), {\tt H\_availability} $=2$ if reverse
communication will be used to obtain Hessian values (see
Section~\ref{reverse}), {\tt H\_availability} $=3$ if an appropriate
Hessian-vector product routine is supplied (see Section~\ref{hvfv}), and
{\tt H\_availability} $=4$ if reverse communication will be used to
get Hessian-vector products (see Section~\ref{reverse}).   Any other
value will result in an error message.

\itt{checkG} is a scalar variable of type default \logical\ that
should be set \true\ if the gradient of the objective function should be
checked.  Otherwise, it should be set \false.  The default is {\tt checkG=\true}.

\itt{checkJ} is a scalar variable of type default \logical\ that
should be set \true\ if the Jacobian of the constraint function should be
checked.  Otherwise, it should be set \false.  The default is {\tt checkJ =\true}.

\itt{checkH} is a scalar variable of type default \logical\ that
should be set \true\ if the Hessian of the Lagrangian function should be
checked.  Otherwise, it should be set \false.  The default is {\tt checkH=\true}.

\itt{deallocate\_error\_fatal} is a scalar variable of type default \logical, 
that must be set \true\ if the user wishes to terminate execution if
a deallocation  fails, and \false\ if an attempt to continue
will be made. The default is {\tt deallocate\_error\_fatal = .FALSE.}.

\end{description}

%%%%%%%%%%% inform type %%%%%%%%%%%

\subsubsection{The derived data type for holding informational
 parameters}\label{typeinform}
The derived data type 
{\tt \packagename\_inform\_type} 
is used to hold parameters that give information about the progress and needs 
of the algorithm. The components of 
{\tt \packagename\_inform\_type} 
are:

\begin{description}

\itt{status} is a scalar variable of type default \integer\ that gives the
exit status of the algorithm.  See Sections~\ref{reverse} and~\ref{galerrors} for further details.

\itt{alloc\_status} is a scalar variable of type default \integer\ that gives
the status of the last attempted array allocation or deallocation.
This will be 0 if {\tt status = 0}.

\itt{bad\_alloc} is a scalar variable of type default \character\
and length 80\ that gives the name of the last internal array 
for which there was an allocation or deallocation error.
This will be the null string if {\tt status = 0}.

\itt{numG\_wrong} is a scalar variable of type default \integer\ that gives the
number of components of the gradient of the objective function that
appear to be wrong.

\itt{numJ\_wrong} is a scalar variable of type default \integer\ that gives the
number of entries of the Jacobian of the constraint function that
appear to be wrong.

\itt{numH\_wrong} is a scalar variable of type default \integer\ that gives the
number of entries of the Hessian of the Lagrangian function that
appear to be wrong.

\itt{derivative\_ok} is a scalar variable of type default \logical\
that is \true\ if all derivatives appear to be correct, and set
\false\ otherwise. 

\end{description}

%%%%%%%%%%% data type %%%%%%%%%%%

\subsubsection{The derived data type for holding problem data}\label{typedata}
The derived data type 
{\tt \packagename\_data\_type} 
is used to hold all the data for a particular problem,
or sequences of problems with the same structure, between calls of 
{\tt \packagename} procedures.  The only data that should be changed
by the user from the initial call to 
{\tt \packagename\_initialize}
to the final call to
{\tt \packagename\_terminate} is the component {\tt RC} of type {\tt
  \packagename\_reverse\_communication\_type} (see Section~\ref{typerc}), and this should be done
only as directed from \solver\ with positive values of {\tt inform\%status} as
described in Section~\ref{reverse}.

%%%%%%%%%%% reverse communication type %%%%%%%%%%%

\subsubsection{The derived data type for holding reverse communication data}\label{typerc}
The derived data type {\tt \packagename\_reverse\_communication\_type}
is used to hold data for reverse communication, when needed. The components of 
{\tt \packagename\_reverse\_communication\_type} are:

\begin{description}

\itt{X} is a rank-one allocatable array of type default \realdp\ that
holds the values of the optimization variables at which the user must
perform function computation.

\itt{Y} is a rank-one allocatable array of type default \realdp\ that
holds the values of the Lagrange multipliers that the user must use
when evaluating the Hessian of the Lagrangian.

\itt{F} is a scalar variable of type default \realdp\ in which the
user places the value of the objective function evaluated at {\tt X},
when required (see Section~\ref{reverse}).

\itt{C} is a rank-one allocatable array of type default \realdp\ in
which the user places the value of the constraint function evaluated
at {\tt X}, when required (see Section~\ref{reverse}).

\itt{G} is a rank-one allocatable array of type default \realdp\ in
which the user places the gradient of the the objective function
evaluated at {\tt X}, when required (see Section~\ref{reverse}).

\itt{V} is a rank-one allocatable array of type default \realdp\ that
holds the vector for which a matrix-vector product is required (see
Section~\ref{reverse}).

\itt{U} is a rank-one allocatable array of type default \realdp\ in
which the user places the result of any required matrix-vector product
with the vector {\tt V} from above (see Section~\ref{reverse}). 

\itt{Jval} is a rank-one allocatable array of type default \realdp\ in
which the user places the entries of the Jacobian matrix evaluated at
{\tt X}, when required (see Section~\ref{reverse}). 

\itt{Hval} is a rank-one allocatable array of type default \realdp\ in
which the user places the {\bf lower triangular} entries of the Hessian matrix of the
Lagrangian evaluated at
{\tt X} and {\tt Y}, when required (see Section~\ref{reverse}). 

\end{description}

%%%%%%%%%%% userdata type %%%%%%%%%%%

\subsubsection{The derived data type for holding user data}\label{typeuserdata}
The derived data type {\tt GALAHAD\_user\-data\_type}
is available from the package {\tt GALAHAD\_user\-data}
to allow the user to pass data to and from user-supplied
subroutines for function and derivative calculations (see Section~\ref{fdv}).
Components of variables of type {\tt GALAHAD\_user\-data\_type} may be 
allocated as necessary. The following components are available:

\begin{description}
\itt{integer} is a rank-one allocatable array of type default \integer.
\ittf{real} is a rank-one allocatable array of type default  \realdp
\itt{complex} is a rank-one allocatable array of type default \complexdp.
\itt{character} is a rank-one allocatable array of type default \character.
\itt{logical} is a rank-one allocatable array of type default \logical.
\itt{integer\_pointer} is a rank-one pointer array of type default \integer.
\itt{real\_pointer} is a rank-one pointer array of type default  \realdp
\itt{complex\_pointer} is a rank-one pointer array of type default \complexdp.
\itt{character\_pointer} is a rank-one pointer array of type default \character.
\itt{logical\_pointer} is a rank-one pointer array of type default \logical.
\end{description}


%%%%%%%%%%%%%%%%%%%%%% argument lists %%%%%%%%%%%%%%%%%%%%%%%%

\galarguments
There are three procedures for user calls
(see Section~\ref{galfeatures} for further features): 

\begin{enumerate}
\item The subroutine 
      {\tt \packagename\_initialize} 
      is used to set default values, and initialize private data, 
      before verifying the derivatives of one or more problems with the
      same sparsity and bound structure.
\item The subroutine 
      {\tt \packagename\_verify} 
      is called to check the derivatives of the given problem.
\item The subroutine 
      {\tt \packagename\_terminate} 
      is provided to allow the user to automatically deallocate array 
       components of the private data, allocated by 
       {\tt \packagename\_verify}, 
       at the end of the verification process. 
       It is important to do this if the data object is re-used for another 
       problem {\bf with a different structure}
       since {\tt \packagename\_initialize} cannot test for this situation, 
       and any existing associated targets will subsequently become unreachable.
\end{enumerate}
We use square brackets {\tt [ ]} to indicate \optional\ arguments.

%%%%%% initialization subroutine %%%%%%

\subsubsection{The initialization subroutine}\label{subinit}
 Default values are provided as follows:
\vspace*{1mm}

\hspace{8mm}
{\tt CALL \packagename\_initialize( control )}

\vspace*{-3mm}
\begin{description}

\itt{control} is a scalar \intentout\ argument of type 
{\tt \packagename\_control\_type}
(see Section~\ref{typecontrol}). 
On exit, {\tt control} contains default values for the components as
described in Section~\ref{typecontrol}.
These values should only be changed after calling 
{\tt \packagename\_initialize}.

\end{description}

%%%%%%%%% main verification subroutine %%%%%%

\subsubsection{The verification  subroutine}
The verification algorithm is called as follows:
\vspace*{1mm}

\hspace{8mm}
{\tt CALL \packagename\_verify( nlp, data, control, inform, userdata[, eval\_F, 
eval\_C, \hspace{4mm}                  \&}
\vspace*{-1mm}

\hspace{41.5mm}
{\tt                        
eval\_G, eval\_J, eval\_H, eval\_Jv, eval\_Hv] )}

\vspace*{-2mm}
\begin{description}
\ittf{nlp} is a scalar \intentinout\ argument of type 
{\tt NLPT\_problem\_type}
(see Section~\ref{typeprob}). 
It is used to hold data about the problem whose derivatives are being verified.
For a new problem, the user must allocate all the array components,
and set values for {\tt nlp\%m}, {\tt nlp\%n},
and the required integer components of {\tt nlp\%J} and {\tt nlp\%H}
that is determined by the values of {\tt checkJ} and {\tt checkH} as described in Section~\ref{typecontrol}.  
Users are free to choose whichever
of the matrix formats described in Section~\ref{galmatrix} 
is appropriate for $\bmJ$ and $\bmH$ for their application.

\noindent
The components {\tt nlp\%X} and {\tt nlp\%Y} must be set to initial
values $\bmx$ and $\bmy$ of the primal and dual variables for the
optimization problem.  Prior to verification of the derivatives, the point {\tt nlp\%X} is modified
internally to ensure feasibility with respect to the bound constraints
$\bmx^l$ and $\bmx^u$; no modification of $\bmy$ is performed.  The
requested derivatives are then checked at the point {\tt nlp\%X} and {\tt nlp\%Y}.

\noindent
\restrictions {\tt nlp\%n} $> 0$ and {\tt nlp\%m} $\geq 0$.
% {\tt nlp\%J\%type} $\in \{
%   \mbox{\tt 'DENSE'}, \mbox{\tt 'COORDINATE'}, \mbox{\tt 'SPARSE\_BY\_ROWS'},
%   \mbox{\tt 'SPARSE\_BY\_COLUMNS'}, \mbox{\tt 'DIAGONAL'} \}$, and
% {\tt nlp\%H\%type} $\in \{
%   \mbox{\tt 'DENSE'}, \mbox{\tt 'COORDINATE'}, \mbox{\tt 'SPARSE\_BY\_ROWS'},
%   \mbox{\tt 'SPARSE\_BY\_COLUMNS'}, \mbox{\tt 'DIAGONAL'} \}$.
\itt{data} is a scalar \intentinout\ argument of type 
{\tt \packagename\_data\_type}
(see Section~\ref{typedata}). It is used to hold data about the
problem derivatives being verified.  With the possible exception of the component 
{\tt RC} (see Sections~\ref{typerc} and \ref{reverse}), 
it must not have been altered {\bf by the user} since the last call to 
{\tt \packagename\_initialize}.

\itt{control} is a scalar \intentin\ argument of type 
{\tt \packagename\_control\_type}
(see Section~\ref{typecontrol}). Default values may be assigned by calling 
{\tt \packagename\_initialize} prior to the first call to 
{\tt \packagename\_solve}.

\itt{inform} is a scalar \intentinout\ argument of type 
{\tt \packagename\_inform\_type}
(see Section~\ref{typeinform}). 
{\bf On initial entry, the  component {\tt status} must be set to the value 1.} 
Other entries need not be set.
A successful call to
{\tt \packagename\_verify}
is indicated when the  component {\tt status} has the value 0. 
For other return values of {\tt status}, see Sections~\ref{reverse} and
\ref{galerrors}.

\itt{userdata} is a scalar \intentinout\ argument of type 
{\tt GALAHAD\_userdata\_type} whose components may be used
to communicate user-supplied data to and from the
\optional\ subroutines
{\tt eval\_F}, {\tt eval\_C}, {\tt eval\_G}, {\tt eval\_J},
{\tt eval\_H}, {\tt eval\_Jv}, and {\tt eval\_Hv}
(see Section~\ref{typeuserdata}). 

\itt{eval\_F} is an \optional\ 
user-supplied subroutine whose purpose is to evaluate the 
objective function $f(\bmx)$ at a given vector $\bmx$.
See Section~\ref{ffv} for details.
If {\tt f\_availability} $=1$ (see Section~\ref{typecontrol}), then {\tt eval\_F}
must be present and declared {\tt EXTERNAL} in the calling program.
If {\tt f\_availability} $=2$, then \solver\ will use reverse communication to
obtain objective function values (see Section~\ref{reverse}).

\itt{eval\_C} is an \optional\ 
user-supplied subroutine whose purpose is to evaluate the 
constraint function $\bmc(\bmx)$ at a given vector $\bmx$.
See Section~\ref{cfv} for details.
If {\tt c\_availability} $=1$ (see Section~\ref{typecontrol}), then {\tt eval\_C} must be present 
and declared {\tt EXTERNAL} in the calling program.
If {\tt c\_availability} $=2$, then \solver\ will use reverse communication to
obtain constraint function values (see Section~\ref{reverse}).

\itt{eval\_G} is an \optional\ 
user-supplied subroutine whose purpose is to evaluate the  
gradient of the objective function $\nabla_x f(\bmx)$ at a given vector $\bmx$.
See Section~\ref{gfv} for details.
If {\tt g\_availability} $=1$ (see Section~\ref{typecontrol}), then {\tt eval\_G} must be present 
and declared {\tt EXTERNAL} in the calling program.
If {\tt g\_availability} $=2$, then \solver\ will use reverse communication to
obtain gradient values (see Section~\ref{reverse}).

\itt{eval\_J} is an \optional\ 
user-supplied subroutine whose purpose is to evaluate the 
Jacobian of the objective function $\nabla_{xx} f(\bmx)$ at a given vector $\bmx$.
See Section~\ref{hfv} for details.
If {\tt J\_availability} $=1$ (see Section~\ref{typecontrol}), then {\tt eval\_J} must be present and  
declared {\tt EXTERNAL} in the calling program.  Otherwise, {\tt eval\_J} need not be supplied.

\itt{eval\_H} is an \optional\ 
user-supplied subroutine whose purpose is to evaluate the
Hessian of the Lagrangian $\nabla_{xx} L(\bmx,\bmy)$ at a given point $(\bmx,\bmy)$.
See Section~\ref{hfv} for details.
If {\tt H\_availability} $=1$ (see Section~\ref{typecontrol}), then
{\tt eval\_H} must be present and  
declared {\tt EXTERNAL} in the calling program.  Otherwise, {\tt eval\_H} need not be supplied.

\itt{eval\_Jv} is an \optional\ 
user-supplied subroutine whose purpose is to evaluate the value of the 
product of the Jacobian of the constraint function 
$\nabla_x \bmc(\bmx)$ with a given vector $\bmv$.
See Section~\ref{hvfv} for details.
If {\tt J\_availability} $=3$ (see Section~\ref{typecontrol}), then {\tt eval\_Jv} must be present and
declared {\tt EXTERNAL} in the calling program.
Otherwise, {\tt eval\_Jv} need not be supplied.

\itt{eval\_Hv} is an \optional\ 
user-supplied subroutine whose purpose is to evaluate the value of the 
product of the Hessian of the Lagrangian function 
$\nabla_{xx} L(\bmx,\bmy)$ with a given vector $\bmv$.
See Section~\ref{hvfv} for details.
If {\tt H\_availability} $=3$ (see Section~\ref{typecontrol}), then {\tt eval\_Hv} must be present and
declared {\tt EXTERNAL} in the calling program.
Otherwise, {\tt eval\_Hv} need not be supplied.

\end{description}

%%%%%%% termination subroutine %%%%%%

\subsubsection{The  termination subroutine}
All previously allocated arrays are deallocated as follows:
\vspace*{1mm}

\hspace{8mm}
{\tt CALL \packagename\_terminate( data, control, inform )}

\vspace*{-1mm}
\begin{description}

\itt{data} is a scalar \intentinout\ argument of type 
{\tt \packagename\_data\_type} 
exactly as for
{\tt \packagename\_verify},
which must not have been altered {\bf by the user} since the last call to 
{\tt \packagename\_initialize} (except possibly {\tt data\%RC} as
described in Section~\ref{reverse}).
On exit, array components will have been deallocated.

\itt{control} is a scalar \intentin\ argument of type 
{\tt \packagename\_control\_type}
exactly as for
{\tt \packagename\_verify}.

\itt{inform} is a scalar \intentout\ argument of type
{\tt \packagename\_inform\_type}
exactly as for
{\tt \packagename\_verify}.
Only the component {\tt status} will be set on exit, and a 
successful call to 
{\tt \packagename\_terminate}
is indicated when this  component {\tt status} has the value 0. 
For other return values of {\tt status} see Section~\ref{galerrors}.

\end{description}

%%%%%%%%%%%%%%%%%%%% Function and derivative values %%%%%%%%%%%%%%%%%%%%%

\subsection{Function and derivative values\label{fdv}}

%%%%%%%%%%%%%%%%%%%%%% Objective function value %%%%%%%%%%%%%%%%%%%%%%

\subsubsection{The objective function value via internal evaluation\label{ffv}}

If the control parameter {\tt f\_availability} $=1$ (see
Section~\ref{typecontrol}), then the argument
{\tt eval\_F} must be present when calling \solver\ and the
user must provide a subroutine of that name to evaluate the
value of the objective function $f(\bmx)$.
The routine must be specified as

\def\baselinestretch{0.8}
{\tt \begin{verbatim}
      SUBROUTINE eval_F( status, X, userdata, F ) \end{verbatim} }
\def\baselinestretch{1.0}
\noindent whose arguments are as follows:

\begin{description}
\itt{status} is a scalar \intentout\ argument of type default \integer,
that should be set to 0 if the routine has been able to evaluate 
the objective function
and to a non-zero value if the evaluation has not been possible.

\ittf{X} is a rank-one \intentin\ array argument of type default \realdp\
whose components contain the vector $\bmx$.

\itt{userdata} is a scalar \intentinout\ argument of type 
{\tt GALAHAD\_userdata\_type} whose components may be used
to communicate user-supplied data to and from the
subroutines {\tt eval\_F}, {\tt eval\_C}, {\tt eval\_G}, {\tt eval\_J}
{\tt eval\_H}, {\tt eval\_Jv}, and {\tt eval\_Hv} 
(see Section~\ref{typeuserdata}). 

\ittf{F} is a scalar \intentout\ argument of type default \realdp\
that should be set to the value of the objective function $f(\bmx)$
evaluated at the vector $\bmx$ input in {\tt X}.

\end{description}

%%%%%%%%%%%%%%%%%%%%%% Constraint function value %%%%%%%%%%%%%%%%%%%%%%

\subsubsection{The constraint function value via internal evaluation\label{cfv}}

If the control parameter {\tt c\_availability} $=1$ (see
Section~\ref{typecontrol}), then the argument
{\tt eval\_C} must be present when calling \solver\ and the
user must provide a subroutine of that name to evaluate the
value of the constraint function $\bmc(\bmx)$.
The routine must be specified as

\def\baselinestretch{0.8}
{\tt \begin{verbatim}
      SUBROUTINE eval_C( status, X, userdata, C ) \end{verbatim} }
\def\baselinestretch{1.0}
\noindent whose arguments are as follows:

\begin{description}
\itt{status} is a scalar \intentout\ argument of type default \integer,
that should be set to 0 if the routine has been able to evaluate 
the constraint function
and to a non-zero value if the evaluation has not been possible.

\ittf{X} is a rank-one \intentin\ array argument of type default \realdp\
whose components contain the vector $\bmx$.

\itt{userdata} is a scalar \intentinout\ argument of type 
{\tt GALAHAD\_userdata\_type} whose components may be used
to communicate user-supplied data to and from the
subroutines {\tt eval\_F}, {\tt eval\_C}, {\tt eval\_G}, {\tt eval\_J}
{\tt eval\_H}, {\tt eval\_Jv}, and {\tt eval\_Hv} 
(see Section~\ref{typeuserdata}). 

\ittf{C} is a rank-one \intentout\ argument of type default \realdp\
that should be set to the value of the constraint function $\bmc(\bmx)$
evaluated at the vector $\bmx$ input in {\tt X}.

\end{description}

%%%%%%%%%%%%%%%%%%%%%% Gradient values %%%%%%%%%%%%%%%%%%%%%%

\subsubsection{Gradient values via internal evaluation\label{gfv}}

If the control parameter {\tt g\_availability} $=1$ (see
Section~\ref{typecontrol}), then the argument {\tt eval\_G} must be
present when calling \solver\ and the
user must provide a subroutine of that name to evaluate the
value of the gradient the objective function $\nabla_x f(\bmx)$.
The routine must be specified as

\def\baselinestretch{0.8}
{\tt \begin{verbatim}
       SUBROUTINE eval_G( status, X, userdata, G ) \end{verbatim} }
\def\baselinestretch{1.0}
\noindent whose arguments are as follows:

\begin{description}
\itt{status} is a scalar \intentout\ argument of type default \integer\
that should be set to 0 if the routine has been able to evaluate
the gradient of the objective function
and to a non-zero value if the evaluation has not been possible.

\ittf{X} is a rank-one \intentin\ array argument of type default \realdp\
whose components contain the vector $\bmx$.

\itt{userdata} is a scalar \intentinout\ argument of type 
{\tt GALAHAD\_userdata\_type} whose components may be used
to communicate user-supplied data to and from the
subroutines {\tt eval\_F}, {\tt eval\_C}, {\tt eval\_G}, {\tt eval\_J}
{\tt eval\_H}, {\tt eval\_Jv}, and {\tt eval\_Hv} 
(see Section~\ref{typeuserdata}).

\itt{G} is a rank-one \intentout\ argument of type default \realdp,
whose components should be set to the values of the gradient 
of the objective function $\nabla_x f(\bmx)$
evaluated at the vector $\bmx$ input in {\tt X}.

\end{description}

%%%%%%%%%%%%%%%%%%%%%% Jacobian values %%%%%%%%%%%%%%%%%%%%%%

\subsubsection{Jacobian values via internal evaluation\label{jfv}}

If the control parameter {\tt J\_availability} $=1$ (see
Section~\ref{typecontrol}), then the argument {\tt eval\_J} must be
present when calling \solver\ and the
user must provide a subroutine of that name to evaluate the
values of the Jacobian of the constraint function $\nabla_x \bmc(\bmx)$.
The routine must be specified as

\def\baselinestretch{0.8}
{\tt \begin{verbatim}
       SUBROUTINE eval_J( status, X, userdata, Jval ) \end{verbatim} }
\def\baselinestretch{1.0}
\noindent whose arguments are as follows:

\begin{description}
\itt{status} is a scalar \intentout\ argument of type default \integer\
that should be set to 0 if the routine has been able to evaluate
the Jacobian of the constraint function
and to a non-zero value if the evaluation has not been possible.

\ittf{X} is a rank-one \intentin\ array argument of type default \realdp\
whose components contain the vector $\bmx$.

\itt{userdata} is a scalar \intentinout\ argument of type 
{\tt GALAHAD\_userdata\_type} whose components may be used
to communicate user-supplied data to and from the
subroutines {\tt eval\_F}, {\tt eval\_C}, {\tt eval\_G}, {\tt eval\_J}
{\tt eval\_H}, {\tt eval\_Jv}, and {\tt eval\_Hv} 
(see Section~\ref{typeuserdata}).

\itt{Jval} is a scalar \intentout\ argument of type default \realdp,
whose components should be set to the values of the Jacobian
of the constraint function $\nabla_x \bmc(\bmx)$
evaluated at the vector $\bmx$ input in {\tt X}. The values should
be input in the same order as that in which the array indices were
given in {\tt nlp\%J}.

\end{description}

%%%%%%%%%%%%%%%%%%%%%% Hessian values %%%%%%%%%%%%%%%%%%%%%%

\subsubsection{Hessian values via internal evaluation\label{hfv}}

If the control parameter {\tt H\_availability} $=1$ (see
Section~\ref{typecontrol}), then the argument {\tt eval\_H} must be
present when calling \solver\ and the
user must provide a subroutine of that name to evaluate the
values of the Hessian of the Lagrangian $\nabla_{xx} L(\bmx,\bmy)$.
The routine must be specified as

\def\baselinestretch{0.8}
{\tt \begin{verbatim}
       SUBROUTINE eval_H( status, X, Y, userdata, Hval ) \end{verbatim} }
\def\baselinestretch{1.0}
\noindent whose arguments are as follows:

\begin{description}

\itt{status} is a scalar \intentout\ argument of type default \integer,
that should be set to 0 if the routine has been able to evaluate
the Hessian of the Lagrangian 
and to a non-zero value if the evaluation has not been possible.

\ittf{X} is a rank-one \intentin\ array argument of type default \realdp\
whose components contain the vector $\bmx$.

\ittf{Y} is a rank-one \intentin\ array argument of type default \realdp\
whose components contain the dual vector $\bmy$.

\itt{userdata} is a scalar \intentinout\ argument of type 
{\tt GALAHAD\_userdata\_type} whose components may be used
to communicate user-supplied data to and from the
subroutines {\tt eval\_F}, {\tt eval\_C}, {\tt eval\_G}, {\tt eval\_J}
{\tt eval\_H}, {\tt eval\_Jv}, and {\tt eval\_Hv} 
(see Section~\ref{typeuserdata}). 

\itt{Hval} is a scalar \intentout\ argument of type default \realdp,
whose components should be set to the values of the Hessian
of the Lagrangian $\nabla_{xx} L(\bmx,\bmy)$
evaluated at the vector $(\bmx,\bmy)$ input in {\tt X} and {\tt Y}. The values should
be input in the same order as that in which the array indices were
given in {\tt nlp\%H}.

\end{description}

%%%%%%%%%%%%%%%%%%%%%% Jacobian-vector products %%%%%%%%%%%%%%%%%%%%%%

\subsubsection{Jacobian-vector products via internal evaluation\label{jvfv}}

If the control parameter {\tt J\_availability} $=3$ (see
Section~\ref{typecontrol}), then the argument {\tt eval\_Jv} must be
present when calling \solver\ and the
user must provide a subroutine of that name to compute
products of the Jacobian of the constraint function (and its transpose)
of the form $\bmu + \nabla_{x} \bmc(\bmx)\bmv$ and
$\bmu + \nabla_{x} \bmc(\bmx)^T\bmv$.
The routine must be specified as

\def\baselinestretch{0.8}
{\tt \begin{verbatim}
       SUBROUTINE eval_Jv( status, X, userdata, transpose, U, V ) \end{verbatim} }
\def\baselinestretch{1.0}
\noindent whose arguments are as follows:

\begin{description}

\itt{status} is a scalar \intentout\ argument of type default \integer\
that should be set to 0 if the routine has been able to perform the
required calculation (see {\tt transpose} below) and to a non-zero
value if the computation has not been possible.

\ittf{X} is a rank-one \intentin\ array argument of type default \realdp\
whose components contain the vector $\bmx$.

\itt{userdata} is a scalar \intentinout\ argument of type 
{\tt GALAHAD\_userdata\_type} whose components may be used
to communicate user-supplied data to and from the
subroutines {\tt eval\_F}, {\tt eval\_C}, {\tt eval\_G}, {\tt eval\_J}
{\tt eval\_H}, {\tt eval\_Jv}, and {\tt eval\_Hv} 
(see Section~\ref{typeuserdata}).

\itt{transpose} is a scalar \intentin\ argument of type default
\logical.  If {\tt transpose} $=$ \false, then the user should evaluate the
sum $\bmu + \nabla_x \bmc(\bmx) \bmv$. If {\tt transpose} $=$ \true,
then the user should evaluate the sum $\bmu + \nabla_x \bmc(\bmx)^T \bmv$.   

\ittf{U} is a rank-one \intentinout\ array argument of type default \realdp\
whose components on input contain the vector $\bmu$ and on output contains
either the sum $\bmu + \nabla_x \bmc(\bmx) \bmv$ or $\bmu + \nabla_x \bmc(\bmx)^T
\bmv$ depending on the value of {\tt transpose} given above.

\ittf{V} is a rank-one \intentin\ array argument of type default \realdp\
whose components contain the vector $\bmv$.

\end{description}

%%%%%%%%%%%%%%%%%%%%%% Hessian-vector products %%%%%%%%%%%%%%%%%%%%%%

\subsubsection{Hessian-vector products via internal evaluation\label{hvfv}}

If the control parameter {\tt H\_availability} $=3$ (see
Section~\ref{typecontrol}), then the argument {\tt eval\_Hv} must be
present when calling \solver\ and the
user must provide a subroutine of that name to evaluate the
sum $\bmu + \nabla_{xx} L(\bmx,\bmy) \bmv$ involving the
product of the Hessian of the Lagrangian $\nabla_{xx} L(\bmx,\bmy)$.
The routine must be specified as

\def\baselinestretch{0.8}
{\tt \begin{verbatim}
       SUBROUTINE eval_Hv( status, X, Y, userdata, U, V ) \end{verbatim} }
\def\baselinestretch{1.0}
\noindent whose arguments are as follows:

\begin{description}
\itt{status} is a scalar \intentout\ argument of type default \integer\
that should be set to 0 if the routine has been able to perform the
required calculation and to a non-zero value if the computation has not
been possible. 

\ittf{X} is a rank-one \intentin\ array argument of type default \realdp\
whose components contain the vector $\bmx$.

\ittf{Y} is a rank-one \intentin\ array argument of type default \realdp\
whose components contain the vector $\bmy$.
\itt{userdata} is a scalar \intentinout\ argument of type 
{\tt GALAHAD\_userdata\_type} whose components may be used
to communicate user-supplied data to and from the
subroutines {\tt eval\_F}, {\tt eval\_C}, {\tt eval\_G}, {\tt eval\_J}
{\tt eval\_H}, {\tt eval\_Jv}, and {\tt eval\_Hv} 
(see Section~\ref{typeuserdata}). 

\ittf{U} is a rank-one \intentinout\ array argument of type default \realdp\
whose components on input contain the vector $\bmu$ and on output the
sum $\bmu + \nabla_{xx} L(\bmx,\bmy) \bmv$.

\ittf{V} is a rank-one \intentin\ array argument of type default \realdp\
whose components contain the vector $\bmv$.


\end{description}

%%%%%%%%%%%%%%%%%%%%%% Reverse communication %%%%%%%%%%%%%%%%%%%%%%%%

\subsection{\label{reverse}Reverse Communication Information}

A positive value of {\tt inform\%status} on exit from 
{\tt \packagename\_verify}
indicates that
\solver\ is seeking further information---this will happen 
if the user has chosen not to evaluate function or 
derivative values internally (see Sections~\ref{typecontrol} and~\ref{fdv}).
The user should compute the required information and re-enter \solver\
with all arguments (except those specifically mentioned below) unchanged.

Possible values of {\tt inform\%status} and the information required are
\begin{description}

\ittf{2.} The user should compute the objective function
     value $f(\bmx)$ at the point $\bmx$ indicated in {\tt data\%RC\%X}.
     The required value should be set in {\tt data\%RC\%F}.  If the user is
     unable to evaluate $f(\bmx)$---for instance, if the function is
     undefined at $\bmx$---the user need not set {\tt data\%RC\%F}, but
     should then set {\tt inform\%status} to any negative value.
     Otherwise, the value of {\tt inform\%status} should remain unchanged.

\ittf{3.} The user should compute the constraint function
     value $\bmc(\bmx)$ at the point $\bmx$ indicated in {\tt data\%RC\%X}.
     The required value should be set in {\tt data\%RC\%C}.  If the user is
     unable to evaluate $\bmc(\bmx)$---for instance, if the function is
     undefined at $\bmx$---the user need not set {\tt data\%RC\%C}, but
     should then set {\tt inform\%status} to any negative value.
     Otherwise, the value of {\tt inform\%status} should remain unchanged.

\ittf{4.} The user should compute the gradient 
     of the objective function $\nabla_x f(\bmx)$ at the point $\bmx$ 
     indicated in {\tt data\%RC\%X}.
     The value of the $i$-th component of the gradient should be set 
     in {\tt data\%RC\%G(i)} for $i = 1, \ldots, n$.  If the user is
     unable to evaluate a component of $\nabla_x f(\bmx)$---for instance, 
     if a component of the gradient is
     undefined at $\bmx$---the user need not set {\tt data\%RC\%G}, but
     should then set {\tt inform\%status} to a negative
     value. Otherwise, the value of {\tt inform\%status} should remain
     unchanged.

\ittf{5.} The user should compute the Jacobian  
     of the constraint function $\nabla_x \bmc(\bmx)$ at the point $\bmx$ 
     indicated in {\tt data\%RC\%X}. The $l$-th component of the Jacobian stored according to the
     scheme used to input {\tt nlp\%J} (see Section~\ref{typeprob})
     should be set in {\tt data\%RC\%Jval(l)} 
     for $l = 1, \ldots,$ {\tt  nlp\%J\%ne}.  If the user is
     unable to evaluate a component of $\nabla_x \bmc(\bmx)$---for instance, 
     if a component of the Jacobian is
     undefined at $\bmx$---the user need not set {\tt data\%RC\%Jval}, but
     should then set {\tt inform\%status} to a negative number.
     Otherwise, the value of {\tt inform\%status} should remain
     unchanged.

\ittf{6.} The user should compute the value 
     $\bmu + \nabla_x \bmc(\bmx)\bmv$, which requires a product of the Jacobian 
     of the constraint function  $\nabla_x \bmc(\bmx)$ at the
     point $\bmx$ with the vector $\bmv$; the vectors
     $\bmx$, $\bmu$, and $\bmv$ are contained in {\tt
       data\%RC\%X}, {\tt data\%RC\%U}, and {\tt
       data\%RC\%V}, respectively.  On exit, the resulting
     vector $\bmu + \nabla_x \bmc(\bmx)\bmv$ should be stored in
     {\tt data\%RC\%U}.  If the user is
     unable to evaluate the product---for instance, 
     if a component of the Jacobian is
     undefined at $\bmx$---the user need not set {\tt data\%RC\%U}, but
     should then set {\tt inform\%status} to a negative value.
     Otherwise, the value of {\tt inform\%status} should remain
     unchanged.

\ittf{7.} The user should compute the value 
     $\bmu + \nabla_x \bmc(\bmx)^T\bmv$, which requires a product of
     the transpose of the Jacobian 
     of the constraint function $\nabla_x \bmc(\bmx)$ at the
     point $\bmx$ with the vector $\bmv$; the vectors
     $\bmx$, $\bmu$, and $\bmv$ are contained in {\tt
       data\%RC\%X}, {\tt data\%RC\%U}, and {\tt
       data\%RC\%V}, respectively.  On exit, the resulting
     vector $\bmu + \nabla_x \bmc(\bmx)^T\bmv$ should be stored in
     {\tt data\%RC\%U}.  If the user is
     unable to evaluate the product---for instance, 
     if a component of the Jacobian is
     undefined at $\bmx$---the user need not set {\tt data\%RC\%U}, but
     should then set {\tt inform\%status} to a negative value.
     Otherwise, the value of {\tt inform\%status} should remain
     unchanged.

\ittf{8.} The user should compute the Hessian 
     of the Lagrangian $\nabla_{xx} L(\bmx,\bmy)$ at the point $(\bmx,\bmy)$ 
     indicated in {\tt data\%RC\%X} and {\tt data\%RC\%Y}.
     The $l$-th component of the Hessian stored according to the
     scheme used to input {\tt nlp\%H} (see Section~\ref{typeprob})
     should be set in {\tt data\%RC\%Hval(l)} 
     for $l = 1, \ldots,$ {\tt  nlp\%H\%ne}.  If the user is
     unable to evaluate a component of $\nabla_{xx} L(\bmx,\bmy)$---for instance, 
     if a component of the Hessian is
     undefined at $(\bmx,\bmy)$---the user need not set {\tt data\%RC\%Hval}, but
     should then set {\tt inform\%status} to a negative value.
     Otherwise, the value of {\tt inform\%status} should remain
     unchanged.

\ittf{9.} The user should compute the value 
     $\bmu + \nabla_{xx} L(\bmx,\bmy)\bmv$, which requires a product of the Hessian 
     of the Lagrangian $\nabla_{xx} L(\bmx,\bmy)$ at the
     point $(\bmx,\bmy)$ with the vector $\bmv$; the vectors
     $\bmx$, $\bmy$, $\bmu$, and $\bmv$ are contained in {\tt
       data\%RC\%X}, {\tt data\%RC\%Y}, {\tt data\%RC\%U}, and {\tt
       data\%RC\%V}, respectively.  On exit, the resulting
     vector $\bmu + \nabla_{xx} L(\bmx,\bmy)\bmv$ should be stored in
     {\tt data\%RC\%U}.  If the user is
     unable to evaluate the product---for instance, 
     if a component of the Hessian is
     undefined at $(\bmx,\bmy)$---the user need not set {\tt data\%RC\%U}, but
     should then set {\tt inform\%status} to a negative value.
     Otherwise, the value of {\tt inform\%status} should remain
     unchanged.

\end{description}

%%%%%%%%%%%%%%%%%%%%%% Warning and error messages %%%%%%%%%%%%%%%%%%%%%%%%

\galerrors
A negative value of {\tt inform\%status} on exit from 
{\tt \packagename\_verify}
or 
{\tt \packagename\_terminate}
indicates that an error has occurred. No further calls should be made
until the error has been corrected. Possible values are:

\begin{description}

\itt{\galerrallocate.} An allocation error occurred. 
A message indicating the offending
array is written on unit {\tt control\%error}, and the returned allocation 
status and a string containing the name of the offending array
are held in {\tt inform\%alloc\_\-status}
and {\tt inform\%bad\_alloc}, respectively.

\itt{\galerrdeallocate.} A deallocation error occurred. 
A message indicating the offending 
array is written on unit {\tt control\%error} and the returned allocation 
status and a string containing the name of the offending array
are held in {\tt inform\%alloc\_\-status}
and {\tt inform\%bad\_alloc}, respectively.

\itt{\galerrrestrictions.} 
Either one of the restrictions {\tt nlp\%n} $> 0$ or {\tt nlp\%m}
$\geq 0$ is violated, or the
requirement that {\tt nlp\%J\_type} and {\tt nlp\%H\_type} contain
a relevant string
{\tt 'DENSE'}, {\tt 'COORDINATE'}, {\tt 'SPARSE\_BY\_ROWS'},
{\tt 'SPARSE\_BY\_COLUMNS'}, or {\tt 'DIAGONAL'}
is not satisfied.

\itt{-50.}
The user has called {\tt \packagename\_verify} with {\tt inform\%status}
$< 0$, which indicates that the user was not able to perform a
requested computation needed during reverse communication.

\itt{-51.}
The user has called {\tt \packagename\_verify} with
{\tt inform\%status} $= 0$, which should never happen.  The user
should only set {\tt inform\%status} in two situations: {\tt inform\%status}
$= 1$ prior to the initial call to {\tt \packagename\_verify}, and
{\tt inform\%status} $< 0$ when reverse communication is being used
and the user is unable to perform the required computation as
indicated by the value of {\tt inform\%status} on return from {\tt \packagename\_verify} (see
Section~\ref{reverse}). The user should not
change {\tt inform\%status} for any other reason.

\itt{-55.}
The user has input an invalid value for at least one of the control
parameters
{\tt f\_availability}, {\tt c\_availability}, {\tt g\_availability},
{\tt J\_availability}, or {\tt H\_availability} as described in
Section~\ref{typecontrol}.

\itt{-56.}
Based on the values of the control parameters
{\tt f\_availability}, {\tt c\_availability}, {\tt g\_availability},
{\tt J\_availa\-bility}, and {\tt H\_availability} (see Section~\ref{typecontrol}), at least one 
optional dummy subroutine is missing in the call to {\tt \packagename\_verify}.

\itt{-57.}
At least one component of {\tt nlp\%X\_l} or {\tt nlp\%X\_u} is
inappropriate (see Section~\ref{typeprob}).

\itt{-58.}
A user supplied function (see Sections~\ref{ffv}--\ref{hvfv}) returned {\tt inform\%status} $\neq 0$,
implying that the function computation could not be performed at the required point. 

\end{description}

%%%%%%%%%%%%%%%%%%%%%% Further features %%%%%%%%%%%%%%%%%%%%%%%%

\galfeatures
\noindent In this section, we describe an alternative means of setting 
control parameters---that is components of the variable {\tt control} of type
{\tt \packagename\_control\_type}
(see Section~\ref{typecontrol})---by reading an appropriate data specification file using the
subroutine {\tt \packagename\_read\_specfile}. This facility
is useful as it allows a user to change  {\tt \packagename} control parameters 
without editing and recompiling programs that call {\tt \packagename}.

A specification file, or specfile, is a data file containing a number of 
"specification commands". Each command occurs on a separate line, 
and comprises a "keyword", 
which is a string (in a close-to-natural language) used to identify a 
control parameter, and 
an (optional) "value", which defines the value to be assigned to the given
control parameter. All keywords and values are case insensitive, 
keywords may be preceded by one or more blanks but
values must not contain blanks, and
each value must be separated from its keyword by at least one blank.
Values must not contain more than 30 characters, and 
each line of the specfile is limited to 80 characters,
including the blanks separating keyword and value.

The portion of the specification file used by 
{\tt \packagename\_read\_specfile}
must start
with a "{\tt BEGIN \packagename}" command and end with an 
"{\tt END}" command.  The syntax of the specfile is thus defined as follows:
\begin{verbatim}
  ( .. lines ignored by CHECK_read_specfile .. )
    BEGIN CHECK
       keyword    value
       .......    .....
       keyword    value
    END 
  ( .. lines ignored by CHECK_read_specfile .. )
\end{verbatim}
where keyword and value are two strings separated by (at least) one blank.
The ``{\tt BEGIN \packagename}'' and ``{\tt END}'' delimiter command lines 
may contain additional (trailing) strings so long as such strings are 
separated by one or more blanks, so that lines such as
\begin{verbatim}
    BEGIN CHECK SPECIFICATION
\end{verbatim}
and
\begin{verbatim}
    END CHECK SPECIFICATION
\end{verbatim}
are acceptable. Furthermore, 
between the
``{\tt BEGIN \packagename}'' and ``{\tt END}'' delimiters,
specification commands may occur in any order.  Blank lines and
lines whose first non-blank character is {\tt !} or {\tt *} are ignored. 
The content 
of a line after a {\tt !} or {\tt *} character is also 
ignored (as is the {\tt !} or {\tt *}
character itself). This provides an easy manner to "comment out" some 
specification commands, or to comment specific values 
of certain control parameters.  

The value of a control parameters may be of three different types, namely
integer, logical, or real.
Integer and real values may be expressed in any relevant Fortran integer and
floating-point formats (respectively). Permitted values for logical
parameters are "{\tt ON}", "{\tt TRUE}", "{\tt .TRUE.}", "{\tt T}", 
"{\tt YES}", "{\tt Y}", or "{\tt OFF}", "{\tt NO}",
"{\tt N}", "{\tt FALSE}", "{\tt .FALSE.}" and "{\tt F}". 
Empty values are also allowed for 
logical control parameters, and are interpreted as "{\tt TRUE}".  

The specification file must be open for 
input when {\tt \packagename\_read\_specfile}
is called, and the associated device number 
passed to the routine in device (see below). 
Note that the corresponding 
file is {\tt REWIND}ed, which makes it possible to combine the specifications 
for more than one program/routine.  For the same reason, the file is not
closed by {\tt \packagename\_read\_specfile}.

\subsubsection{To read control parameters from a specification file}
\label{readspec}

Control parameters may be read from a file as follows:
\hskip0.5in 
\def\baselinestretch{0.8} {\tt \begin{verbatim}
     CALL CHECK_read_specfile( control, device ) \end{verbatim}
}
\def\baselinestretch{1.0}

\begin{description}
\itt{control} is a scalar \intentinout argument of type 
{\tt \packagename\_control\_type}
(see Section~\ref{typecontrol}). 
Default values should have already been set, perhaps by calling 
{\tt \packagename\_initialize}.
On exit, individual components of {\tt control} may have been changed
according to the commands found in the specfile. Specfile commands and 
the components (see Section~\ref{typecontrol}) of {\tt control} 
that they affect are given in Table~\ref{specfile}.

\bctable{|l|l|l|} 
\hline
  command & component of {\tt control} & value type \\ 
\hline
  {\tt error-printout-device} & {\tt \%error} & integer \\
  {\tt printout-device} & {\tt \%out} & integer \\
  {\tt print-level} & {\tt \%print\_level} & integer \\
  {\tt verification-level} & {\tt \%verify\_level} & integer \\
  {\tt f-availability} & {\tt \%f\_availability} & integer \\
  {\tt c-availability} & {\tt \%c\_availability} & integer \\
  {\tt g-availability} & {\tt \%G\_availability} & integer \\
  {\tt J-availability} & {\tt \%J\_availability} & integer \\
  {\tt H-availability} & {\tt \%H\_availability} & integer \\
  {\tt check-gradient} & {\tt \%checkG} & logical \\
  {\tt check-Jacobian} & {\tt \%checkJ} & logical \\
  {\tt check-Hessian} & {\tt \%checkH} & logical \\
  {\tt deallocate-error-fatal}   & {\tt \%deallocate\_error\_fatal} & logical \\
\hline

\ectable{\label{specfile}Specfile commands and associated 
components of {\tt control}.}

\itt{device} is a scalar \intentin argument of type default \integer,
that must be set to the unit number on which the specfile
has been opened. If {\tt device} is not open, {\tt control} will
not be altered and execution will continue, but an error message
will be printed on unit {\tt control\%error}.

\end{description}

%%%%%%%%%%%%%%%%%%%%%% Information printed %%%%%%%%%%%%%%%%%%%%%%%%

\galinfo
If {\tt control\%print\_level} is positive, information about the
derivative verification will be printed on unit {\tt control\-\%out},
provided {\tt control\%out} $> 0$.
If {\tt control\%print\_level} $= 1$, a basic summary of the
derivative checking is produced.  If {\tt control\%print\_level} $=
2$, then in addition to the above there is detailed output of the
derivative verification, control parameters are printed, and basic
matrix data is produced.  If {\tt control\%print\_level} $=
3$, then in addition to the above, full matrix data is printed.
Finally, {\tt control\%print\_level} $\geq 4$ is used for debugging
and in addition to the above also prints private data used during the
verification process.    

%%%%%%%%%%%%%%%%%%%%%% GENERAL INFORMATION %%%%%%%%%%%%%%%%%%%%%%%%

\galgeneral

\galcommon None.
\galworkspace Provided automatically by the module.
\galroutines None. 
\galmodules {\tt \packagename\_verify} and {\tt \packagename\_terminate} call the \galahad\ packages
%{\tt GALAHAD\_CPU\_time},
%{\tt GALAHAD\_SY\-M\-BOLS}, \sloppy
%{\tt GALAHAD\_SPECFILE},
%{\tt GALAHAD\_SPLS},
%{\tt GALAHAD\_GLTR},
%{\tt GALAHAD\_TRS},
{\tt GALAHAD\_MOP}, and
{\tt GALAHAD\_SPACE}.
\galio Output is under control of the arguments
 {\tt control\%error}, {\tt control\%out} and {\tt control\%print\_level}.
\galrestrictions {\tt nlp\%n} $> 0$ and {\tt nlp\%m} $\geq 0$.
%{\tt nlp\%H\_type} $\in \{${\tt 'DENSE'}, 
% {\tt 'COORDINATE'}, {\tt 'SPARSE\_BY\_ROWS'}, {\tt 'DIAGONAL'} $\}$. 
\galportability ISO Fortran~95 + TR 15581 or Fortran~2003. 
The package is thread-safe.

%%%%%%%%%%%%%%%%%%%%%% METHOD %%%%%%%%%%%%%%%%%%%%%%%%

\galmethod
Finite difference approximations are used to numerically ``verify'' requested
derivatives.  If {\tt verify\_level} $=1$, we perform a
``cheap'' check of the gradient of the objective function by
comparing $(f(\bmx) - f(\bmx+\alpha \bms))/\alpha$ with $\nabla_x f(\bmx)^T\bms$
for some appropriately chosen vector $\bms$ and scalar $0 < \alpha \ll
1$.  Similarly, for the Jacobian of the constraints and the Hessian of
the Lagrangian, we compare $(\bmc(\bmx) - \bmc(\bmx+\alpha
\bms))/\alpha$ to $\nabla_x \bmc(\bmx)\bms$ and $(\nabla_xL(\bmx,\bmy)
- \nabla_xL(\bmx+\alpha\bms,\bmy))/\alpha$ to $\nabla_{xx}
L(\bmx,\bmy)\bms$, respectively.  If {\tt verify\_level} $=2$, we perform an
``expensive'' verification of the gradient of the objective function by
comparing $(f(\bmx) - f(\bmx+\alpha\bme_i))/\alpha$ with $[\nabla_x
f(\bmx)]_i$ for $i = 1, \dots, n$, where $\bme_i$ is the $i$th
coordinate vector.  Similarly, for the Jacobian of the constraints and the Hessian of
the Lagrangian, we compare $[(\bmc(\bmx) - \bmc(\bmx+\alpha
\bme_j))]_i/\alpha$ to $[\nabla_x \bmc(\bmx)]_{ij}$ and $[(\nabla_xL(\bmx,\bmy)
- \nabla_xL(\bmx+\alpha\bme_j,\bmy))]_i/\alpha$ to $[\nabla_{xx}
L(\bmx,\bmy)]_{ij}$, respectively.   

%\vspace*{1mm}

%\galreferences
%\vspace*{1mm}

%\noindent
%The generic trust-region method is described in detail in
%\vspace*{1mm}

%\noindent
%A. R. Conn, N. I. M. Gould and Ph. L. Toint (2000).
%Trust-region methods.
%SIAM/MPS Series on Optimization.

%%%%%%%%%%%%%%%%%%%%%% EXAMPLE %%%%%%%%%%%%%%%%%%%%%%%%

\galexamples
Suppose we wish to perform an ``expensive'' check of the derivatives associated with the objective function
$f(\bmx) = x_1 + x_2^3/3$ and the constraint function
$\bmc(\bmx) = (x_1 + x_2^2 + x_3^3 + x_3x_2^2, -x_2^4)$ at the point
$\bmx = (4,3,2)$ and $\bmy = (2,3)$, with bounds $\bmx^l = (-5,-5,-5)$ and
$\bmx^u = (5,5,5)$.
We may use the following code:

{\tt \small
\VerbatimInput{\packageexample}
}
\noindent
The code produces the following output:
{\tt \small
\VerbatimInput{\packageresults}
}

\noindent

\end{document}

