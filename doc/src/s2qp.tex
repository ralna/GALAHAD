\documentclass{galahad}

\usepackage{amsfonts,amssymb,amsmath}

% set the release and package names

\newcommand{\package}{s2qp}
\newcommand{\packagename}{S2QP}
\newcommand{\fullpackagename}{\libraryname\_\packagename}

% macros for this file only

\DeclareMathOperator*{\minim}{minimize}
\DeclareMathOperator*{\subject}{subject}
\newcommand{\minimize}[1]{{\displaystyle\minim_{#1}}}
\newcommand{\into}{\rightarrow}
\newcommand{\Grad}{\nabla}

\begin{document}

\galheader

%%%%%%%%%%%%%%%%%%%%%% SUMMARY %%%%%%%%%%%%%%%%%%%%%%%%

\galsummary

This package uses a {\bf trust-region/linesearch hybrid method to find
  a (local) constrained minimizer} of the following optimization
  problem:
\eqn{prob}{
   \begin{tabular*}{5in}{l@{\extracolsep{\fill}}c@{\extracolsep{0pt}}ll%
                            @{\extracolsep{\fill}}c}
         & $\minimize{\bmx\in\smallRe^n}$&\quad& $f(\bmx)$ &$$ \\ [3pt]
         &  $\subject$        &     &  $\bmc_\ell \leq c(\bmx) \leq \bmc_u$ & \\
         &                    &     &  $\bma_\ell \leq \bmA \bmx \leq \bma_u$ & \\
         &                    &     &  $\bmx_\ell \leq \bmx \leq \bmx_u$
   \end{tabular*}
 } where $f : \Re^n \into \Re$ is a differentiable objective function,
 $c : \Re^n \into \Re^m$ is a differentiable \emph{general} constraint
 function, $\bmA$ is an $\Re^{m_a}$ by $\Re^n$ \emph{linear} constraint
 matrix , $\bmc_\ell$ and $\bmc_u$ are vectors in $\Re^m$ of upper and lower
 bounds for the general constraints $c(\bmx)$, $\bma_\ell$ and $\bma_u$ are
 vectors in $\Re^{m_a}$ of upper and lower bounds for the linear
 constraints $\bmA\bmx$, and $\bmx_\ell$ and $\bmx_u$ are vectors in $\Re^n$ of
 upper and lower bounds on the variables $\bmx$.  First derivatives
 are required and second derivatives may be exploited when they can be
 calculated---if the product of first and/or second derivative matrices with a vector may
 be found but not the derivatives themselves, then this may also be
 exploited.

%  USE GALAHAD_NLPT_double, ONLY: NLPT_problem_type, NLPT_userdata_type, &
%                                  NLPT_write_problem
%   USE GALAHAD_LSQP_double
%   USE GALAHAD_QPC_double
%   USE GALAHAD_EQP_double
%   USE GALAHAD_QPT_double
%   USE GALAHAD_QPD_double, LSQP_data_type => QPD_data_type
%   USE GALAHAD_SMT_double
%   USE GALAHAD_SILS_double
%   USE GALAHAD_SPECFILE_double
%   USE GALAHAD_SPACE_double
%   USE GALAHAD_COPYRIGHT
%   USE GAlAHAD_SYMBOLS
%   USE GALAHAD_CHECK_double
%   USE GALAHAD_SORT_double
%   USE GALAHAD_mop_double
%   USE CUTEr_interface_double
%%%%%%%%%%%%%%%%%%%%%% attributes %%%%%%%%%%%%%%%%%%%%%%%%

\galattributes
\galversions{\tt  \fullpackagename\_single, \fullpackagename\_double},
\galuses
  {\tt GALAHAD\_NLPT},
  {\tt GALAHAD\_LSQP},
  {\tt GALAHAD\_QPC},
  {\tt GALAHAD\_EQP},
  {\tt GALAHAD\_QPT},
  {\tt GALAHAD\_QPD},
  {\tt GALAHAD\_SMT},
  {\tt GALAHAD\_SILS},
  {\tt GALAHAD\_SPECFILE},
  {\tt GALAHAD\_SPACE},
  {\tt GALAHAD\_COPYRIGHT},
  {\tt GALAHAD\_SYMBOLS},
  {\tt GALAHAD\_CHECK},
  {\tt GALAHAD\_SORT},
  {\tt GALAHAD\_MOP},
  {\tt CUTER\_INTERFACE}.
\galdate April 2001.
\galorigin N. I. M. Gould, Rutherford Appleton Laboratory, and
D. P. Robinson, University of Oxford, UK.
\gallanguage Fortran~90. 

%%%%%%%%%%%%%%%%%%%%%% HOW TO USE %%%%%%%%%%%%%%%%%%%%%%%%

\galhowto

The package is available using both single and double precision reals, 
and either 32-bit or 64-bit integers. Access to the 32-bit integer,
single precision version requires the {\tt USE} statement
\medskip

\hspace{8mm} {\tt USE \fullpackagename\_single}

\medskip
\noindent
with the obvious substitution {\tt \fullpackagename\_double},
{\tt \fullpackagename\_single\_64} and 
{\tt \fullpackagename\_double\_64} for the other variants.

\noindent
If it is required to use more than one of the modules at the same time, 
the derived types
{\tt NLPT\_type}, {\tt S2QP\_control\_type}, {\tt S2QP\_data\_type}, 
{\tt S2QP\_inform\_type} (Section~\ref{galtypes})
and the subroutines {\tt S2QP\_initialize}, {\tt S2QP\_solve}, and
{\tt S2QP\_terminate} (Section~\ref{galarguments}) must be renamed on
one of the {\tt USE} statements.

%%%%%%%%%%%%%%%%%%%%%% matrix storage formats %%%%%%%%%%%%%%%%%%%%%%%%

\galmatrix The $m$ by $n$ general constraint Jacobian $\bmJ \eqdef
\Grad c(\bmx)$, the $m_a$ by $n$ linear constraint matrix $\bmA$, and
the symmetric $n$ by $n$ Hessian matrix $\bmH$ (when available) may
be stored in a variety of input formats.

\subsubsection{Dense storage format}\label{dense}
The matrices $\bmA$ and $\bmJ$ are stored as compact dense matrices by
rows, that is, the values of the entries of each row in turn are
stored in order within an appropriate real one-dimensional array.
Component $n \ast (i-1) + j$ of the storage array {\tt A\%val} will
hold the value $\bmA_{ij}$ for $i = 1, \ldots , m_a$, $j = 1, \ldots ,
n$.  Similarly, component $n \ast (i-1) + j$ of the storage array {\tt
  J\%val} will hold the value $\bmJ_{i,j}$ for $i = 1, \ldots , m$, $j =
1, \ldots , n$.  Since $\bmH$ is symmetric, only the lower triangular
part (that is the part $\bmH_{ij}$ for $1 \leq j \leq i \leq n$)
should be stored. In this case the lower triangle will be stored by rows, that is
component $i \ast (i-1)/2 + j$ of the storage array {\tt H\%val} will
hold the value $\bmH_{ij}$ (and, by symmetry, $\bmH_{ji}$) for $1 \leq j
\leq i \leq n$.

\subsubsection{Sparse co-ordinate storage format}\label{coordinate}
Only the nonzero entries of the matrices are stored. For the $l$-th
entry of $\bmA$, its row index $i$, column index $j$ and value
$\bmA_{ij}$ are stored in the $l$-th components of the integer arrays
{\tt A\%row}, {\tt A\%col} and real array {\tt A\%val}.  Similarly,
the $l$-th entry of $\bmJ$, its row index $i$, column index $j$ and
value $\bmJ_{ij}$ are stored in the $l$-th components of the integer
arrays {\tt J\%row}, {\tt J\%col} and real array {\tt J\%val}.  The
order is unimportant, but the total number of entries {\tt A\%ne} and
{\tt J\%ne} are required.  The same scheme is applicable to $\bmH$
(thus requiring integer arrays {\tt H\%row}, {\tt H\%col}, a real
array {\tt H\%val}, and an integer value {\tt H\%ne}), except that only
the entries in the lower triangle should be stored.

We recommend this storage format since it is the most efficient
storage scheme for the underlying linear system solvers.  Any other
storage formats are converted internally to sparse co-ordinate
storage and then handled accordingly.

\subsubsection{Sparse row-wise storage format}\label{rowwise}
Again only the nonzero entries are stored, but this time they are
ordered so that those in row $i$ appear directly before those in row
$i+1$. For the $i$-th row of $\bmA$, the $i$-th component of the integer
array {\tt A\%ptr} holds the position of the first entry in this row,
while {\tt A\%ptr} $(m_a+1)$ holds the total number of entries plus one.
The column indices $j$ and values $\bmA_{ij}$ of the entries in the
$i$-th row are stored in components $l =$ {\tt A\%ptr}$(i)$, \ldots
,{\tt A\%ptr} $(i+1)-1$ of the integer array {\tt A\%col}, and real
array {\tt A\%val}, respectively.  Similarly, for the $i$-th row of
$\bmJ$, the $i$-th component of the integer array {\tt J\%ptr} holds the
position of the first entry in this row, while {\tt J\%ptr} $(m+1)$
holds the total number of entries plus one.  The column indices $j$
and values $\bmJ_{ij}$ of the entries in the $i$-th row are stored in
components $l =$ {\tt J\%ptr}$(i)$, \ldots ,{\tt J\%ptr} $(i+1)-1$ of
the integer array {\tt J\%col}, and real array {\tt J\%val},
respectively.  The same scheme is applicable to
$\bmH$ (thus requiring integer arrays {\tt H\%ptr}, {\tt H\%col}, and 
a real array {\tt H\%val}),
except that only the entries in the lower triangle should be stored.

For sparse matrices, this scheme almost always requires less storage than 
its predecessor.

\subsubsection{Sparse column-wise storage format}\label{columnwise}
Again only the nonzero entries are stored, but this time they are
ordered so that those in column $j$ appear directly before those in
column $j+1$. For the $j$-th column of $\bmA$, the $j$-th component of
the integer array {\tt A\%ptr} holds the position of the first entry in
this column, while {\tt A\%ptr} $(n+1)$ holds the total number of
entries plus one.  The row indices $i$ and values $\bmA_{ij}$ of the
entries in the $j$-th column are stored in components $l =$ {\tt
  A\%ptr}$(j)$, \ldots ,{\tt A\%ptr} $(j+1)-1$ of the integer array
{\tt A\%row}, and real array {\tt A\%val}, respectively.  Similarly,
for the $j$-th column of $\bmJ$, the $j$-th component of the integer array
{\tt J\%ptr} holds the position of the first entry in this column,
while {\tt J\%ptr} $(n+1)$ holds the total number of entries plus one.
The row indices $i$ and values $\bmJ_{ij}$ of the entries in the
$j$-th column are stored in components $l =$ {\tt J\%ptr}$(j)$, \ldots
,{\tt J\%ptr} $(j+1)-1$ of the integer array {\tt J\%row}, and real
array {\tt J\%val}, respectively.  The same scheme is applicable to
$\bmH$ (thus requiring integer arrays {\tt H\%ptr}, {\tt H\%row}, and
a real array {\tt H\%val}), except that only the entries in the lower
triangle should be stored.

\subsubsection{Diagonal storage format}\label{diagonal}
If $\bmH$ is diagonal (i.e., $\bmH_{ij} = 0$ for all $1 \leq i \neq j \leq n$)
only the diagonal entries $\bmH_{ii}$ for $1 \leq i \leq n$ need be stored,
and the first $n$ components of the array {\tt H\%val} may be used for 
the purpose. There is no sensible equivalent for the non-square
matrices $\bmA$ and $\bmJ$.

%%%%%%%%%%%%%%%%%%%%%%%%%%% kinds %%%%%%%%%%%%%%%%%%%%%%%%%%%

%%%%%%%%%%%%%%%%%%%%%% kinds %%%%%%%%%%%%%%%%%%%%%%

\galkinds
We use the terms integer and real to refer to the fortran
keywords {\tt REAL(rp\_)} and {\tt INTEGER(ip\_)}, where 
{\tt rp\_} and {\tt ip\_} are the relevant kind values for the
real and integer types employed by the particular module in use. 
The former are equivalent to
default {\tt REAL} for the single precision versions,
{\tt DOUBLE PRECISION} for the double precision cases
and quadruple-precision if 128-bit reals are available, and 
correspond to {\tt rp\_ = real32}, {\tt rp\_ = real64} and
{\tt rp\_ = real128} respectively as defined by the 
fortran {\tt iso\_fortran\_env} module.
The latter are default (32-bit) and long (64-bit) integers, and
correspond to {\tt ip\_ = int32} and {\tt ip\_ = int64},
respectively, again from the {\tt iso\_fortran\_env} module.


%%%%%%%%%%%%%%%%%%%%%% derived types %%%%%%%%%%%%%%%%%%%%%%%%

\galtypes
Six derived data types are accessible from the package.

%%%%%%%%%%% matrix data type %%%%%%%%%%%

\subsubsection{The derived data type for holding matrices}\label{typesmt}
The derived data type {\tt SMT\_TYPE} is used to hold the linear
constraint matrix $\bmA$, the Jacobian matrix $\bmJ$, and the Hessian
matrix $\bmH$ (if available). The components of {\tt SMT\_TYPE}
used here are:

\begin{description}

\ittf{m} is a scalar component of type \integer, 
that holds the number of rows of the matrix.

\ittf{n} is a scalar component of type \integer, 
that holds the number of columns of the matrix. 
 
\ittf{ne} is a scalar variable of type \integer, that
holds the number of matrix entries.

\ittf{type} is a rank-one allocatable array of type default \character, that
is used to indicate the matrix storage scheme used. Its precise length and
content depends on the type of matrix to be stored (see \S\ref{typeprob}).

\ittf{val} is a rank-one allocatable array of type \realdp\,
and dimension at least {\tt ne}, that holds the values of the entries.
For the {\em symmetric} matrix $\bmH$, each pair of off-diagonal entries
$\bmH_{ij} = \bmH_{ji}$ is represented
as a single entry (see \S\ref{dense}--\ref{columnwise}).

\ittf{row} is a rank-one allocatable array of type \integer\ 
and dimension at least {\tt ne}, that may hold the row indices of the entries 
(see \S\ref{coordinate} and \S\ref{columnwise}).

\ittf{col} is a rank-one allocatable array of type \integer\ 
and dimension at least {\tt ne}, that may hold the column indices of the entries
(see \S\ref{coordinate}--\ref{rowwise}).

\ittf{ptr} for sparse row-wise storage, {\tt ptr} is a rank-one
allocatable array of type \integer\ and dimension at least
{\tt m + 1}, that may hold the pointers to the first entry in each row
(see \S\ref{rowwise}).  For sparse column-wise storage, {\tt ptr}
is a rank-one allocatable array of type \integer\ and
dimension at least {\tt n + 1}, that may hold the pointers to the
first entry in each column (see \S\ref{columnwise}).

\end{description}

%%%%%%%%%%% problem type %%%%%%%%%%%

\subsubsection{The derived data type for holding the problem}\label{typeprob}
The derived data type {\tt NLPT\_problem\_type} is used to hold the
problem. The relevant components are:

\begin{description}

\ittf{n} is a scalar variable of type \integer\ 
 that holds the number of optimization variables $n$.

\ittf{m} is a scalar variable of type \integer\
 that holds the number of {\em general} constraints $m$, i.e., the length
 of the general constraint vector $c(\bmx)$.

\ittf{m\_a} is a scalar variable of type \integer\ 
 that holds the number of {\em linear} constraints $m_a$, i.e., the number
 of rows in the linear constraint matrix $\bmA$.

\ittf{X} is a rank-one allocatable array of dimension {\tt n} and type 
\realdp\ that holds the value $\bmx$ of the optimization variables.
The $j$-th component of {\tt X} contains
$\bmx_{j}$ for $j = 1,  \ldots , n$.

\ittf{X\_l} is a rank-one allocatable array of dimension {\tt n}
and type \realdp\ that holds the vector of lower bounds
$\bmx_\ell$ on the opimization variables {\tt X}. The $i$-th component of
{\tt X\_l} contains $[\bmx_\ell]_i$ for $i = 1, \ldots , n$.  Infinite
bounds are allowed by setting the corresponding components of {\tt
  X\_l} to any value smaller than {\tt -infinity}, where {\tt
  infinity} is a component of the control data type {\tt control} (see
Section~\ref{typecontrol}).

\ittf{X\_u}  is a rank-one allocatable array of dimension {\tt n}
and type \realdp\ that holds the vector of upper bounds
$\bmx_u$ on the optimization variables.  The $i$-th component of
{\tt X\_u} contains $[\bmx_u]_i$ for $i = 1, \ldots , n$.  Infinite
bounds are allowed by setting the corresponding components of {\tt
  X\_u} to any value larger than {\tt infinity}, where {\tt
  infinity} is a component of the control data type {\tt control} (see
Section~\ref{typecontrol}).

\ittf{Z} is a rank-one allocatable array of dimension {\tt n} and type 
\realdp\ that holds an estimate $\bmz$ of a Lagrange
multiplier vector (reduced cost vector) corresponding to the
constraints bounds on $\bmx$ (see Section~\ref{galmethod}).
The $i$-th component of {\tt Z} contains $\bmz_{i}$ for $i = 1,
\ldots , n$.

\ittf{X\_status}

\ittf{f} is a scalar variable of type 
\realdp\ that holds the value of the objective function $f(\bmx)$.

\ittf{Ax} is a rank-one allocatable array of dimension {\tt m\_a}
and type \realdp\ that holds the value of the linear
constraints $\bmA\bmx$ for the vector {\tt X} of optimization variables.   

\ittf{A\_l} is a rank-one allocatable array of dimension {\tt m\_a}
and type \realdp\ that holds the vector of lower bounds
$\bma_\ell$ on the {\em linear} constraints $\bmA\bmx$. The $i$-th component of
{\tt A\_l} contains $[\bma_\ell]_i$ for $i = 1, \ldots , m_a$.  Infinite
bounds are allowed by setting the corresponding components of {\tt
  A\_l} to any value smaller than {\tt -infinity}, where {\tt
  infinity} is a component of the control data type {\tt control} (see
Section~\ref{typecontrol}).

\ittf{A\_u}  is a rank-one allocatable array of dimension {\tt m\_a}
and type \realdp\ that holds the vector of upper bounds
$\bma_u$ on the {\em linear} constraints $\bmA\bmx$. The $i$-th component of
{\tt A\_u} contains $[\bma_u]_i$ for $i = 1, \ldots , m_a$.  Infinite
bounds are allowed by setting the corresponding components of {\tt
  A\_u} to any value larger than {\tt infinity}, where {\tt
  infinity} is a component of the control data type {\tt control} (see
Section~\ref{typecontrol}).

\ittf{Y\_a} is a rank-one allocatable array of dimension {\tt m\_a} and type 
\realdp\ that holds an estimate $\bmy_a$ of a Lagrange
multiplier vector corresponding to the linear constraints $\bmA\bmx$ (see Section~\ref{galmethod}).
The $i$-th component of {\tt Y\_a} contains $[\bmy_a]_{i}$ for $i = 1,
\ldots , m_a$.

\ittf{A\_status}

\ittf{A} is scalar variable of type {\tt SMT\_TYPE} 
that holds the linear constraint matrix $\bmA$. The following components
are used:

\begin{description}

\itt{A\%type} is an allocatable array of rank one and type default 
\character, that
is used to indicate the storage scheme used. If the dense storage scheme 
(see Section~\ref{dense}) is used, 
the first five components of {\tt A\%type} must contain the
string {\tt DENSE}.
For the sparse co-ordinate scheme (see Section~\ref{coordinate}), 
the first ten components of {\tt A\%type} must contain the
string {\tt COORDINATE},  
for the sparse row-wise storage scheme (see Section~\ref{rowwise}),
the first fourteen components of {\tt A\%type} must contain the
string {\tt SPARSE\_BY\_ROWS},
and for the sparse column-wise storage scheme (see Section~\ref{columnwise}),
the first seventeen components of {\tt A\%type} must contain the
string {\tt SPARSE\_BY\_COLUMNS}.

For convenience, the procedure {\tt SMT\_put} 
may be used to allocate sufficient space and insert the required keyword
into {\tt A\%type}.
For example, if {\tt p} is of derived type {\tt NLPT\_problem\_type}
and involves a linear constraint matrix that we wish to store using the co-ordinate scheme,
we may simply
%\vspace*{-2mm}
{\tt 
\begin{verbatim}
        CALL SMT_put( p%A%type, 'COORDINATE' )
\end{verbatim}
}
%\vspace*{-4mm}
\noindent
See the documentation for the \galahad\ package {\tt SMT} 
for further details on the use of {\tt SMT\_put}.

\itt{A\%ne} is a scalar variable of type \integer\ that holds
the number of entries in the matrix $\bmA$ in the sparse co-ordinate
storage scheme (see Section~\ref{coordinate}).  It need not be set for
any of the other schemes.

\itt{A\%val} is a rank-one allocatable array of type \realdp\ that holds
the values of the entries of the matrix $\bmA$ in any of the 
storage schemes discussed in Section~\ref{galmatrix}.

\itt{A\%row} is a rank-one allocatable array of type \integer\
that holds the row indices of the matrix $\bmA$ in the sparse
co-ordinate storage scheme (see Section~\ref{coordinate}), or the
sparse column-wise storage scheme (see Section~\ref{columnwise}).  It
need not be allocated for any of the other schemes.

\itt{A\%col} is a rank-one allocatable array variable of type default
\integer\ that holds the column indices of the matrix $\bmA$ in
either the sparse co-ordinate storage scheme (see Section~\ref{coordinate}), or the
sparse row-wise storage scheme (see Section~\ref{rowwise}).  It need
not be allocated for any of the other schemes.

\itt{A\%ptr} is a rank-one allocatable array of type \integer.
If the sparse row-wise storage scheme~(see Section~\ref{rowwise}) is used,
then {\tt A\%ptr} is of dimension at least {\tt m\_a+1} and holds the
starting position of each row of $\bmA$, as well as the total number
of entries plus one.  If the sparse column-wise storage scheme~(see
Section~\ref{columnwise}) is used, then {\tt A\%ptr} is of dimension
at least {\tt n+1} and holds the starting position of each column of $\bmA$, as well as
the total number of entries plus one.  It need not be allocated for
any of the other schemes.

\end{description}

\ittf{C} is a rank-one allocatable array of dimension {\tt m}
and type \realdp\ that holds the value of the general
constraint vector $c(\bmx)$ for the vector {\tt X} of optimization variables. 

\ittf{C\_l} is a rank-one allocatable array of dimension {\tt m}
and type \realdp\ that holds the vector of lower bounds
$\bmc_\ell$ on the {\em general} constraints $c(\bmx)$. The $i$-th component of
{\tt C\_l} contains $[\bmc_\ell]_i$ for $i = 1, \ldots , m$.  Infinite
bounds are allowed by setting the corresponding components of {\tt
  C\_l} to any value smaller than {\tt -infinity}, where {\tt
  infinity} is a component of the control data type {\tt control} (see
Section~\ref{typecontrol}).

\ittf{C\_u}  is a rank-one allocatable array of dimension {\tt m}
and type \realdp\ that holds the vector of upper bounds
$\bmc_u$ on the {\em general} constraints $c(\bmx)$. The $i$-th component of
{\tt C\_u} contains $[\bmc_u]_i$ for $i = 1, \ldots , m$.  Infinite
bounds are allowed by setting the corresponding components of {\tt
  C\_u} to any value larger than {\tt infinity}, where {\tt
  infinity} is a component of the control data type {\tt control} (see
Section~\ref{typecontrol}).

\ittf{Y} is a rank-one allocatable array of dimension {\tt m} and type 
\realdp\ that holds an estimate $\bmy$ of a Lagrange
multiplier vector corresponding to the general constraints $c(\bmx)$ (see Section~\ref{galmethod}).
The $i$-th component of {\tt Y} contains $\bmy_{i}$ for $i = 1,  \ldots , m$.

\ittf{C\_status}

\ittf{J} is scalar variable of type {\tt SMT\_TYPE} that holds the
nonlinear {\em general} constraint Jacobian matrix $\bmJ = \Grad c(\bmx)$. The
following components are used:

\begin{description}

\itt{J\%type} is an allocatable array of rank one and type default 
\character\ that
is used to indicate the storage scheme used. If the dense storage scheme 
(see Section~\ref{dense}) is used, 
the first five components of {\tt J\%type} must contain the
string {\tt DENSE}.
For the sparse co-ordinate scheme (see Section~\ref{coordinate}), 
the first ten components of {\tt J\%type} must contain the
string {\tt COORDINATE},  
for the sparse row-wise storage scheme (see Section~\ref{rowwise}),
the first fourteen components of {\tt J\%type} must contain the
string {\tt SPARSE\_BY\_ROWS},
and for the sparse column-wise storage scheme (see Section~\ref{columnwise}),
the first seventeen components of {\tt J\%type} must contain the
string {\tt SPARSE\_BY\_COLUMNS}.

For convenience, the procedure {\tt SMT\_put} may be used to allocate
sufficient space and insert the required keyword into {\tt J\%type}.
For example, if {\tt p} is of derived type {\tt NLPT\_problem\_type}
and involves a Jacobian matrix $\bmJ$ that we wish to store using
the sparse row-wise scheme, we may simply
%\vspace*{-2mm}
{\tt 
\begin{verbatim}
        CALL SMT_put( p%J%type, 'SPARSE_BY_ROWS' )
\end{verbatim}
}
%\vspace*{-4mm}
\noindent
See the documentation for the \galahad\ package {\tt SMT} 
for further details on the use of {\tt SMT\_put}.

\itt{J\%ne} is a scalar variable of type \integer\ that holds
the number of entries in the matrix $\bmJ$ in the sparse co-ordinate
storage scheme (see Section~\ref{coordinate}).  It need not be set for
any of the other schemes.

\itt{J\%val} is a rank-one allocatable array of type \realdp\ that holds
the values of the entries of the matrix $\bmJ$ in any of the 
storage schemes discussed in Section~\ref{galmatrix}.

\itt{J\%row} is a rank-one allocatable array of type \integer\
that holds the row indices of the matrix $\bmJ$ in the sparse
co-ordinate storage scheme (see Section~\ref{coordinate}), or the
sparse column-wise storage scheme (see Section~\ref{columnwise}).  It
need not be allocated for any of the other schemes.

\itt{J\%col} is a rank-one allocatable array variable of type default
\integer\ that holds the column indices of the matrix $\bmJ$ in
either the sparse co-ordinate storage scheme (see Section~\ref{coordinate}), or the
sparse row-wise storage scheme (see Section~\ref{rowwise}).  It need
not be allocated for any of the other schemes.

\itt{J\%ptr} is a rank-one allocatable array of type default
\integer.  If the sparse row-wise storage scheme~(see Section~\ref{rowwise})
is used, then {\tt J\%ptr} is of dimension at least {\tt m+1} and holds the
starting position of each row of $\bmJ$, as well as the total number
of entries plus one.  If the sparse column-wise storage scheme~(see
Section~\ref{columnwise}) is used, then {\tt J\%ptr} is of dimension
at least {\tt n+1} and holds the starting position of each column of $\bmJ$, as well as
the total number of entries plus one.  It need not be allocated for
any of the other schemes.

\ittf{G} is a rank-one allocatable array of dimension {\tt n} and type 
\realdp\ that holds the gradient $\bmg = \Grad f(\bmx)$ 
of the objective function.
The $j$-th component of 
{\tt G} contains $\bmg_{j}$ for $j = 1, \ldots, n$.

\end{description}

\ittf{H} is scalar variable of type {\tt SMT\_TYPE} 
that holds the Hessian matrix $\bmH$. The following components
are used:

\begin{description}

\itt{H\%type} is an allocatable array of rank one and type default 
\character\ that
is used to indicate the storage scheme used. If the dense storage scheme 
(see Section~\ref{dense}) is used, 
the first five components of {\tt H\%type} must contain the
string {\tt DENSE}.
For the sparse co-ordinate scheme (see Section~\ref{coordinate}), 
the first ten components of {\tt H\%type} must contain the
string {\tt COORDINATE},  
for the sparse row-wise storage scheme (see Section~\ref{rowwise}),
the first fourteen components of {\tt H\%type} must contain the
string {\tt SPARSE\_BY\_ROWS},
for the sparse column-wise storage scheme (see Section~\ref{columnwise}),
the first seventeen components of {\tt H\%type} must contain the
string {\tt SPARSE\_BY\_COLUMNS},
and for the diagonal storage scheme (see Section~\ref{diagonal}),
the first eight components of {\tt H\%type} must contain the
string {\tt DIAGONAL}.

For convenience, the procedure {\tt SMT\_put} 
may be used to allocate sufficient space and insert the required keyword
into {\tt H\%type}.
For example, if {\tt p} is of derived type {\tt NLPT\_problem\_type}
and involves a Hessian we wish to store using the co-ordinate scheme,
we may simply
%\vspace*{-2mm}
{\tt 
\begin{verbatim}
        CALL SMT_put( p%H%type, 'COORDINATE' )
\end{verbatim}
}
%\vspace*{-4mm}
\noindent
See the documentation for the \galahad\ package {\tt SMT} 
for further details on the use of {\tt SMT\_put}.

\itt{H\%ne} is a scalar variable of type \integer\ that 
holds the number of entries in the {\bf lower triangular} part of $\bmH$
in the sparse co-ordinate storage scheme (see Section~\ref{coordinate}). 
It need not be set for any of the other schemes.

\itt{H\%val} is a rank-one allocatable array of type \realdp\ that holds
the values of the entries of the {\bf lower triangular} part
of the Hessian matrix $\bmH$ in any of the 
storage schemes discussed in Section~\ref{galmatrix}.

\itt{H\%row} is a rank-one allocatable array of type \integer\
that holds the row indices of the {\bf lower triangular} part of
$\bmH$ in the sparse co-ordinate storage scheme (see
Section~\ref{coordinate}), or the sparse column-wise storage scheme
(see Section~\ref{columnwise}).  It need not be allocated for any of
the other schemes.

\itt{H\%col} is a rank-one allocatable array variable of type \integer\
that holds the column indices of the {\bf lower triangular} part of 
$\bmH$ in either the sparse co-ordinate storage scheme 
(see Section~\ref{coordinate}), or the sparse row-wise 
storage scheme (see Section~\ref{rowwise}).
It need not be allocated for any of the other storage schemes.

\itt{H\%ptr} is a rank-one allocatable array of type \integer.
If the sparse row-wise storage scheme~(see Section~\ref{rowwise}) is
used, then {\tt H\%ptr} is of dimension at least {\tt n+1} that holds the starting position
of each row of the {\bf lower triangular} part of $\bmH$, as well as
the total number of entries plus one.  If the sparse column-wise
storage scheme~(see Section~\ref{columnwise}) is used, then {\tt
  H\%ptr} is of dimension at least {\tt n+1} and holds the starting position of each column
of the {\bf lower triangular} part of $\bmH$, as well as the total
number of entries plus one.

\end{description}

\itt{pname} is a scalar variable of type 
default \character\ and length 10, which contains the
``name'' of the problem for printing. The default ``empty'' string is
provided.

\itt{VNAMES} is a rank-one allocatable array of dimension {\tt n} and type 
default \character\ and length 10, whose $j$-th entry contains the
``name'' of the $j$-th variable for printing. This is only used 
if {\tt control\%print\_level > 0}, where {\tt control} is of derived
type {\tt \packagename\_control\_type} (see
Section~\ref{typecontrol}); if not allocated, it will be ignored.

\itt{ANAMES} is a rank-one allocatable array of dimension {\tt m\_a}
and type default \character\ and length 10, whose $j$-th entry
contains the ``name'' of the $j$-th {\em linear} constraint (the
$j$-th row of $\bmA\bmx$) for printing.  This is only used if {\tt
  control\%print\_level > 0}, where {\tt control} is of derived type
{\tt \packagename\_control\_type} (see Section~\ref{typecontrol}); if
not allocated, it will be ignored.

\itt{CNAMES} is a rank-one allocatable array of dimension {\tt m} and
type default \character\ and length 10, whose $j$-th entry contains
the ``name'' of the $j$-th {\em general} constraint $c_j(\bmx)$ for
printing.  This is only used if {\tt control\%print\_level > 0}, where
{\tt control} is of derived type {\tt \packagename\_control\_type}
(see Section~\ref{typecontrol}); if not allocated, it will be ignored.

\end{description}

%%%%%%%%%%% control type %%%%%%%%%%%

\subsubsection{The derived data type for holding control 
 parameters}\label{typecontrol}
The derived data type 
{\tt \packagename\_control\_type} 
is used to hold controlling data. Default values may be obtained by calling 
{\tt \packagename\_initialize}
(see Section~\ref{subinit}). The components of 
{\tt \packagename\_control\_type} 
are:

\begin{description}

\itt{print\_level} is a scalar variable of type \integer\ that is used
to control the amount of informational output which is required. No 
informational output will occur if {\tt print\_level} $\leq$ 0. 
If {\tt print\_level} $= 1$, a single line of output will be produced for each
iteration of the process. If {\tt print\_level} $\geq 2$, this output will be
increased to provide significant detail of each iteration.
The default is {\tt print\_level} $= 0$. 

\itt{out} is a scalar variable of type \integer\ that holds the
stream number for informational messages.
Printing of informational messages in 
{\tt \packagename\_solve}
is suppressed if {\tt out} $\leq 0$.
The default is {\tt out} $= 6$.

\itt{error} is a scalar variable of type \integer\ that holds the
stream number for error messages.
Printing of error messages in 
{\tt \packagename\_solve}
and
{\tt \packagename\_terminate}
is suppressed if {\tt error} $\leq$ 0.
The default is {\tt error} $=$ 6.

\itt{print\_number} is a scalar variable of type \integer\
that holds {\em half} the number of vector components that will be
printed for each print statement.  To be precise, the first {\tt
  print\_number} and the last {\tt print\_number} components of each
vector will be printed.  If $2\ast${\tt print\_number} is greater than the
length of the vector to be printed, then the entire vector is printed.

\itt{alive\_unit} is a scalar variable of type \integer.
If {\tt alive\_unit} $>$ 0, a temporary file named {\tt alive\_file} (see below)
will be created on stream number {\tt alive\_unit} on initial entry to 
{\tt \packagename\_solve}, and execution of {\tt \packagename\_solve} will continue so
long as this file continues to exist. Thus, a user may terminate execution
simply by removing the temporary file from this unit.
If {\tt alive\_unit} $\leq$ 0, no temporary file will be created and
execution cannot be terminated in this way.
The default is {\tt alive\_unit} $=$ 60.

\itt{max\_iterate}

\itt{start\_print}

\itt{stop\_print}
\itt{print\_gap}
\itt{run\_level}
\itt{header\_every}
\itt{fc\_availability}
\itt{gJ\_avialability}
\itt{H\_availability}
\itt{NM\_steps}
\itt{B\_type}
\itt{L\_BFGS\_type}
\itt{BFGS\_curve\_mod}
\itt{stop\_\_abs}
\itt{stop\_\_rel}
\itt{stop\_d\_abs}
\itt{stop\_d\_rel}
\itt{stop\_c\_abs}
\itt{stop\_c\_rel}
\itt{initial\_penalty}
\itt{max\_penalty}
\itt{penalty\_expansion}
\itt{initial\_TRpred}
\itt{max\_TRpred}
\itt{TRsqp\_scale}
\itt{max\_TRsqp}
\itt{infinity}
\itt{eta\_successful}
\itt{eta\_very\_successful}
\itt{eta\_extremely\_successful}
\itt{print\_sol}
\itt{fulsol}
\itt{deallocate\_error\_fatal}
\itt{use\_steering}
\itt{use\_sequ\_min}
\itt{use\_seqp}
\itt{use\_siqp}
\itt{use\_TRpred}
\itt{alive\_file}
\itt{QPfeas\_control}
\itt{QPpred\_control}
\itt{QPsteer\_control}
\itt{QPsiqp\_control}
\itt{QPseqp\_control}
\itt{SILS\_control}

\end{description}

%%%%%%%%%%% inform type %%%%%%%%%%%

\subsubsection{The derived data type for holding informational
 parameters}\label{typeinform}
The derived data type 
{\tt \packagename\_inform\_type} 
is used to hold parameters that give information about the progress and needs 
of the algorithm. The components of
{\tt \packagename\_inform\_type} 
are:

\begin{description}
\itt{status} is a scalar variable of type \integer\ that gives the
exit status of the algorithm. See Sections~\ref{galerrors} and \ref{galinfo}
for details.
\itt{alloc\_status}
\itt{iterate}
\itt{num\_f\_eval}
\itt{num\_g\_eval}
\itt{num\_c\_eval}
\itt{num\_J\_eval}
\itt{num\_H\_eval}
\itt{num\_descent\_active}
\itt{obj}
\itt{primal\_vl}
\itt{dual\_vl}
\itt{comp\_vl}
\itt{bad\_alloc}
\itt{time}
\itt{QPfeas\_inform}
\itt{QPpred\_inform}
\itt{QPsteer\_inform}
\itt{QPsiqp\_inform}
\itt{QPseqp\_inform}
\end{description}

%%%%%%%%%%% time type %%%%%%%%%%%

\subsubsection{The derived data type for holding timing 
 information}\label{typetime}
The derived data type 
{\tt \packagename\_time\_type} 
is used to hold elapsed CPU times for the various parts of the calculation.
The components of 
{\tt \packagename\_time\_type} 
are:
\begin{description}
\itt{total} is a scalar variable of type default \real, that gives
 the total time spent in the package.

\itt{preprocess} is a scalar variable of type default \real, that gives
 the time spent reordering the problem to standard form prior to solution.

\itt{analyse} is a scalar variable of type default \real, that gives
 the time spent analysing required matrices prior to factorization.

\itt{factorize} is a scalar variable of type default \real, that gives
 the time spent factorizing the required matrices.

\itt{solve} is a scalar variable of type default \real, that gives
 the time spent using the factors to solve relevant linear equations.

\itt{feas\_preprocess}
\itt{feas\_depend}
\itt{feas\_analyse}
\itt{feas\_factorize}
\itt{feas\_solve}
\itt{feas\_total}

\itt{pred\_A\_preprocess}
\itt{pred\_A\_analyse}
\itt{pred\_A\_factorize}
\itt{pred\_A\_solve}
\itt{pred\_A\_total}

\itt{pred\_B\_preprocess}
\itt{pred\_B\_analyse}
\itt{pred\_B\_factorize}
\itt{pred\_B\_solve}
\itt{pred\_B\_total}

\itt{pred\_C\_preprocess}
\itt{pred\_C\_depend}
\itt{pred\_C\_analyse}
\itt{pred\_C\_factorize}
\itt{pred\_C\_solve}
\itt{pred\_C\_total}

\itt{steer\_A\_preprocess}
\itt{steer\_A\_analyse}
\itt{steer\_A\_factorize}
\itt{steer\_A\_solve}
\itt{steer\_A\_total}

\itt{steer\_B\_preprocess}
\itt{steer\_B\_analyse}
\itt{steer\_B\_factorize}
\itt{steer\_B\_solve}
\itt{steer\_B\_total}

\itt{siqp\_A\_preprocess}
\itt{siqp\_A\_analyse}
\itt{siqp\_A\_factorize}
\itt{siqp\_A\_solve}
\itt{siqp\_A\_total}

\itt{siqp\_B\_preprocess}
\itt{siqp\_B\_analyse}
\itt{siqp\_B\_factorize}
\itt{siqp\_B\_solve}
\itt{siqp\_B\_total}

\itt{siqp\_C\_preprocess}
\itt{siqp\_C\_depend}
\itt{siqp\_C\_analyse}
\itt{siqp\_C\_factorize}
\itt{siqp\_C\_solve}
\itt{siqp\_C\_total}

\itt{seqp\_factorize}
\itt{seqp\_solve}
\itt{siqp\_total}

\itt{total\_preprocess}
\itt{total\_depend}
\itt{total\_analyse}
\itt{total\_factorise}
\itt{total\_solve}
\itt{total\_total}

\itt{cauchy\_total}
\itt{opt\_test\_total}

\itt{in}
\itt{out}

\itt{enter\_s2qp}
\itt{exit\_s2qp}

\end{description}

%%%%%%%%%%% data type %%%%%%%%%%%

\subsubsection{The derived data type for holding problem data}\label{typedata}
The derived data type 
{\tt \packagename\_data\_type}
is used to hold all the data for a particular problem,
or sequences of problems with the same structure, between calls of 
{\tt \packagename} procedures. 
This data should be preserved, untouched (except as directed on
return from {\tt \packagename\_solve} with positive values of
{\tt inform\%status}, see Section~\ref{reverse}),
from the initial call to 
{\tt \packagename\_initialize}
to the final call to
{\tt \packagename\_terminate}.

%%%%%%%%%%%%%%%%%%%%%% argument lists %%%%%%%%%%%%%%%%%%%%%%%%

\galarguments
There are three procedures for user calls:

\begin{enumerate}
\item The subroutine 
      {\tt \packagename\_initialize} 
      is used to set default values, and initialize private data, 
      before solving \ldots
\item The subroutine 
      {\tt \packagename\_solve} 
      is called to solve \ldots
\item The subroutine 
      {\tt \packagename\_terminate} 
      is provided to allow the user to automatically deallocate array 
       components of the private data, allocated by 
       {\tt \packagename\_solve}, 
       at the end of the solution process. 
\end{enumerate}
We use square brackets {\tt [ ]} to indicate \optional\ arguments.

%%%%%% initialization subroutine %%%%%%

\subsubsection{The initialization subroutine}\label{subinit}
 Default values are provided as follows:
\vspace*{1mm}

\hspace{8mm}
{\tt CALL \packagename\_initialize( data, control )}

\vspace*{-3mm}
\begin{description}

\itt{data} is a scalar \intentinout\ argument of type 
{\tt \packagename\_data\_type}
(see Section~\ref{typedata}). It is used to hold data about the problem being 
solved.

\itt{control} is a scalar \intentout\ argument of type 
{\tt \packagename\_control\_type}
(see Section~\ref{typecontrol}). 
On exit, {\tt control} contains default values for the components as
described in Section~\ref{typecontrol}.
These values should only be changed after calling 
{\tt \packagename\_initialize}.

\end{description}

%%%%%%%%% main solution subroutine %%%%%%

\subsubsection{The minimization subroutine}
The minimization algorithm is called as follows:
\vspace*{1mm}

\hspace{8mm}
{\tt CALL \packagename\_solve( nlp, control, inform, data, userdata,
          [eval\_FC, eval\_GJ, eval\_H, eval\_Jv, eval\_Hv] )}

\vspace*{-3mm}
\begin{description}
\itt{nlp} is a scalar \intentinout\ argument of type 
{\tt \packagename\_problem\_type}
(see Section~\ref{typeprob}). 
It is used to hold data about the problem being solved.

\itt{control} is a scalar \intentin\ argument of type 
{\tt \packagename\_control\_type}
(see Section~\ref{typecontrol}). Default values may be assigned by calling 
{\tt \packagename\_initialize} 
prior to the first call to 
{\tt \packagename\_solve}.

\itt{inform} is a scalar \intentout\ argument of type 
{\tt \packagename\_inform\_type}
(see Section~\ref{typeinform}). A successful call to
{\tt \packagename\_solve}
is indicated when the  component {\tt status} has the value 0. 
For other return values of {\tt status}, see Section~\ref{galerrors}.

\itt{data} is a scalar \intentinout\ argument of type 
{\tt \packagename\_data\_type}
(see Section~\ref{typedata}). It is used to hold data about the problem being 
solved. It must not have been altered {\bf by the user} since the last call to 
{\tt \packagename\_initialize}.

\itt{userdata}

\itt{eval\_FC} is an \optional\ rank-one \intentin\ pointer array of type 
 \realdp,  that holds \ldots

\itt{eval\_GJ} is an \optional\ rank-one \intentin\ pointer array of type 
 \realdp,  that holds \ldots

\itt{eval\_H} is an \optional\ rank-one \intentin\ pointer array of type 
 \realdp,  that holds \ldots

\itt{eval\_Jv} is an \optional\ rank-one \intentin\ pointer array of type 
 \realdp,  that holds \ldots

\itt{eval\_Hv} is an \optional\ rank-one \intentin\ pointer array of type 
 \realdp,  that holds \ldots

\end{description}

%%%%%%% termination subroutine %%%%%%

\subsubsection{The  termination subroutine}
All previously allocated arrays are deallocated as follows:
\vspace*{1mm}

\hspace{8mm}
{\tt CALL \packagename\_terminate( data, control, inform )}

\vspace*{-3mm}
\begin{description}

\itt{data} is a scalar \intentinout\ argument of type 
{\tt \packagename\_data\_type} 
exactly as for
{\tt \packagename\_solve}
that must not have been altered {\bf by the user} since the last call to 
{\tt \packagename\_initialize}.
On exit, array components will have been deallocated.

\itt{control} is a scalar \intentin\ argument of type 
{\tt \packagename\_control\_type}
exactly as for
{\tt \packagename\_solve}.

\itt{inform} is a scalar \intentout\ argument of type 
{\tt \packagename\_inform\_type}
exactly as for
{\tt \packagename\_solve}.
Only the component {\tt status} will be set on exit, and a 
successful call to 
{\tt \packagename\_terminate}
is indicated when this  component {\tt status} has the value 0. 
For other return values of {\tt status}, see Section~\ref{galerrors}.

\end{description}

%%%%%%%%%%%%%%%%%%%%%% function subroutines %%%%%%%%%%%%%%%%%%%%%%%%

\subsection{User function subroutines}\label{functions}


%%%%%%%%%%%%%%%%%%%% function FC %%%%%%%%%%%%%%%%%%%%%%%%%%%

\subsubsection{Subroutine for supplying function values}\label{FC}

%%%%%%%%%%%%%%%%%%%% function GJ %%%%%%%%%%%%%%%%%%%%%%%%%%%

\subsubsection{Subroutine for supplying gradient values}\label{GJ}

%%%%%%%%%%%%%%%%%%%% function H %%%%%%%%%%%%%%%%%%%%%%%%%%%

\subsubsection{Subroutine for supplying Hessian values}\label{H}

%%%%%%%%%%%%%%%%%%%%%% reverse communication %%%%%%%%%%%%%%%%%%%%%%%%

\galreverse

%%%%%%%%%%%%%%%%%%%%%% Warning and error messages %%%%%%%%%%%%%%%%%%%%%%%%

\galerrors
A negative value of {\tt info@status} on exit from 
{\tt \packagename\_solve}
or 
{\tt \packagename\_terminate}
indicates that an error has occurred. No further calls should be made
until the error has been corrected. Possible values are:

\begin{description}
\item{- 1 } One of the restrictions \ldots
          has been violated.
\end{description}
A positive value of {\tt info@status} on exit from 
{\tt \packagename\_solve}
is used to warn the user that the data may be faulty or that 
the subroutine cannot guarantee the solution obtained.
Possible values are:


%%%%%%%%%%%%%%%%%%%%%% Information printed %%%%%%%%%%%%%%%%%%%%%%%%

\galinfo
If {\tt control@print\_level} is positive, information about the progress 
of the algorithm will be printed on unit {\tt control@out}.

%%%%%%%%%%%%%%%%%%%%%% GENERAL INFORMATION %%%%%%%%%%%%%%%%%%%%%%%%

\galgeneral

\galcommon None.
\galworkspace Provided automatically by the module.
\galroutines None. 
\galmodules \ldots
\galio Output is provided under the control of {\tt control@print\_level}.
     If the user supplies a unit number in {\tt control@out}, the message
     is printed to the user supplied unit. However if this unit
     number is negative, printing is suppressed.
\galrestrictions None.
\galportability ISO Fortran 90.

%%%%%%%%%%%%%%%%%%%%%% METHOD %%%%%%%%%%%%%%%%%%%%%%%%

\galmethod
\ldots

\galreference
The method is described in detail in \ldots

%%%%%%%%%%%%%%%%%%%%%% EXAMPLE %%%%%%%%%%%%%%%%%%%%%%%%

\galexample
Suppose we wish to solve the problem \ldots

\noindent
Then we may use the following code
{\tt \small
\VerbatimInput{\packageexample}
}
%\noindent
%with the following data
%{\tt \small
%\VerbatimInput{\packagedata}
%}
\noindent
This produces the following output:
This produces the following output:
{\tt \small
\VerbatimInput{\packageresults}
}
\noindent

\end{document}
