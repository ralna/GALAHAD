\documentclass{galahad}

% set the package name

\newcommand{\package}{rpd}
\newcommand{\packagename}{RPD}
\newcommand{\fullpackagename}{\libraryname\_\packagename}

\begin{document}

\galheader

%%%%%%%%%%%%%%%%%%%%%% SUMMARY %%%%%%%%%%%%%%%%%%%%%%%%

\galsummary
{\bf Read and write data} for the linear program (LP)
\disp{\mbox{minimize}\;\; \bmg^T \bmx + f
  \;\mbox{subject to}\; \bmc_l \leq \bmA \bmx \leq \bmc_u
  \;\mbox{and}\; \bmx_l \leq  \bmx  \leq \bmx_u,}
the linear program with quadratic constraints (QCP)
\disp{\mbox{minimize}\;\; \bmg^T \bmx + f
  \;\mbox{subject to}\; \bmc_l \leq \bmA \bmx +
    \frac{1}{2} \mbox{vec}(\bmx.\bmH_c.\bmx) \leq \bmc_u
  \;\mbox{and}\; \bmx_l \leq  \bmx  \leq \bmx_u,}
the bound-constrained quadratic program (BQP)
\disp{\mbox{minimize}\;\; \frac{1}{2} \bmx^T \bmH \bmx + \bmg^T \bmx + f
  \;\mbox{subject to}\; \bmx_l \leq  \bmx  \leq \bmx_u,}
the quadratic program (QP)
\disp{\mbox{minimize}\;\; \frac{1}{2} \bmx^T \bmH \bmx + \bmg^T \bmx + f
  \;\mbox{subject to}\; \bmc_l \leq \bmA \bmx \leq \bmc_u
  \;\mbox{and}\; \bmx_l \leq  \bmx  \leq \bmx_u,}
or the quadratic program with quadratic constraints (QCQP)
\disp{\mbox{minimize}\;\; \frac{1}{2} \bmx^T \bmH \bmx + \bmg^T \bmx + f
  \;\mbox{subject to}\; \bmc_l \leq \bmA \bmx +
    \frac{1}{2} \mbox{vec}(\bmx.\bmH_c.\bmx) \leq \bmc_u
  \;\mbox{and}\; \bmx_l \leq  \bmx  \leq \bmx_u,}
involving the $n$ by $n$ symmetric matrices $\bmH$
and $(\bmH_c)_i$, $i = 1,\ldots,m$, the $m$ by $n$ matrix $\bmA$,
the vectors $\bmg$, $\bmc^{l}$, $\bmc^{u}$, $\bmx^{l}$,
$\bmx^{u}$, the scalar $f$, and
where vec$( \bmx . \bmH_c . \bmx )$ is the vector whose
$i$-th component is  $\bmx^T (\bmH_c)_i \bmx$ for the $i$-th constraint,
{\bf from and to a QPLIB-format data file}.
Any of the constraint bounds $c_{i}^{l}$, $c_{i}^{u}$,
$x_{j}^{l}$ and $x_{j}^{u}$ may be infinite.
Full advantage is taken of any zero coefficients in the matrices $\bmH$,
$(\bmH_c)_i$ and $\bmA$.

%%%%%%%%%%%%%%%%%%%%%% attributes %%%%%%%%%%%%%%%%%%%%%%%%

\galattributes
\galversions{\tt  \fullpackagename\_single, \fullpackagename\_double}.
\galuses
{\tt \libraryname\_CLOCK},
{\tt \libraryname\_SY\-M\-BOLS},
{\tt \libraryname\_SPACE}, {\tt \libraryname\_\-NORMS},
{\tt \libraryname\_\-SMT},
{\tt \libraryname\_\-QPT},
{\tt \libraryname\_SPECFILE},
{\tt \libraryname\_SORT},
{\tt \libraryname\_LMS}
\galdate January 2006
\galorigin N. I. M. Gould.
\gallanguage Fortran~2003.

%%%%%%%%%%%%%%%%%%%%%% HOW TO USE %%%%%%%%%%%%%%%%%%%%%%%%

\galhowto

Access to the package requires a {\tt USE} statement such as

\medskip\noindent{\em Single precision version}

\hskip0.5in {\tt USE \fullpackagename\_single}

\medskip\noindent{\em Double precision version}

\hskip0.5in {\tt USE  \fullpackagename\_double}

\medskip

\noindent
If it is required to use both modules at the same time, the derived types
{\tt SMT\_TYPE},
{\tt \packagename\_control\_type},
{\tt \packagename\_inform\_type},
{\tt \packagename\_data\_type},
(Section~\ref{galtypes})
and the subroutines
{\tt \packagename\_read\_problem\_data},
must be renamed on one of the {\tt USE} statements.

%%%%%%%%%%%%%%%%%%%%%% matrix formats %%%%%%%%%%%%%%%%%%%%%%%%

\galmatrix
The objective Hessian matrix $\bmH$, the constraint Hessians $(H_c)_i$ and
the constraint Jacobian $\bmA$ will be available in a sparse co-ordinate
storage format.

\subsubsection{Sparse co-ordinate storage format}\label{coordinate}
Only the nonzero entries of the matrices are stored. For the
$l$-th entry of $\bmA$, its row index $i$, column index $j$
and value $a_{ij}$
are stored in the $l$-th components of the integer arrays {\tt A\%row},
{\tt A\%col} and real array {\tt A\%val}, respectively.
The order is unimportant, but the total
number of entries {\tt A\%ne} is also required.
The same scheme is applicable to
$\bmH$ (thus requiring integer arrays {\tt H\%row}, {\tt H\%col}, a real array
{\tt H\%val} and an integer value {\tt H\%ne}),
except that only the entries in the lower triangle need be stored.
For the constraint Hessians, a third index giving the constraint involved
is required for each entry, and is stored in the integer array
{\tt H\%ptr}. Once again, only the lower traingle is stored.

%%%%%%%%%%%%%%%%%%%%%% derived types %%%%%%%%%%%%%%%%%%%%%%%%

\galtypes
Five derived data types are accessible from the package.

%%%%%%%%%%% matrix data type %%%%%%%%%%%

\subsubsection{The derived data type for holding matrices}\label{typesmt}
The derived data type {\tt SMT\_TYPE} is used to hold the matrices
$\bmH$, $(\bmH_c)_i$ and $\bmA$.
The components of {\tt SMT\_TYPE} used here are:

\begin{description}

\ittf{m} is a scalar component of type default \integer,
that holds the number of rows in the matrix.

\ittf{n} is a scalar component of type default \integer,
that holds the number of columns in the matrix.

\ittf{ne} is a scalar variable of type default \integer, that
holds the number of matrix entries.

\ittf{type} is a rank-one allocatable array of type default \character, that
is used to indicate the matrix storage scheme used. Its precise length and
content depends on the type of matrix to be stored.

\ittf{val} is a rank-one allocatable array of type default \realdp\,
and dimension at least {\tt ne}, that holds the values of the entries.
Each pair of off-diagonal entries $h_{ij} = h_{ji}$ of the {\em symmetric}
matrix $\bmH$ is represented as a single entry
(see \S\ref{coordinate}).
Any duplicated entries that appear in the sparse
co-ordinate or row-wise schemes will be summed.

\ittf{row} is a rank-one allocatable array of type default \integer,
and dimension at least {\tt ne}, that may hold the row indices of the entries.
(see \S\ref{coordinate}).

\ittf{col} is a rank-one allocatable array of type default \integer,
and dimension at least {\tt ne}, that may hold the column indices of the entries
(see \S\ref{coordinate}).

\ittf{ptr} is a rank-one allocatable array of type default \integer,
and dimension at least {\tt ne}, that may holds the indices of the
constraints involved when storing $(\bmH_c)_i$
(see \S\ref{coordinate}).
This component is not required when storing $\bmH$ or $\bmA$.

\end{description}

%%%%%%%%%%% problem type %%%%%%%%%%%

\subsubsection{The derived data type for holding the problem}\label{typeprob}
The derived data type {\tt QPT\_problem\_type} is used to hold
the problem. The components of
{\tt QPT\_problem\_type}
are:

\begin{description}

\ittf{n} is a scalar variable of type default \integer,
 that holds the number of optimization variables, $n$.

\ittf{m} is a scalar variable of type default \integer,
 that holds the number of general linear constraints, $m$.

\ittf{H} is scalar variable of type {\tt SMT\_TYPE}
that holds the Hessian matrix $\bmH$, if required,
in the sparse co-ordinate storage scheme (see Section~\ref{coordinate}).
The following components are used:

\begin{description}

\itt{H\%type} is an allocatable array of rank one and type default \character,
that is used to indicate the storage scheme used. Specifically,
the first ten components of {\tt H\%type} will contain the
string {\tt COORDINATE},

\itt{H\%ne} is a scalar variable of type default \integer, that
holds the number of entries in the {\bf lower triangular} part of $\bmH$
in the sparse co-ordinate storage scheme (see Section~\ref{coordinate}).

\itt{H\%val} is a rank-one allocatable array of type default \realdp, that
holds the values of the entries of the {\bf lower triangular} part
of the Hessian matrix $\bmH$ in the sparse co-ordinate storage scheme.

\itt{H\%row} is a rank-one allocatable array of type default \integer,
that holds the row indices of the {\bf lower triangular} part of $\bmH$
in the sparse co-ordinate storage scheme.

\itt{H\%col} is a rank-one allocatable array variable of type default \integer,
that holds the column indices of the {\bf lower triangular} part of the
matrix $(\bmH_c)_i$ in the sparse co-ordinate scheme.

\end{description}

The components of {\tt H} will only be set if the problem has a
nonlinear objective function.

\ittf{G} is a rank-one allocatable array type default \realdp, that
will be allocated to have length {\tt n}, and its $j$-th component
filled with the value $g_{j}$ for $j = 1, \ldots , n$.

\ittf{f} is a scalar variable of type default \realdp, that holds the
constant term, $f$, in the objective function.

\ittf{H\_c} is scalar variable of type {\tt SMT\_TYPE}
that holds the constraint Hessian matrices $(\bmH_c)_i$, if required,
in the sparse co-ordinate storage scheme (see Section~\ref{coordinate}).
The following components are used:

\begin{description}

\itt{H\_c\%type} is an allocatable array of rank one and type default
\character, that is used to indicate the storage scheme used. Specifically,
the first ten components of {\tt H\_c\%type} will contain the
string {\tt COORDINATE},

\itt{H\_c\%ne} is a scalar variable of type default \integer, that
holds the total number of entries in the {\bf lower triangular} part of
the collection of constraint Hessians $(\bmH_c)_i$
in the sparse co-ordinate storage scheme (see Section~\ref{coordinate}).

\itt{H\_c\%val} is a rank-one allocatable array of type default \realdp, that
holds the values of the entries of the {\bf lower triangular} part
of the constraint Hessian matrices $(\bmH_c)_i$
in the sparse co-ordinate storage scheme.

\itt{H\_c\%row} is a rank-one allocatable array of type default \integer,
that holds the row indices of the {\bf lower triangular} part of
$(\bmH_c)_i$ in the sparse co-ordinate storage scheme.

\itt{H\_c\%col} is a rank-one allocatable array variable of type default
\integer, that holds the column indices of the {\bf lower triangular} part of
$(\bmH_c)_i$ in the sparse co-ordinate scheme.

\itt{H\_c\%ptr} is a rank-one allocatable array of variable of type
default \integer, that holds the constraint indices $i$
of the constraint Hessians $(\bmH_c)_i$
in the sparse co-ordinate storage scheme.
\end{description}

The components of {\tt H\_c} will only be set if the problem has a
nonlinear constraints.

\ittf{A} is scalar variable of type {\tt SMT\_TYPE}
that holds the Jacobian matrix $\bmA$, if required,
in the sparse co-ordinate storage scheme (see Section~\ref{coordinate}).
The following components are used:

\begin{description}

\itt{A\%type} is an allocatable array of rank one and type default \character,
that is used to indicate the storage scheme used. Specifically,
the first ten components of {\tt A\%type} will contain the
string {\tt COORDINATE},

\itt{A\%ne} is a scalar variable of type default \integer, that
holds the number of entries in $\bmA$, if any,
in the sparse co-ordinate storage scheme (see Section~\ref{coordinate}).

\itt{A\%val} is a rank-one allocatable array of type default \realdp, that holds
the values of the entries of the Jacobian matrix $\bmA$
in the sparse co-ordinate storage scheme.

\itt{A\%row} is a rank-one allocatable array of type default \integer,
that holds the row indices of $\bmA$ in the sparse co-ordinate storage scheme.

\itt{A\%col} is a rank-one allocatable array variable of type default \integer,
that holds the column indices of $\bmA$ in either the sparse co-ordinate scheme.
\end{description}

The components of {\tt A} will only be set if the problem has general
consraints.

\itt{infinity} is a scalar variable of type default \realdp, that indicates
when a variable or consraint bound is actually infinite. Any component
of {\tt C\_l} or {\tt X\_l} (see below) that is smaller than {\tt -infinity}
should be viewed as $- \infty$, while those of
of {\tt C\_u} or {\tt X\_u} (see below) that are larger than {\tt infinity}
should be viewed as $\infty$,

\ittf{C\_l} is a rank-one allocatable array of dimension {\tt m} and type
default \realdp, that holds the vector of lower bounds $\bmc^{l}$
on the general constraints. The $i$-th component of
{\tt C\_l}, $i = 1, \ldots , m$, contains $\bmc_{i}^{l}$.
Infinite bounds are allowed by setting the corresponding
components of {\tt C\_l} to any value smaller than {\tt -infinity}.

\ittf{C\_u} is a rank-one allocatable array of dimension {\tt m} and type
default \realdp, that holds the vector of upper bounds $\bmc^{u}$
on the general constraints. The $i$-th component of
{\tt C\_u}, $i = 1,  \ldots ,  m$, contains $\bmc_{i}^{u}$.
Infinite bounds are allowed by setting the corresponding
components of {\tt C\_u} to any value larger than {\tt infinity}.

\ittf{X\_l} is a rank-one allocatable array of dimension {\tt n} and type
default \realdp, that holds
the vector of lower bounds $\bmx^{l}$ on the the variables.
The $j$-th component of {\tt X\_l}, $j = 1, \ldots , n$,
contains $\bmx_{j}^{l}$.
Infinite bounds are allowed by setting the corresponding
components of {\tt X\_l} to any value smaller than {\tt -infinity}.

\ittf{X\_u} is a rank-one allocatable array of dimension {\tt n} and type
default \realdp, that holds
the vector of upper bounds $\bmx^{u}$ on the variables.
The $j$-th component of {\tt X\_u}, $j = 1, \ldots , n$,
contains $\bmx_{j}^{u}$.
Infinite bounds are allowed by setting the corresponding
components of {\tt X\_u} to any value larger than that {\tt infinity}.

\ittf{X} is a rank-one allocatable array of dimension {\tt n} and type
default \realdp,
that holds the values $\bmx$ of the optimization variables.
The $j$-th component of {\tt X}, $j = 1,  \ldots , n$, contains $x_{j}$.

\ittf{Y} is a rank-one allocatable array of dimension {\tt m} and type
default \realdp, that holds
the values $\bmy$ of estimates  of the Lagrange multipliers
corresponding to the general linear constraints (see Section~\ref{galmethod}).
The $i$-th component of {\tt Y}, $i = 1,  \ldots ,  m$, contains $y_{i}$.

\ittf{Z} is a rank-one allocatable array of dimension {\tt n} and type default
\realdp, that holds
the values $\bmz$ of estimates  of the dual variables
corresponding to the simple bound constraints (see Section~\ref{galmethod}).
The $j$-th component of {\tt Z}, $j = 1,  \ldots ,  n$, contains $z_{j}$.

\itt{X\_type} is a rank-one allocatable array of dimension {\tt n} and type
default \integer, that defines the types of variables. If {\tt X\_type(i) = 0},
variable $x_i$ is allowed to take continuous values, if {\tt X\_type(i) = 1},
it may only take integer values, and if {\tt X\_type(i) = 2}, it is
restricted to the binary choice, 0 or 1.

\end{description}

%%%%%%%%%%% control type %%%%%%%%%%%

\subsubsection{The derived data type for holding control
 parameters}\label{typecontrol}
The derived data type
{\tt \packagename\_control\_type}
is used to hold controlling data. Default values may be obtained by calling
{\tt \packagename\_initialize}
(see Section~\ref{subinit}). The components of
{\tt \packagename\_control\_type}
are:

\begin{description}
\itt{qplib} is a scalar variable of type default \integer, that holds the
stream number for input QPLIB file.
The default is {\tt qplib = 21}.

\itt{error} is a scalar variable of type default \integer, that holds the
stream number for error messages.
Printing of error messages in
{\tt \packagename\_read\_problem\_data} and {\tt \packagename\_terminate}
is suppressed if ${\tt error} \leq {\tt 0}$.
The default is {\tt error = 6}.

\ittf{out} is a scalar variable of type default \integer, that holds the
stream number for informational messages.
Printing of informational messages in
{\tt \packagename\_read\_problem\_data} is suppressed if ${\tt out} < {\tt 0}$.
The default is {\tt out = 6}.

\itt{print\_level} is a scalar variable of type default \integer,
that is used
to control the amount of informational output which is required. No
informational output will occur if ${\tt print\_level} \leq {\tt 0}$. If
{\tt print\_level = 1} a single line of output will be produced for each
iteration of the process. If {\tt print\_level} $\geq$ {\tt 2} this output
will be increased to provide significant detail of each iteration.
The default is {\tt print\_level = 0}.

\itt{space\_critical} is a scalar variable of type default \logical, that
may be set \true\ if the user wishes the package to allocate as little
internal storage as possible, and \false\ otherwise. The package may
be more efficient if {\tt space\_critical} is set \false.
The default is {\tt space\_critical = \false}.

\itt{deallocate\_error\_fatal} is a scalar variable of type default \logical,
that may be set \true\ if the user wishes the package to return to the user
in the unlikely event that an internal array deallocation fails,
and \false\ if the package should be allowed to try to continue.
The default is {\tt deallocate\_error\_fatal = \false}.

\end{description}

%%%%%%%%%%% inform type %%%%%%%%%%%

\subsubsection{The derived data type for holding informational
 parameters}\label{typeinform}
The derived data type
{\tt \packagename\_inform\_type}
is used to hold parameters that give information about the progress and needs
of the algorithm. The components of
{\tt \packagename\_inform\_type}
are:

\begin{description}
\itt{status} is a scalar variable of type default \integer, that gives the
current status of the algorithm. See Section~\ref{galerrors} for details.

\itt{alloc\_status} is a scalar variable of type default \integer,
that gives the status of the last internal array allocation
or deallocation. This will be 0 if {\tt status = 0}.

\itt{bad\_alloc} is a scalar variable of type default \character\
and length 80, that  gives the name of the last internal array
for which there were allocation or deallocation errors.
This will be the null string if {\tt status = 0}.

\itt{io\_status} is a scalar variable of type default \integer,
that gives the status of the last read attempt.
This will be 0 if {\tt status = 0}.

\itt{line} is a scalar variable of type default \integer,
that gives the number of the last line read from the input file.
This may be used to track an incorrectly-formated file.

\itt{p\_type} is a scalar variable of type default \character\
and length 3 that contains a key that describes the problem.
The first character indicates the type of objective function used.
It will be one of the following:
\begin{description}
   \item {\tt L} a linear objective function.
   \item {\tt D} a convex quadratic objective function whose Hessian is a
           diagonal matrix.
   \item {\tt C} a convex quadratic objective function.
   \item {\tt Q} a quadratic objective function whose Hessian may be indefinite.
\end{description}
The second character indicates the types of variables that are present.
     It will be one of the following:
\begin{description}
    \item {\tt C}  all the variables are continuous.
    \item {\tt B}  all the variables are binary (0-1).
    \item {\tt M}  the variables are a mix of continuous and binary.
    \item {\tt I}  all the variables are integer.
    \item {\tt G}  the variables are a mix of continuous, binary and integer.
\end{description}
The third character indicates the type of the (most extreme)
     constraint function used; other constraints may be of a lesser type.
     It will be one of the following:
\begin{description}
  \item {\tt N}  there are no constraints.
  \item {\tt B}  some of the variables lie between lower and upper bounds
         (box constraint).
  \item {\tt L}  the constraint functions are linear.
  \item {\tt D}  the constraint functions are convex quadratics with diagonal
         Hessians.
  \item {\tt C}  the constraint functions are convex quadratics.
  \item {\tt Q}  the constraint functions are quadratics whose Hessians
         may be indefinite.
\end{description}
     Thus for continuous problems, we would have
\begin{description}
     \item {\tt LCL}              a linear program.
     \item {\tt LCC} or {\tt LCQ} a linear program with quadratic constraints.
     \item {\tt CCB} or {\tt QCB} a bound-constrained quadratic program.
     \item {\tt CCL} or {\tt QCL} a quadratic program.
     \item {\tt CCC} or {\tt CCQ} or {\tt QCC} or {\tt QCQ}
           a quadratic program with quadratic constraints.
\end{description}
For integer problems, the second character would be {\tt I} rather
than {\tt C}, and for mixed integer problems, the second character
would by {\tt M} or {\tt G}.

\end{description}

%%%%%%%%%%%%%%%%%%%%%% argument lists %%%%%%%%%%%%%%%%%%%%%%%%

\galarguments
There are three procedures for user calls
(see Section~\ref{galfeatures} for further features):

\begin{enumerate}
\item The subroutine
      {\tt \packagename\_initialize}
      is used to set default values and initialize private data.
\item The subroutine
      {\tt \packagename\_read\_problem\_data}
      is called to read the prolem from a specified
      QPLIB file into a {\tt QPT\_problem\_type} structure.
\item The subroutine
      {\tt \packagename\_terminate}
      is provided to allow the user to automatically deallocate array
       components of the problem structure set by
       {\tt \packagename\_read\_problem\_data} once
       the input file has been proccessed.
\end{enumerate}

%%%%%% initialization subroutine %%%%%%

\subsubsection{The initialization subroutine}\label{subinit}
 Default values are provided as follows:

\hskip0.5in
{\tt CALL \packagename\_initialize( control, inform )}

\begin{description}

\itt{control} is a scalar \intentout argument of type
{\tt \packagename\_control\_type}
(see Section~\ref{typecontrol}).
On exit, {\tt control} contains default values for the components as
described in Section~\ref{typecontrol}.
These values should only be changed after calling
{\tt \packagename\_initialize}.

\itt{inform} is a scalar \intentout\ argument of type
{\tt \packagename\_inform\_type}
(see Section~\ref{typeinform}). A successful call to
{\tt \packagename\_initialize}
is indicated when the  component {\tt status} has the value 0.
For other return values of {\tt status}, see Section~\ref{galerrors}.

\end{description}

%%%%%%%%% problem solution subroutine %%%%%%

\subsubsection{Subroutine to extract the data from a QPLIB format file}
Extract the data from a QPLIB format file as follows:

\hskip0.5in
{\tt CALL \packagename\_read\_problem\_data( problem, control, inform )}

\begin{description}

\itt{problem} is a scalar \intentinout\ argument of type
{\tt qpt\_problem\_type} (see Section~\ref{typeprob}) whose
components will be filled with problem data extracted from the
QPLIB file.

\itt{control} is a scalar \intentin\ argument of type
{\tt \packagename\_control\_type}
(see Section~\ref{typecontrol}).
Default values may be assigned by calling {\tt \packagename\_initialize}
prior to the first call to {\tt \packagename\_read\_problem\_data}.
Of particular note, the component {\tt control\%qplib} specifies
the stream number for input QPLIB file.

\itt{inform} is a scalar \intentinout argument of type
{\tt \packagename\_inform\_type}
(see Section~\ref{typeinform}) whose components need not be set on entry.
A successful call to
{\tt \packagename\_read\_problem\_data}
is indicated when the  component {\tt status} has the value 0.
For other return values of {\tt status}, see Section~\ref{galerrors}.

\end{description}

%%%%%%% termination subroutine %%%%%%

\subsubsection{The  termination subroutine}
All previously allocated arrays are deallocated as follows:

\hskip0.5in
{\tt CALL \packagename\_terminate( data, control, inform )}

\begin{description}

\itt{data} is a scalar \intentinout argument of type
{\tt \packagename\_data\_type}
exactly as for
{\tt \packagename\_read\_problem\_data}
that must not have been altered {\bf by the user} since the last call to
{\tt \packagename\_initialize}.
On exit, array components will have been deallocated.

\itt{control} is a scalar \intentin argument of type
{\tt \packagename\_control\_type}
exactly as for
{\tt \packagename\_read\_problem\_data}.

\itt{inform} is a scalar \intentout argument of type
{\tt \packagename\_inform\_type}
exactly as for
{\tt \packagename\_read\_problem\_data}.
Only the component {\tt status} will be set on exit, and a
successful call to
{\tt \packagename\_terminate}
is indicated when this  component {\tt status} has the value 0.
For other return values of {\tt status}, see Section~\ref{galerrors}.

\end{description}

%%%%%%%%%%%%%%%%%%%%%% Warning and error messages %%%%%%%%%%%%%%%%%%%%%%%%

\galerrors
A negative value of  {\tt inform\%status} on exit from
{\tt \packagename\_read\_problem\_data}
or
{\tt \packagename\_terminate}
indicates that an error might have occurred. No further calls should be made
until the error has been corrected. Possible values are:

\begin{description}
\itt{\galerrallocate.} An allocation error occurred. A message indicating
the offending
array is written on unit {\tt control\%error}, and the returned allocation
status and a string containing the name of the offending array
are held in {\tt inform\%alloc\_\-status}
and {\tt inform\%bad\_alloc}, respectively.

\itt{\galerrdeallocate.} A deallocation error occurred.
A message indicating the offending
array is written on unit {\tt control\%error} and the returned allocation
status and a string containing the name of the offending array
are held in {\tt inform\%alloc\_\-status}
and {\tt inform\%bad\_alloc}, respectively.

\itt{\galerrio.}
An input/output error occurred.

\itt{\galerrinput.}
The end of the input file was encountered before the problem specification
was complete.

\itt{\galerrunavailable.}
The problem type was not recognised.

\end{description}

%%%%%%%%%%%%%%%%%%%%%% Further features %%%%%%%%%%%%%%%%%%%%%%%%

\galfeatures
\noindent In this section, we describe an alternative means of setting
control parameters, that is components of the variable {\tt control} of type
{\tt \packagename\_control\_type}
(see Section~\ref{typecontrol}),
by reading an appropriate data specification file using the
subroutine {\tt \packagename\_read\_specfile}. This facility
is useful as it allows a user to change  {\tt \packagename} control parameters
without editing and recompiling programs that call {\tt \packagename}.

A specification file, or specfile, is a data file containing a number of
"specification commands". Each command occurs on a separate line,
and comprises a "keyword",
which is a string (in a close-to-natural language) used to identify a
control parameter, and
an (optional) "value", which defines the value to be assigned to the given
control parameter. All keywords and values are case insensitive,
keywords may be preceded by one or more blanks but
values must not contain blanks, and
each value must be separated from its keyword by at least one blank.
Values must not contain more than 30 characters, and
each line of the specfile is limited to 80 characters,
including the blanks separating keyword and value.

The portion of the specification file used by
{\tt \packagename\_read\_specfile}
must start
with a "{\tt BEGIN \packagename}" command and end with an
"{\tt END}" command.  The syntax of the specfile is thus defined as follows:
\begin{verbatim}
  ( .. lines ignored by RPD_read_specfile .. )
    BEGIN RPD
       keyword    value
       .......    .....
       keyword    value
    END
  ( .. lines ignored by RPD_read_specfile .. )
\end{verbatim}
where keyword and value are two strings separated by (at least) one blank.
The ``{\tt BEGIN \packagename}'' and ``{\tt END}'' delimiter command lines
may contain additional (trailing) strings so long as such strings are
separated by one or more blanks, so that lines such as
\begin{verbatim}
    BEGIN RPD SPECIFICATION
\end{verbatim}
and
\begin{verbatim}
    END RPD SPECIFICATION
\end{verbatim}
are acceptable. Furthermore,
between the
``{\tt BEGIN \packagename}'' and ``{\tt END}'' delimiters,
specification commands may occur in any order.  Blank lines and
lines whose first non-blank character is {\tt !} or {\tt *} are ignored.
The content
of a line after a {\tt !} or {\tt *} character is also
ignored (as is the {\tt !} or {\tt *}
character itself). This provides an easy manner to "comment out" some
specification commands, or to comment specific values
of certain control parameters.

The value of a control parameters may be of three different types, namely
integer, logical or real.
Integer and real values may be expressed in any relevant Fortran integer and
floating-point formats (respectively). Permitted values for logical
parameters are "{\tt ON}", "{\tt TRUE}", "{\tt .TRUE.}", "{\tt T}",
"{\tt YES}", "{\tt Y}", or "{\tt OFF}", "{\tt NO}",
"{\tt N}", "{\tt FALSE}", "{\tt .FALSE.}" and "{\tt F}".
Empty values are also allowed for
logical control parameters, and are interpreted as "{\tt TRUE}".

The specification file must be open for
input when {\tt \packagename\_read\_specfile}
is called, and the associated device number
passed to the routine in device (see below).
Note that the corresponding
file is {\tt REWIND}ed, which makes it possible to combine the specifications
for more than one program/routine.  For the same reason, the file is not
closed by {\tt \packagename\_read\_specfile}.

Control parameters corresponding to the components
{\tt SLS\_control}
and
{\tt IR\_control} may be changed by including additional sections enclosed by
``{\tt BEGIN SLS}'' and
``{\tt END SLS}'', and
``{\tt BEGIN IR}'' and
``{\tt END IR}'', respectively.
See the specification sheets for the packages
{\tt \libraryname\_SLS}
and
{\tt \libraryname\_IR}
for further details.

\subsubsection{To read control parameters from a specification file}
\label{readspec}

Control parameters may be read from a file as follows:
\hskip0.5in
\def\baselinestretch{0.8} {\tt \begin{verbatim}
     CALL RPD_read_specfile( control, device )
\end{verbatim}}
\def\baselinestretch{1.0}

\begin{description}
\itt{control} is a scalar \intentinout argument of type
{\tt \packagename\_control\_type}
(see Section~\ref{typecontrol}).
Default values should have already been set, perhaps by calling
{\tt \packagename\_initialize}.
On exit, individual components of {\tt control} may have been changed
according to the commands found in the specfile. Specfile commands and
the component (see Section~\ref{typecontrol}) of {\tt control}
that each affects are given in Table~\ref{specfile}.

\bctable{|l|l|l|}
\hline
  command & component of {\tt control} & value type \\
\hline
  {\tt qplib-file-device} & {\tt \%qplib} & integer \\
  {\tt error-printout-device} & {\tt \%error} & integer \\
  {\tt printout-device} & {\tt \%out} & integer \\
  {\tt print-level} & {\tt \%print\_level} & integer \\
  {\tt space-critical} & {\tt \%space\_critical} & logical \\
  {\tt deallocate-error-fatal} & {\tt \%deallocate\_error\_fatal} & logical \\
\hline

\ectable{\label{specfile}Specfile commands and associated
components of {\tt control}.}

\itt{device} is a scalar \intentin argument of type default \integer,
that must be set to the unit number on which the specfile
has been opened. If {\tt device} is not open, {\tt control} will
not be altered and execution will continue, but an error message
will be printed on unit {\tt control\%error}.

\end{description}

%%%%%%%%%%%%%%%%%%%%%% Information printed %%%%%%%%%%%%%%%%%%%%%%%%

\galinfo
If {\tt control\%print\_level} is positive, information about the progress
of the algorithm may be printed on unit {\tt control\-\%\-out}.

%%%%%%%%%%%%%%%%%%%%%% GENERAL INFORMATION %%%%%%%%%%%%%%%%%%%%%%%%

\galgeneral

\galcommon None.
\galworkspace Provided automatically by the module.
\galroutines None.
\galmodules {\tt \packagename\_read\_problem\_data}
calls the \galahad\ packages
{\tt \libraryname\_CLOCK},
{\tt \libraryname\_SY\-M\-BOLS},
{\tt \libraryname\_\-SPACE},
{\tt \libraryname\-\_SMT},
{\tt \libraryname\-\_QPT},
{\tt \libraryname\_SPECFILE},
{\tt \libraryname\_SORT}
and
{\tt \libraryname\_LMS}.
\galio Output is under control of the arguments
{\tt control\%error}, {\tt control\%out} and {\tt control\%print\_level}.
\galportability ISO Fortran~2003.
The package is thread-safe.

%%%%%%%%%%%%%%%%%%%%%% METHOD %%%%%%%%%%%%%%%%%%%%%%%%

\galmethod

\noindent
The QPBLIB format is defined in
\vspace*{1mm}

\noindent
F. Furini, E. Traversi, P. Belotti, A. Frangioni, A. Gleixner, N. Gould,
L. Liberti, A. Lodi, R. Misener, H. Mittelmann, N. V. Sahinidis,
S. Vigerske and A. Wiegele  (2019).
QPLIB: a library of quadratic programming instances,
Mathematical Programming Computation {\bf 11} 237–265.

%%%%%%%%%%%%%%%%%%%%%% EXAMPLE %%%%%%%%%%%%%%%%%%%%%%%%

\galexample
Suppose we wish to read the data encoded in the QPLIB file
{\tt ALLINIT.qplib} that may be found in the directory {\tt examples} of
the \libraryname\ distribution.
Then we may use the following code:

{\tt \small
\VerbatimInput{\packageexample}
}
\noindent
This produces the following output:
{\tt \small
\VerbatimInput{\packageresults}
}

\end{document}
