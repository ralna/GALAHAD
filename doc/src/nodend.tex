\documentclass{galahad}

% set the package name

\newcommand{\package}{nodend}
\newcommand{\packagename}{NODEND}
\newcommand{\fullpackagename}{\libraryname\_\-\packagename}

\begin{document}

\galheader

%%%%%%%%%%%%%%%%%%%%%% SUMMARY %%%%%%%%%%%%%%%%%%%%%%%%

\galsummary

This package
{\bf finds a symmetric row and column permutation $\bmP \bmA \bmP^T$ 
of a symmetric, sparse matrix $\bmA$ with the aim of limiting 
the fill-in during subsequent Cholesky-like factorization}. 
The package is actually a wrapper to the {\tt METIS\_NodeND} 
procedure from versions 4.0, 5.1 and 5.2 of the
{\tt METIS} package from the Karypis Lab; Versions 5 are freely 
available under an open-source licence, and included here, 
while Version 4 requires a more restrictive licence, and a separate download, 
see {\tt https://github.com/KarypisLab}; if Version 4 is not provided,
a dummy will be substituted.

%%%%%%%%%%%%%%%%%%%%%% attributes %%%%%%%%%%%%%%%%%%%%%%%%

\galattributes
\galversions{\tt  \fullpackagename\_single, \fullpackagename\_double}.
\galcalls
{\tt GALAHAD\-\_KINDS},
{\tt GALA\-HAD\_CLOCK},
{\tt GAL\-AHAD\-\_SYMBOLS},
{\tt GALAHAD\_SMT},
{\tt GALAHAD\_\-SORT}
and 
{\tt GALAHAD\_SPECFILE},
\galdate March 2025.
\galorigin N. I. M. Gould, Rutherford Appleton Laboratory.
\gallanguage Fortran~95 + TR 15581 or Fortran~2003.

%%%%%%%%%%%%%%%%%%%%%% HOW TO USE %%%%%%%%%%%%%%%%%%%%%%%%

\galhowto

%\subsection{Calling sequences}

%The package is available with either 32-bit or 64-bit integers, 
%and the subsidiary {\tt SMT\_type} package may use
%single, double and (if available) quadruple precision reals.
%Access to the 32-bit integer,
%single precision version requires the {\tt USE} statement
%\medskip

%\hspace{8mm} {\tt USE \fullpackagename\_single}

%\medskip
%\noindent
%with the obvious substitution 
%{\tt \fullpackagename\_double},
%{\tt \fullpackagename\_quadruple},
%{\tt \fullpackagename\_single\_64},
%{\tt \fullpackagename\_double\_64} and
%{\tt \fullpackagename\_quadruple\_64} for the other variants.

The package is available with single, double and (if available) 
quadruple precision reals, and either 32-bit or 64-bit integers. 
Access to the 32-bit integer,
single precision version requires the {\tt USE} statement
\medskip

\hspace{8mm} {\tt USE \fullpackagename\_single}

\medskip
\noindent
with the obvious substitution 
{\tt \fullpackagename\_double},
{\tt \fullpackagename\_quadruple},
{\tt \fullpackagename\_single\_64},
{\tt \fullpackagename\_double\_64} and
{\tt \fullpackagename\_quadruple\_64} for the other variants.


\noindent
If it is required to use more than one of the modules at the same time, 
the derived types
{\tt SMT\_type},
{\tt \packagename\_control\-\_type}
and
{\tt \packagename\_inform\_type}
(\S\ref{galtypes}),
and the subroutines
{\tt \packagename\-\_initialize},
{\tt \packagename\-\_order}
and
{\tt \packagename\-\_order\-\_adjacency},
(\S\ref{galarguments})
must be renamed on one of the {\tt USE} statements.

\noindent There are three principal subroutines for user calls.

\begin{description}
\item[]{\tt \packagename\_initialize} is used to set or re-initialize 
 default control and information values.

\item[] {\tt \packagename\_order} takes the (symmetric) pattern of $\bmA$
 and finds a symmetric permutation $\bmP$ so that the fill-in
 during Cholesky-like factorizations of $\bmP \bmA \bmP^T$
 is kept small.

\item[] {\tt \packagename\_order\_adjacency} takes the adjacency graph of 
 the (whole) pattern of $\bmA$, and performs the same task as
 {\tt \packagename\_order}. This package is actually called
 by its predecessor, but is provided for use by experts, and for
 those whose application is naturally in adjacency form.
\end{description}

%%%%%%%%%%%%%%%%%%%%%% matrix formats %%%%%%%%%%%%%%%%%%%%%%%%

\galmatrix
The sparsity pattern of the matrix $\bmA$ may be stored in 
a variety of input formats.

\subsubsection{Sparse co-ordinate storage format}\label{coordinate}
Only the nonzero entries of the lower-triangular part of $\bmA$ are stored.
For the $l$-th entry of the lower-triangular portion of $\bmA$,
its row index $i$ and column index $j$
are stored in the $l$-th components of the integer arrays 
{\tt row} and {\tt col}, respectively.
The order is unimportant, but the total number of entries
{\tt ne} is also required.

\subsubsection{Sparse row-wise storage format}\label{rowwise}
Again only the nonzero entries of the lower-triangular part are stored,
but this time they are ordered so that those in row $i$ appear directly
before those in row $i+1$. For the $i$-th row of $\bmA$, the $i$-th component
of an integer array {\tt ptr} holds the position of the first entry in this row,
while {\tt ptr} $(m+1)$ holds the total number of entries plus one.
The column indices $j$ of the entries in the $i$-th row are 
stored in components $l =$ {\tt ptr}$(i)$, \ldots ,{\tt ptr} $(i+1)-1$ 
of the integer array {\tt col}.

For sparse matrices, this scheme almost always requires less storage than
its predecessor.

\subsubsection{Dense storage format}\label{dense}
The matrix $\bmA$ is stored as a compact
dense matrix by rows, that is, the values of the entries of each row in turn are
stored in order within an appropriate real one-dimensional array.
Since no indexing information is needed, no integer arrays are required.
Indeed, there no point in reordering a dense matrix, and this option
is simply included for completeness.

%%%%%%%%%%%%%%%%%%%%%% graph format %%%%%%%%%%%%%%%%%%%%%%%%

\galgraph
The sparsity pattern of $\bmA$ may also be stored as an adjacency graph.
For each column of $\bmA$, a list of indices of rows of the whole of 
$\bmA$ (that is, both triangles) that correspond to nonzero entries are 
recorded; by convention for column $j$, if row $j$ occurs, it is omitted. 
Two integer arrays {\tt IND} and {\tt PTR}
are used, and the row indices of column $j$ are stored as
{\tt IND(l)}, {\tt l} $=$ {\tt PTR}$(j)$, \ldots ,{\tt PTR} $(j+1)-1$.
for $j = 1,\ldots,n$.

%%%%%%%%%%%%%%%%%%%%%%%%%%% kinds %%%%%%%%%%%%%%%%%%%%%%%%%%%

%%%%%%%%%%%%%%%%%%%%%% kinds %%%%%%%%%%%%%%%%%%%%%%

\galkinds
We use the terms integer and real to refer to the fortran
keywords {\tt REAL(rp\_)} and {\tt INTEGER(ip\_)}, where 
{\tt rp\_} and {\tt ip\_} are the relevant kind values for the
real and integer types employed by the particular module in use. 
The former are equivalent to
default {\tt REAL} for the single precision versions,
{\tt DOUBLE PRECISION} for the double precision cases
and quadruple-precision if 128-bit reals are available, and 
correspond to {\tt rp\_ = real32}, {\tt rp\_ = real64} and
{\tt rp\_ = real128} respectively as defined by the 
fortran {\tt iso\_fortran\_env} module.
The latter are default (32-bit) and long (64-bit) integers, and
correspond to {\tt ip\_ = int32} and {\tt ip\_ = int64},
respectively, again from the {\tt iso\_fortran\_env} module.


%%%%%%%%%%%%%%%%%%%%%% derived types %%%%%%%%%%%%%%%%%%%%%%%%

\galtypes
Four derived data types are used by the package.

%%%%%%%%%%% problem type %%%%%%%%%%%

\subsubsection{The derived data type for holding the matrix}\label{typeprob}
The derived data type {\tt SMT\_type} is used to hold the matrix $\bmA$.
The components of {\tt SMT\_type} used are:

\begin{description}

\ittf{n} is a scalar variable of type \integer, that holds
the order $n$ of the matrix  $\bmA$.
\restriction {\tt n} $\geq$ {\tt 1}.

\itt{type} is an allocatable array of rank one and type default \character, that
indicates the storage scheme used. If the
sparse co-ordinate scheme (see \S\ref{coordinate}) is used
the first ten components of {\tt type} must contain the
string {\tt COORDINATE}.
For the sparse row-wise storage scheme (see \S\ref{rowwise}),
the first fourteen components of {\tt type} must contain the
string {\tt SPARSE\_BY\_ROWS}, and
for dense storage scheme (see \S\ref{dense})
the first five components of {\tt type} must contain the
string {\tt DENSE}.
%and for the diagonal storage scheme (see \S\ref{diagonal}),
%the first eight components of {\tt type} must contain the
%string {\tt DIAGONAL}.

For convenience, the procedure {\tt SMT\_put}
may be used to allocate sufficient space and insert the required keyword
into {\tt type}.
For example, if $\bmA$ is to be stored in the structure {\tt A}
of derived type {\tt SMT\_type} and we wish to use
the co-ordinate scheme, we may simply
%\vspace*{-2mm}
{\tt
\begin{verbatim}
        CALL SMT_put( A%type, 'COORDINATE', istat )
\end{verbatim}
}
%\vspace*{-4mm}
\noindent
See the documentation for the \galahad\ package {\tt SMT}
for further details on the use of {\tt SMT\_put}.

\itt{ne} is a scalar variable of type \integer, that
holds the number of entries in the {\bf lower triangular} part of $\bmA$
in the sparse co-ordinate storage scheme (see \S\ref{coordinate}).
It need not be set for any of the other schemes.

\itt{row} is a rank-one allocatable array of type \integer,
that holds the row indices of the {\bf lower triangular} part of $\bmA$
in the sparse co-ordinate storage
scheme (see \S\ref{coordinate}).
It need not be allocated for any of the other schemes.
Any entry whose row index lies out of the range $[$1,n$]$ will be ignored.

\itt{col} is a rank-one allocatable array variable of type \integer,
that holds the column indices of the {\bf lower triangular} part of
$\bmA$ in either the sparse co-ordinate
(see \S\ref{coordinate}), or the sparse row-wise
(see \S\ref{rowwise}) storage scheme.
It need not be allocated when the dense
%or diagonal storage schemes are used.
storage scheme is used.
Any entry whose column index lies out of the range $[$1,n$]$ will be ignored,
while the row and column indices of any entry from the
{\bf strict upper triangle} will implicitly be swapped.

\itt{ptr} is a rank-one allocatable array of size {\tt n+1} and type
\integer, that holds the starting position of
each row of the {\bf lower triangular} part of $\bmA$, as well
as the total number of entries plus one, in the sparse row-wise storage
scheme (see \S\ref{rowwise}). It need not be allocated for the
other schemes.

\end{description}
The derived type also has a {\tt val} component that may hold real values
that are not used here, but can be by other applications that share
a {\tt SMT\_type} variable.

%%%%%%%%%%% control type %%%%%%%%%%%

\subsubsection{The derived data type for holding control
 parameters}\label{typecontrol}
The derived data type
{\tt \packagename\_control\_type}
is used to hold controlling data. Values specifically for the desired solver
may be changed at run time by calling
{\tt \packagename\_read\-\_specfile}
(see \S\ref{readspec}).
The components of
{\tt \packagename\_control\_type}
are:

\begin{description}

\itt{version} is a scalar variable of type default \character\
and length 30, that specifies the desired version of METIS. Possible
values are '4.0', '5.1' and '5.2'.
The default is {\tt version = '5.2'}.

\itt{error} is a scalar variable of type \integer, that holds the
unit number for error messages.
Printing of error messages is suppressed if ${\tt error} < {\tt 0}$.
The default is {\tt error = 6}.

\ittf{out} is a scalar variable of type \integer, that holds the
unit number for informational messages.
Printing of informational messages is suppressed if ${\tt out} < {\tt 0}$.
The default is {\tt out = 6}.

\itt{print\_level} is a scalar variable of type \integer,
that is used
to control the amount of informational output that is required. No
informational output will occur if ${\tt print\_level} \leq {\tt 0}$. If
{\tt print\_level} $\geq$ {\tt 1} details of the ordering process will 
be produced. The default is {\tt print\_level = 0}.

\itt{metis4\_ptype} is a scalar variable of type \integer, that specifies
the partitioning method employed. 0 = multilevel recursive bisectioning:
 1 = multilevel k-way partitioning
The default is {\tt metis4\_ptype = 0}, and any invalid value will be
replaced by this default.

\itt{metis4\_ctype} is a scalar variable of type \integer, that specifies
the matching scheme to be used during coarsening: {\tt 1} = random matching, 
 {\tt 2} = heavy-edge matching, {\tt 3} = sorted heavy-edge matching, and
 {\tt 4} = k-way sorted heavy-edge matching.
The default is {\tt metis4\_ctype = 3}, and any invalid value will be
replaced by this default.

\itt{metis4\_itype} is a scalar variable of type \integer, that specifies
the algorithm used during initial partitioning: 
 {\tt 1} = edge-based region growing and {\tt 2} = node-based region growing.
The default is {\tt metis4\_itype = 1}, and any invalid value will be
replaced by this default.

\itt{metis4\_rtype} is a scalar variable of type \integer, that specifies
the algorithm used for refinement: 
 {\tt 1} = two-sided node Fiduccia-Mattheyses (FM) refinement, and
 {\tt 2} = one-sided node FM refinement.
The default is {\tt metis4\_rtype = 1}, and any invalid value will be
replaced by this default.

\itt{metis4\_dbglvl} is a scalar variable of type \integer, that specifies
the amount of progress/debugging information printed: 
{\tt 0} = nothing, {\tt 1} = timings, and $>$ {\tt 1} increasingly more.
The default is {\tt metis4\_dbglvl = 0}, and any invalid value will be
replaced by this default.

\itt{metis4\_oflags} is a scalar variable of type \integer, that specifies
select whether or not to compress the graph, and to order connected 
 components separately: {\tt 0} = do neither, 
 {\tt 1} = try to compress the graph, 
 {\tt 2} = order each connected component separately, and {\tt 3} = do both.
The default is {\tt metis4\_oflags = 1}, and any invalid value will be
replaced by this default.

\itt{metis4\_pfactor } is a scalar variable of type \integer, that specifies
the minimum degree of the vertices that will be ordered last. 
More specifically, any vertices with a degree greater than 
0.1 {\tt metis4\_pfactor} times the average degree are removed from
the graph, an ordering of the rest of the vertices is computed, and an
overall ordering is computed by ordering the removed vertices at the end
of the overall ordering. Any value
smaller than 1 means that no vertices will be ordered last.
The default is {\tt metis4\_pfactor =-1 }.

\itt{metis4\_nseps} is a scalar variable of type \integer, that specifies
the number of different separators that the algorithm will compute
at each level of nested dissection.
The default is {\tt metis4\_nseps = 1}, and any smaller value
will be replaced by this default.

\itt{metis5\_ptype} is a scalar variable of type \integer, that specifies
the partitioning method. The value {\tt 0} gives multilevel recursive 
bisectioning, while {\tt 1} corresponds to multilevel $k$-way partitioning.
The default is {\tt metis5\-\_ptype = 0}, and any invalid value will be
replaced by this default.

\itt{metis5\_objtype} is a scalar variable of type \integer, that specifies
the type of the objective. Currently the only and default value
{\tt metis5\_objtype = 2}, specifies node-based nested dissection, 
and any invalid value will be replaced by this default.

\itt{metis5\_ctype} is a scalar variable of type \integer, that specifies
the matching scheme to be used during coarsening: {\tt 0} = random matching, 
 and {\tt 1} = sorted heavy-edge matching.
The default is {\tt metis5\_ctype = 1}, and any invalid value will be
replaced by this default.

\itt{metis5\_iptype} is a scalar variable of type \integer, that specifies
the algorithm used during initial partitioning:
 {\tt 2} = derive separators from edge cuts, and
 {\tt 3} = grow bisections using a greedy node-based strategy.
The default is {\tt metis5\_iptype = 2}, and any invalid value will be
replaced by this default.

\itt{metis5\_rtype} is a scalar variable of type \integer, that specifies
the algorithm used for refinement: {\tt 2} = Two-sided node FM refinement,
 and {\tt 3} = One-sided node FM refinement.
The default is {\tt metis5\_rtype = 2}, and any invalid value will be
replaced by this default.

\itt{metis5\_dbglvl} is a scalar variable of type \integer, that specifies
the amount of progress/debugging information printed: {\tt 0} = nothing, 
 {\tt 1} = diagnostics, {\tt 2} = plus timings, and $>$ 2 plus more.
The default is {\tt metis5\_dbglvl = 0}, and any invalid value will be
replaced by this default.

\itt{metis5\_niparts} is a scalar variable of type \integer, that specifies
the number of initial partitions used by MeTiS 5.2.
The default is {\tt metis5\_niparts = -1}, and any invalid value will be
replaced by this default.

\itt{metis5\_niter} is a scalar variable of type \integer, that specifies
the number of iterations used by the refinement algorithm.
The default is {\tt metis5\_niter = 10}, and any non-positive value will be
replaced by this default.

\itt{metis5\_ncuts} is a scalar variable of type \integer, that specifies
the number of different partitionings that it will compute: {\tt -1} = not used.
The default is {\tt metis5\_ncuts = -1}, and any invalid value will be
replaced by this default.

\itt{metis5\_seed} is a scalar variable of type \integer, that specifies
the seed for the random number generator.
The default is {\tt metis5\_seed = -1}.

\itt{metis5\_ondisk} is a scalar variable of type \integer, that specifies
whether on-disk storage is used ({\tt 0} = no, {\tt 1} = yes) by MeTiS 5.2.
The default is {\tt metis5\_ondisk = 0}, and any invalid value will be
replaced by this default.

\itt{metis5\_minconn} is a scalar variable of type \integer, that specifies
specify that the partitioning routines should try to minimize the maximum 
degree of the subdomain graph: {\tt 0} = no, {\tt 1} = yes, and 
{\tt -1} = not used. 
The default is {\tt metis5\_minconn =-1  }, and any invalid value will be
replaced by this default.

\itt{metis5\_contig} is a scalar variable of type \integer, that specifies
specify that the partitioning routines should try to produce partitions 
that are contiguous: {\tt 0} = no, {\tt 1} = yes, and {\tt -1} = not used.
The default is {\tt metis5\_contig = 1}, and any invalid value will be
replaced by this default.

\itt{metis5\_compress} is a scalar variable of type \integer, that specifies
specify that the graph should be compressed by combining together vertices 
that have identical adjacency lists: {\tt 0} = no, and {\tt 1} = yes.
The default is {\tt metis5\_compress = 1}, and any invalid value will be
replaced by this default.

\itt{metis5\_ccorder} is a scalar variable of type \integer, that specifies
specify if the connected components of the graph should first be
identified and ordered separately: {\tt 0} = no, and {\tt 1} = yes.
The default is {\tt metis5\_ccorder = 0}, and any invalid value will be
replaced by this default.

\itt{metis5\_pfactor} is a scalar variable of type \integer, that specifies
the minimum degree of the vertices that will be ordered last.
More specifically, any vertices with a degree greater than 
0.1 {\tt metis4\_pfactor} times the average degree are removed from
the graph, an ordering of the rest of the vertices is computed, and an
overall ordering is computed by ordering the removed vertices at the end
of the overall ordering. The default is {\tt metis5\_pfactor = 0}, and any 
negative value will be replaced by this default.

\itt{metis5\_nseps} is a scalar variable of type \integer, that specifies
the number of different separators that the algorithm will compute
at each level of nested dissection.
The default is {\tt metis5\_nseps = 1}, and any non-positive value will be
replaced by this default.

\itt{metis5\_ufactor} is a scalar variable of type \integer, that specifies
the maximum allowed load imbalance {\tt (1 +  metis5\_ufactor)/1000} 
among the partitions.
The default is {\tt metis5\_ufactor = 200}, and any negative value will be
replaced by this default.

\itt{metis5\_dropedges} is a scalar variable of type \integer, that specifies
whether edges will be dropped ({\tt 0} = no, {\tt 1} = yes) by MeTiS 5.2.
The default is {\tt metis5\_dropedges = 0}, and any invalid value will be
replaced by this default.

\itt{metis5\_no2hop} is a scalar variable of type \integer, that specifies
specify that the coarsening will not perform any 2–hop matchings when 
the standard matching approach fails to sufficiently coarsen the graph:
{\tt 0} = no, and {\tt 1} = yes.
The default is {\tt metis5\_no2hop = 0}, and any invalid value will be
replaced by this default.

\itt{metis5\_twohop} is a scalar variable of type \integer, 
that is reserved for future use but ignored at present.
The default is {\tt metis5\_twohop = -1}.

\itt{metis5\_fast} is a scalar variable of type \integer, 
that is reserved for future use but ignored at present.
The default is {\tt metis5\_fast = -1}.
replaced by this default.

\itt{no\_metis\_4\_use\_5\_instead} is a scalar variable of type \logical, 
that specifies whether to use {\tt METIS} version 5 (specifically, 5.2) 
if {\tt METIS} is unavailable.
The default is {\tt no\_metis\_4\_use\_5\_instead = \true}.

\itt{prefix} is a scalar variable of type default \character\
and length 30, that may be used to provide a user-selected
character string to preface every line of printed output.
Specifically, each line of output will be prefaced by the string
{\tt prefix(2:LEN(TRIM(prefix))-1)},
thus ignoring the first and last non-null components of the
supplied string. If the user does not want to preface lines by such
a string, the default {\tt prefix = ""} should be used.

\end{description}

%%%%%%%%%%% time type %%%%%%%%%%%

\subsubsection{The derived data type for holding timing
 information}\label{typetime}
The derived data type
{\tt \packagename\_time\_type}
is used to hold elapsed CPU and system clock times for the various parts of
the calculation. The components of
{\tt \packagename\_time\_type}
are:
\begin{description}
\itt{total} is a scalar variable of type \realdp, that gives
 the total CPU time spent in the package.

\itt{metis} is a scalar variable of type \realdp, that gives
 the CPU time spent in the {\tt METIS} package.

\itt{clock\_total} is a scalar variable of type \realdp, that gives
 the total elapsed system clock time spent in the package.

\itt{clock\_metis} is a scalar variable of type \realdp, that gives
 the elapsed system clock time spent in the {\tt METIS} package.

\end{description}


%%%%%%%%%%% inform type %%%%%%%%%%%

\subsubsection{The derived data type for holding informational
 parameters}\label{typeinform}
The derived data type
{\tt \packagename\_inform\_type}
is used to hold parameters that give information about the progress and needs
of the algorithm. The components of
{\tt \packagename\_inform\_type}
are as follows---any component that is not relevant to the solver being used
will have the value {\tt -1} or {\tt -1.0} as appropriate:

\begin{description}

\itt{status} is a scalar variable of type \integer, that gives the
exit status of the algorithm.
%See Sections~\ref{galerrors} and \ref{galinfo}
See \S\ref{galerrors}
for details.

\itt{alloc\_status} is a scalar variable of type \integer, that gives
the status of the last attempted array allocation or deallocation.

\itt{bad\_alloc} is a scalar variable of type default \character\
and length 80, that  gives the name of the last internal array
for which there were allocation or deallocation errors.
This will be the null string if there have been no
allocation or deallocation errors.

\itt{version} is a scalar variable of type default \character\
and length 3, that contains the actual version of {\tt METIS} used.

\ittf{time} is a scalar variable of type {\tt \packagename\_time\_type}
whose components are used to hold elapsed CPU and system clock times for the
various parts of the calculation (see Section~\ref{typetime}).

\end{description}

%%%%%%%%%%% data type %%%%%%%%%%%

\subsubsection{The derived data type for holding problem data}\label{typedata}
The derived data type
{\tt \packagename\_data\_type}
is used to hold all the data for a particular problem,
or sequences of problems with the same structure, between calls to
{\tt \packagename} procedures.
%This data should be preserved, untouched, from the initial call to
%{\tt \packagename\_initialize}
%to the final call to
%{\tt \packagename\_terminate}.
All components are private.

%%%%%%%%%%%%%%%%%%%%%% argument lists %%%%%%%%%%%%%%%%%%%%%%%%

\galarguments

%%%%%% initialization subroutine %%%%%%

\subsubsection{The initialization subroutine}\label{subinit}
 Default values are provided as follows:
\vspace*{1mm}

\hspace{8mm}
{\tt CALL \packagename\_initialize( control, inform )}

\vspace*{-3mm}
\begin{description}

\itt{control} is a scalar \intentout\ argument of type
{\tt \packagename\_control\_type}
(see Section~\ref{typecontrol}).
On exit, {\tt control} contains default values for the components as
described in Section~\ref{typecontrol}.
These values should only be changed after calling
{\tt \packagename\_initialize}.

\itt{inform} is a scalar \intentout\ argument of type
{\tt \packagename\_inform\_type}
(see Section~\ref{typeinform}). A successful call to
{\tt \packagename\_initialize}
is indicated when the  component {\tt status} has the value 0.
For other return values of {\tt status}, see Section~\ref{galerrors}.

\end{description}

%%%%%% order subroutine %%%%%%

\subsubsection{The basic ordering subroutine}\label{suborder}
A nested-dissection-based ordering of the sparsity pattern of $\bmA$ 
may be obtained as follows:
\vskip2mm

\hskip0.5in
{\tt CALL \packagename\_order( A, PERM, control, inform )}
\begin{description}
\itt{A} is scalar \intentin\ argument of type {\tt SMT\_type}
that is used to specify $\bmA$.
The user must set all of the relevant components of {\tt matrix} according
to the storage scheme desired (see \S\ref{typeprob}.
Incorrectly-set components will result in errors
flagged in {\tt inform\%status}, see \S\ref{galerrors}.

\itt{perm} is an 
\itt{PERM} is a rank-one \integer\ \intentout\ array argument
of \intentout\ and length $A\%n$.
{\tt PERM} will be set to the permutation array, so that the 
{\tt PERM(i)}-th rows and columns in the permuted matrix 
$\bmP \bmA \bmP^T$ correspond to those labelled {\tt i} in $\bmA$.

\itt{control} is a scalar \intentout argument of type
{\tt \packagename\_control\_type}. Its components control the action
of the analysis phase, as explained in
\S\ref{typecontrol}.

\itt{inform} is a scalar \intentout argument of type
{\tt \packagename\_inform\_type}
(see \S\ref{typeinform}).
A successful call is indicated when the  component {\tt status} has the value 0.
For other return values of {\tt status}, see \S\ref{galerrors}.

\end{description}

\subsubsection{The graph ordering subroutine}\label{subgraphorder}
A nested-dissection-based ordering of the adjacency graph (see 
\S\ref{galgraph}) of $\bmA$ may be obtained as follows:
\vskip2mm

\hskip0.5in
{\tt CALL \packagename\_order\_adjacency( n, PTR, IND, PERM, control, inform )}
\begin{description}
\itt{n} is an \intentin\ scalar of type  {\tt INTEGER} that gives the
number of rows (and columns) of $\bmA$.

\itt{PTR} is a rank-one \integer\ array argument of \intentin\ and length 
 at least $n+1$. Its $j$ entry, {\tt PTR}$(j)$, must be set to the position
 in {\tt IND} of the first entry in column $j$ of the whole of $\bmA$,
 while {\tt PTR}$(n+1)$ points to the first unoccupied position in {\tt IND}.

\itt{IND} is a rank-one \integer\ array argument of \intentin\ and length 
 at least  {\tt PTR}$(n+1)-1$. Components
 {\tt IND(}$(l)${\tt )}, $l =$ {\tt PTR}$(j)$, \ldots ,{\tt PTR} $(j+1)-1$
 must hold the row indices of non-diagonal entries in column $j$ of $\bmA$.

\itt{PERM} is a rank-one \integer\ \intentout\ array argument
 of \intentout\ and length {\tt n}.
 {\tt PERM} will be set to the permutation array, so that the 
 {\tt PERM(i)}-th rows and columns in the permuted matrix 
 $\bmP \bmA \bmP^T$ correspond to those labelled {\tt i} in $\bmA$.

\itt{control} is a scalar \intentout argument of type
{\tt \packagename\_control\_type}. Its components control the action
of the analysis phase, as explained in
\S\ref{typecontrol}.

\itt{inform} is a scalar \intentout argument of type
{\tt \packagename\_inform\_type}
(see \S\ref{typeinform}).
A successful call is indicated when the  component {\tt status} has the value 0.
For other return values of {\tt status}, see \S\ref{galerrors}.
\end{description}

%%%%%%%%%%%%%%%%%%%%%% Warning and error messages %%%%%%%%%%%%%%%%%%%%%%%%

\galerrors
A negative value of {\tt inform\%status} on exit from the subroutines
indicates that an error has occurred. No further calls should be made
until the error has been corrected. Possible values are:

\begin{description}

\itt{\galerrallocate} An allocation error occurred. A message indicating
the offending
array is written on unit {\tt control\%error}, and the returned allocation
status and a string containing the name of the offending array
are held in {\tt inform\%alloc\_\-status}
and {\tt inform\%bad\_alloc} respectively.

\itt{\galerrdeallocate} A deallocation error occurred.
A message indicating the offending
array is written on unit {\tt control\%error} and the returned allocation
status and a string containing the name of the offending array
are held in {\tt inform\%alloc\_\-status}
and {\tt inform\%bad\_alloc} respectively.

\itt{\galerrrestrictions} One of the restrictions
 {\tt n} $> 0$,
 {\tt A\%n} $> 0$ or
 {\tt A\%ne} $< 0$, for co-ordinate entry,
  or requirements that {\tt A\%type}
  contain its relevant string
 {\tt 'COORDINATE'}, {\tt 'SPARSE\_BY\_ROWS'} or {\tt 'DENSE'}, and
 {\tt control\%version} in one of {\tt '4.0'}, {\tt '5.1'} or {\tt '5.2'}
  has been violated.

\itt{\galunknownsolver} The requested version of {\tt METIS} is not available.

\itt{\galerrmetismemory} {\tt METIS} has insufficient memory to continue.

\itt{\galerrmetis} An internal {\tt METIS} error occurred.

\end{description}

%%%%%%%%%%%%%%%%%%%%%% Further features %%%%%%%%%%%%%%%%%%%%%%%%

\galcontrolfeatures
\noindent In this section, we describe an alternative means of setting
control parameters, that is components of the variable {\tt control} of type
{\tt \packagename\_control\_type}
(see \S\ref{typecontrol}),
by reading an appropriate data specification file using the
subroutine {\tt \packagename\_read\_specfile}. This facility
is useful as it allows a user to change  {\tt \packagename} control parameters
without editing and recompiling programs that call {\tt \packagename}.

A specification file, or specfile, is a data file containing a number of
``specification commands''. Each command occurs on a separate line,
and comprises a ``keyword'',
that is a string (in a close-to-natural language) used to identify a
control parameter, and
an (optional) "value", which defines the value to be assigned to the given
control parameter. All keywords and values are case insensitive,
keywords may be preceded by one or more blanks but
values must not contain blanks, and
each value must be separated from its keyword by at least one blank.
Values must not contain more than 30 characters, and
each line of the specification file is limited to 80 characters,
including the blanks separating keyword and value.

The portion of the specification file used by
{\tt \packagename\_read\_specfile}
must start
with a ``{\tt BEGIN \packagename}'' command and end with an
``{\tt END}'' command.  The syntax of the specfile is thus defined as follows:
\begin{verbatim}
  ( .. lines ignored by METIS_read_specfile .. )
    BEGIN METIS
       keyword    value
       .......    .....
       keyword    value
    END
  ( .. lines ignored by METIS_read_specfile .. )
\end{verbatim}
where keyword and value are two strings separated by (at least) one blank.
The ``{\tt BEGIN \packagename}'' and ``{\tt END}'' delimiter command lines
may contain additional (trailing) strings so long as such strings are
separated by one or more blanks, so that lines such as
\begin{verbatim}
    BEGIN METIS SPECIFICATION
\end{verbatim}
and
\begin{verbatim}
    END METIS SPECIFICATION
\end{verbatim}
are acceptable. Furthermore,
between the
``{\tt BEGIN \packagename}'' and ``{\tt END}'' delimiters,
specification commands may occur in any order.  Blank lines and
lines whose first non-blank character is {\tt !} or {\tt *} are ignored.
The content
of a line after a {\tt !} or {\tt *} character is also
ignored (as is the {\tt !} or {\tt *}
character itself). This provides an easy way to ``comment out'' some
specification commands, or to comment specific values
of certain control parameters.

The value of a control parameter may be of three different types, namely
integer, character or real.
Integer and real values may be expressed in any relevant Fortran integer and
floating-point formats (respectively).
%Permitted values for logical
%parameters are "{\tt ON}", "{\tt TRUE}", "{\tt .TRUE.}", "{\tt T}",
%"{\tt YES}", "{\tt Y}", or "{\tt OFF}", "{\tt NO}",
%"{\tt N}", "{\tt FALSE}", "{\tt .FALSE.}" and "{\tt F}".
%Empty values are also allowed for
%logical control parameters, and are interpreted as "{\tt TRUE}".

The specification file must be open for
input when {\tt \packagename\_read\_specfile}
is called, and the associated unit number
passed to the routine in {\tt device} (see below).
Note that the corresponding
file is rewound, which makes it possible to combine the specifications
for more than one program/routine.  For the same reason, the file is not
closed by {\tt \packagename\_read\_specfile}.

\subsubsection{To read control parameters from a specification file}
\label{readspec}

Control parameters may be read from a file as follows:
\hskip0.5in

\def\baselinestretch{0.8}
{\tt
\begin{verbatim}
     CALL METIS_read_specfile( control, device )
\end{verbatim}
}
\def\baselinestretch{1.0}

\begin{description}
\itt{control} is a scalar \intentinout\ argument of type
{\tt \packagename\_control\_type}
(see \S\ref{typecontrol}).
Default values should have already been set, perhaps by calling
{\tt \packagename\_initialize}.
On exit, individual components of {\tt control} may have been changed
according to the commands found in the specfile. Specfile commands and
the component (see \S\ref{typecontrol}) of {\tt control}
that each affects are given in Table~\ref{specfile}.

\pctable{|l|l|l|}
\hline
  command & component of {\tt control} & value type \\
\hline


{\tt version} & {\tt \%version} & {\tt character} \\
{\tt error-printout-device} & {\tt \%error} & {\tt integer} \\
{\tt printout-device} & {\tt \%out} & {\tt integer} \\
{\tt print-level} & {\tt \%print\_level} & {\tt integer} \\
{\tt metis4-ptype} & {\tt \%metis4\_ptype} & {\tt integer} \\
{\tt metis4-ctype} & {\tt \%metis4\_ctype} & {\tt integer} \\
{\tt metis4-itype} & {\tt \%metis4\_itype} & {\tt integer} \\
{\tt metis4-rtype} & {\tt \%metis4\_rtype} & {\tt integer} \\
{\tt metis4-dbglvl} & {\tt \%metis4\_dbglvl} & {\tt integer} \\
{\tt metis4-oflags} & {\tt \%metis4\_oflags} & {\tt integer} \\
{\tt metis4-pfactor} & {\tt \%metis4\_pfactor} & {\tt integer} \\
{\tt metis4-nseps} & {\tt \%metis4\_nseps} & {\tt integer} \\
{\tt metis5-ptype} & {\tt \%metis5\_ptype} & {\tt integer} \\
{\tt metis5-objtype} & {\tt \%metis5\_objtype} & {\tt integer} \\
{\tt metis5-ctype} & {\tt \%metis5\_ctype} & {\tt integer} \\
{\tt metis5-iptype} & {\tt \%metis5\_iptype} & {\tt integer} \\
{\tt metis5-rtype} & {\tt \%metis5\_rtype} & {\tt integer} \\
{\tt metis5-dbglvl} & {\tt \%metis5\_dbglvl} & {\tt integer} \\
{\tt metis5-niparts} & {\tt \%metis5\_niparts} & {\tt integer} \\
{\tt metis5-niter} & {\tt \%metis5\_niter} & {\tt integer} \\
{\tt metis5-ncuts} & {\tt \%metis5\_ncuts} & {\tt integer} \\
{\tt metis5-seed} & {\tt \%metis5\_seed} & {\tt integer} \\
{\tt metis5-ondisk} & {\tt \%metis5\_ondisk} & {\tt integer} \\
{\tt metis5-minconn} & {\tt \%metis5\_minconn} & {\tt integer} \\
{\tt metis5-contig} & {\tt \%metis5\_contig} & {\tt integer} \\
{\tt metis5-compress} & {\tt \%metis5\_compress} & {\tt integer} \\
{\tt metis5-ccorder} & {\tt \%metis5\_ccorder} & {\tt integer} \\
{\tt metis5-pfactor} & {\tt \%metis5\_pfactor} & {\tt integer} \\
{\tt metis5-nseps} & {\tt \%metis5\_nseps} & {\tt integer} \\
{\tt metis5-ufactor} & {\tt \%metis5\_ufactor} & {\tt integer} \\
{\tt metis5-dropedges} & {\tt \%metis5\_dropedges} & {\tt integer} \\
{\tt metis5-no2hop} & {\tt \%metis5\_no2hop} & {\tt integer} \\
{\tt metis5-twohop} & {\tt \%metis5\_twohop} & {\tt integer} \\
{\tt metis5-fast} & {\tt \%metis5\_fast} & {\tt integer} \\
{\tt no-metis-4-use-5-instead} & {\tt \%no\_metis\_4\_use\_5\_instead} 
 & {\tt logical} \\
{\tt output-line-prefix} & {\tt \%prefix} & {\tt character} \\
\hline

\ectable{\label{specfile}Specfile commands and associated
components of {\tt control}.}

\itt{device} is a scalar \intentin\ argument of type \integer,
that must be set to the unit number on which the specification file
has been opened. If {\tt device} is not open, {\tt control} will
not be altered and execution will continue, but an error message
will be printed on unit {\tt control\%error}.

\end{description}

%%%%%%%%%%%%%%%%%%%%%% GENERAL INFORMATION %%%%%%%%%%%%%%%%%%%%%%%%

\galgeneral

%\galcommon None.
\galworkspace Provided automatically by the module.
\galmodules {\tt GALAHAD\_CLOCK},
{\tt GALAHAD\_KINDS},
{\tt GALAHAD\_SYMBOLS},
{\tt GALAHAD\_SORT\_single/double},
{\tt GALAHAD\_SMT\-\_sin\-gle/double}
and
{\tt GALAHAD\_SPECFILE\_single/double},
\galio Output is under control of the arguments
{\tt control\%error},
{\tt control\%out}
\galrestrictions {\tt n} $\geq$ {\tt 1}, {\tt A\%n} $\geq$ {\tt 1},
{\tt A\%ne} $\geq$ {\tt 0} if {\tt A\%type = 'COORDINATE'},
{\tt A\%type} one of
{\tt 'COORDINATE'}, {\tt 'SPARSE\_BY\_ROWS'} or   {\tt 'DENSE'}.
{\tt control\%version} one of
{\tt '4.0'}, {\tt '5.1'} or   {\tt '5.2'}.
\galportability ISO Fortran~95 + TR 15581 or Fortran~2003.
The package is thread-safe.

%%%%%%%%%%%%%%%%%%%%%% METHOD %%%%%%%%%%%%%%%%%%%%%%%%

\galmethod
Variants of node-based nested-dissection ordering are used.

\noindent
The package relies crucially on the ordering package {\tt METIS} from 
the Karypis Lab. To obtain {\tt METIS} 4.0, see

  {\tt https://github.com/KarypisLab}.

\noindent
or 

  {\tt https://github.com/CIBC-Internal/metis-4.0.3}

\noindent
Versions 5.1 and 5.2 are open-source software, and included.

\vspace*{1mm}

\galreferences
\vspace*{1mm}

\noindent
The methods used are described in the user-documentation
\vspace*{1mm}

\noindent
G. Karypis.
METIS, A software package for partitioning unstructured
graphs, partitioning meshes, and computing
fill-reducing orderings of sparse matrices, Version 5,
Department of Computer Science \& Engineering, University of Minnesota
Minneapolis, MN 55455, USA (2013), see

   {\tt https://github.com/KarypisLab/METIS/blob/master/manual/manual.pdf}

\noindent
and paper

\noindent
G. Karypis and V. Kumar (1999). 
A fast and high quality multilevel scheme for partitioning irregular graphs,
SIAM Journal on Scientific Computing. {\bf 20(1)} (1999) 359--392.

%%%%%%%%%%%%%%%%%%%%%% EXAMPLE %%%%%%%%%%%%%%%%%%%%%%%%

\galexample
We illustrate the use of the package on the symmetric matrix with structure
\disp{\mat{ccccc}{ \ast &      & \ast &      & \ast \\
                        & \ast &      &      &      \\
                   \ast &      & \ast &      &      \\
                        &      &      & \ast & \ast \\
                   \ast &      &      & \ast & \ast },}
where $\ast$ denotes a nonzero.
Then, we may use the following code to find a suitable nested-dissection 
permutation prior to Cholesky-like factorization.

{\tt \small
\VerbatimInput{\packageexample}
}
\noindent
%with the following data
%{\tt \small
%\VerbatimInput{\packagedata}
%}
%\noindent
This produces the following output:
{\tt \small
\VerbatimInput{\packageresults}
}
\noindent
Alternatively, we may use the adjacency graph format to produce the same
ordering.

{\tt \small
\VerbatimInput{\packageexampleb}
}
\noindent
This produces the following output:
{\tt \small
\VerbatimInput{\packageresultsb}
}
\noindent
\end{document}
