\documentclass{galahad}
%  PRESOLVE documentation (Ph.L. Toint)
%  This version 13 V 2003.

% set the package name

\newcommand{\package}{pre}
\newcommand{\packagename}{PRESOLVE}
\newcommand{\fullpackagename}{\libraryname\_\packagename}
\newcommand{\sym}{\tt\small}
\newcommand{\fwbox}[1]{\fbox{\rule[-0.1cm]{0cm}{0.35cm}#1}}
\begin{document}

\galheader

%%%%%%%%%%%%%%%%%%%%%% SUMMARY %%%%%%%%%%%%%%%%%%%%%%%%

\galsummary

Presolving aims to {\bf improve the formulation of a given optimization
  problem by applying a sequence of simple transformations}, and thereby
to produce a ``reduced'' problem in a ``standard form'' that should be
simpler to solve.  This reduced problem may then be passed to an
appropriate solver.  Once the reduced problem has been solved, it is
then ``restored'' to recover the solution for the original formulation.

%\noindent
The package {\tt \fullpackagename} applies presolving techniques to a
{\bf linear}
\begin{equation}\label{objlp}
{\rm minimize}\;  \ell(\bmx) = f + \bmg^T \bmx
\end{equation}
or {\bf quadratic program}
\begin{equation}\label{objqp}
{\rm minimize}\;  q(\bmx) = f + \bmg^T \bmx + \half \bmx^T \bmH \bmx
\end{equation}
subject to the general linear constraints
\begin{equation}\label{gcqp}
c_{i}^{l}  \leq  \bma_{i}^{T}\bmx  \leq  c_{i}^{u}, \;\;\; i = 1, \ldots , m,
\end{equation}
and simple bounds
\begin{equation}\label{vbqp}
x_{j}^{l}  \leq  x_{j}^{ } \leq  x_{j}^{u} , \;\;\; j = 1, \ldots , n,
\end{equation}
where the scalar $f$, the $n$-dimensional vectors $\bmg$, $\bmx^l$ and
$\bmx^u$, the $m$-dimensional vectors $\bmc^l$ and $\bmc^u$,
the $n \times n$ symmetric matrix $\bmH$ and the $m \times n$ matrix $\bmA$
(whose rows are the vectors $\bma_{i}^T$) are
given. Furthermore, bounds on the Lagrange multipliers $\bmy$ associated with
the general linear constraints and on the dual variables $\bmz$ associated
with the simple bound constraints
\[
y_{i}^{l}  \leq  \bmy_{i}  \leq  y_{i}^{u}, \;\;\;  i = 1, \ldots , m,
\]
and
\[
z_{i}^{l}  \leq  \bmz_{i}  \leq  z_{i}^{u}, \;\;\;  i = 1, \ldots , n,
\]
are also provided, where the $m$-dimensional vectors $\bmy^l$ and
$\bmy^u$, as well as the $n$-dimensional vectors $\bmx^l$ and $\bmx^u$
are given.  Any component of $\bmc^l$, $\bmc^u$, $\bmx^l$, $\bmx^u$,
$\bmy^l$, $\bmy^u$, $\bmz^l$ or $\bmz^u$ may be infinite.

%%%%%%%%%%%%%%%%%%%%%% attributes %%%%%%%%%%%%%%%%%%%%%%%%

\galattributes
\galversions{\tt  \fullpackagename\_single, \fullpackagename\_double}.
\galuses {\tt GALAHAD\_SMT}, {\tt GAL\-AHAD\_QPT}, {\tt GALAHAD\_SPECFILE},
{\tt GALAHAD\_SORT}, {\tt GALAHAD\_SYMBOLS}.
\galdate March 2002.
\galorigin N. I. M. Gould, Rutherford Appleton Laboratory, and
Ph. L. Toint, The University of Namur, Belgium.
\gallanguage Fortran~95 + TR 15581 or Fortran~2003.

%%%%%%%%%%%%%%%%%%%%%% HOW TO USE %%%%%%%%%%%%%%%%%%%%%%%%

\galhowto

Access to the package requires a {\tt USE} statement such as

\medskip\noindent{\em Single precision version}

\hspace{8mm} {\tt USE \fullpackagename\_single}

\medskip\noindent{\em Double precision version}

\hspace{8mm} {\tt USE  \fullpackagename\_double}

\medskip

\noindent
If it is required to use both modules at the same time, the derived types
{\tt SMT\_problem\_type},
{\tt QPT\_problem\_type},
{\tt \packagename\_control\_type},
{\tt \packagename\_inform\_type}
and
{\tt \packagename\_data\_type}
(Section~\ref{galtypes})
and the five subroutines
{\tt \packagename\_initialize},
{\tt \packagename\_read\_specfile},
{\tt \packagename\_apply},
{\tt \packagename\_restore},
{\tt \packagename\_terminate},
(Section~\ref{galarguments})
must be renamed on one of the {\tt USE} statements.

%%%%%%%%%%%%%%%%%%%%%% matrix formats %%%%%%%%%%%%%%%%%%%%%%%%

\galmatrix

Both the Hessian matrix $\bmH$ and the constraint Jacobian $\bmA$
may be stored in a variety of input formats.

\subsubsection{Dense storage format}\label{dense}
The matrix $\bmA$ is stored as a compact
dense matrix by rows, that is, the values of the entries of each row in turn are
stored in order within an appropriate real one-dimensional array.
Component $n \ast (i-1) + j$ of the storage array {\tt A\%val} will hold the
value $a_{ij}$ for $i = 1, \ldots , m$, $j = 1, \ldots , n$.
Since $\bmH$ is symmetric, only the lower triangular part (that is the part
$h_{ij}$ for $1 \leq j \leq i \leq n$) need be held. In this case
the lower triangle will be stored by rows, that is
component $i \ast (i-1)/2 + j$ of the storage array {\tt H\%val}
will hold the value $h_{ij}$ (and, by symmetry, $h_{ji}$)
for $1 \leq j \leq i \leq n$.

\subsubsection{Sparse co-ordinate storage format}\label{coordinate}
Only the nonzero entries of the matrices are stored. For the
$l$-th entry of $\bmA$, its row index $i$, column index $j$
and value $a_{ij}$
are stored in the $l$-th components of the integer arrays {\tt A\%row},
{\tt A\%col} and real array {\tt A\%val}, respectively.
The order is unimportant, but the total
number of entries {\tt A\%ne} is also required.
The same scheme is applicable to
$\bmH$ (thus requiring integer arrays {\tt H\%row}, {\tt H\%col}, a real array
{\tt H\%val} and an integer value {\tt H\%ne}),
except that only the entries in the lower triangle need be stored.

\subsubsection{Sparse row-wise storage format}\label{rowwise}
Again only the nonzero entries are stored, but this time
they are ordered so that those in row $i$ appear directly before those
in row $i+1$. For the $i$-th row of $\bmA$, the $i$-th component of a
integer array {\tt A\%ptr} holds the position of the first entry in this row,
while {\tt A\%ptr} $(m+1)$ holds the total number of entries plus one.
The column indices $j$ and values $a_{ij}$ of the entries in the $i$-th row
are stored in components
$l =$ {\tt A\%ptr}$(i)$, \ldots ,{\tt A\%ptr} $(i+1)-1$ of the
integer array {\tt A\%col}, and real array {\tt A\%val}, respectively.
The same scheme is applicable to
$\bmH$ (thus requiring integer arrays {\tt H\%ptr}, {\tt H\%col}, and
a real array {\tt H\%val}),
except that only the entries in the lower triangle need be stored.

For sparse matrices, this scheme almost always requires less storage than
its predecessor.

\subsubsection{Diagonal storage format}\label{diagonal}
If $\bmH$ is diagonal (i.e., $h_{ij} = 0$ for all $1 \leq i \neq j \leq n$)
only the diagonals entries $h_{ii}$, $1 \leq i \leq n$,  need be stored,
and the first $n$ components of the array {\tt H\%val} may be used for
the purpose. There is no sensible equivalent for the non-square $\bmA$.

\subsubsection{Zero storage format}\label{zero}
If $\bmH$ is the zero matrix (i.e., $h_{ij} = 0$ for all
$1 \leq i \leq j \leq n$), no entries need be stored.

%%%%%%%%%%%%%%%%%%%%%% symbols %%%%%%%%%%%%%%%%%%%%%%%%

%\galsymbols

%The following description make use of ``symbols'' that are publicly available
%in the GALAHAD\_SYMBOLS module. These symbols are conventional names given to
%specific integer values, that allow a more natural specification of the
%various options and parameters of the package.  Each symbol provided in the
%SYMBOLS module is of the form {\tt GALAHAD\_NAME}, where {\tt NAME} is the name
%of the symbol.  For clarify and conciseness, we will represent such a symbol
%by {\sym GALAHAD\_NAME} (in sans-serif upper case font) in what follows. See
%Section~\ref{galexample} to see how symbols may be used in the program unit
%that calls the \packagename\ subroutines.

%%%%%%%%%%%%%%%%%%%%%% derived types %%%%%%%%%%%%%%%%%%%%%%%%

\galtypes

Six derived data types are accessible from the package.

%%%%%%%%%%% matrix data type %%%%%%%%%%%

\subsubsection{The derived data type for holding matrices}\label{typesmt}
The derived data type {\tt SMT\_TYPE} is used to hold the matrices $\bmA$
and $\bmH$. The components of {\tt SMT\_TYPE} used here are:

\begin{description}

\ittf{m} is a scalar component of type default \integer,
that holds the number of rows in the matrix.

\ittf{n} is a scalar component of type default \integer,
that holds the number of columns in the matrix.

\ittf{ne} is a scalar variable of type default \integer, that either
holds the number of matrix entries or is used to flag the storage scheme
used.

\ittf{val} is a rank-one allocatable array of type default \realdp\,
and dimension at least {\tt ne}, that holds the values of the entries.
Each pair of off-diagonal entries $h_{ij} = h_{ji}$ of a {\em symmetric}
matrix $\bmH$ is represented as a single entry
(see \S\ref{dense}--\ref{rowwise}).
Any duplicated entries that appear in the sparse
co-ordinate or row-wise schemes will be summed.

\ittf{row} is a rank-one allocatable array of type default \integer,
and dimension at least {\tt ne}, that may hold the row indices of the entries.
(see \S\ref{coordinate}).

\ittf{col} is a rank-one allocatable array of type default \integer,
and dimension at least {\tt ne}, that may the column indices of the entries
(see \S\ref{coordinate}--\ref{rowwise}).

\ittf{ptr} is a rank-one allocatable array of type default \integer,
and dimension at least {\tt m + 1}, that may hold the pointers to
the first entry in each row (see \S\ref{rowwise}).

\end{description}

%%%%%%%%%%% problem type %%%%%%%%%%%

\subsubsection{The derived data type for holding the problem}\label{typeprob}
The derived data type {\tt QPT\_problem\_type} is used to hold
the problem. The components of {\tt QPT\_problem\_type}
are:

\begin{description}

\itt{new\_problem\_structure} is a scalar variable of type default \logical,
 that is \true\ if this is the first (or only) problem in a sequence of
 problems with identical ``structure'' to be attempted, and \false\ if
 a previous problem with the same ``structure'' (but different
 numerical data) has been solved. We say that two problems have the same
 structure if they differ only in their components of type {\tt REAL}, which
 means that they share the same dimensions and matrix sparsity patterns.
 See Section~\ref{galmethod} for a description of how the package may be
 applied to  more than one problem with the same structure.

\noindent
 When a {\tt .TRUE.} value is specified for this component on entry in {\tt
 \packagename\_apply}\ (see below), this routine performs extensive checks on
 the consistency of the {\tt problem} structure and also allocates the
 necessary problem dependent workspace.  It is thus mandatory that the {\tt
 .TRUE.} value is used on the first call to {\tt \packagename\_apply}, but
 the {\tt .FALSE.} value should be used for any subsequent call to this
 routine for problems with the same structure.

\ittf{n} is a scalar variable of type default \integer,
 that holds the number of optimization variables, $n$.

\ittf{m} is a scalar variable of type default \integer,
 that holds the number of general linear constraints, $m$.

\ittf{H} is scalar variable of type {\tt SMT\_TYPE}
that holds the Hessian matrix $\bmH$. The following components
are used:

\begin{description}

\itt{H\%type} is an allocatable array of rank one and type default \character, that
is used to indicate the storage scheme used. If the dense storage scheme
(see Section~\ref{dense}) is used,
the first five components of {\tt H\%type} must contain the
string {\tt DENSE}.
For the sparse co-ordinate scheme (see Section~\ref{coordinate}),
the first ten components of {\tt H\%type} must contain the
string {\tt COORDINATE},
for the sparse row-wise storage scheme (see Section~\ref{rowwise}),
the first fourteen components of {\tt H\%type} must contain the
string {\tt SPARSE\_BY\_ROWS},
for the diagonal storage scheme (see Section~\ref{diagonal}),
the first eight components of {\tt H\%type} must contain the
string {\tt DIAGONAL},
and for the zero storage scheme (see Section~\ref{diagonal}),
the first four components of {\tt H\%type} must contain the
string {\tt ZERO}.

For convenience, the procedure {\tt SMT\_put}
may be used to allocate sufficient space and insert the required keyword
into {\tt H\%type}.
For example, if {\tt prob} is of derived type {\tt \packagename\_problem\_type}
and involves a Hessian we wish to store using the co-ordinate scheme,
we may simply
%\vspace*{-2mm}
{\tt
\begin{verbatim}
        CALL SMT_put( prob%H%type, 'COORDINATE' )
\end{verbatim}
}
%\vspace*{-4mm}
\noindent
See the documentation for the \galahad\ package {\tt SMT}
for further details on the use of {\tt SMT\_put}.

\itt{H\%ne} is a scalar variable of type default \integer, that
holds the number of entries in the {\bf lower triangular} part of $\bmH$
in the sparse co-ordinate storage scheme (see Section~\ref{coordinate}).
It need not be set for any of the other four schemes.

\itt{H\%val} is a rank-one allocatable array of type default \realdp, that holds
the values of the entries of the {\bf lower triangular} part
of the Hessian matrix $\bmH$ in any of the
storage schemes discussed in Section~\ref{galmatrix}.

\itt{H\%row} is a rank-one allocatable array of type default \integer,
that holds the row indices of the {\bf lower triangular} part of $\bmH$
in the sparse co-ordinate storage
scheme (see Section~\ref{coordinate}).
It need not be allocated for any of the other four schemes.

\itt{H\%col} is a rank-one allocatable array variable of type default \integer,
that holds the column indices of the {\bf lower triangular} part of
$\bmH$ in either the sparse co-ordinate
(see Section~\ref{coordinate}), or the sparse row-wise
(see Section~\ref{rowwise}) storage scheme.
It need not be allocated when the dense or diagonal storage schemes are used.

\itt{H\%ptr} is a rank-one allocatable array of dimension {\tt n+1} and type
default \integer, that holds the starting position of
each row of the {\bf lower triangular} part of $\bmH$, as well
as the total number of entries plus one, in the sparse row-wise storage
scheme (see Section~\ref{rowwise}). It need not be allocated when the
other schemes are used.

\end{description}

\itt{gradient\_kind} is a scalar variable of type default \integer,
that is used to indicate whether the components of the gradient $\bmg$
have special or general values. Possible values for {\tt gradient\_kind} are:
\begin{description}
\itt{0}  In this case, $\bmg = 0$.

\itt{1} In this case, $g_{i} = 1$ for $i = 1, \ldots ,n$.

\itt{$\neq$ 0,1} In this case, general values of $\bmg$ will be used,
     and will be provided by the user in the component {\tt G}.
\end{description}

\ittf{G} is a rank-one allocatable array of dimension {\tt n} and type
default \realdp, that holds the gradient $\bmg$
of the linear term of the quadratic objective function.
The $j$-th component of {\tt G}, $j = 1,  \ldots ,  n$, contains $\bmg_{j}$.
If {\tt gradient\_kind} {= 0, 1}, {\tt G} need not be allocated.

\ittf{f} is a scalar variable of type
default \realdp, that holds
the constant term, $f$, in the objective function.

\ittf{A} is scalar variable of type {\tt SMT\_TYPE}
that holds the Jacobian matrix $\bmA$. The following components are used:

\begin{description}

\itt{A\%type} is an allocatable array of rank one and type default \character, that
is used to indicate the storage scheme used. If the dense storage scheme
(see Section~\ref{dense}) is used,
the first five components of {\tt A\%type} must contain the
string {\tt DENSE}.
For the sparse co-ordinate scheme (see Section~\ref{coordinate}),
the first ten components of {\tt A\%type} must contain the
string {\tt COORDINATE}, while
for the sparse row-wise storage scheme (see Section~\ref{rowwise}),
the first fourteen components of {\tt A\%type} must contain the
string {\tt SPARSE\_BY\_ROWS}.

Just as for {\tt H\%type} above, the procedure {\tt SMT\_put}
may be used to allocate sufficient space and insert the required keyword
into {\tt A\%type}.
Once again, if {\tt prob} is of derived type {\tt \packagename\_problem\_type}
and involves a Jacobian we wish to store using the sparse row-wise
storage scheme, we may simply
%\vspace*{-2mm}
{\tt
\begin{verbatim}
        CALL SMT_put( prob%A%type, 'SPARSE_BY_ROWS' )
\end{verbatim}
}
%\vspace*{-4mm}
\noindent

\itt{A\%ne} is a scalar variable of type default \integer, that
holds the number of entries in $\bmA$
in the sparse co-ordinate storage scheme (see Section~\ref{coordinate}).
It need not be set for either of the other two schemes.

\itt{A\%val} is a rank-one allocatable array of type default \realdp, that holds
the values of the entries of the Jacobian matrix $\bmA$ in any of the
storage schemes discussed in Section~\ref{galmatrix}.

\itt{A\%row} is a rank-one allocatable array of type default \integer,
that holds the row indices of $\bmA$ in the sparse co-ordinate storage
scheme (see Section~\ref{coordinate}).
It need not be allocated for either of the other two schemes.

\itt{A\%col} is a rank-one allocatable array variable of type default \integer,
that holds the column indices of $\bmA$ in either the sparse co-ordinate
(see Section~\ref{coordinate}), or the sparse row-wise
(see Section~\ref{rowwise}) storage scheme.
It need not be allocated when the dense storage scheme is used.

\itt{A\%ptr} is a rank-one allocatable array of dimension {\tt m+1} and type
default \integer, that holds the
starting position of each row of $\bmA$, as well
as the total number of entries plus one, in the sparse row-wise storage
scheme (see Section~\ref{rowwise}). It need not be allocated when the
other schemes are used.

\end{description}

\ittf{C\_l} is a rank-one allocatable array of dimension {\tt m} and type
default \realdp, that holds the vector of lower bounds $\bmc^{l}$
on the general constraints. The $i$-th component of
{\tt C\_l}, $i = 1, \ldots , m$, contains $\bmc_{i}^{l}$.
Infinite bounds are allowed by setting the corresponding
components of {\tt C\_l} to any value smaller than {\tt -infinity},
where {\tt infinity} is a component of the control array {\tt control}
(see Section~\ref{typecontrol}).

\ittf{C\_u} is a rank-one allocatable array of dimension {\tt m} and type
default \realdp, that holds the vector of upper bounds $\bmc^{u}$
on the general constraints. The $i$-th component of
{\tt C\_u}, $i = 1,  \ldots ,  m$, contains $\bmc_{i}^{u}$.
Infinite bounds are allowed by setting the corresponding
components of {\tt C\_u} to any value larger than {\tt infinity},
where {\tt infinity} is a component of the control array {\tt control}
(see Section~\ref{typecontrol}).

\itt{C\_status} is a rank-one allocatable array of dimension {\tt m} and type
default \integer, that holds the status of the problem constraints (active or
inactive). A constraint is said to be inactive if it is not included in the
formulation of the considered quadratic program.

\ittf{X\_l} is a rank-one allocatable array of dimension {\tt n} and type
default \realdp, that holds
the vector of lower bounds $\bmx^{l}$ on the the variables.
The $j$-th component of {\tt X\_l}, $j = 1, \ldots , n$,
contains $\bmx_{j}^{l}$.
Infinite bounds are allowed by setting the corresponding
components of {\tt X\_l} to any value smaller than {\tt -infinity},
where {\tt infinity} is a component of the control array {\tt control}
(see Section~\ref{typecontrol}).

\ittf{X\_u} is a rank-one allocatable array of dimension {\tt n} and type
default \realdp, that holds
the vector of upper bounds $\bmx^{u}$ on the variables.
The $j$-th component of {\tt X\_u}, $j = 1, \ldots , n$,
contains $\bmx_{j}^{u}$.
Infinite bounds are allowed by setting the corresponding
components of {\tt X\_u} to any value larger than that {\tt infinity},
where {\tt infinity} is a component of the control array {\tt control}
(see Section~\ref{typecontrol}).

\itt{X\_status} is a rank-one allocatable array of dimension {\tt n} and type
default \integer, that holds the status of the problem variables (active or
inactive). Variable $j$ is said to be inactive if its value is fixed to the
current value of {\tt problem\%X(j)}, in which case it can be seen as a
parameter of the quadratic program.

\ittf{Y\_l} is a rank-one allocatable array of dimension {\tt m} and type
default \realdp, that holds
the vector of lower bounds $\bmy^{l}$ on the the multipliers.
The $j$-th component of {\tt Y\_l}, $j = 1, \ldots , m$,
contains $\bmy_{j}^{l}$.
Infinite bounds are allowed by setting the corresponding
components of {\tt Y\_l} to any value smaller than {\tt -infinity},
where {\tt infinity} is a component of the control array {\tt control}
(see Section~\ref{typecontrol}).

\ittf{Y\_u} is a rank-one allocatable array of dimension {\tt m} and type
default \realdp, that holds
the vector of upper bounds $\bmy^{u}$ on the multipliers.
The $j$-th component of {\tt Y\_u}, $j = 1, \ldots , m$,
contains $\bmy_{j}^{u}$.
Infinite bounds are allowed by setting the corresponding
components of {\tt Y\_u} to any value larger than that {\tt infinity},
where {\tt infinity} is a component of the control array {\tt control}
(see Section~\ref{typecontrol}).

\ittf{Z\_l} is a rank-one allocatable array of dimension {\tt n} and type
default \realdp, that holds
the vector of lower bounds $\bmz^{l}$ on the the dual variables.
The $j$-th component of {\tt Z\_l}, $j = 1, \ldots , n$,
contains $\bmz_{j}^{l}$.
Infinite bounds are allowed by setting the corresponding
components of {\tt Z\_l} to any value smaller than {\tt -infinity},
where {\tt infinity} is a component of the control array {\tt control}
(see Section~\ref{typecontrol}).

\ittf{Z\_u} is a rank-one allocatable array of dimension {\tt n} and type
default \realdp, that holds
the vector of upper bounds $\bmz^{u}$ on the dual variables.
The $j$-th component of {\tt Z\_u}, $j = 1, \ldots , n$,
contains $\bmz_{j}^{u}$.
Infinite bounds are allowed by setting the corresponding
components of {\tt Z\_u} to any value larger than that {\tt infinity},
where {\tt infinity} is a component of the control array {\tt control}
(see Section~\ref{typecontrol}).

\ittf{X} is a rank-one allocatable array of dimension {\tt n} and type
default \realdp,
that holds the values $\bmx$ of the optimization variables.
The $j$-th component of {\tt X}, $j = 1,  \ldots , n$, contains $x_{j}$.

\ittf{Z} is a rank-one allocatable array of dimension {\tt n} and type default
\realdp, that holds
the values $\bmz$ of estimates  of the dual variables
corresponding to the simple bound constraints (see Section~\ref{galmethod}).
The $j$-th component of {\tt Z}, $j = 1,  \ldots ,  n$, contains $z_{j}$.

\ittf{C} is a rank-one allocatable array of dimension {\tt m} and type default
\realdp, that holds
the values $\bmA \bmx$ of the constraints.
The $i$-th component of {\tt C}, $i = 1,  \ldots ,  m$, contains
$\bma_{i}^{T}\bmx \equiv (\bmA \bmx)_{i}$.

\ittf{Y} is a rank-one allocatable array of dimension {\tt m} and type
default \realdp, that holds
the values $\bmy$ of estimates  of the Lagrange multipliers
corresponding to the general linear constraints (see Section~\ref{galmethod}).
The $i$-th component of {\tt Y}, $i = 1,  \ldots ,  m$, contains $y_{i}$.

\end{description}

%%%%%%%%%%% control type %%%%%%%%%%%


\subsubsection{The derived data type for holding control
 parameters}\label{typecontrol}
The derived data type
{\tt \packagename\_control\_type}
is used to hold controlling data. Default values may be obtained by calling
{\tt \packagename\_initialize} (see Section~\ref{subinit}),
while individual components may also be changed by calling
{\tt \packagename\_read\-\_specfile}
(see Section~\ref{readspec}).
The components of
{\tt \packagename\_control\_type}
are:

\begin{description}
\itt{termination} is a scalar variable of type default \integer, that
determines the strategy for terminating the presolve analysis.
Possible values are:
\begin{description}
\item[\sym \galsymreducedsize.]  presolving continues so long as one of
            the sizes of the problem ($n$, $m$, sizes of $\bmA$ and $\bmH$) is
            being reduced.
\item[\sym \galsymfullpresolve.] presolving continues so long as further
            problem transformations are possible.
\end{description}
Note that the maximum number of analysis passes ({\tt max\_nbr\_passes})  and
the maximum number of problem transformations ({\tt max\_nbr\_transforms})
set an upper limit on the  presolving effort irrespective of the choice of
{\tt termination}. The only effect of this latter parameter is to allow for
early termination.
The default is {\tt termination} = {\sym \galsymreducedsize}.

\itt{max\_nbr\_transforms} is a scalar variable of type default \integer, that
determines the maximum number of problem transformations.
The default is {\tt max\_nbr\_transforms} = 1000000.

\itt{max\_nbr\_passes} is a scalar variable of type default \integer, that
determines the maximum number of analysis passes for problem analysis
during a single call to {\tt \packagename\_apply}.
The default is {\tt max\_nbr\_passes} = 25.

\itt{c\_accuracy} is a scalar variable of type default \realdp, that holds the
relative accuracy at which the general linear constraints are satisfied at
the exit of the solver. Note that this value is not used before the restoration
of the problem.
The default is {\tt c\_accuracy} = $10^{-4}$ in single precision, and
{\tt c\_accuracy} = $10^{-6}$ in double precision.

\itt{z\_accuracy} is a scalar variable of type default \realdp, that holds the
relative accuracy at which the dual feasibility constraints are satisfied at
the exit of the solver. Note that this value is not used before the restoration
of the problem.
The default is {\tt z\_accuracy} = $10^{-4}$ in single precision, and
{\tt z\_accuracy} = $10^{-6}$ in double precision.

\itt{infinity} is a scalar variable of type default \realdp, that holds the
value beyond which a number is deemed equal to plus infinity
(minus infinity being defined as its opposite)
The default is {\tt infinity} = $10^{19}$.

\ittf{out} is a scalar variable of type default \integer, that holds the
unit number associated with the device used for printout.
The default is {\tt out} = 6.

\itt{errout} is a scalar variable of type default \integer, that holds the
unit number associated with the device used for error ouput.
The default is {\tt errout} = 6.

\itt{print\_level} is a scalar variable of type default \integer, that holds the
level of printout requested by the user. See Section~\ref{galinfo}.
The default is {\tt print\_level} = {\sym \galsymsilent}.

\itt{dual\_transformations} is a scalar variable of type default \logical,
that has the value {\tt .TRUE.} if dual transformations of the problem are
allowed. Note that this implies that the reduced problem is solved
accurately (for the dual feasibility condition to hold)
as to be able to restore the problem to the original
constraints and variables. The value {\tt .FALSE.} prevents dual
transformations to be applied, thus allowing for inexact
solution of the reduced problem. The setting of this control
parameter overides the values of {\tt get\_z}, {\tt get\_z\_bounds}, {\tt
get\_y}, {\tt get\_y\_bounds}, {\tt dual\_constraints\_freq}, {\tt
singleton\_columns\_freq}, {\tt doubleton\_columns\_freq}, {\tt z\_accuracy}
and {\tt  check\_dual\_feasibility}.
The default is {\tt dual\_transformations} = {\tt .TRUE.}.

\itt{redundant\_xc} is a scalar variable of type default \logical,
that has the value {\tt .TRUE.} if redundant variables and constraints
(that is variables that don't occur in the objective function and are
either unbounded above with all their coefficients in $\bmA$
being positive for constraints that are unbounded above and negative for
constraints that are unbounded below, or unbounded below
with all their coefficients in $\bmA$ being positive for constraints that are
unbounded below or negative for all constraints that are unbounded above)
are to be removed from the problem with their associated constraints before
any other problem transformation is attempted.
The default is {\tt redundant\_xc} = {\tt .TRUE.}.

\itt{primal\_constraints\_freq} is a scalar variable of type default \integer,
that holds the  frequency of primal constraints analysis in terms of
presolving passes.  A value of 2 indicates that primal
constraints are analyzed every 2 presolving passes. A zero
value indicates that they are never analyzed.
The default is {\tt primal\_constraints\_freq} = 1.

\itt{dual\_constraints\_freq} is a scalar variable of type default \integer,
that holds the frequency of dual constraints analysis in terms of
presolving passes.  A value of 2 indicates that dual
constraints are analyzed every 2 presolving passes.  A zero
value indicates that they are never analyzed.
The default is {\tt dual\_constraints\_freq} = 1.

\itt{singleton\_columns\_freq} is a scalar variable of type default \integer,
that holds the frequency of singleton column analysis in terms of
presolving passes.  A value of 2 indicates that
singleton columns are analyzed every 2 presolving passes.
A zero value indicates that they are never analyzed.
The default is {\tt singleton\_columns\_freq} = 1.

\itt{doubleton\_columns\_freq} is a scalar variable of type default \integer,
that holds the frequency of doubleton column analysis in terms of
presolving passes.  A value of j indicates that doubleton
columns are analyzed every 2 presolving passes.  A zero
value indicates that they are never analyzed.
The default is {\tt doubleton\_columns\_freq} = 1.

\itt{unc\_variables\_freq} is a scalar variable of type default \integer,
that holds the frequency of the attempts to fix linearly unconstrained
variables, expressed in terms of presolving passes.  A
value of 2 indicates that attempts are made every 2
presolving passes.  A zero value indicates that no attempt is ever made.
The default is {\tt unc\_variables\_freq} = 1.

\itt{dependent\_variables\_freq} is a scalar variable of type default
\integer, that holds the frequency of search for dependent variables in terms
of  presolving passes.  A value of 2 indicates that
dependent variables are searched for every 2 presolving
passes.  A zero value indicates that no attempt is ever made to detect such
variables. The default is {\tt dependent\_variables\_freq} = 1.

\itt{sparsify\_rows\_freq} is a scalar variable of type default \integer,
that holds the frequency of the attempts to make A sparser in terms of
presolving passes.  A value of 2 indicates that attempts
are made every 2 presolving passes.  A zero value indicates
that no attempt is ever made.
The default is {\tt sparsify\_rows\_freq} = 1.

\itt{max\_fill} is a scalar variable of type default \integer, that holds
the maximum percentage of fill in each row of $\bmA$. Note that
this is a row-wise measure: globally fill never exceeds
the storage initially used for $\bmA$, no matter how large
{\tt max\_fill} is chosen. If {\tt max\_fill} is negative,
no limit is put on row fill.
The default is {\tt max\_fill} = -1 (no limit).

\itt{transf\_file\_nbr} is a scalar variable of type default \integer, that
holds the unit number to be associated with the file(s) used
for saving problem transformations on a disk file.
The default is {\tt transf\_file\_nbr} = 52.

\itt{transf\_buffer\_size} is a scalar variable of type default \integer, that
holds the number of transformations that can be kept in memory
at once (that is without being saved on a disk file).
The default is {\tt transf\_buffer\_size} = 50000.

\itt{transf\_file\_status} is a scalar variable of type default \integer, that
holds the exit status of the file where problem transformations are saved:
\begin{description}
\item[\sym \galsymkeep.] the file is not deleted after program termination,
\item[\sym \galsymdelete.] the file is not deleted after program termination.
\end{description}
The default is {\tt transf\_file\_status} = {\sym \galsymkeep}.

\itt{transf\_file\_name} is a scalar variable of type default \integer, that
holds the name of the file (to be) used for storing problem transformation on
disk. Note that this parameter must be identical for all calls to
{\tt \packagename} that follows {\tt \packagename\_read\_specfile}. It can
then only be changed after calling {\tt \packagename\_terminate}.
The default is {\tt transf\_file\_name} = {\tt transf.sav}.

\itt{y\_sign} is a scalar variable of type default \integer, that
determines the convention of sign used for the multipliers
associated with the general linear constraints. Possible values are:
\begin{description}
\item[\sym \galsympositive.] all multipliers corresponding to active
                inequality constraints are non-negative for
                lower bound constraints and non-positive for
                upper bounds constraints;
\item[\sym \galsymnegative.] all multipliers corresponding to active
                inequality constraints are non-positive for
                lower bound constraints and non-negative for
                upper bounds constraints.
\end{description}
The default is {\tt y\_sign} = {\sym \galsympositive}.

\itt{inactive\_y} is a scalar variable of type default \integer, that
determines whether or not the multipliers corresponding
to general linear constraints that are inactive at the unreduced point
corresponding to the reduced point on input of {\tt \packagename\_restore}
must be set to zero. Possible values are:
\begin{description}
\item[\sym \galsymforcetozero.] all multipliers corresponding to inactive
                     general linear constraints are forced to zero,
                     possibly at the expense of deteriorating the
                     dual feasibility condition.
                     Note that this option is inactive unless
                     {\tt get\_y = get\_c get\_c\_bounds = .TRUE.}.
\item[\sym \galsymleaveasis.] multipliers corresponding to inactive
                   general linear constraints are left unaltered.
\end{description}
The default is {\tt inactive\_y} = {\sym \galsymleaveasis}.

\itt{z\_sign}is a scalar variable of type default \integer, that
determines the convention of sign used for the dual
variables associated with the bound constraints. Possible values are:
\begin{description}
\item[\sym \galsympositive.] all dual variables corresponding to
                active lower bounds are non-negative, and
                non-positive for active upper bounds;
\item[\sym \galsymnegative.] all dual variables corresponding to
                active lower bounds are non-positive, and
               non-negative for active upper bounds.
\end{description}
The default is {\tt z\_sign} = {\sym \galsympositive}.

\itt{inactive\_z} is a scalar variable of type default \integer, that
determines whether or not the dual variables corresponding
to bound constraints that are inactive at the unreduced point
corresponding to the reduced point on input of {\tt \packagename\_restore}
must be set to zero. Possible values are:
\begin{description}
\item[\sym \galsymforcetozero.] all dual variables corresponding to inactive
                     bounds are forced to zero,
                     possibly at the expense of deteriorating the
                     dual feasibility condition.
                     Note that this option is inactive unless
                     {\tt get\_z = get\_x get\_x\_bounds = .TRUE.}.
\item[\sym \galsymleaveasis.] dual variables corresponding to inactive
                   bounds are left unaltered.
\end{description}
The default is {\tt inactive\_z} = {\sym \galsymleaveasis}.

\itt{final\_x\_bounds} is a scalar variable of type default \integer, that
holds the type of final bounds on the variables returned by the
package.  This parameter can take the values:
\begin{description}
\item[\sym \galsymtightest.] the final bounds are the tightest bounds
                known on the variables (at the risk of
                being redundant with other constraints,
                which may cause degeneracy);
\item[\sym \galsymnondegenerate.] the best known bounds that are known to
                      be non-degenerate. This option implies
                      that an additional real workspace of size
                      {\tt 2 * problem\%n} must be allocated;
\item[\sym \galsymloosest.] the loosest bounds that are known to
               keep the problem equivalent to the
               original problem. This option also
               implies that an additional real workspace of size
               {\tt 2 * problem\%n} must be allocated;
\end{description}
Note that his parameter must be identical for all calls to
{\tt \packagename} following {\tt \packagename\_read\_specfile}.
The default is {\tt final\_x\_bounds} = {\sym \galsymtightest}.

\itt{final\_z\_bounds} is a scalar variable of type default \integer, that
holds the type of final bounds on the dual variables returned by
the package.  This parameter can take the values:
\begin{description}
\item[\sym \galsymtightest.] the final bounds are the tightest bounds
                known on the dual variables (at the risk of
                being redundant with other constraints,
                which may cause degeneracy);
\item[\sym \galsymnondegenerate.] the best known bounds that are known to
                      be non-degenerate. This option implies
                      that an additional real workspace of size
                      {\tt 2 * problem\%n} must be allocated;
\item[\sym \galsymloosest.] the loosest bounds that are known to
               keep the problem equivalent to the
               original problem. This option also
               implies that an additional real workspace of size
               {\tt 2 * problem\%n} must be allocated;
\end{description}
Note that his parameter must be identical for all calls to
{\tt \packagename} following {\tt \packagename\_read\_specfile}.
The default is {\tt final\_z\_bounds} = {\sym \galsymtightest}.

\itt{final\_c\_bounds} is a scalar variable of type default \integer, that
holds the type of final bounds on the constraints returned by
the package.  This parameter can take the values:
\begin{description}
\item[\sym \galsymtightest.] the final bounds are the tightest bounds
                known on the constraints (at the risk of
                being redundant with other constraints,
                which may cause degeneracy);
\item[\sym \galsymnondegenerate.] the best known bounds that are known to
                      be non-degenerate. This option implies
                      that an additional real workspace of size
                      {\tt 2 * problem\%n} must be allocated;
\item[\sym \galsymloosest.] the loosest bounds that are known to
               keep the problem equivalent to the
               original problem. This option also
               implies that an additional real workspace of size
               {\tt 2 * problem\%n} must be allocated;
\end{description}
Note that his parameter must be identical for all calls to
{\tt \packagename} following {\tt \packagename\_read\_specfile}.
If different from {\tt \galsymtightest}, its value must be equal to
that of {\tt final\_x\_bounds}.
The default is {\tt final\_c\_bounds} = {\sym \galsymtightest}.

\itt{final\_y\_bounds} is a scalar variable of type default \integer, that
holds the type of final bounds on the multipliers returned by
the package.  This parameter can take the values:
\begin{description}
\item[\sym \galsymtightest.] the final bounds are the tightest bounds
                known on the multipliers (at the risk of
                being redundant with other constraints,
                which may cause degeneracy);
\item[\sym \galsymnondegenerate.] the best known bounds that are known to
                      be non-degenerate. This option implies
                      that an additional real workspace of size
                      {\tt 2 * problem\%n} must be allocated;
\item[\sym \galsymloosest.] the loosest bounds that are known to
               keep the problem equivalent to the
               original problem. This option also
               implies that an additional real workspace of size
               {\tt 2 * problem\%n} must be allocated;
\end{description}
Note that his parameter must be identical for all calls to
{\tt \packagename} following {\tt \packagename\_read\_specfile}.
The default is {\tt final\_y\_bounds} = {\sym \galsymtightest}.

\itt{check\_primal\_feasibility} is a scalar variable of type default
\integer, that holds the level of feasibility check (on the values of $\bmx$)
at the start of the restoration phase.  This parameter can
take the values:
\begin{description}
\item[\sym \galsymnone.] no check at all;
\item[\sym \galsymbasic.] the primal constraints are recomputed at $\bmx$
             and a message issued if the computed value
             does not match the input value, or if it is
             out of bounds (if {\tt print\_level} $\geq$ {\sym \galsymaction});
\item[\sym \galsymsevere.] the same as for {\sym \galsymbasic}, but {\tt \packagename} is
              terminated if an incompatibilty is detected.
\end{description}
The default is {\tt check\_primal\_feasibility} = {\sym \galsymnone}.

\itt{check\_dual\_feasibility} is a scalar variable of type default
\integer, that holds the level of dual feasibility check (on the values of
$\bmx$, $\bmy$ and $\bmz$) at the start of the restoration phase.  This
parameter can take the values:
\begin{description}
\item[\sym \galsymnone.] no check at all;
\item[\sym \galsymbasic.] the primal constraints are recomputed at $(\bmx,\bmy,\bmz)$
             and a message issued if the computed value
             does not match the input value, or if it is
             out of bounds (if {\tt print\_level} $\geq$ {\sym \galsymaction});
\item[\sym \galsymsevere.] the same as for {\sym \galsymbasic}, but {\tt \packagename} is
              terminated if an incompatibilty is detected.
\end{description}
The default is {\tt check\_dual\_feasibility} = {\sym \galsymnone}.

\itt{get\_q} is a scalar variable of type default \logical, that
must be set to {\tt .TRUE.} if the value of the objective
function must be reconstructed by {\tt \packagename\_restore} from the
(solved) reduced problem.
The default is {\tt get\_q} = {\tt .TRUE.}.

\itt{get\_f} is a scalar variable of type default \logical, that
must be set to {\tt .TRUE.} if the value of the objective
function's independent term is to be be reconstructed
by{\tt \packagename\_ restore} from the (solved) reduced problem.
The default is {\tt get\_f} = {\tt .TRUE.}.

\itt{get\_g} is a scalar variable of type default \logical, that
must be set to {\tt .TRUE.} if the values of the objective
function's gradient is to be be reconstructed
by {\tt \packagename\_restore} from the (solved) reduced problem.
The default is {\tt get\_g} = {\tt .TRUE.}.

\itt{get\_H} is a scalar variable of type default \logical, that
must be set to {\tt .TRUE.} if the values of the objective
function's Hessian is to be be reconstructed
by {\tt \packagename\_restore} from the (solved) reduced problem.
The default is {\tt get\_H} = {\tt .TRUE.}.

\itt{get\_A} is a scalar variable of type default \logical, that
must be set to {\tt .TRUE.} if the values of the constraints'
Jacobian is to be be reconstructed
by {\tt \packagename\_restore} from the (solved) reduced problem.
The default is {\tt get\_A} = {\tt .TRUE.}.

\itt{get\_x} is a scalar variable of type default \logical, that
must be set to {\tt .TRUE.} if the value of the variables
must be reconstructed by {\tt
\packagename\_restore} from the (solved) reduced problem.
The default is {\tt get\_x} = {\tt .TRUE.}.

\itt{get\_x\_bounds} is a scalar variable of type default \logical, that
must be set to {\tt .TRUE.} if the values of the bounds on the
problem variables must be reconstructed by {\tt \packagename\_restore}
from the (solved) reduced problem.
The default is {\tt get\_x\_bounds} = {\tt .TRUE.}.

\itt{get\_z} is a scalar variable of type default \logical, that
must be set to {\tt .TRUE.} if the value of the dual variables
must be reconstructed by {\tt
\packagename\_restore} from the (solved) reduced problem.
The default is {\tt get\_z} = {\tt .TRUE.}.

\itt{get\_z\_bounds} is a scalar variable of type default \logical, that
must be set to {\tt .TRUE.} if the values of the bounds on the
problem dual variables must be reconstructed by {\tt \packagename\_restore}
from the (solved) reduced problem.
If set, this may require to store specific
additional information on the problem transformations,
therefore increasing the storage needed for these transformations.
Note that this parameter must be identical for all calls to
{\tt \packagename} following {\tt \packagename\_read\_specfile}.
The default is {\tt get\_z\_bounds} = {\tt .TRUE.}.

\itt{get\_c} is a scalar variable of type default \logical, that
must be set to {\tt .TRUE.} if the values of the constraints
must be reconstructed by {\tt
\packagename\_restore} from the (solved) reduced problem.
The default is {\tt get\_c} = {\tt .TRUE.}.

\itt{get\_c\_bounds} is a scalar variable of type default \logical, that
must be set to {\tt .TRUE.} if the values of the bounds on the
problem constraints must be reconstructed by {\tt \packagename\_restore}
from the (solved) reduced problem.
The default is {\tt get\_c\_bounds} = {\tt .TRUE.}.

\itt{get\_y} is a scalar variable of type default \logical, that
must be set to {\tt .TRUE.} if the values of the multipliers
must be reconstructed by {\tt
\packagename\_restore} from the (solved) reduced problem.
The default is {\tt get\_y} = {\tt .TRUE.}.

\itt{get\_y\_bounds} is a scalar variable of type default \logical, that
must be set to {\tt .TRUE.} if the values of the bounds on the
problem multipliers must be reconstructed by {\tt \packagename\_restore}
from the (solved) reduced problem.
If set, this may require to store specific
additional information on the problem transformations,
therefore increasing the storage needed for these transformations.
Note that this parameter must be identical for all calls to
{\tt \packagename} following {\tt \packagename\_read\_specfile}.
The default is {\tt get\_y\_bounds} = {\tt .TRUE.}.

\itt{pivot\_tol} is a scalar variable of type default \realdp, that
holds the relative pivot tolerance above which pivoting is
considered as numerically stable in transforming the
coefficient matrix $\bmA$.  A zero value corresponds to a
totally unsafeguarded pivoting strategy (potentially unstable).
The default is {\tt pivot\_tol} = $10^{-6}$ in single precision, and
{\tt pivot\_tol} = $10^{-10}$ in double precision.

\itt{min\_rel\_improve}  is a scalar variable of type default
\realdp, that holds the minimum relative improvement in the bounds on $\bmx$,
$\bmy$ and $\bmz$ for a tighter bound on these quantities to be
accepted in the course of the analysis.  More formally,
if {\tt lower} is the current value of the lower bound on one
of the $\bmx$, $\bmy$ or $\bmz$, and if {\tt newlower} is a tentative tighter
lower bound on the same quantity, it is only accepted if
\[
{\tt newlower} \geq
{\tt lower} + {\tt min\_rel\_improve} * \max( 1, |{\tt lower}|).
\]
Similarly, a tentative tighter upper bound {\tt newupper}
only replaces the current upper bound {\tt upper} if
\[
{\tt newupper} \leq
{\tt upper} - {\tt min\_rel\_improve} * \max( 1, |{\tt upper}|).
\]
Note that this parameter must exceed the machine precision significantly.
The default is {\tt min\_rel\_improve} = $10^{-6}$ in single
precision, and {\tt min\_rel\_improve} = $10^{-10}$ in double
precision.

\itt{max\_growth\_factor} is a scalar variable of type default
\realdp, that holds the maximum ratio that is allowed for the absolute
value of any data item of the reduced problem compared to the maximum
absolute value of any data item of the original problem.  In the
course of the presolving process, any transformation that would result in
violating this bound is skipped.
The default is {\tt min\_growth\_factor} = $10^{4}$ in single
precision, and {\tt min\_growth\_factor} = $10^{8}$ in double
precision.

\end{description}

%%%%%%%%%%% info type %%%%%%%%%%%

\subsubsection{The derived data type for holding informational
 parameters}\label{typeinform}
The derived data type
{\tt \packagename\_inform\_type}
is used to hold parameters that give information about the progress and needs
of the algorithm. The components of
{\tt \packagename\_inform\_type}
are:

\begin{description}

\itt{status} is a scalar variable of type default \integer, that gives the
exit status of the algorithm. See Sections~\ref{galerrors} and \ref{galinfo}
for details.

\itt{message} is a character array of 3 lines of 80 characters each,
containing a description of the exit condition
on exit, typically including more information
than indicated in the description of {\tt status} above.
It is printed out on device {\tt errout} at the end of execution
unless {\tt print\_level} is {\sym \galsymsilent}.

\itt{nbr\_transforms} is a scalar variable of type default \integer, that
gives the the final number of problem transformations, as reported
to the user at exit.
\end{description}

%%%%%%%%%%% data type %%%%%%%%%%%

\subsubsection{The derived data type for holding problem data}\label{typedata}
The derived data type
{\tt \packagename\_data\_type}
is used to hold all the data for a particular problem,
or sequences of problems with the same structure, between calls of
{\tt \packagename} procedures.
This data should be preserved, untouched, from the initial call to
{\tt \packagename\_initialize}
to the final call to
{\tt \packagename\_terminate}.

%%%%%%%%%%%%%%%%%%%%%% argument lists %%%%%%%%%%%%%%%%%%%%%%%%

\galarguments
There are five procedures for user calls
(see Section~\ref{galfeatures} for further features):

\begin{enumerate}
\item The subroutine {\tt \packagename\_initialize}
      is used to set default values, and initialize private data,
      before presolving one or more problems with the
      same sparsity and bound structure.
\item The subroutine {\tt packagename\_read\_specfile}
      is used to read the {\tt packagename} specfile in order to
      possibly modify the algoritmic default parameters (see
      Section~\ref{readspec}).
\item The subroutine {\tt \packagename\_apply}
      is called to presolve the problem, that is to reduce it by applying
      suitable problem transformations and permute it to standard form.
\item The subroutine {\tt \packagename\_restore}
      restores the (solved) reduced problem to the original definition
      of variables and constraints;
\item The subroutine {\tt \packagename\_terminate}
      is provided to allow the user to automatically deallocate array
      components of the private data, allocated by
      {\tt \packagename}, at the end of the solution process.
      It is important to do this if the data object is re-used for another
      problem {\bf with a different structure}
      since {\tt \packagename\_initialize} cannot test for this situation,
      and any existing associated targets will subsequently become unreachable.
\end{enumerate}

%%%%%% initialization subroutine %%%%%%

\subsubsection{The initialization subroutine}\label{subinit}


Default values for the control parameters are provided as follows:
\vspace*{1mm}

\hspace{8mm}
{\tt CALL \packagename\_initialize( control, inform, data )}

\begin{description}
\itt{control} is a scalar \intentout\ argument of type
{\tt \packagename\_control\_type}
(see Section~\ref{typecontrol}).
On exit, {\tt control} contains default values for the components as
described in Section~\ref{typecontrol}.
These values should only be changed after calling
{\tt \packagename\_initialize}.

\itt{inform} is a scalar \intentout\ argument of type
{\tt \packagename\_inform\_type} (see Section~\ref{typeinform}).
A successful call to the routine {\tt \packagename\_initialize}
is indicated when the  component {\tt status} has the value 0.
For other return values of {\tt status}, see Section~\ref{galerrors}.

\itt{data} is a scalar \intentinout\ argument of type
{\tt \packagename\_data\_type}
(see Section~\ref{typedata}). It is used to hold data about the problem being
solved. It should never be altered by the user.
\end{description}

%%%%%%%%% main presolving subroutine %%%%%%

\subsubsection{The subroutine that applies presolving transformations to the problem}
The presolving algorithm is called as follows:
\vspace*{1mm}

\hspace{8mm}
{\tt CALL \packagename\_apply( problem, control, inform, data )}
\vspace*{1mm}

\noindent
Such a call must always be preceded by a call to {\tt \packagename\_initialize}.

\begin{description}
\itt{problem} is a scalar \intentinout\ argument of type
{\tt QPT\_problem\_type}
(see Section~\ref{typeprob})
that contains the problem statement.
It is used to hold data about the problem being solved.
Users are free to choose whichever
of the three matrix formats described in Section~\ref{galmatrix}
is appropriate for $\bmA$ and $\bmH$ for their application.
Some components of the problem structure need not be allocated or set on
input, in which case they will be assigned suitable default values.
The components in question, their size and the associated default values
are given in
Table~\ref{defvec}.
\bctable{|l|c|c|l|c|c|}
\hline
component & size & default & component & size & default \\
\hline
{\tt X\_l}      & $n$ & $-\infty$      &{\tt C\_l}      & $m$ & $-\infty$     \\
{\tt X\_u}      & $n$ & $+\infty$      &{\tt C\_u}      & $m$ & $+\infty$     \\
{\tt X}    & $n$ & (problem dependent) &{\tt C}   & $m$ & (problem dependent) \\
{\tt X\_status} & $n$ & {\sym \galsymactive } &{\tt C\_status} & $m$ & {\sym \galsymactive }\\
{\tt Z\_l}      & $n$ & $-\infty$      &{\tt Y\_l}      & $m$ & $-\infty$     \\
{\tt Z\_u}      & $n$ & $+\infty$      &{\tt Y\_u}      & $m$ & $+\infty$     \\
{\tt Z}    & $n$ & (problem dependent) &{\tt Y}   & $m$ & (problem dependent) \\
\hline
\ectable{\label{defvec}Defaults for unallocated array components of {\tt
problem}.}
\vspace*{-3mm}

\noindent
If the array {\tt problem\%X\_status} is allocated on entry, then possible
value of its $j$-th component are as follows:
\begin{description}
\item[\sym \galsyminactive.] the $j$-th variable is inactive in the sense that
the quadratic program under consideration ignores it (this is equivalent of
fixing $x_j$ to {\tt problem\%X(j)}; which obviously requires {\tt problem\%X}
to be allocated);
\item[\sym \galsymactive. ] the $j$-th variable is active (i.e.\ not
inactive).
\end{description}
The meaning of the $i$-th component of {\tt problem\%C\_status} is identical,
except that it relates to the $i$-th constraint:
\begin{description}
\item[\sym \galsyminactive.] the $i$-th constraint is inactive in the sense
that the quadratic program under consideration ignores it;
\item[\sym \galsymactive. ] the $j$-th variable is active (i.e.\ not
inactive).
\end{description}

\noindent
On exit, the problem structure will contain the reduced problem, {\bf with its
Hessian and Jacobian matrices stored in sparse row-wise format}; exceptionally,
if {\tt problem\%H\%ne = 0}, no values or indices of the Hessian will be
returned.
Values for $\bmx$, $\bmz$, $\bmc$, $\bmy$ and $f$ will be provided, that are
feasible for the reduced problem. Note that frequently not
all the space allocated for the original problem is used by the
reduced one.  However, crucial information that is necessary to restore the
problem to its original variables/constraints remains stored in the problem
structure, beyond that specified by the dimensions of the reduced problem.
Thus modification (for instance by a QP algorithm) of the reduced problem data
is possible (except for {\tt problem\%X\_status} and {\tt problem\%C\_status},
which should always remain unchanged), but no other data within the problem
structure should be altered before calling {\tt \packagename\_restore}.

\itt{control} is a scalar \intentinout\ argument of type
{\tt \packagename\_control\_type}
(see Section~\ref{typecontrol}). Default values may be assigned by calling
{\tt \packagename\_initialize} prior to the first call to
{\tt \packagename\_apply}.

\itt{inform} is a scalar \intentinout\ argument of type {\tt
\packagename\_inform\_type}
(see Section~\ref{typeinform}). A successful call to the routine
{\tt \packagename\_apply}
is indicated when the  component {\tt status} has the value 0.
For other return values of {\tt status}, see Section~\ref{galerrors}.

\itt{data} is a scalar \intentinout\ argument of type
{\tt \packagename\_data\_type}
(see Section~\ref{typedata}). It is used to hold data about the problem being
solved. It must never be  altered by the user since the last call to
any of the {\tt \packagename} routines.

\end{description}

%%%%%%% restoration subroutine %%%%%%

\subsubsection{The restoration subroutine}

The (solved) reduced problem is restored in the original variables/constraints
and matrix format by calling
\vspace*{1mm}

\hspace{8mm}
{\tt CALL \packagename\_restore( problem, control, inform, data )}
\vspace*{1mm}

The choice of which components of the problem to restore is governed by the
different {\tt \%get\_*} components of the {\tt control} structure (see
Section~\ref{typecontrol}).

\begin{description}
\itt{problem} is a scalar \intentinout\ argument of type
{\tt QPT\_problem\_type}
(see Section~\ref{typeprob}).
On entry, it is used to hold data about the solved reduced problem. The values
of {\tt problem\%X\_status} and {\tt problem\%C\_status} should not have been
altered since the exit from {\tt \packagename\_apply}.

\noindent
On exit, the problem structure will contain selected components of the solved
problem restored to the original variables/constraints and/or matrix format.
The selection of these elements is specified by setting the {\tt get\_*}
components of the {\tt control} dummy argument (see
Section~\ref{typecontrol}).

\itt{control} is a scalar \intentinout\ argument of type
{\tt \packagename\_control\_type}
(see Section~\ref{typecontrol}).  In particular, its {\tt get\_*} components
specify which elements of the (solved) reduced problem must be restored to the
original formulation.

\itt{inform} is a scalar \intentinout\ argument of type
{\tt \packagename\_inform\_\_type}
(see Section~\ref{typeinform}). A successful call to the routine
{\tt \packagename\_restoire}
is indicated when the  component {\tt status} has the value 0.
For other return values of {\tt status}, see Section~\ref{galerrors}.

\itt{data} is a scalar \intentinout\ argument of type
{\tt \packagename\_data\_type}
(see Section~\ref{typedata}). It is used to hold data about the problem being
solved. It must not have been altered by the user since the last call to
any of the {\tt \packagename} routines.

\end{description}

%%%%%%% termination subroutine %%%%%%

\subsubsection{The  termination subroutine}
All previously allocated workspace arrays are deallocated as follows:
\vspace*{1mm}

\hspace{8mm}
{\tt CALL \packagename\_terminate( control, inform, data )}

\begin{description}
\itt{control} is a scalar \intentin\ argument of type
{\tt \packagename\_control\_type}
exactly as for
{\tt \packagename\_initialize}.

\itt{inform} is a scalar \intentout\ argument of type
{\tt \packagename\_inform\_type}
exactly as for
{\tt \packagename\_initialize}.
A successful call to {\tt \packagename\_terminate}
is indicated when the component {\tt status} has the value 0.
For other return values of {\tt status}, see Section~\ref{galerrors}.

\itt{data} is a scalar \intentinout\ argument of type
{\tt \packagename\_data\_type}
exactly as for
{\tt \packagename\_solve},
which must not have been altered by the user since the last call to
{\tt \packagename\_initialize}.
On exit, array components will have been deallocated.
\end{description}

\noindent
Note that a call to this routine is mandatory before {\tt \packagename\_apply}
is called for a new quadratic program whose structure differs from the current
one.

%%%%%%%%%%%%%%%%%%%%%% Warning and error messages %%%%%%%%%%%%%%%%%%%%%%%%

\galerrors
A negative value of {\tt info\%status} on exit from
{\tt \packagename\_initialize},
{\tt \packagename\_read\_specfile},
{\tt \packagename\_apply},
{\tt \packagename\_restore},
or
{\tt \packagename\_terminate}
indicates that an error has occurred. No further calls should be made
to the four three of these routines until the error has been
corrected. Possible values are:

\begin{description}

\itt{~~1.} The maximum number of problem transformation has been reached.
Note that this exit is not really an error, since the problem
can  nevertheless be permuted and  solved.  It merely
signals that further problem reduction could possibly
be obtained with a larger value of the parameter
{\tt max\_nbr\_transforms}.

\itt{~-1.} A workspace allocation failed.

\itt{~-2.} A file intended for saving problem transformations could not be
opened.

\itt{~-3.} An input-output error occurred while saving transformations on
the relevant disk file.

\itt{~-4.} The machine/compiler only supports less than 8 bits in a
single integer (this error is thus very unlikely).

\itt{-21.} The problem appears to be primal infeasible.

\itt{-22.} The problem appears to be dual infeasible.

\itt{-23.} The dimension of the gradient {\tt problem\%G} is not equal to
the number of variables in the problem {\tt problem\%n}.

\itt{-24.} The dimension of the vector {\tt problem\%H\_val} containing the
entries of the Hessian is erroneously specified.

\itt{-25.} The dimension of the vector {\tt problem\%H\_ptr} containing the
addresses of the first entry of each Hessian row is erroneously specified.

\itt{-26.} The dimension of the vector {\tt problem\%H\_col} containing the
column  indices of the nonzero Hessian entries is erroneously specified.

\itt{-27.} The dimension of the vector {\tt problem\%H\_row} containing the
row indices of the nonzero Hessian entries is erroneously specified.

\itt{-28.} The dimension of the vector {\tt problem\%A\_val} containing the
entries of the Jacobian is erroneously specified.

\itt{-29.} The dimension of the vector {\tt problem\%A\_ptr} containing the
addresses of the first entry of each Jacobian row is erroneously specified.

\itt{-30.} The dimension of the vector {\tt problem\%A\_col} containing
the column indices of the nonzero Jacobian entries is erroneously specified.

\itt{-31.} The dimension of the vector {\tt problem\%A\_row} containing the
row indices of the nonzero Jacobian entries is erroneously specified;

\itt{-32.} The dimension of the vector {\tt problem\%X} of variables is
incompatible with the problem dimension {\tt problem\%n}.

\itt{-33.} The dimension of the vector {\tt problem\%X\_l} of lower bounds on
the variables is incompatible with the problem dimension {\tt problem\%n}.

\itt{-34.} The dimension of the vector {\tt problem\%X\_u} of upper bounds on
the variables is incompatible with the problem dimension {\tt problem\%n}.

\itt{-35.} The dimension of the vector {\tt problem\%Z} of dual variables is
incompatible with the problem dimension {\tt problem\%n}.

\itt{-36.} The dimension of the vector {\tt problem\%Z\_l} of lower bounds on
the dual variables is incompatible with the problem dimension {\tt problem\%n}.

\itt{-37.} The dimension of the vector {\tt problem\%Z\_u} of upper bounds on
the dual variables is incompatible with the problem dimension {\tt problem\%n}.

\itt{-38.} The dimension of the vector {\tt problem\%C} of constraints values is
incompatible with the problem dimension {\tt problem\%m}.

\itt{-39.} The dimension of the vector {\tt problem\%C\_l} of lower bounds on
the constraints is incompatible with the problem dimension {\tt problem\%m}.

\itt{-40.} The dimension of the vector {\tt problem\%C\_u} of upper bounds on
the constraints is incompatible with the problem dimension {\tt problem\%m}.

\itt{-41.} The dimension of the vector {\tt problem\%Y} of multipliers values is
incompatible with the problem dimension {\tt problem\%m}.

\itt{-42.} The dimension of the vector {\tt problem\%Y\_l} of lower bounds on
the multipliers is incompatible with the problem dimension {\tt problem\%m}.

\itt{-43.} The dimension of the vector {\tt problem\%Y\_u} of upper bounds on
the multipliers is incompatible with the problem dimension {\tt problem\%m}.

\itt{-44.} The problem structure has not been set or has been
cleaned up before an attempt to apply {\tt \packagename\_apply}.

\itt{-45.} The problem has not been analyzed before an attempt to permute it.

\itt{-46.} The problem has not been permuted or fully reduced
before an attempt to restore it.

\itt{-47.} The column indices of a row of the sparse Hessian are
not in increasing order, in that they specify an entry above the diagonal.

\itt{-48.} One of the files containing saved problem transformations has been
corrupted between  writing and reading.

\itt{-49.} The dimension of the vector {\tt problem\%X\_status} of
variables' status is incompatible with the problem dimension {\tt problem\%n}.

\itt{-50.} The dimension of the vector {\tt problem\%C\_status} of
constraints' status is incompatible with the problem dimension
{\tt problem\%m}.

\itt{-52.} The problem does not contain any (active) variable ({\tt problem\%n}
$\leq 0$).

\itt{-53.} The problem contains a negative number of constraints ({\tt
problem\%m} $< 0$).

\itt{-54.} The vectors are too long for the quicksort sorting routine (see the
{\tt GALAHAD SORT} module).

\itt{-55.} The value of a variable that is obtained in {\tt
\packagename\_restore} by substitution from a constraint is incoherent with
the variable's bounds.  This may be due to a relatively loose accuracy on the
linear constraints. Try to increase {\tt control\%c\_accuracy}.

\itt{-56.} The value of a constraint that is obtained by
recomputing its value on input of {\tt \packagename\_restore} from the
current $\bmx$ is incompatible with its declared value
or its bounds. This may caused the restored problem to be infeasible.

\itt{-57.} The value of a dual variable that is obtained by
recomputing its value on input of {\tt \packagename\_restore} (assuming
dual feasibility) from the current values of $( \bmx, \bmy, \bmz )$ is
incompatible with its declared value. This may caused the restored problem to
be infeasible or suboptimal.

\itt{-58.} A dual variable whose value is nonzero because the
corresponding primal is at an artificial bound cannot
be zeroed while maintaining dual feasibility
(in {\tt \packagename\_restore}). This can happen when $( \bmx, \bmy, \bmz )$
on  input of this routine are not (sufficiently) optimal.

\itt{-60.} A keyword was not recognized in the analysis of the specification
file.

\itt{-61.} A value was not recognized in the analysis of the specification file.

\itt{-63.} The vector {\tt problem\%G} has not been allocated although it
has general values.

\itt{-65.} The vector {\tt problem\%A\_val} has not been allocated although
{\tt problem\%m} $> 0$.

\itt{-66.} The vector {\tt problem\%A\_ptr} has not been allocated although
{\tt problem\%m} $> 0$ and $\bmA$ is stored in row-wise sparse format.

\itt{-67.} The vector {\tt problem\%A\_col} has not been allocated although
{\tt problem\%m} $> 0$ and $\bmA$ is stored in row-wise sparse format
or sparse coordinate format.

\itt{-68.} The vector {\tt problem\%A\_row} has not been allocated although
{\tt problem\%m} $> 0$ and $\bmA$ is stored in sparse coordinate format.

\itt{-69.} The vector {\tt problem\%H\_val} has not been allocated although
{\tt problem\%H\_ne} $= -2$ or {\tt problem\%H\_ne} $= -1$ or {\tt
problem\%H\_ne} $> 0$.

\itt{-70.} The vector {\tt problem\%H\_ptr} has not been allocated although
$\bmH$ is stored in row-wise sparse format.

\itt{-71.} The vector {\tt problem\%H\_col} has not been allocated although
$\bmH$ is stored in row-wise sparse format or {\tt problem\%H\_ne} $> 0$ and
$\bmH$ is stored sparse coordinate format.

\itt{-72.} The vector {\tt problem\%H\_row} has not been allocated although
{\tt problem\%H\_ne} $> 0$ and $\bmH$ is stored in sparse coordinate format.

\itt{-73.} The value of {\tt problem\%A\_ne} is erroneously specified.

\itt{-74.} The value of {\tt problem\%H\_ne} is erroneously specified.

\end{description}

%%%%%%%%%%%%%%%%%%%%%% Further features %%%%%%%%%%%%%%%%%%%%%%%%

\galfeatures
\noindent In this section, we describe an alternative means of setting
control parameters, that is components of the variable {\tt control} of type
{\tt \packagename\_control\_type}
(see Section~\ref{typecontrol}),
by reading an appropriate data specification file using the
subroutine {\tt \packagename\_read\_specfile}. This facility
is useful as it allows a user to change  {\tt \packagename} control parameters
without editing and recompiling programs that call {\tt \packagename}.

A specification file, or specfile, is a data file containing a number of
"specification commands". Each command occurs on a separate line,
and comprises a "keyword",
which is a string (in a close-to-natural language) used to identify a
control parameter, and
an (optional) "value", which defines the value to be assigned to the given
control parameter. All keywords and values are case insensitive,
keywords may be preceded by one or more blanks but
values must not contain blanks, and
each value must be separated from its keyword by at least one blank.
Values must not contain more than 30 characters, and
each line of the specfile is limited to 80 characters,
including the blanks separating keyword and value.

The portion of the specification file used by {\tt \packagename\_read\_specfile}
must start with a "{\tt BEGIN \packagename}" command and end with an
"{\tt END}" command.  The syntax of the specfile is thus defined as follows:
\begin{verbatim}
  ( .. lines ignored by PRESOLVE_read_specfile .. )
    BEGIN PRESOLVE
       keyword    value
       .......    .....
       keyword    value
    END
  ( .. lines ignored by PRESOLVE_read_specfile .. )
\end{verbatim}
where {\tt keyword} and {tt value} are two strings separated by (at least) one
blank. The ``{\tt BEGIN \packagename}'' and ``{\tt END}'' delimiter command
lines  may contain additional (trailing) strings so long as such strings are
separated by one or more blanks, so that lines such as
\begin{verbatim}
    BEGIN PRESOLVE SPECIFICATION
\end{verbatim}
and
\begin{verbatim}
    END PRESOLVE SPECIFICATION
\end{verbatim}
are acceptable. Furthermore, between the
``{\tt BEGIN \packagename}'' and ``{\tt END}'' delimiters,
specification commands may occur in any order.  Blank lines and
lines whose first non-blank character is {\tt !} or {\tt *} are ignored.
The content of a line after a {\tt !} or {\tt *} character is also
ignored (as is the {\tt !} or {\tt *}
character itself). This provides an easy manner to "comment out" some
specification commands, or to comment specific values
of certain control parameters.

The value of a control parameters may be of five different types, namely
integer, logical, real, string or symbol.
Integer and real values may be expressed in any relevant Fortran integer and
floating-point formats (respectively). Permitted values for logical
parameters are "{\tt ON}", "{\tt TRUE}", "{\tt .TRUE.}", "{\tt T}",
"{\tt YES}", "{\tt Y}", or "{\tt OFF}", "{\tt NO}",
"{\tt N}", "{\tt FALSE}", "{\tt .FALSE.}" and "{\tt F}".
Empty values are also allowed for logical control parameters, and are
interpreted as "{\tt TRUE}". String are specified as a sequence of characters.
%A symbolic value is a special string obtained from one of the predefined
%symbols of the SYMBOLS module by deleting the leading {\tt GALAHAD\_}
%characters in its name. Thus, the specification command
%\begin{verbatim}
%   termination REDUCED_SIZE
%\end{verbatim}
%im plies that the value {\sym \galsymreducedsize} is assigned to {\tt
%control\%termination}.  This technique is intended to help expressing an
%(integer) control parameter for an algorithm  in a "language" that is close to
%natural (see Section~\ref{galsymbols}).

The specification file must be open for input when {\tt
\packagename\_read\_specfile} is called, and the associated device number
passed to the routine in device (see below).  Note that the corresponding
file is {\tt REWIND}ed, which makes it possible to combine the specifications
for more than one program/routine.  For the same reason, the file is not
closed by {\tt \packagename\_read\_specfile}.

\subsubsection{To read control parameters from a specification file}
\label{readspec}

Control parameters may be read from a file as follows:
\hskip0.5in
\def\baselinestretch{0.8} {\tt \begin{verbatim}
     CALL PRESOLVE_read_specfile( device, control, inform )
\end{verbatim}}
\def\baselinestretch{1.0}

\begin{description}
\itt{device} is a scalar \intentin argument of type default \integer,
that must be set to the unit number on which the specfile
has been opened. If {\tt device} is not open, {\tt control} will
not be altered and execution will continue, but an error message
will be printed on unit {\tt control\%error}.

\itt{control} is a scalar \intentinout argument of type
{\tt \packagename\_control\_type} (see Section~\ref{typecontrol}).
Default values should have already been set, perhaps by calling
{\tt \packagename\_initialize}. On exit, individual components of {\tt
control} may have been changed according to the commands found in the
specfile. Specfile commands and  the component (see Section~\ref{typecontrol})
of {\tt control}  that each affects are given in
Tables~\ref{specfile_1} and \ref{specfile_2}.

\itt{inform} is a scalar \intentout\ argument of type
{\tt \packagename\_inform\_type} (see Section~\ref{typeinform}).
\end{description}

\bctable{|l|l|l|}
\hline
  command & component of {\tt control} & value type    \\
\hline
{\tt error-printout-device} & {\tt \%error} & integer \\
{\tt printout-device} & {\tt \%out} & integer \\
{\tt print-level} & {\tt \%print\_level} & integer \\
{\tt presolve-termination-strategy} &  {\tt \%termination} & integer \\
{\tt maximum-number-of-transformations} &
     {\tt \%max\_nbr\_transforms} & integer\\
{\tt maximum-number-of-passes} & {\tt \%max\_nbr\_passes} & integer \\
{\tt constraints-accuracy} & {\tt \%c\_accuracy} & real \\
{\tt dual-variables-accuracy} & {\tt \%z\_accuracy} & real \\
{\tt allow-dual-transformations} & {\tt \%dual\_transformations} & logical \\
{\tt remove-redundant-variables-constraints} & {\tt \%redundant\_xc}& logical \\
{\tt primal-constraints-analysis-frequency} &
     {\tt \%primal\_constraints\_freq} & integer \\
{\tt dual-constraints-analysis-frequency} &
     {\tt \%dual\_constraints\_freq} & integer \\
{\tt singleton-columns-analysis-frequency} &
     {\tt \%singleton\_columns\_freq} & integer \\
{\tt doubleton-columns-analysis-frequency} &
     {\tt \%doubleton\_columns\_freq} & integer \\
{\tt unconstrained-variables-analysis-frequency} &
     {\tt \%unc\_variables\_freq} & integer \\
{\tt dependent-variables-analysis-frequency} &
     {\tt \%dependent\_variables\_freq} & integer \\
{\tt row-sparsification-frequency} & {\tt \%sparsify\_rows\_freq} & integer \\
{\tt maximum-percentage-row-fill} & {\tt \%max\_fill} & integer \\
{\tt transformations-buffer-size} & {\tt \%transf\_buffer\_size} & integer \\
{\tt transformations-file-device} & {\tt \%transf\_file\_nbr} & integer \\
{\tt transformations-file-status} & {\tt \%transf\_file\_status} & integer \\
{\tt transformations-file-name} & {\tt \%transf\_file\_name} & string \\
{\tt primal-feasibility-check} & {\tt \%check\_primal\_feasibility} & integer \\
{\tt dual-feasibility-check} & {\tt \%check\_dual\_feasibility} & integer \\
{\tt active-multipliers-sign} & {\tt \%y\_sign} & integer \\
{\tt inactive-multipliers-value} & {\tt \%inactive\_y} & integer \\
{\tt active-dual-variables-sign} & {\tt \%z\_sign} & integer \\
{\tt inactive-dual-variables-value} & {\tt \%inactive\_z} & integer \\
{\tt primal-variables-bound-status} & {\tt \%final\_x\_bounds} & integer \\
\hline
\ectable{\label{specfile_1}Specfile commands and associated
components of {\tt control} .}
\bctable{|l|l|l|}
\hline
  command & component of {\tt control} & value type  \\
\hline
{\tt dual-variables-bound-status} & {\tt \%final\_z\_bounds} & integer \\
{\tt constraints-bound-status} & {\tt \%final\_c\_bounds} & integer \\
{\tt multipliers-bound-status} & {\tt \%final\_y\_bounds} & integer \\
{\tt infinity-value} & {\tt \%infinity} & real \\
{\tt pivoting-threshold} & {\tt \%pivot\_tol} & real \\
{\tt minimum-relative-bound-improvement} &
     {\tt \%min\_rel\_improve} & real \\
{\tt maximum-growth-factor}      & {\tt \%max\_growth\_factor} & real \\
{\tt compute-quadratic-value}    & {\tt \%get\_q} & logical \\
{\tt compute-objective-constant} & {\tt \%get\_f} & logical \\
{\tt compute-gradient}           & {\tt \%get\_g} & logical \\
{\tt compute-Hessian}            & {\tt \%get\_H} & logical \\
{\tt compute-constraints-matrix} & {\tt \%get\_A} & logical \\
{\tt compute-primal-variables-values} & {\tt \%get\_x} & logical \\
{\tt compute-primal-variables-bounds} & {\tt \%get\_x\_bounds} & logical \\
{\tt compute-dual-variables-values} & {\tt \%get\_z} & logical \\
{\tt compute-dual-variables-bounds} & {\tt \%get\_z\_bounds} & logical \\
{\tt compute-contraints-values} & {\tt \%get\_c} & logical \\
{\tt compute-constraints-bounds} & {\tt \%get\_c\_bounds} & logical \\
{\tt compute-multipliers-values} & {\tt \%get\_y} & logical \\
{\tt compute-multipliers-bounds} & {\tt \%get\_y\_bounds} & logical \\
\hline
\ectable{\label{specfile_2}Specfile commands and associated
components of {\tt control} (continued).}

%%%%%%%%%%%%%%%%%%%%%% Information printed %%%%%%%%%%%%%%%%%%%%%%%%

\galinfo
The level of output produced by values of {\tt control\%print\_level} is as
follows:
\begin{description}
\item[\sym \galsymsilent.] no printout is produced,
\item[\sym \galsymtrace.] only the major steps in the analysis is
 reported, that is headers of the main preprocessing phases and, for each pass,
  the number of transformations of each type applied,
\item[\sym \galsymaction.] in addition, reports the nature of each problem
 transformation,
\item[\sym \galsymdetails.] in addition, reports more details on each of the
  main presolve loops
constituents,
\item[\sym \galsymdebug.] reports considerable detail, including information on
unsuccessful attempts to apply presolving transformations,
\item[\sym \galsymcrazy.] reports a completely silly amount of information.
\end{description}

%%%%%%%%%%%%%%%%%%%%%% GENERAL INFORMATION %%%%%%%%%%%%%%%%%%%%%%%%

\galgeneral

\galcommon None.
\galworkspace Provided automatically by the module.
\galroutines None.
\galmodules {\tt \packagename} calls the \galahad\ modules
{\tt GALAHAD\_SMT}, {\tt GALAHAD\_QPT},
{\tt GALAHAD\_SPECFILE}, {\tt GALAHAD\_SORT},
and {\tt GALAHAD\_SYMBOLS}.
\galio Output is under control of the arguments
 {\tt control\%error}, {\tt control\%out} and {\tt control\%print\_level}.
\galrestrictions {\tt prob\%n} $> 0$, {\tt prob\%m} $\geq  0$,
{\tt prob\%A\_type} and {\tt prob\%H\_type} $\in \{${\tt 'DENSE'},
 {\tt 'COORDINATE'}, {\tt 'SPARSE\_BY\_ROWS'}$\}$.
\galportability ISO Fortran~95 + TR 15581 or Fortran~2003.
The package is thread-safe.

%%%%%%%%%%%%%%%%%%%%%% METHOD %%%%%%%%%%%%%%%%%%%%%%%%

\galmethod
The required solution $\bmx$ of the problem necessarily satisfies
the primal optimality conditions
\[
\bmA \bmx = \bmc
\]
and
\[
\bmc^{l} \leq \bmc \leq \bmc^{u}, \;\;
\bmx^{l} \leq \bmx \leq \bmx^{u},
\]
the dual optimality conditions
\[
 \bmH \bmx + \bmg =  \bmA^{T} \bmy + \bmz, \;\;
 \bmy = \bmy^{l} + \bmy^{u} \tim{and}
 \bmz = \bmz^{l} + \bmz^{u},
\]
and
\[
 \bmy^{l} \geq 0 , \;\;
 \bmy^{u} \leq 0 , \;\;
 \bmz^{l} \geq 0 \;\; \mbox{and} \;\;
 \bmz^{u} \leq 0 ,
\]
and the complementary slackness conditions
\[
( \bmA \bmx - \bmc^{l} )^{T} \bmy^{l} = 0,  \;\;
( \bmA \bmx - \bmc^{u} )^{T} \bmy^{u} = 0,  \;\;
(\bmx -\bmx^{l} )^{T} \bmz^{l} = 0   \tim{and}
(\bmx -\bmx^{u} )^{T} \bmz^{u} = 0,
\]
where the vectors $\bmy$ and $\bmz$ are known as the Lagrange multipliers for
the general linear constraints, and the dual variables for the bounds,
respectively, and where the vector inequalities hold componentwise.
The purpose of presolving is to exploit these equations in order to reduce the
problem to the standard form defined as follows:
\begin{itemize}
\item The variables are ordered so that their bounds appear in the order
\begin{center}
\begin{tabular}{lccccc}
free                &          &        & $\bmx$ &        &          \\
non-negativity      &   0      & $\leq$ & $\bmx$ &        &          \\
lower               & $\bmx^l$ & $\leq$ & $\bmx$ &        &          \\
range               & $\bmx^l$ & $\leq$ & $\bmx$ & $\leq$ & $\bmx^u$ \\
upper               &          &        & $\bmx$ & $\leq$ & $\bmx^u$ \\
non-positivity      &          &        & $\bmx$ & $\leq$ &      0   \\
\end{tabular}
\end{center}

Fixed variables are removed. Within each category, the variables
are further ordered so that those with non-zero diagonal Hessian
entries occur before the remainder.

\item
The constraints are ordered so that their bounds appear in the order
\begin{center}
\begin{tabular}{lccccc}
non-negativity      &     0    & $\leq$ & $\bmA \bmx$ &        &          \\
equality            & $\bmc^l$ &   $=$  & $\bmA \bmx$ &        &          \\
lower               & $\bmc^l$ & $\leq$ & $\bmA \bmx$ &        &          \\
range               & $\bmc^l$ & $\leq$ & $\bmA \bmx$ & $\leq$ & $\bmc^u$ \\
upper               &          &        & $\bmA \bmx$ & $\leq$ & $\bmc^u$ \\
non-positivity      &          &        & $\bmA \bmx$ & $\leq$ &     0    \\
\end{tabular}
\end{center}
Free constraints are removed.

\item
In addition, constraints may be removed or bounds tightened, to reduce the
size of the feasible region or simplify the problem if this is possible, and
bounds may be tightened on the dual variables and the multipliers
associated  with the problem.
\end{itemize}

The presolving algorithm proceeds by applying a (potentially long) series of
simple transformations to the problem, each transformation introducing a
further simplification of the problem. These involve the removal of empty and
singleton rows, the removal of redundant and forcing primal constraints, the
tightening of primal and dual bounds, the exploitation of linear singleton,
linear doubleton and linearly unconstrained columns, the merging dependent
variables, row sparsification and split equalities. Transformations are
applied in successive passes, each pass involving the following actions:
\begin{enumerate}
\item remove empty and singletons rows,
\item try to eliminate variables that are linearly unconstrained,
\item attempt to exploit the presence of linear singleton columns,
\item attempt to exploit the presence of linear doubleton columns,
\item complete the analysis of the dual constraints,
\item remove empty and singletons rows,
\item possibly remove dependent variables,
\item analyze the primal constraints,
\item try to make $A$ sparser by combining its rows,
\item check the current status of the variables, dual variables
      and multipliers.
\end{enumerate}
All these transformations are applied to the structure of the original
problem, which is only permuted to standard form after all transformations are
completed. \emph{Note that the Hessian and Jacobian of the resulting reduced
problem are always stored in sparse row-wise format.} The reduced problem is
then solved by a quadratic or linear programming solver, thus ensuring
sufficiently small primal-dual feasibility and complementarity. Finally, the
solution of the simplified problem is re-translated in the
variables/constraints/format of the original problem formulation by a
``restoration'' phase.

If the number of problem transformations exceeds {\tt
control\%transf\_buffer\_size},  the transformation buffer size (see
Section~\ref{typecontrol}), then they are saved in a ``history'' file, whose
name may be chosen by specifying the {\tt control\%transf\_file\_name} control
parameter (see Section~\ref{typecontrol}).  When this is the case, this file
is subsequently reread by {\tt \packagename\_restore}. It must not be altered
by the user.

At the overall level, the presolving process follows one of the two
sequences:

\vspace*{0.2cm}

\noindent
\fwbox{initialize}
$\rightarrow$ $\bigg[$ \fwbox{apply transformations}
$\rightarrow$ (solve problem)
$\rightarrow$ \fwbox{restore}  $\bigg]$
$\rightarrow$ \fwbox{terminate}
\vspace*{0.2cm}

\noindent or

\vspace*{0.2cm}
\noindent
\fwbox{initialize}
$\rightarrow$ $\bigg[$ \fwbox{read specfile}
$\rightarrow$ \fwbox{apply transformations}
$\rightarrow$ (solve problem)
$\rightarrow$ \fwbox{restore}  $\bigg]$
$\rightarrow$ \fwbox{terminate}
\vspace*{0.2cm}

\noindent
where the procedure's control parameter may be modified by reading the
specfile (see Section~\ref{galfeatures}), and where (solve problem) indicates
that the reduced problem is solved. Each of the ``boxed'' steps in these
sequences corresponds to calling a specific routine of the package (see
Section~\ref{galarguments}). In the above diagrams, brackated subsequence of
steps means that they can be repeated with problem having the same
structure. The value of the {\tt new\_problem\_structure} component of {\tt
problem} must be {\tt .TRUE.} on entry of {\tt \packagename\_apply} on the
first time it is used in this repeated subsequence. Such a subsequence must be
terminated by a call to {\tt \packagename\_terminate} before presolving is
applied to a problem with a different structure.

Note that the values of the multipliers and dual variables (and thus of
their respective bounds) depend on the functional form assumed for the
Lagrangian function associated with the problem.  This form is given by
\[
L( \bmx, \bmy, \bmz ) =
q( \bmx ) - {\tt y\_sign} * \bmy^T ( \bmA \bmx - \bmc ) - {\tt z\_sign} * \bmz,
\]
(considering only active constraints $\bmA \bmx = \bmc$), where the parameters
{\tt y\_sign} and {\tt z\_sign} are +1 or -1 and can be chosen by the user.
Thus, if {\tt y\_sign = +1}, the multipliers associated to active constraints
originally posed as inequalities are non-negative if the inequality is a lower
bound and non-positive if it is an upper bound. Obvioulsy they are not
constrained in sign for constraints originally posed as equalities. These
sign conventions are reversed if {\tt y\_sign = -1}.
Similarly, if {\tt z\_sign = +1}, the dual variables associated to active
bounds are non-negative if the original bound is an lower bound, non-positive
if it is an upper bound, or unconstrained in sign if the variables is fixed;
and this convention is reversed in {\tt z\_sign = -1}. The values of {\tt
z\_sign} and {\tt y\_sign} may be chosen by setting the corresponding
components of the {\tt control} structure to {\sym \galsympositive} or {\sym
\galsymnegative} (see Section~\ref{typecontrol}).

\vspace*{1mm}

\galreferences
\vspace*{1mm}

\noindent
The algorithm is described in more detail in
\vspace*{1mm}

\noindent
N. I. M. Gould and Ph. L. Toint (2004).
Presolving for quadratic programming.
Mathematical Programming {\bf 100}(1), pp 95--132.
\vspace*{1mm}

%%%%%%%%%%%%%%%%%%%%%% EXAMPLE %%%%%%%%%%%%%%%%%%%%%%%%

%=========================

\galexample
Suppose that we wish to solve the quadratic program
(\ref{objqp})--(\ref{vbqp}) with the data $n = 6$, $m = 5$, $f = 1$,
$\bmg = ( 1 \; 1 \; 1 \; 1 \; 1 \; 1)^T$,
\[
\bmH =\left(\begin{array}{cccccc}
        1 & 0 & 0 & 0 & 0 & 0 \\
        0 & 0 & 0 & 0 & 0 & 0 \\
        0 & 0 & 0 & 0 & 0 & 0 \\
        0 & 0 & 0 & 0 & 0 & 0 \\
        0 & 0 & 0 & 0 & 0 & 0 \\
        0 & 0 & 0 & 0 & 0 & 0 \\
        \end{array}\right),
\hspace*{2cm}
\bmA = \left(\begin{array}{cccccc}
        0 & 0 & 0 & 0 & 0 & 0 \\
        0 & 0 & 0 & 0 & 0 & 0 \\
        0 & 0 & 1 & 1 & 1 & 0 \\
        0 & 0 & 1 & 0 & 0 & 1 \\
        0 & 0 & 0 & 1 & 1 & 1 \\
        \end{array}\right),
\]
$\bmx^l = ( 0\; 0\; 0\; 0\; 0\; 0 )^T$,
$\bmx^u = ( 1 \; 1 \; 1 \; 1 \; 1 \; 1 )^T$,
$\bmc^l = ( 0 \; 0 \; 2 \; 1 \; 3 )^T$ and
$\bmc^u = ( 1 \; 1 \; 3 \; 3 \; 3 )^T$,
using the quadratic programming solver {\tt QPSOLVER} after applying
the {\tt \packagename} package and then restoring the final solution to the
original
variable formulation. We may use the following code---note that we require
some output from {\tt \packagename} by setting {\tt control\%print\_level} to
{\sym \galsymtrace}, and that calling {\tt QPSOLVER} is actually
unnecessary since the problem of our example is completely reduced to a single
feasible point (which must then be the solution) after presolving.
{\tt \small
\begin{verbatim}
   PROGRAM GALAHAD_PRESOLVE_EXAMPLE
   USE GALAHAD_QPT_double                           ! Double precision
   USE GALAHAD_PRESOLVE_double                      ! Double precision
   USE GALAHAD_SYMBOLS                              ! The GALAHAD symbols
   IMPLICIT NONE
   INTEGER, PARAMETER :: wp = KIND( 1.0D0 )         ! Set precision
   REAL ( KIND = wp ), PARAMETER  :: infinity = 10._wp ** 20
   REAL ( KIND = wp ), PARAMETER  :: r0 = 0.0_wp, r1 = 1.0_wp
   REAL ( KIND = wp ), PARAMETER  :: r2 = 2.0_wp, r3 = 3.0_wp
   TYPE ( QPT_problem_type )      :: problem
   TYPE ( PRESOLVE_control_type ) :: control
   TYPE ( PRESOLVE_inform_type )  :: inform
   TYPE ( PRESOLVE_data_type )    :: data
   INTEGER :: j, n, m, a_ne, h_ne
! start problem data
   n = 6; m = 5; h_ne = 1; a_ne = 8
   problem%new_problem_structure = .TRUE.
   problem%n = n; problem%m = m; problem%f = r1
   ALLOCATE( problem%G( n )  , problem%X_l( n ), problem%X_u( n ) )
   ALLOCATE( problem%C_l( m ), problem%C_u( m ) )
   problem%gradient_kind = 1
   problem%C_l = (/  r0, r0, r2, r1, r3  /)
   problem%C_u = (/  r1, r1, r3, r3, r3  /)
   problem%X_l = (/ -r3, r0, r0, r0, r0, r0 /)
   problem%X_u = (/  r3, r1, r1, r1, r1, r1 /)
! sparse coordinate format
   CALL SMT_put( problem%H%type, 'COORDINATE' )
   CALL SMT_put( problem%A%type, 'COORDINATE' )
   ALLOCATE( problem%H%val( h_ne ) )
   ALLOCATE( problem%H%col( h_ne ), problem%H%row( h_ne ) )
   ALLOCATE( problem%A%val( a_ne ) )
   ALLOCATE( problem%A%col( a_ne ), problem%A%row( a_ne ) )
   problem%H%val = (/ r1  /)
   problem%H%row = (/  1  /)
   problem%H%col = (/  1  /)
   problem%A%val = (/ r1, r1, r1, r1, r1, r1, r1, r1 /)
   problem%A%row = (/  3,  3,  3,  4,  4,  5,  5,  5 /)
   problem%A%col = (/  3,  4,  5,  3,  6,  4,  5,  6 /)
   problem%a_ne  = a_ne; problem%h_ne = h_ne
! problem data complete
! write the original formulation
   CALL QPT_write_problem( 6, problem )
! set the default PRESOLVE control parameters
   CALL PRESOLVE_initialize( control, inform, data )
   IF ( inform%status /= 0 ) STOP
   control%print_level = 1  ! Ask for some output
! apply presolving to reduce the problem
   CALL PRESOLVE_apply( problem, control, inform, data )
   IF ( inform%status /= 0 ) STOP
! write the reduced problem
   CALL QPT_write_problem( 6, problem )
! solve the reduced problem
   ! CALL QPSOLVER (unnecessary here, because the reduced problem has a
   ! single feasible point in this example)
! restore the solved reduced problem to the original formulation
   CALL PRESOLVE_restore( problem, control, inform, data )
   IF ( inform%status /= 0 ) STOP
! write the final solution in the original variables
   WRITE( 6, "( /, '  The problem solution X is', /, )" )
   DO j = 1, n
      WRITE( 6, '(3x,''x('',I1,'') = '', ES12.4)' ) j, problem%X( j )
   END DO
! deallocate internal workspace
   CALL PRESOLVE_terminate( control, inform, data )
   END PROGRAM GALAHAD_PRESOLVE_EXAMPLE
\end{verbatim}
}
\noindent
This produces the following output:
{\tt \small
\begin{verbatim}

    =============== PROBLEM =====================

    n =  6

    variables

                    lower       upper

   x(   1) =    -3.0000E+00  3.0000E+00
   x(   2) =     0.0000E+00  1.0000E+00
   x(   3) =     0.0000E+00  1.0000E+00
   x(   4) =     0.0000E+00  1.0000E+00
   x(   5) =     0.0000E+00  1.0000E+00
   x(   6) =     0.0000E+00  1.0000E+00

    m =  5

    constraints

                    lower       upper

   c(   1) =     0.0000E+00  1.0000E+00
   c(   2) =     0.0000E+00  1.0000E+00
   c(   3) =     2.0000E+00  3.0000E+00
   c(   4) =     1.0000E+00  3.0000E+00
   c(   5) =     3.0000E+00  3.0000E+00

    Jacobian

   A(   3,   3) =   1.0000E+00
   A(   3,   4) =   1.0000E+00
   A(   3,   5) =   1.0000E+00
   A(   4,   3) =   1.0000E+00
   A(   4,   6) =   1.0000E+00
   A(   5,   4) =   1.0000E+00
   A(   5,   5) =   1.0000E+00
   A(   5,   6) =   1.0000E+00


    objective function constant term =   1.0000E+00


    gradient

   g(   1) =     1.0000E+00
   g(   2) =     1.0000E+00
   g(   3) =     1.0000E+00
   g(   4) =     1.0000E+00
   g(   5) =     1.0000E+00
   g(   6) =     1.0000E+00

    Hessian

   H(   1,   1) =   1.0000E+00

    ============ END OF PROBLEM =================


  ********************************************
  *                                          *
  *         GALAHAD presolve for QPs         *
  *                                          *
  *            problem analysis              *
  *                                          *
  ********************************************


  ============ starting problem analysis ============

  checking bounds on x, y, z, and c: 0 transformations
  redundant variables and constraints: 0 transformations

  ============= main processing loop 1  =============
    ( n = 6 , m = 5 , a_ne = 8 , h_ne = 1 )

  removing empty and singleton rows: 2 transformations
  analyzing special linear columns: 3 transformations
  analyzing dual constraints: 0 transformations
  removing empty and singleton rows: 0 transformations
  checking dependent variables: 2  transformations
  analyzing primal constraints: 5 transformations
  checking bounds on x, y, z, and c: 0 transformations

  ============= main processing loop 2  =============
    ( n = 1 , m = 2 , a_ne = 2 , h_ne = 0 )

  removing empty and singleton rows: 2 transformations
  analyzing special linear columns: 2 transformations

  ======== end of the main processing loop ( loop = 2 ) ========

  all variables and constraints have been eliminated!

  No permutation necessary.

  ******************** Bye *******************


    =============== PROBLEM =====================

    n =  0

    m =  0

    current objective function value =   3.5000E+00


    objective function constant term =   3.5000E+00


    ============ END OF PROBLEM =================


  ********************************************
  *                                          *
  *         GALAHAD PRESOLVE for QPs         *
  *                                          *
  *            problem restoration           *
  *                                          *
  ********************************************

  verifying user-defined presolve control parameters
  ===  starting historical loop
  ===  end of the historical loop

  Problem successfully restored.

  ******************** Bye *******************


   The problem solution X is

   x(1) =  -1.0000E+00
   x(2) =   0.0000E+00
   x(3) =   0.0000E+00
   x(4) =   1.0000E+00
   x(5) =   1.0000E+00
   x(6) =   1.0000E+00

  ********************************************
  *                                          *
  *         GALAHAD PRESOLVE for QPs         *
  *                                          *
  *            workspace cleanup             *
  *                                          *
  ********************************************



  ******************** Bye *******************

\end{verbatim}
}
\noindent
The same problem may be solved holding the data in
a sparse row-wise storage format by replacing the lines
{\tt \small
\begin{verbatim}
! sparse coordinate format
  ......
! problem data complete
\end{verbatim}
}
\noindent
by
{\tt \small
\begin{verbatim}
! sparse row-wise storage format
   CALL SMT_put( problem%H%type, 'SPARSE_BY_ROWS' )
   CALL SMT_put( problem%A%type, 'SPARSE_BY_ROWS' )
   ALLOCATE( problem%H%val( h_ne ) )
   ALLOCATE( problem%H%ptr( n+1 ), problem%H%col( h_ne ) )
   ALLOCATE( problem%A%val( a_ne ) )
   ALLOCATE( problem%A%ptr( m+1 ), problem%A%col( a_ne ) )
   problem%H%val = (/ r1  /)
   problem%H%ptr = (/  1,  2,  2,  2,  2,  2,  2  /)
   problem%H%col = (/  1  /)
   problem%A%val = (/ r1, r1, r1, r1, r1, r1, r1, r1 /)
   problem%A%ptr = (/  1,  1,  1,  4,  6,  9  /)
   problem%A%col = (/  3,  4,  5,  3,  6,  4,  5,  6 /)
! problem data complete
\end{verbatim}
}
\noindent
or using a dense storage format with the replacement lines
{\tt \small
\begin{verbatim}
! dense storage format
   CALL SMT_put( problem%H%type, 'DENSE' )
   CALL SMT_put( problem%A%type, 'DENSE' )
   ALLOCATE( problem%H%val( n*(n+1)/2 ) )
   ALLOCATE( problem%A%val( n*m ) )
   problem%H%val = (/ r1,                           &
                      r0, r0,                       &
                      r0, r0, r0,                   &
                      r0, r0, r0, r0,               &
                      r0, r0, r0, r0, r0,           &
                      r0, r0, r0, r0, r0, r0  /)
   problem%A%val = (/ r0, r0, r0, r0, r0, r0,       &
                      r0, r0, r0, r0, r0, r0,       &
                      r0, r0, r1, r1, r1, r0,       &
                      r0, r0, r1, r0, r0, r1,       &
                      r0, r0, r0, r1, r1, r1  /)
! problem data complete
\end{verbatim}
}
\noindent
respectively.
(If instead $\bmH$ had been the diagonal matrix
\disp{\bmH = \mat{ccc}{1 & &   \\ & 0 & \\  &  & 3}}
but the other data is as before, the diagonal storage scheme
might be used for $\bmH$, and in this case we would instead
{\tt \small
\begin{verbatim}
   CALL SMT_put( prob%H%type, 'DIAGONAL' )  ! Specify dense storage for H
   ALLOCATE( p%H%val( n ) )
   p%H%val = (/ r1, r0, r0, r0, r0, r0, r0 /) ! Hessian values
\end{verbatim}
}
\noindent
Notice here that zero diagonal entries are stored.)
We could also make use of the {\tt \packagename\_read\_specfile} routine to
set the printing level, in which case the statement
{\small
\begin{verbatim}
   control%print_level = GALAHAD_TRACE              ! Ask for some output
\end{verbatim}
}
\noindent
is replaced by
{\small
\begin{verbatim}
! open specfile
   OPEN( 57, FILE = 'PRESOLVE.SPC', STATUS = 'OLD' )
! read its content (asking for some output)
   CALL PRESOLVE_read_specfile( 57, control, inform )
! close it
   CLOSE( 57 )
\end{verbatim}
}
\noindent
where we assume that the file {\tt PRESOLVE.SPC} exists in the current
directory and contains the lines
{\small
\begin{verbatim}
BEGIN PRESOLVE SPECIFICATION
   print-level 1
END PRESOLVE SPECIFICATION
\end{verbatim}
}
\end{document}
