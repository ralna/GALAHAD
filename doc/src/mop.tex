\documentclass{galahad}

% set the release and package names

\newcommand{\package}{mop}
\newcommand{\packagename}{MOP}
\newcommand{\fullpackagename}{\libraryname\_\packagename}

\begin{document}

\galheader

%%%%%%%%%%%%%%%%%%%%%% SUMMARY %%%%%%%%%%%%%%%%%%%%%%%%

\galsummary

{\tt \fullpackagename} is a suite of Fortran~90 procedures for
{\bf performing operations on/with a matrix $\mathbf{A}$} of derived data type {\tt
  SMT\_type} (Section~\ref{typesmt}), which allows for multiple storage
types~(Section~\ref{galmatrix}).  In particular, this package contains the following
subroutines for a given $m$ by $n$ matrix $\bmA$:
\begin{itemize}
  \item subroutine {\tt mop\_Ax} computes matrix-vector products of
    the form
    \disp{\bmr \gets \alpha \bmA\bmx + \beta \bmr \quad {\rm and}
          \quad
          \bmr \gets \alpha \bmA^T\bmx + \beta \bmr}
    for given scalars $\alpha$ and $\beta$, and vectors $\bmx
    \in \Re^n$ and $\bmr \in \Re^m$;
  \item subroutine {\tt mop\_getval} obtains the $(i,j)$-element of
    the matrix $\bmA$ for given integers $i$ and $j$;
  \item subroutine {\tt mop\_scaleA} scales the rows of $\bmA$ by a
    given vector $\bmu\in\Re^m$ and the columns by a vector $\bmv\in\Re^n$.
\end{itemize}

%%%%%%%%%%%%%%%%%%%%%% attributes %%%%%%%%%%%%%%%%%%%%%%%%

\galattributes
\galversions{\tt  \fullpackagename\_single, \fullpackagename\_double},
\galuses {\tt GALAHAD\_SMT\_double.}
\galdate November 2009.
\galorigin N. I. M. Gould, Rutherford Appleton Laboratory, and
D. P. Robinson, University of Oxford, UK.
\gallanguage Fortran~90. 

%%%%%%%%%%%%%%%%%%%%%% HOW TO USE %%%%%%%%%%%%%%%%%%%%%%%%

\galhowto

The package is available with single, double and (if available) 
quadruple precision reals, and either 32-bit or 64-bit integers. 
Access to the 32-bit integer,
single precision version requires the {\tt USE} statement
\medskip

\hspace{8mm} {\tt USE \fullpackagename\_single}

\medskip
\noindent
with the obvious substitution 
{\tt \fullpackagename\_double},
{\tt \fullpackagename\_quadruple},
{\tt \fullpackagename\_single\_64},
{\tt \fullpackagename\_double\_64} and
{\tt \fullpackagename\_quadruple\_64} for the other variants.


\noindent
If it is required to use more than one of the modules at the same time, 
the derived types
{\tt SMT\_type} (Section~\ref{typesmt}) 
and the subroutines 
{\tt \packagename\_Ax}, {\tt \packagename\_getval}, and 
{\tt \packagename\_scaleA} (Sections~\ref{Ax}, \ref{getval},
\ref{scaleA}) must be renamed on one of the {\tt USE} statements.

%%%%%%%%%%% Matrix Storage Formats %%%%%%%%%%%

\galmatrix

The matrix $\bmA$ may be stored in a variety of input formats.

\subsubsection{Dense storage format}\label{dense}
The matrix $\bmA$ is stored as a compact 
dense matrix by rows, that is, the values of the entries of each row in turn are
stored in order within an appropriate real one-dimensional array.
Component $n \ast (i-1) + j$ of the storage array {\tt A\%val} will hold the 
value $a_{ij}$ for $i = 1, \ldots , m$, $j = 1, \ldots , n$.
If $\bmA$ is symmetric, only the lower triangular part (that is the part 
$a_{ij}$ for $1 \leq j \leq i \leq n$) should be stored.  In this case
the lower triangle will be stored by rows, that is 
component $i \ast (i-1)/2 + j$ of the storage array {\tt A\%val}  
will hold the value $a_{ij}$ (and, by symmetry, $a_{ji}$)
for $1 \leq j \leq i \leq n$.

\subsubsection{Sparse co-ordinate storage format}\label{coordinate}
Only the nonzero entries of the matrices are stored. For the $l$-th
entry of $\bmA$, its row index $i$, column index $j$ and value
$a_{ij}$ are stored in the $l$-th components of the integer arrays
{\tt A\%row}, {\tt A\%col} and real array {\tt A\%val}.  The order is
unimportant, but the total number of entries {\tt A\%ne} is also
required.  If $\bmA$ is symmetric, the same scheme is applicable,
except that only the entries in the lower triangle should be stored.

\subsubsection{Sparse row-wise storage format}\label{rowwise}
Again only the nonzero entries are stored, but this time they are
ordered so that those in row $i$ appear directly before those in row
$i+1$. For the $i$-th row of $\bmA$, the $i$-th component of a integer
array {\tt A\%ptr} holds the position of the first entry in this row,
while {\tt A\%ptr} $(m+1)$ holds the total number of entries plus one.
The column indices $j$ and values $a_{ij}$ of the entries in the
$i$-th row are stored in components $l =$ {\tt A\%ptr}$(i)$, \ldots
,{\tt A\%ptr} $(i+1)-1$ of the integer array {\tt A\%col}, and real
array {\tt A\%val}, respectively.  If $\bmA$ is symmetric, the same
scheme is applicable, except that only the entries in the lower
triangle should be stored.

For sparse matrices, this scheme almost always requires less storage than 
its predecessor.

\subsubsection{Sparse column-wise storage format}\label{columnwise}
Again only the nonzero entries are stored, but this time they are
ordered so that those in column $j$ appear directly before those in column
$j+1$. For the $j$-th column of $\bmA$, the $j$-th component of the integer
array {\tt A\%ptr} holds the position of the first entry in this column,
while {\tt A\%ptr} $(n+1)$ holds the total number of entries plus one.
The row indices $i$ and values $a_{ij}$ of the entries in the
$j$-th column are stored in components $l =$ {\tt A\%ptr}$(j)$, \ldots
,{\tt A\%ptr} $(j+1)-1$ of the integer array {\tt A\%row}, and real
array {\tt A\%val}, respectively.  If $\bmA$ is symmetric, the same
scheme is applicable, except that only the entries in the lower
triangle should be stored.

\subsubsection{Diagonal storage format}\label{diagonal}
If $\bmA$ is diagonal (i.e., $a_{ij} = 0$ for all $1 \leq i \neq j
\leq n$) only the diagonals entries $a_{ii}$ for $1 \leq i \leq n$ should
be stored, and the first $n$ components of the array {\tt A\%val}
should be used for this purpose.

%%%%%%%%%%%%%%%%%%%%%%%%%%% kinds %%%%%%%%%%%%%%%%%%%%%%%%%%%

%%%%%%%%%%%%%%%%%%%%%% kinds %%%%%%%%%%%%%%%%%%%%%%

\galkinds
We use the terms integer and real to refer to the fortran
keywords {\tt REAL(rp\_)} and {\tt INTEGER(ip\_)}, where 
{\tt rp\_} and {\tt ip\_} are the relevant kind values for the
real and integer types employed by the particular module in use. 
The former are equivalent to
default {\tt REAL} for the single precision versions,
{\tt DOUBLE PRECISION} for the double precision cases
and quadruple-precision if 128-bit reals are available, and 
correspond to {\tt rp\_ = real32}, {\tt rp\_ = real64} and
{\tt rp\_ = real128} respectively as defined by the 
fortran {\tt iso\_fortran\_env} module.
The latter are default (32-bit) and long (64-bit) integers, and
correspond to {\tt ip\_ = int32} and {\tt ip\_ = int64},
respectively, again from the {\tt iso\_fortran\_env} module.


%%%%%%%%%%% derived type for holding A %%%%%%%%%%%

\subsection{The derived data type for holding the matrix $\bmA$}\label{typesmt}
The matrix $\bmA$ is stored using the derived data type {\tt SMT\_type}
whose components are:

\begin{description}

\ittf{m} is a scalar component of type \integer, 
that holds the number of rows in the matrix. 
 
\ittf{n} is a scalar component of type \integer, 
that holds the number of columns in the matrix. 
 
\ittf{ne} is a scalar variable of type \integer, that
holds the number of matrix entries.

\ittf{type} is a rank-one allocatable array of type default \character, that
is used to indicate the matrix storage scheme used.  If the dense storage scheme 
(see Section~\ref{dense}) is used, 
the first five components of {\tt H\%type} must contain the
string {\tt DENSE}.
For the sparse co-ordinate scheme (see Section~\ref{coordinate}), 
the first ten components of {\tt H\%type} must contain the
string {\tt COORDINATE},  
for the sparse row-wise storage scheme (see Section~\ref{rowwise}),
the first fourteen components of {\tt H\%type} must contain the
string {\tt SPARSE\_BY\_ROWS},
for the sparse column-wise storage scheme (see Section~\ref{columnwise}),
the first seventeen components of {\tt H\%type} must contain the
string {\tt SPARSE\_BY\_COLUMNS},
and for the diagonal storage scheme (see Section~\ref{diagonal}),
the first eight components of {\tt H\%type} must contain the
string {\tt DIAGONAL}.

For convenience, the procedure {\tt SMT\_put} 
may be used to allocate sufficient space and insert the required keyword
into {\tt H\%type}.
For example, if we wish to store $\bmA$ using the co-ordinate scheme,
we may simply
%\vspace*{-2mm}
{\tt 
\begin{verbatim}
        CALL SMT_put( A%type, 'COORDINATE' )
\end{verbatim}
}
%\vspace*{-4mm}
\noindent
See the documentation for the \galahad\ package {\tt SMT} 
for further details on the use of {\tt SMT\_put}. 

\ittf{val} is a rank-one allocatable array of type \realdp\, 
and dimension at least {\tt ne}, that holds the values of the entries. 
Each pair of off-diagonal entries $a_{ij} = a_{ji}$ of a {\em symmetric}
matrix $\bmA$ is represented as a single entry 
(see \S\ref{dense}--\ref{diagonal}).

\ittf{row} is a rank-one allocatable array of type \integer, 
and dimension at least {\tt ne}, that may hold the row indices of the entries 
(see \S~\ref{coordinate} and \ref{columnwise}).

\ittf{col} is a rank-one allocatable array of type \integer, 
and dimension at least {\tt ne}, that may hold the column indices of the entries
(see \S\ref{coordinate}--\ref{rowwise}).

\ittf{ptr} is a rank-one allocatable array of type \integer.
If sparse row-wise storage is used, then {\tt ptr} should be of
dimension at least {\tt m + 1} and should contain pointers to the
first entry in each row (see \S\ref{rowwise}). If sparse column-wise
storage is used, then {\tt ptr} should be of dimension at least {\tt n
  + 1} and should contain pointers to the first entry in each column (see
\S\ref{columnwise}).

\end{description}

%%%%%%%%%%%%%%%%%%%%% Subroutine mop_Ax %%%%%%%%%%%%%%%%%%%%%%%%

%\subsection{Subroutine {\tt \packagename\_Ax}}\label{Ax}
\subsection{The subroutine to form matrix-vector products}\label{Ax}

The subroutine {\tt \packagename\_Ax} may be called to compute matrix
vector products with $\bmA$ of the form
\eqn{r1}{\bmr \gets \alpha \bmA \bmx + \beta \bmr}
or
\eqn{r2}{\bmr \gets \alpha \bmA^T \bmx + \beta \bmr}
by using
\vspace*{1mm}

\hspace{8mm}
{\tt CALL \packagename\_Ax( alpha, A, X, beta, R, [out, error,
                            print\_level, symmetric, transpose] )}

\vspace*{1mm}
\noindent where square brackets indicate \optional\ arguments. 
%\vspace*{-3mm}

\begin{description}
  \itt{alpha} is a scalar \intentin\ argument of type \realdp\ that
     must hold the value of $\alpha$.
  \itt{A} is a scalar \intentin\ argument of type {\tt SMT\_type} (see
     Section~\ref{typesmt}) that must hold the matrix $\bmA$.
  \itt{X} is a rank-one \intentin\ array of type \realdp\ that must
     contain the components of the vector $\bmx$.
  \itt{beta} is a scalar \intentin\ argument of type \realdp\ that
     must hold the value of $\beta$.
  \itt{R} is a rank-one \intentinout\ array of type \realdp\ that must
  contain the components of the vector $\bmr$. {\tt R} need not be set
  on entry if {\tt beta} is zero.
  \ittf{out} is an \optional\ scalar \intentin\ argument of type \integer, that holds the
     stream number for informational messages. If this argument is not
     provided, then the default value {\tt out = 6} is used.
  \itt{error} is an \optional\ scalar \intentin\ argument of type \integer, that holds the
     stream number for error messages. If this argument is not
     provided, then the default value {\tt error = 6} is used.
  \itt{print\_level} is an \optional\ scalar \intentin\ argument of type \integer, that is used
     to control the amount of informational output which is required. No 
     informational output will occur if {\tt print\_level} $\leq 0$. If 
     {\tt print\_level} $= 1$, minimal output will be produced and if
     {\tt print\_level} $\geq 2$ then output will be
     increased to provide full details.
     The default is {\tt print\_level = 0}.
  \itt{symmetric} is an \optional\ scalar \intentin\ argument of type default
     \logical\ that should be set \true\ if the matrix $\bmA$ is
     symmetric, and set \false\ otherwise.  If this argument is not
     provided, then the dafault value of \false\ is used. 
  \itt{transpose} is an \optional\ scalar \intentin\ argument of type default
     \logical\ that should be set \false\ if the user wishes to compute
     \req{r1}, an set \true\ if the user wishes to compute \req{r2}.
     If {\tt transpose} is not provided, then the dafault value of
     \false\ is used. 
\end{description}

\subsubsection{Warning and error messages}\label{Ax-error}

All warning and error messages will be printed on unit {\tt error} as
discussed in the previous section.

\subsubsection{Information printed}\label{Ax-info}

If {\tt print\_level} is positive, information about the calculation
will be printed on unit {\tt out} as discussed previously.  In
particular, if {\tt print\_level} $= 1$, then the values {\tt
  symmetric}, {\tt transpose}, {\tt A\%m}, {\tt A\%n}, {\tt A\%type},
{\tt A\%id}, {\tt alpha}, and {\tt beta} are printed.  If {\tt
  print\_level} $= 2$, then additionally {\tt A\%ptr}, {\tt A\%val},
{\tt A\%row}, and {\tt A\%col} are printed.  If {\tt print\_level}
$\geq 3$, then additionally the input {\tt X} and {\tt R} as well as
the result {\tt R} will be printed.

%%%%%%%%%%%%%%%%%%%%% Subroutine mop_getval %%%%%%%%%%%%%%%%%%%%%%%%

%\subsection{Subroutine {\tt \packagename\_getval}}\label{getval}
\subsection{The subroutine to get matrix values}\label{getval}

The subroutine {\tt \packagename\_getval} may be used to get the
$(i,j)$-th element of the matrix $\bmA$ by using
\vspace*{1mm}

\hspace{8mm}
{\tt CALL \packagename\_getval( A, row, col, val, [symmetric, out,
                                error, print\_level] )}

\vspace*{1mm}
\noindent where square brackets indicate \optional\ arguments. 
\vspace*{0mm}
\begin{description}
  \itt{A} is a scalar \intentin\ argument of type {\tt SMT\_type} (see
     Section~\ref{typesmt}) that must contain the matrix $\bmA$.
  \itt{row} is a scalar \intentin\ argument of type \integer\ that
     specifies the row index $i$ of the requested element of the
     matrix $\bmA$.
  \itt{col} is a scalar \intentin\ argument of type \integer\ that
     specifies the column index $j$ of the requested element of the
     matrix $\bmA$.
  \itt{val} is a scalar \intentout\ argument of type \realdp\ that
     holds the value of the $(i,j)$-th element of the matrix $\bmA$ on return.
  \itt{symmetric} is an \optional\ scalar \intentin\ argument of type default
     \logical\ that should be set \true\ if the matrix $\bmA$ is
     symmetric, and set \false\ otherwise.  If {\tt symmetric} is not
     provided, then the dafault value of \false\ is used. 
  \ittf{out} is an \optional\ scalar \intentin\ argument of type \integer, that holds the
     stream number for informational messages. If this argument is not
     provided, then the default value {\tt out = 6} is used.
  \itt{error} is an \optional\ scalar \intentin\ argument of type \integer, that holds the
     stream number for error messages. If this argument is not
     provided, then the default value {\tt error = 6} is used.
  \itt{print\_level} is an \optional\ scalar \intentin\ argument of type \integer, that is used
     to control the amount of informational output which is required. No 
     informational output will occur if {\tt print\_level} $\leq 0$. If 
     {\tt print\_level} $= 1$, minimal output will be produced and if
     {\tt print\_level} $\geq 2$ then output will be
     increased to provide full details.
     The default is {\tt print\_level = 0}.
\end{description}

\subsubsection{Warning and error messages}\label{getval-error}

All warning and error messages will be printed on unit {\tt error} as
discussed in the previous section.

\subsubsection{Information printed}\label{getval-info}

If {\tt print\_level} is positive, information about the subroutine
data will be printed on unit {\tt out} as discussed
previously.  In particular, if {\tt print\_level} $= 1$, then the
values {\tt A\%m}, {\tt A\%n}, {\tt A\%type}, {\tt
  A\%id}, {\tt row}, {\tt col}, and the resulting value {\tt val} are
printed.  If {\tt print\_level} $\geq 2$, then
additionally {\tt A\%ptr}, {\tt A\%val}, {\tt A\%row},
and {\tt A\%col} are printed.

%%%%%%%%%%%%%%%%%%%%% Subroutine mop_scaleA %%%%%%%%%%%%%%%%%%%%%%%%

%\subsection{Subroutine {\tt \packagename\_scaleA}}\label{scaleA}
\subsection{The subroutine to scale the matrix}\label{scaleA}

The subroutine {\tt \packagename\_scaleA} may be called to scale the
rows of the $m$ by $n$ matrix $\bmA$ by the vector $\bmu\in\Re^m$
and the columns by the vector $\bmv\in\Re^n$.  In other words, it
forms the scaled matrix whose $(i,j)$-th element is $u_ia_{i,j}v_j$.
This scaled matrix is stored in $\bmA$ on return.  If the \optional\
argument {\tt symmetric} is set \true\ , then the rows and columns of
$\bmA$ are scaled by the vector $\bmu$.  The calling
sequence is given by \vspace*{1mm}

\hspace{8mm}
{\tt CALL \packagename\_scaleA( A, [u, v, out, error,
                                print\_level, symmetric] )}

\vspace*{1mm}
\noindent where square brackets indicate \optional\ arguments. 

\vspace*{-1mm}
\begin{description}
  \itt{A} is a scalar \intentinout\ argument of type {\tt SMT\_type} (see
     Section~\ref{typesmt}) that must contain the matrix $\bmA$.
  \itt{u} is an \optional\ rank-one \intentin\ argument of type \realdp\ of length
     {\tt A\%m} whose $i$-th component is used to scale the $i$-th
     row of the matrix $\bmA$.
  \itt{v} is an \optional\ rank-one \intentin\ argument of type \realdp\ of length
     {\tt A\%n} whose $i$-th component is used to scale the $i$-th
     column of the matrix $\bmA$.
  \ittf{out} is an \optional\ scalar \intentin\ argument of type \integer, that holds the
     stream number for informational messages. If this argument is not
     provided, then the default value {\tt out = 6} is used.
  \itt{error} is an \optional\ scalar \intentin\ argument of type \integer, that holds the
     stream number for error messages. If this argument is not
     provided, then the default value {\tt error = 6} is used.
  \itt{print\_level} is an \optional\ scalar \intentin\ argument of type \integer, that is used
     to control the amount of informational output which is required. No 
     informational output will occur if {\tt print\_level} $\leq 0$. If 
     {\tt print\_level} $= 1$, minimal output will be produced and if
     {\tt print\_level} $\geq 2$ then output will be
     increased to provide full details.
     The default is {\tt print\_level = 0}.
  \itt{symmetric} is an \optional\ scalar \intentin\ argument of type default
     \logical\ that should be set \true\ if the matrix $\bmA$ is
     symmetric, and set \false\ otherwise.  If {\tt symmetric} is not
     provided, then the dafault value of \false\ is used. 
\end{description}

\subsubsection{Warning and error messages}\label{scaleA-error}

All warning and error messages will be printed on unit {\tt error} as
discussed in the previous section.

\subsubsection{Information printed}\label{scaleA-info}

If {\tt print\_level} is positive, information about the arguments
will be printed on unit {\tt out} as discussed previously.  In
particular, if {\tt print\_level} $\geq 1$, then the values {\tt
  A\%m}, {\tt A\%n}, {\tt A\%type}, {\tt A\%id},
{\tt A\%ptr}, {\tt A\%val}, {\tt A\%row}, {\tt
  A\%col}, {\tt u} and {\tt v} will be printed.

% %%%%%%%%%%%%%%%%%%%%%% GENERAL INFORMATION %%%%%%%%%%%%%%%%%%%%%%%%

\galgeneral % There is a \begin{description} in here!

\galcommon None.
\galworkspace Provided automatically by the module.
\galroutines None. 
\galmodules The \galahad\ package {\tt GALAHAD\_SMT} is used by the
    subroutines {\tt \packagename\_Ax}, {\tt \packagename\_getval},
    and {\tt \packagename\_scaleA}.
\galio Output is provided under the control of the \optional\ input
     arguments {\tt print\_level}, {\tt out}, and {\tt error}.  The
     argument {\tt print\_level} controls the amount of information
     printed to the device with unit number {\tt out}; all error
     messages will be printed to the device with unit number {\tt
       error}.  If the user does not supply any of these optional
     arguments, then the default values {\tt print\_level} $= 0$,
     {\tt out} $= 6$, and {\tt error} $= 6$ are used. 
\galrestrictions {\tt A\%n} $> 0$, {\tt A\%m} $> 0$, and 
     {\tt A\%type} $\in \{${\tt 'DENSE'}, 
     {\tt 'COORDINATE'}, {\tt 'SPARSE\_BY\_ROWS'}, {\tt 'SPARSE\_BY\_COLUMNS'}
      {\tt 'DIAGONAL'} $\}$. 
\galportability ISO Fortran 90.
\end{description}

%%%%%%%%%%%%%%%%%%%%%% EXAMPLE %%%%%%%%%%%%%%%%%%%%%%%%

\galexample
Suppose we wish to perform the following operations.  We first compute
\disp{ \bmr \gets \alpha \bmA \bmx + \beta \bmr}
where
\disp{\bmA   = \mat{ccc}{1 & 2 & 3 \\ 4 & 5 & 6}, \quad
      \bmx   = \mat{c}{1 \\ 1 \\ 1}, \quad
      \bmr   = \mat{c}{3 \\ 3}, \quad
      \alpha = 3, \quad {\rm and} \quad
      \beta  = 2.}
Next we scale the rows of $\bmA$ by the vector $\bmu$ and columns of $\bmA$ by the
vector $\bmv$, where
\disp{\bmu = \mat{c}{\phantom{-}2 \\ -1} \quad {\rm and} \quad
      \bmv = \mat{c}{3 \\ 1 \\ 2}.}
In other words, we over-write the matrix $\bmA$ with the scaled matrix
whose $(i,j)$th element is $u_ia_{i,j}v_j$.  Finally, we retrieve the
$(1,2)$ element of the scaled matrix.  

We may use the following code
{\tt \small
\VerbatimInput{\packageexample}
}
%\noindent
%with the following data
%{\tt \small
%\VerbatimInput{\packagedata}
%}
\noindent
This produces the following output:
{\tt \small
\VerbatimInput{\packageresults}
}
\noindent

\end{document}
