\documentclass{galahad}

% set the package name

\newcommand{\package}{rand}
\newcommand{\packagename}{RAND}
\newcommand{\fullpackagename}{\libraryname\_\packagename}

\begin{document}

\galheader

%%%%%%%%%%%%%%%%%%%%%% SUMMARY %%%%%%%%%%%%%%%%%%%%%%%%

\galsummary

{\tt \fullpackagename} is a suite of Fortran procedures for
generating {\bf uniformly distributed pseudo-random
 numbers. } Random reals are generated in the range $0 < \xi < 1$ or
 the range $-1 < \eta < 1$ and random integers in the range
 $1 \leq k \leq N$ where $N$ is specified by the user.

A multiplicative congruent method is used where a 31 bit generator
 word $g$ is maintained. On each call to a procedure of the package,
 $g_{n+1} $ is updated to $7^5 g_n mod (2^{31} - 1)$; the
 initial value of $g$ is $2^{16} - 1$.
 Depending upon the type of random number
 required the following are computed $\xi = g_{n+1} /(2^{31} - 1)$;
 $\eta = 2 \xi - 1$ or $k = \mbox{integer part}\, \xi N +1$.

The package also provides the facility for saving the current
 value of the generator word and for restarting with any specified
 value.


%%%%%%%%%%%%%%%%%%%%%% attributes %%%%%%%%%%%%%%%%%%%%%%%%

\galattributes
\galversions{\tt  \fullpackagename\_single, \fullpackagename\_double},
\galuses None.
\galdate March 2001.
\galorigin
N. I. M. Gould and J. K. Reid, Rutherford Appleton Laboratory.
\gallanguage Fortran~95 + TR 15581 or Fortran~2003.

%%%%%%%%%%%%%%%%%%%%%% HOW TO USE %%%%%%%%%%%%%%%%%%%%%%%%

\galhowto

Access to the package requires a {\tt USE} statement such as

\medskip\noindent{\em Single precision version}

\hskip0.5in {\tt USE \fullpackagename\_single}

\medskip\noindent{\em Double precision version}

\hskip0.5in {\tt USE  \fullpackagename\_double}

\medskip
If it is required to use both modules at the same time, the derived type
{\tt \packagename\_seed}
(Section~\ref{galtypes})
and the subroutines{\tt \packagename\_random\_real},
{\tt \packagename\_random\_integer},
{\tt \packagename\_get\_seed},
and {\tt \packagename\_set\_seed}
(Section~\ref{galarguments})
must be renamed on one of the {\tt USE} statements.
Their seeds will be independent.

%%%%%%%%%%%%%%%%%%%%%% derived types %%%%%%%%%%%%%%%%%%%%%%%%

\galtypes
The user must provide a variable of derived type
{\tt \packagename\_seed}
to hold the current seed value which must be passed to all calls of
{\tt \packagename}.
The seed value component is private and can only be set and retrieved
through the
{\tt \packagename\_set\_seed}  and {\tt \packagename\_get\_seed} entries.

%%%%%%%%%%%%%%%%%%%%%% argument lists %%%%%%%%%%%%%%%%%%%%%%%%

\galarguments
There are five procedures for user calls.
%When using the Fortran 90 version
The initialization entry must be called before any call to the
{\tt \packagename\_random\_real},
{\tt \packagename\_random\_integer} and
{\tt \packagename\_get\_seed}  entries.

%%%%%% Subroutine to initialize the generator word %%%%%%

\subsubsection{Subroutine to initialize the generator word}\label{subinit}
This entry must be called first to initialize the generator word.
\vspace*{1mm}

\hskip 0.5in
{\tt CALL \packagename\_initialize( seed )}

\vspace*{-2mm}
\begin{description}
\ittf{seed} is a scalar \intentout\ argument of derived type
{\tt \packagename\_seed}
that holds the seed value.
\end{description}

%%%%%%%%% Subroutine to obtain a random real value %%%%%%

\subsubsection{Subroutine to obtain random real values}
A random real value or values may be obtained as follows:
\vspace*{1mm}

\hskip 0.5in
{\tt CALL \packagename\_random\_real( seed, positive, random\_real )}

\vspace*{-2mm}
\begin{description}
\ittf{seed} is a scalar \intentinout\ argument of derived type
{\tt \packagename\_seed} that holds the seed value.

\itt{positive} is a scalar \intentin\ argument of type default
\logical. If {\tt positive} is \true,
the generated random number is a real value in the range $0 < \xi < 1$,
while if {\tt positive} is \false, the generated random number
is a real value in the range  $-1 < \eta < 1$.

\itt{random\_real} is a scalar or rank 1 or 2 array
\intentout\ argument of type \realdp.
It is set to the required random number(s).

\end{description}

%%%%%%% Subroutine to obtain a random integer value %%%%%%

\subsubsection{Subroutine to obtain random integer values}
A random integer value or values may be obtained as follows:
\vspace*{1mm}

\hskip0.5in
{\tt CALL \packagename\_random\_integer( seed, n, random\_integer )}

\vspace*{-2mm}
\begin{description}
\itt{seed} is a scalar \intentinout\ argument of derived type
{\tt \packagename\_seed}  that holds the seed value.

\ittf{n} is a scalar \intentin\ argument of type default \integer.
It must be  set by the user to specify the upper bound for the range
$1 \leq k \leq ${\tt n}  within which the generated random number(s) $k$
is/are required to lie.
{\bf Restriction:}  {\tt n} must be positive.

\itt{random\_integer} is a scalar  or rank 1 or 2 array
\intentout\ argument of type default  \integer.
It is set to the required random integer $k$
or an array of such integers.

\end{description}

%%%%%%% Subroutine to obtain the current generator word %%%%%%

\subsubsection{Subroutine to obtain the current generator word}
The current generator word may be obtained as follows:
\vspace*{1mm}

\hskip0.5in
{\tt CALL \packagename\_get\_seed( seed, value )}

\vspace*{-2mm}
\begin{description}
\itt{seed} is a scalar \intentin\ argument of derived type
{\tt \packagename\_seed} that must be provided to hold the seed value.

\itt{value} is a scalar \intentout\ argument of type default
{\tt INTEGER}.
It is set to the current value of the generator \linebreak word $g$.
\end{description}

%%%%%%% Subroutine to reset the current value of the generator word %%%%%%

\subsubsection{Subroutine to reset the current value of the generator word}
The current value of the generator word may be reset as follows:
\vspace*{1mm}

\hskip0.5in
{\tt CALL \packagename\_set\_seed( seed, value )}

\vspace*{-2mm}
\begin{description}
\itt{seed} is a scalar \intentout\ argument of derived type
{\tt \packagename\_seed}  that holds the seed value.

\itt{value} is a scalar \intentin\ argument of type default
{\tt INTEGER} that
 must be set by the user to the required value of the generator word. It
 is recommended that the value
 should have been obtained by a previous call of {\tt \packagename\_get\_seed}.
 It should have a value in the range 0 $<$ {\tt value} $\leq$ {\tt P},
 where {\tt P} =
 $2^{31}  - 1$  = 2147483647. If it is outside this range, the value
 {\tt value} mod($2^{31}  - 1$) is used.

\end{description}

%%%%%%%%%%%%%%%%%%%%%% GENERAL INFORMATION %%%%%%%%%%%%%%%%%%%%%%%%

\galgeneral

\galcommon None.
\galworkspace None.
\galroutines None.
\galmodules None.
\galio None.
\galrestrictions {\tt n} $> 0$.
\galportability ISO Fortran~95 + TR 15581 or Fortran~2003.
The package is thread-safe.

%%%%%%%%%%%%%%%%%%%%%% METHOD %%%%%%%%%%%%%%%%%%%%%%%%

\galmethod
The code is based on that of L.Schrage, ``A More
 Portable Fortran Random Number Generator'', TOMS, {\bf 5}(2) June 1979.
 The method employed is a multiplicative congruential method. The
 generator word $g$ is held as an integer and is updated on each call
 as follows
 \disp{g_{n+1} = 7^5 g_n mod (2^{31} - 1)}
 The result returned from {\tt \packagename\_random\_real},
 for a non-negative argument, is $\xi$, where
 \disp{\xi = g_{n+1} /(2^{31} - 1)}
 and for a negative argument is
 \disp{2 \xi - 1.}
 The value of $k$ returned by {\tt \packagename\_random\_integer} is
 \disp{\mbox{integer part}\, \xi N  +1.}
Arrays or random reals and integers are formed by calling the above
sequentially in Fortran column order.

%%%%%%%%%%%%%%%%%%%%%% EXAMPLE %%%%%%%%%%%%%%%%%%%%%%%%

\galexample
Suppose we wish to generate two random real numbers lying between plus
and minus one, reset the generator word to its original value, and then
find two positive random integers with values no larger than one hundred.
Then we might use the following piece of code.
{\tt \small
\VerbatimInput{\packageexample}
}
This produces the following output:
{\tt \small
\VerbatimInput{\packageresults}
}

\end{document}

