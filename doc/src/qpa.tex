\documentclass{galahad}

% set the package name

\newcommand{\package}{qpa}
\newcommand{\packagename}{QPA}
\newcommand{\fullpackagename}{\libraryname\_\packagename}

\begin{document}

\galheader

%%%%%%%%%%%%%%%%%%%%%% SUMMARY %%%%%%%%%%%%%%%%%%%%%%%%

\galsummary
This package uses a working-set method
to solve the {\bf $\mathbf{\ell_1}$ quadratic programming problem}
\eqn{1.1}{\minin{\bmx \in \smallRe^n} q(\bmx) + \rho_g v_g(\bmx) +
\rho_b v_b(\bmx)}
involving the quadratic objective
\disp{q(\bmx) = \half \bmx^T \bmH \bmx + \bmg^T \bmx + f}
and the infeasibilities
\disp{v_g(\bmx) = 
   \sum_{i=1}^{m} \max ( c_{i}^{l} - \bma_i^T \bmx, 0 )
 + \sum_{i=1}^{m} \max ( \bma_i^T \bmx - c_{i}^{u}, 0 )}
and
\disp{v_b(\bmx) = 
   \sum_{j=1}^{n} \max ( x_{j}^{l} - x_{j}  , 0 )
 + \sum_{j=1}^{n} \max ( x_{j}  - x_{j}^{u} , 0 ) ,}
where the $n$ by $n$ symmetric matrix $\bmH$, the 
vectors $\bmg$, $\bma_{i}$, $\bmc^{l}$, $\bmc^{u}$, $\bmx^{l}$, $\bmx^{u}$ 
and the scalars $f$, $\rho_g$ and $\rho_b$ are given.
Full advantage is taken of any zero coefficients in the matrix $\bmH$ or the
vectors $\bma_{i}$. 
Any of the constraint bounds $c_{i}^{l}$, $c_{i}^{u}$, 
$x_{j}^{l}$ and $x_{j}^{u}$ may be infinite.

The package may also be used to solve the {\bf quadratic programming problem}
\eqn{1.2}{\minin{\bmx \in \smallRe^n} q(\bmx),}
subject to the general linear constraints
\eqn{1.3}{c_{i}^{l}  \leq  \bma_{i}^{T}\bmx  \leq  c_{i}^{u}, \;\;\; 
 i = 1, \ldots , m,}
and the simple bound constraints
\eqn{1.4}{x_{j}^{l}  \leq  x_{j}^{ } \leq  x_{j}^{u} , \;\;\; 
 j = 1, \ldots , n,}
by automatically adjusting the parameters $\rho_g$ and $\rho_b$ in \req{1.1}.
Similarly, the package is capable of solving the 
{\bf bound-constrained $\mathbf{\ell_1}$ quadratic programming problem}
\eqn{1.5}{\minin{\bmx \in \smallRe^n} q(\bmx) + \rho_g v_g(\bmx),}
subject to the simple bound constraints \req{1.4},
by automatically adjusting $\rho_b$ in \req{1.1}.

If the matrix $\bmH$ is positive semi-definite, a global
solution is found. However, if $\bmH$ is indefinite,
the procedure may find  a (weak second-order) critical point 
that is not the global solution to the given problem.

{\bf N.B.} In many cases, the alternative \galahad\ quadratic 
programming package {\tt GALAHAD\_QPB} is faster, and thus 
to be preferred.

%%%%%%%%%%%%%%%%%%%%%% attributes %%%%%%%%%%%%%%%%%%%%%%%%

\galattributes
\galversions{\tt  \fullpackagename\_single, \fullpackagename\_double}.
\galuses 
%{\tt GALAHAD\_CPU\_time},
{\tt GALAHAD\_\-CLOCK},
{\tt GALAHAD\_SY\-M\-BOLS}, 
{\tt GALAHAD\-\_NORMS}, 
{\tt GALAHAD\-\_SPACE}, 
{\tt GALAHAD\_RA\-ND}, 
{\tt GALAHAD\_TOOLS}, 
{\tt GALAHAD\_ROOTS}, 
{\tt GALAHAD\_SORT},
{\tt GALAHAD\_\-SMT}, 
{\tt GALAHAD\-\_QPT},
{\tt GALAHAD\_QPP},
{\tt GALAHAD\_QPD},
{\tt GALAHAD\_SLS}, 
{\tt GALAHAD\_SPECFILE}, 
{\tt GALAHAD\_SCU}.
\galdate October 2001.
\galorigin N. I. M. Gould, Rutherford Appleton Laboratory, and
Ph. L. Toint, University of Namur, Belgium.
\gallanguage Fortran~95 + TR 15581 or Fortran~2003. 
\galparallelism Some options may use OpenMP and its runtime library.

%%%%%%%%%%%%%%%%%%%%%% HOW TO USE %%%%%%%%%%%%%%%%%%%%%%%%

\galhowto

%\subsection{Calling sequences}

Access to the package requires a {\tt USE} statement such as

\medskip\noindent{\em Single precision version}

\hspace{8mm} {\tt USE \fullpackagename\_single}

\medskip\noindent{\em Double precision version}

\hspace{8mm} {\tt USE  \fullpackagename\_double}

\medskip

\noindent
If it is required to use both modules at the same time, the derived types 
{\tt SMT\_type}, 
{\tt QPT\_problem\_type}, 
{\tt \packagename\_time\_type}, 
{\tt \packagename\_control\_type}, 
{\tt \packagename\_inform\_type} 
and
{\tt \packagename\_data\_type}
(Section~\ref{galtypes})
and the subroutines
{\tt \packagename\_initialize}, 
{\tt \packagename\_\-solve},
{\tt \packagename\_terminate},
(Section~\ref{galarguments})
and 
{\tt \packagename\_read\_specfile}
(Section~\ref{galfeatures})
must be renamed on one of the {\tt USE} statements.

%%%%%%%%%%%%%%%%%%%%%% matrix formats %%%%%%%%%%%%%%%%%%%%%%%%

\galmatrix
Both the Hessian matrix $\bmH$ and
the constraint Jacobian $\bmA$, the matrix 
whose rows are the vectors $\bma_{i}^{T}$, $i = 1, \ldots , m$, 
may be stored in a variety of input formats.

\subsubsection{Dense storage format}\label{dense}
The matrix $\bmA$ is stored as a compact 
dense matrix by rows, that is, the values of the entries of each row in turn are
stored in order within an appropriate real one-dimensional array.
Component $n \ast (i-1) + j$ of the storage array {\tt A\%val} will hold the 
value $a_{ij}$ for $i = 1, \ldots , m$, $j = 1, \ldots , n$.
Since $\bmH$ is symmetric, only the lower triangular part (that is the part 
$h_{ij}$ for $1 \leq j \leq i \leq n$) need be held. In this case
the lower triangle will be stored by rows, that is 
component $i \ast (i-1)/2 + j$ of the storage array {\tt H\%val}  
will hold the value $h_{ij}$ (and, by symmetry, $h_{ji}$)
for $1 \leq j \leq i \leq n$.

\subsubsection{Sparse co-ordinate storage format}\label{coordinate}
Only the nonzero entries of the matrices are stored. For the 
$l$-th entry of $\bmA$, its row index $i$, column index $j$ 
and value $a_{ij}$
are stored in the $l$-th components of the integer arrays {\tt A\%row}, 
{\tt A\%col} and real array {\tt A\%val}. 
The order is unimportant, but the total
number of entries {\tt A\%ne} is also required. 
The same scheme is applicable to
$\bmH$ (thus requiring integer arrays {\tt H\%row}, {\tt H\%col}, a real array 
{\tt H\%val} and an integer value {\tt H\%ne}),
except that only the entries in the lower triangle need be stored.

\subsubsection{Sparse row-wise storage format}\label{rowwise}
Again only the nonzero entries are stored, but this time
they are ordered so that those in row $i$ appear directly before those
in row $i+1$. For the $i$-th row of $\bmA$, the $i$-th component of a 
integer array {\tt A\%ptr} holds the position of the first entry in this row,
while {\tt A\%ptr} $(m+1)$ holds the total number of entries plus one.
The column indices $j$ and values $a_{ij}$ of the entries in the $i$-th row 
are stored in components 
$l =$ {\tt A\%ptr}$(i)$, \ldots ,{\tt A\%ptr} $(i+1)-1$ of the 
integer array {\tt A\%col}, and real array {\tt A\%val}, respectively. 
The same scheme is applicable to
$\bmH$ (thus requiring integer arrays {\tt H\%ptr}, {\tt H\%col}, and 
a real array {\tt H\%val}),
except that only the entries in the lower triangle need be stored.

For sparse matrices, this scheme almost always requires less storage than 
its predecessor.

\subsubsection{Diagonal storage format}\label{diagonal}
If $\bmH$ is diagonal (i.e., $h_{ij} = 0$ for all $1 \leq i \neq j \leq n$)
only the diagonals entries $h_{ii}$, $1 \leq i \leq n$,  need be stored,
and the first $n$ components of the array {\tt H\%val} may be used for 
the purpose. There is no sensible equivalent for the non-square $\bmA$.

%%%%%%%%%%%%%%%%%%%%%% long integers %%%%%%%%%%%%%%%%%%%%%%

\subsection{Integer kinds}\label{Integer kinds}
We use the term
long \integer\ to denote {\tt INTEGER\-(kind=long)}, where 
{\tt long = selected\_int\_kind(18))}.

%%%%%%%%%%%%%%%%%%%%%% OpenMP usage %%%%%%%%%%%%%%%%%%%%%%%%

\subsection{OpenMP}
OpenMP may be used by the {\tt \fullpackagename} package to provide 
parallelism for some solver options in shared memory environments.  
See the documentation for the \galahad\ package {\tt SLS} for more details.
To run in parallel, OpenMP 
must be enabled at compilation time by using the correct compiler flag 
(usually some variant of {\tt -openmp}). 
The number of threads may be controlled at runtime
by setting the environment variable {\tt OMP\_NUM\_THREADS}.

\noindent
The code may be compiled and run in serial mode.

%%%%%%%%%%%%%%%%%%%%%% derived types %%%%%%%%%%%%%%%%%%%%%%%%

\galtypes
Six derived data types are accessible from the package.

%%%%%%%%%%% matrix data type %%%%%%%%%%%

\subsubsection{The derived data type for holding matrices}\label{typesmt}
The derived data type {\tt SMT\_TYPE} is used to hold the matrices $\bmA$
and $\bmH$. The components of {\tt SMT\_TYPE} used here are:

\begin{description}

\ittf{m} is a scalar component of type default \integer, 
that holds the number of rows in the matrix. 
 
\ittf{n} is a scalar component of type default \integer, 
that holds the number of columns in the matrix. 
 
\ittf{ne} is a scalar variable of type default \integer, that
holds the number of matrix entries.

\ittf{type} is a rank-one allocatable array of type default \character, that
is used to indicate the matrix storage scheme used. Its precise length and
content depends on the type of matrix to be stored (see \S\ref{typeprob}).

\ittf{val} is a rank-one allocatable array of type default \realdp\, 
and dimension at least {\tt ne}, that holds the values of the entries. 
Each pair of off-diagonal entries $h_{ij} = h_{ji}$ of a {\em symmetric}
matrix $\bmH$ is represented as a single entry 
(see \S\ref{dense}--\ref{rowwise}).
Any duplicated entries that appear in the sparse 
co-ordinate or row-wise schemes will be summed. 

\ittf{row} is a rank-one allocatable array of type default \integer, 
and dimension at least {\tt ne}, that may hold the row indices of the entries. 
(see \S\ref{coordinate}).

\ittf{col} is a rank-one allocatable array of type default \integer, 
and dimension at least {\tt ne}, that may hold the column indices of the entries
(see \S\ref{coordinate}--\ref{rowwise}).

\ittf{ptr} is a rank-one allocatable array of type default \integer, 
and dimension at least {\tt m + 1}, that may hold the pointers to
the first entry in each row (see \S\ref{rowwise}).

\end{description}

%%%%%%%%%%% problem type %%%%%%%%%%%

\subsubsection{The derived data type for holding the problem}\label{typeprob}
The derived data type {\tt QPT\_problem\_type} is used to hold 
the problem. The components of 
{\tt QPT\_problem\_type} 
are:

\begin{description}

\itt{new\_problem\_structure} is a scalar variable of type default \logical, 
 that is \true\ if this is the first (or only) problem in a sequence of 
 problems with identical "structure" to be attempted, and \false\ if 
 a previous problem with the same "structure" (but different
 numerical data) has been solved. Here, the term "structure" refers both to 
 the sparsity patterns of the Jacobian matrices $\bmA$ involved 
 (but not their numerical values), to the zero/nonzero/infinity patterns 
 (a bound is either zero, $\pm$ infinity, or a finite but arbitrary 
 nonzero) of each of the constraint bounds, and to the variables and constraints
 that are fixed (both bounds are the same) or free (the lower and upper
 bounds are $\pm$ infinity, respectively).

\ittf{n} is a scalar variable of type default \integer, 
 that holds the number of optimization variables, $n$.  
              
\ittf{m} is a scalar variable of type default \integer, 
 that holds the number of general linear constraints, $m$.
              
\ittf{H} is scalar variable of type {\tt SMT\_TYPE} 
that holds the Hessian matrix $\bmH$. The following components
are used:

\begin{description}

\itt{H\%type} is an allocatable array of rank one and type default \character, 
that
is used to indicate the storage scheme used. If the dense storage scheme 
(see Section~\ref{dense}) is used, 
the first five components of {\tt H\%type} must contain the
string {\tt DENSE}.
For the sparse co-ordinate scheme (see Section~\ref{coordinate}), 
the first ten components of {\tt H\%type} must contain the
string {\tt COORDINATE},  
for the sparse row-wise storage scheme (see Section~\ref{rowwise}),
the first fourteen components of {\tt H\%type} must contain the
string {\tt SPARSE\_BY\_ROWS},
and for the diagonal storage scheme (see Section~\ref{diagonal}),
the first eight components of {\tt H\%type} must contain the
string {\tt DIAGONAL}.

For convenience, the procedure {\tt SMT\_put} 
may be used to allocate sufficient space and insert the required keyword
into {\tt H\%type}.
For example, if {\tt prob} is of derived type {\tt \packagename\_problem\_type}
and involves a Hessian we wish to store using the co-ordinate scheme,
we may simply
%\vspace*{-2mm}
{\tt 
\begin{verbatim}
        CALL SMT_put( prob%H%type, 'COORDINATE', istat )
\end{verbatim}
}
%\vspace*{-4mm}
\noindent
See the documentation for the \galahad\ package {\tt SMT} 
for further details on the use of {\tt SMT\_put}.

\itt{H\%ne} is a scalar variable of type default \integer, that 
holds the number of entries in the {\bf lower triangular} part of $\bmH$
in the sparse co-ordinate storage scheme (see Section~\ref{coordinate}). 
It need not be set for any of the other three schemes.

\itt{H\%val} is a rank-one allocatable array of type default \realdp, that holds
the values of the entries of the {\bf lower triangular} part
of the Hessian matrix $\bmH$ in any of the 
storage schemes discussed in Section~\ref{galmatrix}.

\itt{H\%row} is a rank-one allocatable array of type default \integer,
that holds the row indices of the {\bf lower triangular} part of $\bmH$ 
in the sparse co-ordinate storage
scheme (see Section~\ref{coordinate}). 
It need not be allocated for any of the other three schemes.

\itt{H\%col} is a rank-one allocatable array variable of type default \integer,
that holds the column indices of the {\bf lower triangular} part of 
$\bmH$ in either the sparse co-ordinate 
(see Section~\ref{coordinate}), or the sparse row-wise 
(see Section~\ref{rowwise}) storage scheme.
It need not be allocated when the dense or diagonal storage schemes are used.

\itt{H\%ptr} is a rank-one allocatable array of dimension {\tt n+1} and type 
default \integer, that holds the starting position of 
each row of the {\bf lower triangular} part of $\bmH$, as well
as the total number of entries plus one, in the sparse row-wise storage
scheme (see Section~\ref{rowwise}). It need not be allocated when the
other schemes are used.

\end{description}

\ittf{G} is a rank-one allocatable array of dimension {\tt n} and type 
default \realdp, that holds the gradient $\bmg$ 
of the linear term of the quadratic objective function.
The $j$-th component of 
{\tt G}, $j = 1,  \ldots ,  n$, contains $\bmg_{j}$.

\ittf{f} is a scalar variable of type default \realdp, that holds 
the constant term, $f$, in the objective function.

\itt{rho\_g} is a scalar variable of type default \realdp, that holds 
the parameter, $\rho_g$, used to weight the infeasibility term $v_g(\bmx)$.

\itt{rho\_b} is a scalar variable of type default \realdp, that holds 
the parameter, $\rho_b$, used to weight the infeasibility term $v_b(\bmx)$.

\ittf{A} is scalar variable of type {\tt SMT\_TYPE} 
that holds the Jacobian matrix $\bmA$. The following components are used:

\begin{description}

\itt{A\%type} is an allocatable array of rank one and type default \character, that
is used to indicate the storage scheme used. If the dense storage scheme 
(see Section~\ref{dense}) is used, 
the first five components of {\tt A\%type} must contain the
string {\tt DENSE}.
For the sparse co-ordinate scheme (see Section~\ref{coordinate}), 
the first ten components of {\tt A\%type} must contain the
string {\tt COORDINATE}, while 
for the sparse row-wise storage scheme (see Section~\ref{rowwise}),
the first fourteen components of {\tt A\%type} must contain the
string {\tt SPARSE\_BY\_ROWS}.

Just as for {\tt H\%type} above, the procedure {\tt SMT\_put} 
may be used to allocate sufficient space and insert the required keyword
into {\tt A\%type}.
Once again, if {\tt prob} is of derived type {\tt \packagename\_problem\_type}
and involves a Jacobian we wish to store using the sparse row-wise 
storage scheme, we may simply
%\vspace*{-2mm}
{\tt 
\begin{verbatim}
        CALL SMT_put( prob%A%type, 'SPARSE_BY_ROWS', istat )
\end{verbatim}
}
%\vspace*{-4mm}
\noindent

\itt{A\%ne} is a scalar variable of type default \integer, that 
holds the number of entries in $\bmA$
in the sparse co-ordinate storage scheme (see Section~\ref{coordinate}). 
It need not be set for either of the other two schemes.

\itt{A\%val} is a rank-one allocatable array of type default \realdp, that holds
the values of the entries of the Jacobian matrix $\bmA$ in any of the 
storage schemes discussed in Section~\ref{galmatrix}.

\itt{A\%row} is a rank-one allocatable array of type default \integer,
that holds the row indices of $\bmA$ in the sparse co-ordinate storage
scheme (see Section~\ref{coordinate}). 
It need not be allocated for either of the other two schemes.

\itt{A\%col} is a rank-one allocatable array variable of type default \integer,
that holds the column indices of $\bmA$ in either the sparse co-ordinate 
(see Section~\ref{coordinate}), or the sparse row-wise 
(see Section~\ref{rowwise}) storage scheme.
It need not be allocated when the dense storage scheme is used.

\itt{A\%ptr} is a rank-one allocatable array of dimension {\tt m+1} and type 
default \integer, that holds the 
starting position of each row of $\bmA$, as well
as the total number of entries plus one, in the sparse row-wise storage
scheme (see Section~\ref{rowwise}). It need not be allocated when the
other schemes are used.

\end{description}

\ittf{C\_l} is a rank-one allocatable array of dimension {\tt m} and type 
default \realdp, that holds the vector of lower bounds $\bmc^{l}$ 
on the general constraints. The $i$-th component of 
{\tt C\_l}, $i = 1, \ldots , m$, contains $\bmc_{i}^{l}$.
Infinite bounds are allowed by setting the corresponding 
components of {\tt C\_l} to any value smaller than {\tt -infinity}, 
where {\tt infinity} is a component of the control array {\tt control} 
(see Section~\ref{typecontrol}).

\ittf{C\_u} is a rank-one allocatable array of dimension {\tt m} and type 
default \realdp, that holds the vector of upper bounds $\bmc^{u}$ 
on the general constraints. The $i$-th component of 
{\tt C\_u}, $i = 1,  \ldots ,  m$, contains $\bmc_{i}^{u}$.
Infinite bounds are allowed by setting the corresponding 
components of {\tt C\_u} to any value larger than {\tt infinity}, 
where {\tt infinity} is a component of the control array {\tt control} 
(see Section~\ref{typecontrol}).

\ittf{X\_l} is a rank-one allocatable array of dimension {\tt n} and type 
default \realdp, that holds
the vector of lower bounds $\bmx^{l}$ on the the variables.
The $j$-th component of {\tt X\_l}, $j = 1, \ldots , n$, 
contains $\bmx_{j}^{l}$.
Infinite bounds are allowed by setting the corresponding 
components of {\tt X\_l} to any value smaller than {\tt -infinity}, 
where {\tt infinity} is a component of the control array {\tt control} 
(see Section~\ref{typecontrol}).

\ittf{X\_u} is a rank-one allocatable array of dimension {\tt n} and type 
default \realdp, that holds
the vector of upper bounds $\bmx^{u}$ on the variables.
The $j$-th component of {\tt X\_u}, $j = 1, \ldots , n$, 
contains $\bmx_{j}^{u}$.
Infinite bounds are allowed by setting the corresponding 
components of {\tt X\_u} to any value larger than that {\tt infinity}, 
where {\tt infinity} is a component of the control array {\tt control} 
(see Section~\ref{typecontrol}).

\ittf{X} is a rank-one allocatable array of dimension {\tt n} and type 
default \realdp, 
that holds the values $\bmx$ of the optimization variables.
The $j$-th component of {\tt X}, $j = 1,  \ldots , n$, contains $x_{j}$.  

\ittf{Z} is a rank-one allocatable array of dimension {\tt n} and type default 
\realdp, that holds
the values $\bmz$ of estimates  of the dual variables 
corresponding to the simple bound constraints (see Section~\ref{galmethod}).
The $j$-th component of {\tt Z}, $j = 1,  \ldots ,  n$, contains $z_{j}$.  

\ittf{C} is a rank-one allocatable array of dimension {\tt m} and type default 
\realdp, that holds
the values $\bmA \bmx$ of the constraints.
The $i$-th component of {\tt C}, $i = 1,  \ldots ,  m$, contains 
$\bma_{i}^{T}\bmx \equiv (\bmA \bmx)_{i}$.  

\ittf{Y} is a rank-one allocatable array of dimension {\tt m} and type 
default \realdp, that holds
the values $\bmy$ of estimates  of the Lagrange multipliers
corresponding to the general linear constraints (see Section~\ref{galmethod}).
The $i$-th component of {\tt Y}, $i = 1,  \ldots ,  m$, contains $y_{i}$.  

\end{description}

%%%%%%%%%%% control type %%%%%%%%%%%

\subsubsection{The derived data type for holding control 
 parameters}\label{typecontrol}
The derived data type 
{\tt \packagename\_control\_type} 
is used to hold controlling data. Default values may be obtained by calling 
{\tt \packagename\_initialize}
(see Section~\ref{subinit}),
while components may also be changed by calling 
{\tt \fullpackagename\_read\-\_spec}
(see Section~\ref{readspec}). 
The components of 
{\tt \packagename\_control\_type} 
are:

\begin{description}

\itt{error} is a scalar variable of type default \integer, that holds the
stream number for error messages. Printing of error messages in
{\tt QPA\_solve} and {\tt QPA\_terminate}
is suppressed if {\tt error} $\leq 0$.
The default is {\tt error = 6}.

\ittf{out} is a scalar variable of type default \integer, that holds the
stream number for informational messages. Printing of informational messages in
{\tt QPA\_solve} is suppressed if {\tt out} $< 0$.
The default is {\tt out = 6}.

\itt{print\_level} is a scalar variable of type default \integer, that is used
to control the amount of informational output which is required. No
informational output will occur if {\tt print\_level} $\leq 0$. If
{\tt print\_level} $= 1$, a single line of output will be produced for each
iteration of the process. If {\tt print\_level} $\geq 2$, this output will be
increased to provide significant detail of each iteration.
The default is {\tt print\_level = 0}.

\itt{maxit} is a scalar variable of type default \integer, that holds the
maximum number of iterations which will be allowed in {\tt QPA\_solve}.
The default is {\tt maxit = 1000}.

\itt{start\_print} is a scalar variable of type default \integer, that specifies
the first iteration for which printing will occur in {\tt QPA\_solve}.
If {\tt start\_print} is negative, printing will occur from the outset.
The default is {\tt start\_print = -1}.

\itt{stop\_print} is a scalar variable of type default \integer, that specifies
the last iteration for which printing will occur in  {\tt QPA\_solve}.
If {\tt stop\_print} is negative, printing will occur once it has been
started by {\tt start\_print}.
The default is {\tt stop\_print = -1}.

\itt{factor} is a scalar variable of type default \integer, that indicates
the type of factorization of the preconditioner to be used.
Possible values are:

\begin{description}
\itt{0} the type is chosen automatically on the basis of which option looks
        likely to be the most efficient.
\itt{1} a Schur-complement factorization will be used.
\itt{2} an augmented-system factorization will be used.
\end{description}
The default is {\tt factor = 0}.

\itt{max\_col} is a scalar variable of type default \integer, that specifies
the maximum number of nonzeros in a column of $\bmA$ which is permitted
by the Schur-complement factorization.
The default is {\tt max\_col = 35}.

\itt{max\_sc} is a scalar variable of type default \integer, that specifies
the maximum number of columns permitted in the Schur complement of the
reference matrix (see Section~\ref{galmethod})
before a refactorization is triggered.
The default is {\tt max\_sc = 75}.

%\itt{indmin} is a scalar variable of type default \integer, that specifies
%an initial estimate as to the amount of integer workspace required by
%the factorization package {\tt SILS.
%The} default is {\tt indmin = 1000}.

%\itt{valmin} is a scalar variable of type default \integer, that specifies
%an initial estimate as to the amount of real workspace required by
%the factorization package {\tt SILS.
%The} default is {\tt valmin = 1000}.

\itt{itref\_max} is a scalar variable of type default \integer, that specifies
the maximum number of iterative refinements allowed with each application
of the preconditioner.
The default is {\tt itref\_max = 1}.

\itt{cg\_maxit} is a scalar variable of type default \integer, that holds the
maximum number of conjugate-gradient inner iterations that may be performed
during the computation of each search direction in {\tt QPA\_solve}.
If {\tt cg\_itmax} is set to a negative number, it will be reset by
{\tt QPA\_solve} to the dimension of the relevant linear system $+ 1$.
The default is {\tt cg\_itmax = -1}.

\itt{precon} is a scalar variable of type default \integer, that specifies
which preconditioner to be used to accelerate the conjugate-gradient
inner iteration.  Possible values are:

\begin{description}
\itt{0} the type is chosen automatically on the basis of which option looks
        likely to be the most efficient at any given stage of the solution
        process. Different preconditioners may be used at different stages.
\itt{1}  a full factorization using the Hessian, which is equivalent to
        replacing the conjugate gradient inner iteration by a direct method.
        The Hessian may be perturbed to ensure that the resultant matrix
        is a preconditioner.
\itt{2} the Hessian matrix is replaced by the identity matrix.
\itt{3} the Hessian matrix is replaced by a band of given semi-bandwidth
        (see {\tt nsemib} below).
\itt{4}  the Hessian matrix terms in the current reference matrix
        (see Section~\ref{galmethod}) are replaced by the identity matrix.
\itt{5}  the Hessian matrix terms outside a band of given semi-bandwidth
        in the current reference matrix are replaced by zeros
        (see {\tt nsemib} below).
\end{description}
The default is {\tt precon = 0}.

\itt{nsemib} is a scalar variable of type default \integer, that specifies
the semi-bandwidth of the band preconditioner when {\tt precon = 3},
if appropriate.
The default is {\tt nsemib = 5}.

\itt{full\_max\_fill} is a scalar variable of type default \integer.
If the ratio of the number of nonzeros in the factors
of the reference matrix (see Section~\ref{galmethod}) to the number of
nonzeros in the matrix itself exceeds {\tt full\_max\_fil},
and the preconditioner is being selected
automatically ({\tt precon = 0}), a banded approximation (see {\tt precon = 3})
will be used instead.
The default is {\tt full\_max\_fill = 10}.

\itt{deletion\_strategy} is a scalar variable of type default \integer, that
specifies the rules used to determine which constraint to remove
from the working set (see Section~\ref{galmethod}) when necessary to ensure
further progress towards the solution. Possible values are:
\begin{description}
\itt{0} the constraint whose Lagrange multiplier most violates its
required optimality bound will be removed.
\itt{1}the most-recently added constraint whose Lagrange multiplier violates its
required optimality bound will be removed.
\itt{k $>$ 1} among the $k$ most-recently added constraints whose
Lagrange multipliers violates their required optimality bounds,
the one which most violates its bound will be removed.
\end{description}
The default is {\tt deletion\_strategy = 0}.

\itt{restore\_problem} is a scalar variable of type default \integer, that
specifies how much of the input problem is to be retored on output.
Possible values are:
\begin{description}
\itt{0} nothing is restored.
\itt{1} the vector data $\bmg$,
   $\bmc^{l}$, $\bmc^{u}$, $\bmx^{l}$, and $\bmx^{u}$
   will be restored to their input values.
\itt{2} the entire problem, that is the above vector data along with
the Hessian matrix $\bmH$ and the Jacobian matrix $\bmA$, will be restored.
\end{description}
The default is {\tt restore\_problem = 2}.

\itt{monitor\_residuals} is a scalar variable of type default \integer, that
specifies the frequency at which working constraint residuals will be monitored
to ensure feasibility. The  residuals will be monitored every
{\tt monitor\_residu\-al} iterations.
The default is {\tt monitor\_residuals = 1}.

\itt{cold\_start} is a scalar variable of type default \integer, that controls
the initial working set (see Section~\ref{galmethod})). Possible values are:
\begin{description}
\itt{0} a "warm start" will be performed. The values set in
     {\tt C\_stat} and {\tt B\_stat} indicate which
     constraints will be included in the initial working set.
     (see {\tt C\_stat} and {\tt B\_stat} in Section~\ref{qps}).
\itt{1} the constraints "active" at {\tt p\%X} (see Section~\ref{qps})
     will determine the initial working set.
\itt{2} the initial working set will be empty.
\itt{3} the initial working set will only contain equality constraints.
\itt{4} the initial working set will contain as many active constraints as
     possible, chosen (in order) from equality constraints, simple bounds, and
     finally general linear constraints.
\end{description}
The default is {\tt cold\_start = 3}.

\itt{infeas\_check\_interval} is a scalar variable of type default \integer,
that gives the number of iterations that are permitted before the
infeasibility is checked for improvement.
The default is {\tt infeas\_check\_interval = 100}.

\itt{infinity} is a scalar variable of type default \realdp, that is used to
specify which constraint bounds are infinite.
Any bound larger than {\tt infinity} in modulus will be regarded as infinite.
The default is {\tt infinity =} $10^{19}$.

\itt{feas\_tol} is a scalar variable of type default \realdp, that specifies the
maximum amount by which a constraint may be violated and yet still be
considered to be satisfied.
The default is {\tt feas\_tol =} $u^{3/4}$,
where $u$ is {\tt EPSILON(1.0)} ({\tt EPSILON(1.0D0)} in
{\tt \fullpackagename\_dou\-ble}).

\itt{obj\_unbounded}  is a scalar variable of type default
\realdp, that specifies smallest
value of the objective function that will be tolerated before the problem
is declared to be unbounded from below.
The default is {\tt potential\_u\-nbounded =} $-u^{-2}$,
where $u$ is {\tt EPSILON(1.0)} ({\tt EPSILON(1.0D0)} in
{\tt \fullpackagename\_double}).

\itt{increase\_rho\_g\_factor} is a scalar variable of type default \realdp,
that gives the factor by which the current penalty parameter $\rho_g$
for the general constraints may be increased when solving quadratic programs.
The default is {\tt increase\_rho\_g\_factor = 2}.

\itt{increase\_rho\_b\_factor} is a scalar variable of type default \realdp,
that gives the factor by which the current penalty parameter $\rho_b$ for
the simple bound constraints may be increased when solving quadratic programs.
The default is {\tt increase\_rho\_b\_factor = 2}.

\itt{infeas\_g\_improved\_by\_factor} is a scalar variable of type default
\realdp, that specifies the relative improvement in the infeasibility
that must be achieved when solving quadratic programs if the current
value of $\rho_g$ is to be maintained.
Specifically if the infeasibility of the general constraints
has not fallen by at least a factor {\tt infeas\_g\_improved\_by\_factor}
during the previous {\tt infeas\_check\_interval} iterations, the
penalty parameter will be increased by a factor
{\tt increase\_rho\_g\_factor}.
The default is {\tt infeas\_im\-proved\_g\_by\_factor = 0.75}.

\itt{infeas\_b\_improved\_by\_factor} is a scalar variable of type default
\realdp, that specifies the relative improvement in the infeasibility
that must be achieved when solving quadratic programs if the current
value of $\rho_b$ is to be maintained.
Specifically if the infeasibility of the simple bound constraints
has not fallen by at least a factor {\tt infeas\_b\_improved\_by\_factor}
during the previous {\tt infeas\_check\_in\-terval} iterations, the
penalty parameter will be increased by a factor
{\tt increase\_rho\_b\_factor}.
The default is {\tt infeas\_im\-proved\_b\_by\_factor = 0.75}.

%\itt{pivot\_tol}  is a scalar variable of type default
%\realdp, that holds the
%threshold pivot tolerance used by the matrix factorization.  See
%the documentation for the package {\tt SILS} for details.
%The default is {\tt pivot\_tol =} $0.1 \sqrt{u}$,
%where $u$ is {\tt EPSILON(1.0)} ({\tt EPSILON(1.0D0)} in
%{\tt \fullpackagename\_double}).
%%The default is {\tt pivot\_tol = 0.01}.

\itt{pivot\_tol\_for\_dependencies} is a scalar variable of type default
\realdp, that holds the relative
threshold pivot  tolerance used by the matrix factorization when
attempting to detect linearly dependent constraints. A value larger
than  {\tt pivot\_tol} is appropriate. See
the documentation for the package {\tt SLS} for details.
The default is {\tt pivot\_tol\_for\_dependencies = 0.5}.

%\itt{zero\_pivot} is a scalar variable of type default \realdp.
%Any pivots smaller than  {\tt zero\_pivot} in absolute value will be regarded
%to be zero when attempting to detect linearly dependent constraints.
%%The default is {\tt zero\_pivot =} $u^{3/4}$,
%where $u$ is {\tt EPSILON(1.0)} ({\tt EPSILON(1.0D0)} in
%{\tt \fullpackagename\_dou\-ble}).

\itt{multiplier\_tol} is a scalar variable of type default \realdp.
Any dual variable or Lagrange multiplier which is less than
{\tt multiplier\_tol} outside its optimal interval will be regarded
as being acceptable when checking for optimality.
The default is {\tt zero\_pivot =} $\sqrt{u}$,
where $u$ is {\tt EPSILON(1.0)} ({\tt EPSILON(1.0D0)} in
{\tt \fullpackagename\_dou\-ble}).

\itt{inner\_stop\_relative} and {\tt inner\_stop\_absolute}
are scalar variables of type default \realdp,
that hold the relative and absolute convergence tolerances for the
inner iteration (search direction) problem.
and correspond to the values {\tt control\%stop\_relative} and
{\tt control\%stop\_absolute} in the \galahad\ package
{\tt \libraryname\_GLTR}.
The defaults are {\tt inner\_stop\_relative = 0.0}
and \sloppy {\tt inner\_stop\_absolute =} $\sqrt{u}$,
where $u$ is {\tt EPSILON(1.0)} ({\tt EPSILON(1.0D0)} in
{\tt \fullpackagename\_double}).

\itt{cpu\_time\_limit} is a scalar variable of type default \realdp,
that is used to specify the maximum permitted CPU time. Any negative
value indicates no limit will be imposed. The default is
{\tt cpu\_time\_limit = - 1.0}.

\itt{clock\_time\_limit} is a scalar variable of type default \realdp,
that is used to specify the maximum permitted elapsed system clock time.
Any negative value indicates no limit will be imposed. The default is
{\tt clock\_time\_limit = - 1.0}.

\itt{treat\_zero\_bounds\_as\_general} is a scalar variable of type
default \logical.
If it is set to \false, variables which
are only bounded on one side, and whose bound is zero,
will be recognised as non-negativities/non-positivities rather than simply as
lower- or upper-bounded variables.
If it is set to \true, any variable bound
$x_{j}^{l}$ or $x_{j}^{u}$ which has the value 0.0 will be
treated as if it had a general value.
Setting {\tt treat\_zero\_bounds\_as\_general} to \true\ has the advantage
that if a sequence of problems are reordered, then bounds which are
``accidentally'' zero will be considered to have the same structure as
those which are nonzero. However, {\tt \fullpackagename} is
able to take special advantage of non-negativities/non-positivities, so
if a single problem, or if a sequence of problems whose
bound structure is known not to change, is/are to be solved,
it will pay to set the variable to \false.
The default is {\tt treat\_zero\_bounds\_as\_general = .FALSE.}.

\ifthenelse{\equal{\package}{qpa}}{
\itt{solve\_qp} is a scalar variable of type default \logical, that
must be set \true\ if the algorithm will aim to solve the quadratic
programming problem \req{1.2}--\req{1.4}
(by adjusting $\rho_g$ and $\rho_b$ as necessary), and
and \false\ if the solution of the $l_1$-quadratic program
for the specified values of $\rho_g$ and $\rho_b$ is required.
The default is {\tt solve\_qp = .FALSE.}.

\itt{solve\_within\_bounds} is a scalar variable of type default \logical, that
must be set \true\ if the algorithm will aim to solve
the bound-constrained $\mathbf{\ell_1}$ quadratic programming problem
\req{1.4}--\req{1.5} (by adjusting $\rho_b$ as necessary), and
and \false\ otherwise.
If {\tt solve\_qp} is \true, the value of {\tt solve\_within\_bounds}
will be ignored. The default is {\tt solve\_within\_bounds = .FALSE.}.
}{}

\itt{ramdomize} is a scalar variable of type default \logical, that
must be set \true\ if the user wishes to perturb the constraint bounds
by small random quantities during the first stage of the solution process,
and \false\ otherwise.
Any randomization will ultimately be removed. Randomization
helps when solving degenerate problems and is usually to be recommended.
The default is {\tt randomize = .TRUE.}.

%\itt{array\_syntax\_worse\_than\_do\_loop} is a scalar variable of type
%default \logical, that should be set \true\ if the compiler is
%better able to optimize Fortran 77-style do-loops than to exploit
%Fortran 95-style array syntax when performing vector operations,
%and \false\ otherwise.
%The default is {\tt array\_syntax\_worse\_than\_do\_loop = .FALSE.}.

\itt{symmetric\_linear\_solver} is a scalar variable of type default \character\
and length 30, that specifies the external package to be used to
solve any symmetric linear system that might arise. Current possible
choices are {\tt 'sils'}, {\tt 'ma27'}, {\tt 'ma57'}, {\tt 'ma77'},
{\tt 'ma86'}, {\tt 'ma97'}, {\tt ssids}, {\tt 'pardiso'}
and {\tt 'wsmp'},
although only {\tt 'sils'} and, for OMP 4.0-compliant compilers,
{\tt 'ssids'} are installed by default.
See the documentation for the \galahad\ package {\tt SLS} for further details.
The default is {\tt symmetric\_linear\_solver = 'sils'}.

\itt{definite\_linear\_solver} is a scalar variable of type default \character\
and length 30, that specifies the external package to be used to
solve any symmetric positive-definite linear system that might arise.
Current possible
choices are {\tt 'sils'}, {\tt 'ma27'}, {\tt 'ma57'}, {\tt 'ma77'},
{\tt 'ma86'}, {\tt 'ma87'}, {\tt 'ma97'}, {\tt ssids}, {\tt 'pardiso'}
and {\tt 'wsmp'},
although only {\tt 'sils'} and, for OMP 4.0-compliant compilers,
{\tt 'ssids'} are installed by default.
See the documentation for the \galahad\ package {\tt SLS} for further details.
The default is {\tt definite\_linear\_solver = 'sils'}.

\itt{prefix} is a scalar variable of type default \character\
and length 30, that may be used to provide a user-selected
character string to preface every line of printed output.
Specifically, each line of output will be prefaced by the string
{\tt prefix(2:LEN(TRIM(prefix))-1)},
thus ignoring the first and last non-null components of the
supplied string. If the user does not want to preface lines by such
a string, they may use the default {\tt prefix = ""}.

\itt{SLS\_control} is a scalar variable argument of type
{\tt SLS\_control\_type} that is used to pass control
options to external packages used to
factorize relevant symmetric matrices that arise.
See the documentation for the \galahad\ package {\tt SLS} for further details.
In particular, default values are as for {\tt SLS}, except that
{\tt SLS\_control\%rela\-tive\_pivot\_tolerance} is reset to
{\tt pivot\_tol}.

\end{description}


%%%%%%%%%%% time type %%%%%%%%%%%

\subsubsection{The derived data type for holding timing 
 information}\label{typetime}
The derived data type 
{\tt \packagename\_time\_type} 
is used to hold elapsed CPU and system clock times for the various parts of 
the calculation. The components of 
{\tt \packagename\_time\_type} 
are:
\begin{description}
\itt{total} is a scalar variable of type default \realdp, that gives
 the total CPU time spent in the package.

\itt{preprocess} is a scalar variable of type default \realdp, that gives
 the CPU time spent reordering the problem to standard form prior to solution.

\itt{analyse} is a scalar variable of type default \realdp, that gives
 the CPU time spent analysing the required matrices prior to factorization.

\itt{factorize} is a scalar variable of type default \realdp, that gives
 the CPU time spent factorizing the required matrices.

\itt{solve} is a scalar variable of type default \realdp, that gives
 the CPU time spent computing the search direction.

\itt{clock\_total} is a scalar variable of type default \realdp, that gives
 the total elapsed system clock time spent in the package.

\itt{clock\_preprocess} is a scalar variable of type default \realdp, that gives
 the elapsed system clock time spent reordering the problem to standard form 
prior to solution.

\itt{clock\_analyse} is a scalar variable of type default \realdp, that gives
 the elapsed system clock time spent analysing the required matrices prior to 
factorization.

\itt{clock\_factorize} is a scalar variable of type default \realdp, that gives
 the elapsed system clock time spent factorizing the required matrices.

\itt{clock\_solve} is a scalar variable of type default \realdp, that gives
 the elapsed system clock time spent computing the search direction.

\end{description}

%%%%%%%%%%% info type %%%%%%%%%%%

\subsubsection{The derived data type for holding informational
 parameters}\label{typeinform}
The derived data type 
{\tt \packagename\_inform\_type} 
is used to hold parameters that give information about the progress and needs 
of the algorithm. The components of 
{\tt \packagename\_inform\_type} 
are:

\begin{description}

\itt{status} is a scalar variable of type default \integer, that gives the
exit status of the algorithm. 
%See Sections~\ref{galerrors} and \ref{galinfo}
See Section~\ref{galerrors}
for details.

\itt{alloc\_status} is a scalar variable of type default \integer, that gives
the status of the last attempted array allocation or deallocation.
This will be 0 if {\tt status = 0}.

\itt{bad\_alloc} is a scalar variable of type default \character\
and length 80, that  gives the name of the last internal array 
for which there were allocation or deallocation errors.
This will be the null string if {\tt status = 0}. 

\itt{major\_iter} is a scalar variable of type default \integer, that gives the
total number of major iterations required.

\ittf{iter} is a scalar variable of type default \integer, that gives the
total number of iterations required.

\itt{cg\_iter} is a scalar variable of type default \integer, that gives the
total number of conjugate-gradient inner iterations required.

\itt{factorization\_status} is a scalar variable of type default \integer, that 
gives the return status from the matrix factorization.

\itt{factorization\_integer} is a scalar variable of type long
\integer, that gives the amount of integer storage used for the matrix 
factorization.

\itt{factorization\_real} is a scalar variable of type long \integer, 
that gives the amount of real storage used for the matrix factorization.

\itt{nfacts} is a scalar variable of type default \integer, that gives the
total number of factorizations performed.

\itt{nmods} is a scalar variable of type default \integer, that gives the
total number of factorizations which were modified to 
ensure that the matrix is an appropriate preconditioner. 

\itt{num\_g\_infeas} is a scalar variable of type default \integer, that 
gives the total number of general constraints that are violated.

\itt{num\_b\_infeas} is a scalar variable of type default \integer, that 
gives the total number of simple bound constraints that are violated.

\ittf{obj} is a scalar variable of type default \realdp, that holds the
value of the objective function at the best estimate of the solution found.

\itt{infeas\_g} is a scalar variable of type default \realdp, that holds the
value of the infeasibility $v_g(\bmx)$.

\itt{infeas\_b} is a scalar variable of type default \realdp, that holds the
value of the infeasibility $v_b(\bmx)$.

\itt{merit} is a scalar variable of type default \realdp, that holds the
value of the merit function $q(\bmx) + \rho_g v_g(\bmx) + \rho_b v_b(\bmx)$ 
at the best estimate of the solution found.

\ittf{time} is a scalar variable of type {\tt \packagename\_time\_type} 
whose components are used to hold elapsed CPU  and system clock 
times for the various parts of the calculation (see Section~\ref{typetime}).

\itt{SLS\_inform} is a scalar variable argument of type 
{\tt SLS\_inform\_type} that is used to pass information
concerning the progress of the external packages used to 
factorize relevant symmetric matrices that arise. 
See the documentation for the \galahad\ package {\tt SLS} for further details. 

\end{description}



%%%%%%%%%%% data type %%%%%%%%%%%

\subsubsection{The derived data type for holding problem data}\label{typedata}
The derived data type 
{\tt \packagename\_data\_type} 
is used to hold all the data for a particular problem,
or sequences of problems with the same structure, between calls of 
{\tt \packagename} procedures. 
This data should be preserved, untouched, from the initial call to 
{\tt \packagename\_initialize}
to the final call to
{\tt \packagename\_terminate}.

%%%%%%%%%%%%%%%%%%%%%% argument lists %%%%%%%%%%%%%%%%%%%%%%%%

\galarguments
There are three procedures for user calls
(see Section~\ref{galfeatures} for further features): 

\begin{enumerate}
\item The subroutine 
      {\tt \packagename\_initialize} 
      is used to set default values, and initialize private data, 
      before solving one or more problems with the
      same sparsity and bound structure.
\item The subroutine 
      {\tt \packagename\_solve} 
      is called to solve the problem.
\item The subroutine 
      {\tt \packagename\_terminate} 
      is provided to allow the user to automatically deallocate array 
       components of the private data, allocated by 
       {\tt \packagename\_solve}, 
       at the end of the solution process. 
       It is important to do this if the data object is re-used for another 
       problem {\bf with a different structure}
       since {\tt \packagename\_initialize} cannot test for this situation, 
       and any existing associated targets will subsequently become unreachable.
\end{enumerate}
%We use square brackets {\tt [ ]} to indicate \optional\ arguments.

%%%%%% initialization subroutine %%%%%%

\subsubsection{The initialization subroutine}\label{subinit}
 Default values are provided as follows:
\vspace*{1mm}

\hspace{8mm}
{\tt CALL \packagename\_initialize( data, control, inform )}

\vspace*{-3mm}
\begin{description}

\itt{data} is a scalar \intentinout\ argument of type 
{\tt \packagename\_data\_type}
(see Section~\ref{typedata}). It is used to hold data about the problem being 
solved. 

\itt{control} is a scalar \intentout\ argument of type 
{\tt \packagename\_control\_type}
(see Section~\ref{typecontrol}). 
On exit, {\tt control} contains default values for the components as
described in Section~\ref{typecontrol}.
These values should only be changed after calling 
{\tt \packagename\_initialize}.

\itt{inform} is a scalar \intentinout\ argument of type 
{\tt \packagename\_inform\_type}
(see Section~\ref{typeinform}). A successful call to
{\tt \packagename\_initialize}
is indicated when the  component {\tt status} has the value 0. 
For other return values of {\tt status}, see Section~\ref{galerrors}.

\end{description}

%%%%%%%%% main solution subroutine %%%%%%

\subsubsection{The quadratic programming subroutine}\label{qps}
The quadratic programming solution algorithm is called as follows:
\vspace*{1mm}

\hspace{8mm}
{\tt CALL \packagename\_solve( p, C\_stat, B\_stat, data, control, info )}

\vspace*{-3mm}
\begin{description}
\ittf{p} is a scalar \intentinout\ argument of type 
{\tt QPT\_problem\_type}
(see Section~\ref{typeprob}). 
It is used to hold data about the problem being solved.
For a new problem, the user must allocate all the array components,
and set values for all components except {\tt p\%C}.
{\tt p\%new\_problem\_structure} must be set \true, but will have been reset
to \false\ on exit from {\tt \packagename\_solve}.
Users are free to choose whichever
of the matrix formats described in Section~\ref{galmatrix} 
is appropriate for $\bmA$ and $\bmH$ for their application---different 
formats may be used for the two matrices.

For a problem with the same structure as one that has just been 
solved, the user may set {\tt p\%new\_problem\_str\-u\-cture} to \false,
so long as {\tt \packagename\_terminate} has not been called in the interim.
The \integer\ components must be unaltered since the
previous call to {\tt \packagename\_solve}, but the \realdp\ 
may be altered to reflect the new problem.

The components {\tt p\%X}, {\tt p\%Y} and {\tt p\%Z}
must be set to initial estimates, $\bmx^{0}$, of the primal variables, 
$\bmx$, Lagrange multipliers for the general constraints, $\bmy$, 
and dual variables for the bound constraints, $\bmz$, respectively.
Inappropriate initial values will be altered, so the user should
not be overly concerned if suitable values are not apparent, and may be
content with merely setting {\tt p\%X=0.0},
{\tt p\%Y=0.0} and {\tt p\%Z=0.0}.
The component {\tt p\%C} need not be set on entry.

On exit, the components {\tt p\%X}, {\tt p\%Y}, {\tt p\%Z} and {\tt p\%C}
will contain the best estimates of the primal variables $\bmx$, 
Lagrange multipliers for the general
constraints $\bmy$, dual variables for the bound constraints $\bmz$, 
and values of the constraints $\bmA \bmx$ respectively.
The components {\tt p\%rho\_g} and {\tt p\%rho\_b} may have been altered if 
either of {\tt control\%solve\_qp} or {\tt control\%solve\_within\_bounds} 
have been set \true\ and will reflect the final values of $\rho_g$ and $\rho_b$
used. What of the remaining problem data has been restored depends upon
the input value of the control parameter {\tt control\%restore\_problem}.
The return format for a restored array component will be the same as its
input format.
\restrictions {\tt p\%n} $> 0$, {\tt p\%m} $\geq 0$,
              {\tt p\%A\%ne} $\geq -2$ and {\tt p\%H\%ne} $\geq -2$.

\itt{C\_stat} is a rank-one \intentinout\ array argument of dimension {\tt p\%m}
and type default \integer, that indicates which of the general linear
constraints are in the current working set. Possible values for 
{\tt C\_stat(i)}, {\tt i}$=1, \ldots ,$ {\tt p\%m}, and their meanings are
\begin{description}
\itt{<0} the $i$-th general constraint 
is in the working set, on its lower bound,
\itt{>0} the $i$-th general constraint 
is in the working set, on its upper bound, and
\itt{0}  the $i$-th general constraint is not in the working set.
\end{description}
Suitable values must be supplied if  {\tt control\%cold\_start = 0} on entry,
but need not be provided for other input values of {\tt control\%cold\_start}.
Inappropriate values will be ignored.
On exit, {\tt C\_stat} will contain values appropriate for the ultimate
working set.

\itt{B\_stat} is a rank-one \intentinout\ array argument of dimension {\tt p\%n}
and type default \integer, that indicates which of the simple bound 
constraints are in the current working set. Possible values for 
{\tt B\_stat(j)}, {\tt j}$=1, \ldots ,$ {\tt p\%n}, and their meanings are
\begin{description}
\itt{<0} the $j$-th simple bound constraint 
is in the working set, on its lower bound,
\itt{>0} the $j$-th simple bound constraint 
is in the working set, on its upper bound, and
\itt{0}  the $j$-th simple bound constraint is not in the working set.
\end{description}
Suitable values must be supplied if  {\tt control\%cold\_start = 0} on entry,
but need not be provided for other input values of {\tt control\%cold\_start}.
Inappropriate values will be ignored.
On exit, {\tt B\_stat} will contain values appropriate for the ultimate
working set.

\itt{data} is a scalar \intentinout\ argument of type 
{\tt \packagename\_data\_type}
(see Section~\ref{typedata}). It is used to hold data about the problem being 
solved. It must not have been altered {\bf by the user} since the last call to 
{\tt \packagename\_initialize}.

\itt{control} is a scalar \intentin\ argument of type 
{\tt \packagename\_control\_type}
(see Section~\ref{typecontrol}). Default values may be assigned by calling 
{\tt \packagename\_initialize} prior to the first call to 
{\tt \packagename\_solve}.

\itt{inform} is a scalar \intentout\ argument of type 
{\tt \packagename\_inform\_type}
(see Section~\ref{typeinform}). A successful call to
{\tt \packagename\_solve}
is indicated when the  component {\tt status} has the value 0. 
For other return values of {\tt status}, see Section~\ref{galerrors}.

%\itt{v} is an \optional rank-one \intentin\ allocatable array of type 
% default \realdp,  that holds \ldots

\end{description}

%%%%%%% termination subroutine %%%%%%

\subsubsection{The  termination subroutine}
All previously allocated arrays are deallocated as follows:
\vspace*{1mm}

\hspace{8mm}
{\tt CALL \packagename\_terminate( data, control, info )}

\vspace*{-3mm}
\begin{description}

\itt{data} is a scalar \intentinout\ argument of type 
{\tt \packagename\_data\_type} 
exactly as for
{\tt \packagename\_solve},
which must not have been altered {\bf by the user} since the last call to 
{\tt \packagename\_initialize}.
On exit, array components will have been deallocated.

\itt{control} is a scalar \intentin\ argument of type 
{\tt \packagename\_control\_type}
exactly as for
{\tt \packagename\_solve}.

\itt{inform} is a scalar \intentout\ argument of type 
{\tt \packagename\_inform\_type}
exactly as for
{\tt \packagename\_solve}.
Only the component {\tt status} will be set on exit, and a 
successful call to 
{\tt \packagename\_terminate}
is indicated when this  component {\tt status} has the value 0. 
For other return values of {\tt status}, see Section~\ref{galerrors}.

\end{description}

%%%%%%%%%%%%%%%%%%%%%% Warning and error messages %%%%%%%%%%%%%%%%%%%%%%%%

\galerrors
A negative value of {\tt inform\%status} on exit from 
{\tt \packagename\_solve}
or 
{\tt \packagename\_terminate}
indicates that an error has occurred. No further calls should be made
until the error has been corrected. Possible values are:

\begin{description}

\itt{\galerrallocate.} An allocation error occurred. 
A message indicating the offending 
array is written on unit {\tt control\%error}, and the returned allocation 
status and a string containing the name of the offending array
are held in {\tt inform\%alloc\_\-status}
and {\tt inform\%bad\_alloc} respectively.

\itt{\galerrdeallocate.} A deallocation error occurred. 
A message indicating the offending 
array is written on unit {\tt control\%error} and the returned allocation 
status and a string containing the name of the offending array
are held in {\tt inform\%alloc\_\-status}
and {\tt inform\%bad\_alloc} respectively.

\itt{\galerrrestrictions.} One of the restrictions 
    {\tt prob\%n} $> 0$ or {\tt prob\%m} $\geq  0$
    or requirements that {\tt prob\%A\_type} 
    and {\tt prob\%H\_type} contain its relevant string
    {\tt 'DENSE'}, {\tt 'COORDINATE'},{\tt 'SPARSE\_BY\_ROWS'} 
    or {\tt 'DIAGONAL'}
    has been violated.

%\itt{-3.} At least one of the arrays 
% {\tt p\%A\_val}, {\tt p\%A\_row}, {\tt p\%A\_col},
% {\tt p\%H\_val}, {\tt p\%H\_row} or {\tt p\%H\_col},
% is not large enough to hold the original, or reordered, matrices $\bmA$
% or $\bmH$.

\itt{\galerrprimalinfeasible.} The constraints appear to have no feasible point.

\itt{\galerrunbounded.} 
   The objective function appears to be unbounded from below on the
    feasible set.

\itt{\galerranalysis.} The analysis phase of the factorization failed; 
 the return status from the factorization
    package is given in the component {\tt inform\%fac\-t\-or\_status}.

\itt{\galerrfactorization.} 
   The factorization failed; the return status from the factorization
    package is given in the component {\tt inform\%fac\-t\-or\_status}.

\itt{\galerrillconditioned.} The problem is so ill-conditioned that 
  further progress is impossible.

\itt{\galerrmaxiterations.} 
    Too many iterations have been performed. This may happen if
    {\tt control\%maxit} is too small, but may also be symptomatic of 
    a badly scaled problem.

\itt{\galerrcpulimit.} The elapsed CPU or system clock time limit has been 
    reached. This may happen if either {\tt control\%cpu\_time\_limit} or
    {\tt control\%clock\_time\_limit} is too small, but may also be symptomatic 
    of a badly scaled problem.

\itt{\galerrupperentry.} An entry from the strict upper triangle of $\bmH$ 
 has been specified.

\end{description}

%%%%%%%%%%%%%%%%%%%%%% Further features %%%%%%%%%%%%%%%%%%%%%%%%

\galfeatures
\noindent In this section, we describe an alternative means of setting 
control parameters, that is components of the variable {\tt control} of type
{\tt \packagename\_control\_type}
(see Section~\ref{typecontrol}), 
by reading an appropriate data specification file using the
subroutine {\tt \packagename\_read\_specfile}. This facility
is useful as it allows a user to change  {\tt \packagename} control parameters 
without editing and recompiling programs that call {\tt \packagename}.

A specification file, or specfile, is a data file containing a number of 
"specification commands". Each command occurs on a separate line, 
and comprises a "keyword", 
which is a string (in a close-to-natural language) used to identify a 
control parameter, and 
an (optional) "value", which defines the value to be assigned to the given
control parameter. All keywords and values are case insensitive, 
keywords may be preceded by one or more blanks but
values must not contain blanks, and
each value must be separated from its keyword by at least one blank.
Values must not contain more than 30 characters, and 
each line of the specfile is limited to 80 characters,
including the blanks separating keyword and value.

The portion of the specification file used by 
{\tt \packagename\_read\_specfile}
must start
with a "{\tt BEGIN \packagename}" command and end with an 
"{\tt END}" command.  The syntax of the specfile is thus defined as follows:
\begin{verbatim}
  ( .. lines ignored by QPA_read_specfile .. )
    BEGIN QPA
       keyword    value
       .......    .....
       keyword    value
    END 
  ( .. lines ignored by QPA_read_specfile .. )
\end{verbatim}
where keyword and value are two strings separated by (at least) one blank.
The ``{\tt BEGIN \packagename}'' and ``{\tt END}'' delimiter command lines 
may contain additional (trailing) strings so long as such strings are 
separated by one or more blanks, so that lines such as
\begin{verbatim}
    BEGIN QPA SPECIFICATION
\end{verbatim}
and
\begin{verbatim}
    END QPA SPECIFICATION
\end{verbatim}
are acceptable. Furthermore, 
between the
``{\tt BEGIN \packagename}'' and ``{\tt END}'' delimiters,
specification commands may occur in any order.  Blank lines and
lines whose first non-blank character is {\tt !} or {\tt *} are ignored. 
The content 
of a line after a {\tt !} or {\tt *} character is also 
ignored (as is the {\tt !} or {\tt *}
character itself). This provides an easy manner to "comment out" some 
specification commands, or to comment specific values 
of certain control parameters.  

The value of a control parameters may be of three different types, namely
integer, logical or real.
Integer and real values may be expressed in any relevant Fortran integer and
floating-point formats (respectively). Permitted values for logical
parameters are "{\tt ON}", "{\tt TRUE}", "{\tt .TRUE.}", "{\tt T}", 
"{\tt YES}", "{\tt Y}", or "{\tt OFF}", "{\tt NO}",
"{\tt N}", "{\tt FALSE}", "{\tt .FALSE.}" and "{\tt F}". 
Empty values are also allowed for 
logical control parameters, and are interpreted as "{\tt TRUE}".  

The specification file must be open for 
input when {\tt \packagename\_read\_specfile}
is called, and the associated device number 
passed to the routine in device (see below). 
Note that the corresponding 
file is {\tt REWIND}ed, which makes it possible to combine the specifications 
for more than one program/routine.  For the same reason, the file is not
closed by {\tt \packagename\_read\_specfile}.

\subsubsection{To read control parameters from a specification file}
\label{readspec}

Control parameters may be read from a file as follows:
\hskip0.5in 
\def\baselinestretch{0.8} {\tt 
\begin{verbatim}
     CALL QPA_read_specfile( control, device )
\end{verbatim}}
\def\baselinestretch{1.0}

\begin{description}
\itt{control} is a scalar \intentinout argument of type 
{\tt \packagename\_control\_type}
(see Section~\ref{typecontrol}). 
Default values should have already been set, perhaps by calling 
{\tt \packagename\_initialize}.
On exit, individual components of {\tt control} may have been changed
according to the commands found in the specfile. Specfile commands and 
the component (see Section~\ref{typecontrol}) of {\tt control} 
that each affects are given in Table~\ref{specfile}.

\pctable{|l|l|l|} 
\hline
  command & component of {\tt control} & value type \\ 
\hline
  {\tt error-printout-device} & {\tt \%error} & integer \\
  {\tt printout-device} & {\tt \%out} & integer \\
  {\tt print-level} & {\tt \%print\_level} & integer \\
  {\tt maximum-number-of-iterations} & {\tt \%maxit} & integer \\
  {\tt start-print} & {\tt \%start\_print} & integer \\
  {\tt stop-print} & {\tt \%stop\_print} & integer \\
  {\tt factorization-used} & {\tt \%factor} & integer \\
  {\tt maximum-column-nonzeros-in-schur-complement} & {\tt \%max\_col} & integer \\
  {\tt maximum-dimension-of-schur-complement} & {\tt \%max\_sc} & integer \\
%  {\tt initial-integer-workspace} & {\tt \%indmin} & integer \\
%  {\tt initial-real-workspace} & {\tt \%valmin} & integer \\
  {\tt maximum-refinements} & {\tt \%itref\_max} & integer \\
  {\tt maximum-infeasible-iterations-before-rho-increase} & {\tt \%infeas\_check\_interval} & integer \\
  {\tt maximum-number-of-cg-iterations} & {\tt \%cg\_maxit} & integer \\
  {\tt preconditioner-used} & {\tt \%precon} & integer \\
  {\tt semi-bandwidth-for-band-preconditioner} & {\tt \%nsemib} & integer \\
  {\tt full-max-fill-ratio} & {\tt \%full\_max\_fill} & integer \\
  {\tt deletion-strategy} & {\tt \%deletion\_strategy} & integer \\
  {\tt restore-problem-on-output} & {\tt \%restore\_problem} & integer \\
  {\tt residual-monitor-interval} & {\tt \%monitor\_residuals} & integer \\
  {\tt cold-start-strategy} & {\tt \%cold\_start} & integer \\
  {\tt infinity-value} & {\tt \%infinity} & real \\
  {\tt feasibility-tolerance} & {\tt \%feas\_tol} & real \\
  {\tt minimum-objective-before-unbounded} & {\tt \%obj\_unbounded} & real \\
  {\tt increase-rho-g-factor} & {\tt \%increase\_rho\_g\_factor} & real \\
  {\tt increase-rho-b-factor} & {\tt \%increase\_rho\_b\_factor} & real \\
  {\tt infeasible-g-required-improvement-factor} & {\tt \%infeas\_g\_improved\_by\_factor} & real \\
  {\tt infeasible-b-required-improvement-factor} & {\tt \%infeas\_b\_improved\_by\_factor} & real \\
%  {\tt pivot-tolerance-used} & {\tt \%pivot\_tol} & real \\
   {\tt pivot-tolerance-used-for-dependencies} & {\tt \%pivot\_tol\_for\_dependencies} & real \\
%  {\tt zero-pivot-tolerance} & {\tt \%zero\_pivot} & real \\
  {\tt multiplier-tolerance} & {\tt \%multiplier\_tol} & real \\
  {\tt inner-iteration-relative-accuracy-required} & {\tt \%inner\_stop\_relative} & real \\
  {\tt inner-iteration-absolute-accuracy-required} & {\tt \%inner\_stop\_absolute} & real \\
  {\tt maximum-cpu-time-limit} & {\tt \%cpu\_time\_limit} & real \\
  {\tt maximum-clock-time-limit} & {\tt \%clock\_time\_limit} & real \\
  {\tt treat-zero-bounds-as-general} & {\tt \%treat\_zero\_bounds\_as\_general} & logical \\
  {\tt solve-qp} & {\tt \%solve\_qp} & logical \\
  {\tt solve-within-bounds} & {\tt \%solve\_within\_bounds} & logical \\
  {\tt temporarily-perturb-constraint-bounds} & {\tt \%randomize} & logical \\
%  {\tt array-syntax-worse-than-do-loop} & {\tt \%array\_syntax\_worse\_than\_do\_loop} & logical \\
\hline

\ectable{\label{specfile}Specfile commands and associated 
components of {\tt control}.}

\itt{device} is a scalar \intentin argument of type default \integer,
that must be set to the unit number on which the specfile
has been opened. If {\tt device} is not open, {\tt control} will
not be altered and execution will continue, but an error message
will be printed on unit {\tt control\%error}.

\end{description}

%%%%%%%%%%%%%%%%%%%%%% Information printed %%%%%%%%%%%%%%%%%%%%%%%%

\galinfo
If {\tt control\%print\_level} is positive, information about the progress 
of the algorithm will be printed on unit {\tt control\-\%out}.
If {\tt control\%print\_level} $= 1$, a single line of output will be produced 
for each iteration of the process. 
The overall algorithm, for a fixed value of the penalty parameter, 
is divided into major iterations, at which 
a factorization of the current reference matrix is performed, and
minor iterations, at which the the working set that defines the
reference matrix is gradually modified.
For the major iterations, details about the factorization performed are given.
Each minor iteration results in a single line summary of the
progress of the method. This includes the value of the merit function,
the step size taken, the size and number of the infeasibility, the
number of conjugate-gradient steps taken, the dimension of the
Schur complement, along with the numbers of positive and negative eigenvalues,
an brief indication of what happened during the iteration (the index of the
general constraint (c) or simple bound (b) which joined (+) or left (-) 
the active set), and the 
and the elapsed CPU time in seconds. 

If {\tt control\%print\_level} $\geq 2$ this 
output will be increased to provide significant detail of each iteration. 
This extra output includes detailed progress of the linesearch and the
residuals of the linear systems solved, and,
for larger values of {\tt control\%print\_level}, values of the primal and dual
variables and Lagrange multipliers. 

%%%%%%%%%%%%%%%%%%%%%% GENERAL INFORMATION %%%%%%%%%%%%%%%%%%%%%%%%

\galgeneral

\galcommon None.
\galworkspace Provided automatically by the module.
\galroutines None. 
\galmodules {\tt \packagename\_solve} calls the \galahad\ packages
\sloppy
{\tt GALAHAD\_\-CLOCK},
{\tt GALAHAD\_SY\-M\-BOLS}, 
{\tt GALAHAD\_NORMS}, 
{\tt GALAHAD\_SPACE}, 
{\tt GALAHAD\_RA\-ND}, 
{\tt GALAHAD\_TOOLS}, 
{\tt GALAHAD\_ROOTS}, 
{\tt GALAHAD\_SORT},
{\tt GALAHAD\_SMT}, 
{\tt GALAHAD\_QPT},
{\tt GALAHAD\_QPP},
{\tt GALAHAD\_QPD},
{\tt GALAHAD\_SLS}, 
{\tt GALAHAD\_SPECFILE}
and
{\tt GALAHAD\_SCU}.

\galio Output is under control of the arguments
 {\tt control\%error}, {\tt control\%out} and {\tt control\%print\_level}.
\galrestrictions {\tt prob\%n} $> 0$, {\tt prob\%m} $\geq  0$, 
{\tt prob\%A\_type} and {\tt prob\%H\_type} $\in \{${\tt 'DENSE'}, 
 {\tt 'COORDINATE'}, {\tt 'SPARSE\_BY\_ROWS'}, {\tt 'DIAGONAL'} $\}$. 
\galportability ISO Fortran~95 + TR 15581 or Fortran~2003. 
The package is thread-safe.

%%%%%%%%%%%%%%%%%%%%%% METHOD %%%%%%%%%%%%%%%%%%%%%%%%

\galmethod
At the $k$-th iteration of the method, an improvement to the value
of the merit function 
$m(\bmx, \rho_g, \rho_b ) = q(\bmx) + \rho_g v_g(\bmx) + \rho_b v_b(\bmx)$
at $\bmx = \bmx^{(k)}$ is sought. This is achieved by first 
computing a search direction $\bms^{(k)}$,
and then setting $\bmx^{(k+1)} = \bmx^{(k)} + \alpha^{(k)} \bms^{(k)}$,
where the stepsize $\alpha^{(k)}$ is chosen as the first local minimizer of 
$\phi ( \alpha ) = m( \bmx^{(k)} + \alpha \bms^{(k)} , \rho_g, \rho_b )$
as $\alpha$ incesases from zero. 
The stepsize calculation is straightforward, and exploits the fact that
$\phi ( \alpha )$ is a piecewise quadratic function of $\alpha$.

The search direction is defined by a subset of the "active" terms in 
$v(\bmx)$, i.e., those for which 
$\bma_i^T \bmx = c_i^l$ or $c_i^u$ (for $i=1,\ldots ,m$) or 
$x_j = x_j^l$ or $x_j^u$ (for {j=1,\ldots ,n}).
The "working" set $W^{(k)}$ is chosen from the active terms, and is such 
that its members have linearly independent gradients. 
The search direction $\bms^{(k)}$ is chosen as an approximate solution of 
the equality-constrained quadratic program
\eqn{4.1}{\minin{\bms \in \smallRe^n} q(\bmx^{(k)} + \bms) + 
 \rho_g l_g^{(k)} (\bms) + \rho_b l_b^{(k)} (\bms),}
subject to 
\eqn{4.2}{\bma_i^T \bms = 0,\;\;  i \in \{ 1, \ldots , m \} \cap W^{(k)},
\tim{and}
x_j = 0, \;\;  i  \in \{1, \ldots , n \} \cap W^{(k)},}
where
\disp{l_g^{(k)} (\bms) = 
     - \sum_{\stackrel{i=1}{\bma_i^T \bmx < c_i^l}}^m \bma_i^T \bms 
    \; + \sum_{\stackrel{i=1}{\bma_i^T \bmx > c_i^u}}^m \bma_i^T \bms}
and
\disp{l_b^{(k)} (\bms) = 
    - \sum_{\stackrel{j=1}{x_j < x_j^l}}^n s_j
    \; + \sum_{\stackrel{j=1}{x_j > x_j^u}}^n s_j .}
The equality-constrained quadratic program \req{4.1}--\req{4.2} is solved by
a projected preconditioned conjugate gradient method. The method terminates
either after a prespecified number of iterations, or if the solution is found,
or if a direction of infinite descent, along which 
$q(\bmx^{(k)} + \bms) + \rho_g l_g^{(k)} (\bms) + \rho_b l_b^{(k)} (\bms)$
decreases without bound within the feasible region \req{4.2}, is located. 
Succesively more accurate approximations are required as suspected 
solutions of \req{1.1} are approached.

Preconditioning of the conjugate gradient iteration
requires the solution of one or more linear systems of the form
\eqn{4.3}{\mat{cc}{\bmM^{(k)} & \bmA^{(k)T} \\ \bmA^{(k)} & 0 }
\vect{ \bmp \\ \bmu} = \vect{ \bmg \\ 0 },}
where $\bmM^{(k)}$ is a "suitable" approximation to $\bmH$
and the rows of $\bmA^{(k)}$ comprise the gradients of the
terms in the current working set. Rather than recomputing a
factorization of the preconditioner at every iteration, a
Schur complement method is used, recognising the fact that
gradual changes occur to successive working sets. The main
iteration is divided into a sequence of "major" iterations.
At the start of each major iteration (say, the overall iteration $l$), 
a factorization of the
current "reference" matrix, that is the matrix
\eqn{4.4}{\mat{cc}{\bmM^{(l)} & \bmA^{(l)T} \\ \bmA^{(l)} & 0 }}
is obtained using the \galahad\ matrix factorization package 
{\tt GALAHAD\_SLS}.
This reference matrix may be factorized as a whole (the so-called 
"augmented system" approach), or by performing a block elimination first 
(the "Schur-complement" approach). The latter is usually to be preferred 
when a (non-singular) diagonal 
preconditioner is used, but may be inefficient if any of the columns
of $\bmA^{(l)}$ is too dense.
Subsequent iterations within the current major
iteration obtain solutions to \req{4.3} via the factors of \req{4.4}
and an appropriate (dense) Schur complement,
obtained from the \galahad\ package {\tt GALAHAD\_SCU}.
The major iteration terminates
once the space required to hold the factors of the (growing) Schur
complement exceeds a given threshold.

The working set changes by (a) adding an active term encountered during 
the determination of the stepsize, or (b) the removal of a term if $\bms = 0$
solves \req{4.1}--\req{4.2}. The  decision on which to remove in the latter 
case is based upon the expected decrease upon the removal of an individual term,
and this information is available from the magnitude and sign of the components
of the auxiliary vector $\bmu$ computed in \req{4.3}. At optimality, the
components of $\bmu$ for $\bma_i$ terms will all lie between 
$0$ and $\rho_g$---and those for the other terms 
between $0$ and $\rho_b$---and any violation
of this rule indicates further progress is possible. The components
of $\bmu$ corresonding to the terms involving $\bma_i^T \bmx$
are sometimes known as Lagrange multipliers (or generalized gradients) and
denoted by $\bmy$, while those for the remaining $x_j$ terms are dual variables
and denoted by $\bmz$.

To solve \req{1.2}--\req{1.4}, a sequence of problems of the form \req{1.1} are
solved, each with a larger value of $\rho_g$ and/or $\rho_b$ 
than its predecessor. The
required solution has been found once the infeasibilities 
$v_g(\bmx)$ and $v_b(\bmx)$ have been reduced to zero at the solution of 
\req{1.1} for the given $\rho_g$ and $\rho_b$.

The required solution $\bmx$ to \req{1.2}--\req{1.4} necessarily satisfies 
the primal optimality conditions
\disp{\bmA \bmx = \bmc}
and 
\disp{
 \bmc^{l} \leq \bmc \leq \bmc^{u}, \;\;
\bmx^{l} \leq\bmx \leq\bmx^{u},}
the dual optimality conditions
\disp{
 \bmH \bmx + \bmg =
 \bmA^{T} \bmy + \bmz,\;\;
 \bmy = \bmy^{l} + \bmy^{u} \tim{and}
 \bmz = \bmz^{l} + \bmz^{u} ,}
and 
\disp{
 \bmy^{l} \geq 0 , \;\;
 \bmy^{u} \leq 0 , \;\;
 \bmz^{l} \geq 0 \;\; \mbox{and} \;\;
 \bmz^{u} \leq 0 ,}
and the complementary slackness conditions 
\disp{
( \bmA \bmx - \bmc^{l} )^{T} \bmy^{l} = 0  ,\;\;
( \bmA \bmx - \bmc^{u} )^{T} \bmy^{u} = 0  ,\;\;
(\bmx -\bmx^{l} )^{T} \bmz^{l} = 0   \tim{and}
(\bmx -\bmx^{u} )^{T} \bmz^{u} = 0 ,}
where, as before, the vectors $\bmy$ and $\bmz$ are 
known as the Lagrange multipliers for
the general linear constraints, and the dual variables for the bounds,
respectively, and where the vector inequalities hold componentwise.

In order to make the solution as efficient as possible, the 
variables and constraints are reordered internally
by the \galahad\ package {\tt GALAHAD\_QPP} prior to solution. 
In particular, fixed variables and 
free (unbounded on both sides) constraints are temporarily removed.

\vspace*{1mm}

\galreferences
\vspace*{1mm}

\noindent
The method is described in detail in
\vspace*{1mm}

\noindent
N. I. M. Gould and Ph. L. Toint (2001).
``An iterative working-set method 
for large-scale non-convex quadratic programming''.
\em Applied Numerical Mathematics
\bf 43 (1--2) \rm (2002) 109--128.

%%%%%%%%%%%%%%%%%%%%%% EXAMPLE %%%%%%%%%%%%%%%%%%%%%%%%

\galexample
Suppose we wish to minimize
$\half x_1^2 + x_2^2 + \threehalves x_3^2 + 4 x_1 x_3 + 2 x_2 + 1$
subject to the the general linear constraints
$1 \leq  2 x_{1}  +  x_{2}  \leq  2$ and
$x_{2}  +  x_{3}  =  2$, and simple bounds
$-1  \leq  x_{1}  \leq  1$ and $x_{3}  \leq  2$,
Then, on writing the data for this problem as
\disp{\bmH = \mat{ccc}{1 & & 4 \\ & 2 & \\ 4 &  & 3}, \;\;
 \bmg = \vect{ 0 \\ 2 \\ 0 }, \;\;
\bmx^{l} = \vect{ -1 \\ - \infty \\ - \infty } \tim{and}
\bmx^{u} = \vect{ 1 \\ \infty \\ 2 },}
and
\disp{
 \bmA = \mat{ccc}{ 2 & 1 & \\ & 1 & 1},\;\;
 \bmc^{l} = \vect{ 1 \\ 2 }, \tim{and}
 \bmc^{u} = \vect{ 2 \\ 2 }}
in sparse co-ordinate format,
we may use the following code:

{\tt \small
\VerbatimInput{\packageexample}
}
\noindent
This produces the following output:
{\tt \small
\VerbatimInput{\packageresults}
}
\noindent
The same problem may be solved holding the data in 
a sparse row-wise storage format by replacing the lines
{\tt \small
\begin{verbatim}
!  sparse co-ordinate storage format
...
! problem data complete   
\end{verbatim}
}
\noindent
by
{\tt \small
\begin{verbatim}
! sparse row-wise storage format
   CALL SMT_put( p%H%type, 'SPARSE_BY_ROWS', s )  ! Specify sparse-by-row
   CALL SMT_put( p%A%type, 'SPARSE_BY_ROWS', s )  ! storage for H and A
   ALLOCATE( p%H%val( h_ne ), p%H%col( h_ne ), p%H%ptr( n + 1 ) )
   ALLOCATE( p%A%val( a_ne ), p%A%col( a_ne ), p%A%ptr( m + 1 ) )
   p%H%val = (/ 1.0_wp, 2.0_wp, 3.0_wp, 4.0_wp /) ! Hessian H
   p%H%col = (/ 1, 2, 3, 1 /)                     ! NB lower triangular
   p%H%ptr = (/ 1, 2, 3, 5 /)                     ! Set row pointers
   p%A%val = (/ 2.0_wp, 1.0_wp, 1.0_wp, 1.0_wp /) ! Jacobian A
   p%A%col = (/ 1, 2, 2, 3 /)
   p%A%ptr = (/ 1, 3, 5 /)                        ! Set row pointers  
! problem data complete   
\end{verbatim}
}
\noindent
or using a dense storage format with the replacement lines
{\tt \small
\begin{verbatim}
! dense storage format
   CALL SMT_put( p%H%type, 'DENSE', s )  ! Specify dense
   CALL SMT_put( p%A%type, 'DENSE', s )  ! storage for H and A
   ALLOCATE( p%H%val( n * ( n + 1 ) / 2 ) )
   ALLOCATE( p%A%val( n * m ) )
   p%H%val = (/ 1.0_wp, 0.0_wp, 2.0_wp, 4.0_wp, 0.0_wp, 3.0_wp /) ! Hessian
   p%A%val = (/ 2.0_wp, 1.0_wp, 0.0_wp, 0.0_wp, 1.0_wp, 1.0_wp /) ! Jacobian
! problem data complete   
\end{verbatim}
}
\noindent
respectively.

If instead $\bmH$ had been the diagonal matrix
\disp{\bmH = \mat{ccc}{1 & &   \\ & 0 & \\  &  & 3}}
but the other data is as before, the diagonal storage scheme 
might be used for $\bmH$, and in this case we would instead 
{\tt \small
\begin{verbatim}
   CALL SMT_put( prob%H%type, 'DIAGONAL', s )  ! Specify dense storage for H
   ALLOCATE( p%H%val( n ) )
   p%H%val = (/ 1.0_wp, 0.0_wp, 3.0_wp /) ! Hessian values
\end{verbatim}
}
\noindent
Notice here that zero diagonal entries are stored.

\end{document}
