\documentclass{galahad}

% set the package name

\newcommand{\package}{wcp}
\newcommand{\packagename}{WCP}
\newcommand{\fullpackagename}{\libraryname\_\-\packagename}

\begin{document}

\galheader

%%%%%%%%%%%%%%%%%%%%%% SUMMARY %%%%%%%%%%%%%%%%%%%%%%%%

\galsummary
This package uses a primal-dual interior-point method
to {\bf find a well-centered interior point} $\bmx$ for a set of
general linear constraints
\eqn{lc}{c_{i}^{l}  \leq  \bma_{i}^{T}\bmx  \leq  c_{i}^{u}, \;\;\;
 i = 1, \ldots , m,}
and simple bounds
\eqn{bc}{x_{j}^{l}  \leq  x_{j}^{ } \leq  x_{j}^{u}, \;\;\; j = 1, \ldots , n,}
where the vectors 
$\bma_{i}$, $\bmc^{l}$, $\bmc^{u}$, $\bmx^{l}$ and $\bmx^{u}$ are given.
More specifically, if possible, the package finds a solution to the 
system of  primal optimality equations
\eqn{prim}{\bmA \bmx = \bmc,}
dual optimality equations
\eqn{dual}{\bmg = 
 \bmA^{T} \bmy + \bmz,\;\;
 \bmy = \bmy^{l} + \bmy^{u} \tim{and}
 \bmz = \bmz^{l} + \bmz^{u} ,}
and perturbed complementary slackness equations
\eqn{slknessc}{
( c_i^{} - c^l_i ) y^{l}_i = (\mu_c^{l})_i^{} \tim{and}
( c_i^{} - c_i^u ) y^{u}_i = (\mu_c^{u})_i^{}, \;\;\;
 i = 1, \ldots , m,}
and
\eqn{slknessx}{
((x^{}_j - x^l_j ) z_j^{l} = (\mu_x^{l})_j^{}  \tim{and}
( x^{}_j - x^u_j ) z_j^{u} = (\mu_x^{u})_j^{}, \;\;\; 
 j = 1, \ldots , n,}
for which
\eqn{boundary}{
 \bmc^{l} \leq \bmc \leq \bmc^{u}, \;\; \bmx^{l} \leq \bmx \leq \bmx^{u}, \;\;
 \bmy^{l} \geq \bmzero , \;\;  \bmy^{u} \leq \bmzero , \;\;
 \bmz^{l} \geq \bmzero \;\; \mbox{and} \;\;  \bmz^{u} \leq \bmzero .}
Here $\bmA$ is the matrix whose rows are the $\bma_i^T$, $i = 1, \ldots , m$,
$\bmmu_c^{l}$, $\bmmu_c^{u}$, $\bmmu_x^{l}$ and $\bmmu_x^{u}$ are
vectors of strictly positive {\em targets}, $\bmg$ is another given 
target vector (which is often zero),
and $(\bmy^{l}, \bmy^{u})$ and $(\bmz^{l}, \bmz^{u})$ are dual variables
for the linear constraints and simple bounds respectively; 
$\bmc$ gives the constraint value $\bmA \bmx$.
Since \req{slknessc}--\req{boundary} normally imply that
\eqn{interior}{
 \bmc^{l} < \bmc < \bmc^{u}, \;\; \bmx^{l} < \bmx < \bmx^{u}, \;\;
 \bmy^{l} > \bmzero , \;\;  \bmy^{u} < \bmzero , \;\;
 \bmz^{l} > \bmzero \;\; \mbox{and} \;\;  \bmz^{u} < \bmzero,}
such a primal-dual point $(\bmx, \bmc, \bmy^{l}, \bmy^{u}, \bmz^{l}, \bmz^{l})$
may be used, for example, as a feasible starting point for primal-dual 
interior-point methods for solving the linear programming problem 
of minimizing $\bmg^T \bmx$ subject to \req{lc} and \req{bc}.

Full advantage is taken of any zero coefficients in the 
vectors $\bma_{i}$. 
Any of the constraint bounds $c_{i}^{l}$, $c_{i}^{u}$, 
$x_{j}^{l}$ and $x_{j}^{u}$ may be infinite.
The package identifies infeasible problems, and problems for which there
is no strict interior, that is one or more of \req{boundary}
only holds as an equality for all feasible points.

%%%%%%%%%%%%%%%%%%%%%% attributes %%%%%%%%%%%%%%%%%%%%%%%%

\galattributes
\galversions{\tt  \fullpackagename\_single, \fullpackagename\_double}.
\galuses 
{\tt GALAHAD\_\-CLOCK},
{\tt GALAHAD\_SY\-M\-BOLS}, 
{\tt GALAHAD\_SMT},
{\tt GALAHAD\_QPT}, {\tt GALAHAD\_QPD}, {\tt GALAHAD\_SPECFILE}, 
{\tt GALAHAD\_QPP}, {\tt GALAHAD\_SPACE}, {\tt GALAHAD\_\-SORT}, 
{\tt GALAHAD\_STRING}, {\tt GALAHAD\_\-ROOTS}, {\tt GALAHAD\_\-SBLS},
{\tt GALAHAD\_FDC}.
\galdate July 2006.
\galorigin C. Cartis and N. I. M. Gould, Rutherford Appleton Laboratory.
\gallanguage Fortran~95 + TR 15581 or Fortran~2003. 
\galparallelism Some options may use OpenMP and its runtime library.

%%%%%%%%%%%%%%%%%%%%%% HOW TO USE %%%%%%%%%%%%%%%%%%%%%%%%

\galhowto

%\subsection{Calling sequences}

Access to the package requires a {\tt USE} statement such as

\medskip\noindent{\em Single precision version}

\hspace{8mm} {\tt USE \fullpackagename\_single}

\medskip\noindent{\em Double precision version}

\hspace{8mm} {\tt USE  \fullpackagename\_double}

\medskip

\noindent
If it is required to use both modules at the same time, the derived types 
{\tt SMT\_type}, 
{\tt QPT\_problem\_type}, 
{\tt \packagename\_time\_type}, 
{\tt \packagename\_control\_type}, 
{\tt \packagename\_inform\_type} 
and
{\tt \packagename\_data\_type}
(Section~\ref{galtypes})
and the subroutines
{\tt \packagename\_initialize}, 
{\tt \packagename\_\-solve},
{\tt \packagename\_terminate},
(Section~\ref{galarguments})
and 
{\tt \packagename\_read\_specfile}
(Section~\ref{galfeatures})
must be renamed on one of the {\tt USE} statements.

%%%%%%%%%%%%%%%%%%%%%% matrix formats %%%%%%%%%%%%%%%%%%%%%%%%

\galmatrix
The constraint Jacobian $\bmA$, that is, the matrix 
whose rows are the vectors $\bma_{i}^{T}$, $i = 1, \ldots , m$, 
may be stored in one of three input formats.

\subsubsection{Dense storage format}\label{dense}
The matrix is stored as a compact 
dense matrix by rows, that is, the values of the entries of each row in turn are
stored in order within an appropriate real one-dimensional array.
Component $n \ast (i-1) + j$ of the storage array {\tt A\%val} will hold the 
value $a_{ij}$ for $i = 1, \ldots , m$, $j = 1, \ldots , n$.

\subsubsection{Sparse co-ordinate storage format}\label{coordinate}
Only the nonzero entries of the matrix are stored. For the 
$l$-th entry of $\bmA$, its row index $i$, column index $j$ 
and value $a_{ij}$
are stored in the $l$-th components of the integer arrays {\tt A\%row}, 
{\tt A\%col} and real array {\tt A\%val}, respectively.
The order is unimportant, but the total
number of entries {\tt A\%ne} is also required. 

\subsubsection{Sparse row-wise storage format}\label{rowwise}
Again only the nonzero entries are stored, but this time
they are ordered so that those in row $i$ appear directly before those
in row $i+1$. For the $i$-th row of $\bmA$, the $i$-th component of a 
integer array {\tt A\%ptr} holds the position of the first entry in this row,
while {\tt A\%ptr} $(m+1)$ holds the total number of entries plus one.
The column indices $j$ and values $a_{ij}$ of the entries in the $i$-th row 
are stored in components 
$l =$ {\tt A\%ptr}$(i)$, \ldots ,{\tt A\%ptr} $(i+1)-1$ of the 
integer array {\tt A\%col}, and real array {\tt A\%val}, respectively. 

For sparse matrices, this scheme almost always requires less storage than 
its predecessor.

%%%%%%%%%%%%%%%%%%%%%% long integers %%%%%%%%%%%%%%%%%%%%%%

\subsection{Integer kinds}\label{Integer kinds}
We use the term
long \integer\ to denote {\tt INTEGER\-(kind=long)}, where 
{\tt long = selected\_int\_kind(18))}.

%%%%%%%%%%%%%%%%%%%%%% parallel usage %%%%%%%%%%%%%%%%%%%%%%%%

%%%%%%%%%%%%%%%%%%%%%% parallel usage %%%%%%%%%%%%%%%%%%%%%%%%

\subsection{Parallel usage}
OpenMP may be used by the {\tt \fullpackagename} package to provide
parallelism for some solvers in shared memory environments.
See the documentation for the \galahad\ package {\tt SLS} for more details.
To run in parallel, OpenMP
must be enabled at compilation time by using the correct compiler flag
(usually some variant of {\tt -openmp}).
The number of threads may be controlled at runtime
by setting the environment variable {\tt OMP\_NUM\_THREADS}.

\noindent
MPI may also be used by the package to provide
parallelism for some solvers in a distributed memory environment.
To use this form of parallelism, MPI must be enabled at runtime
by using the correct compiler flag
(usually some variant of {\tt -lmpi}).
Although the MPI process will be started automatically when required,
it should be stopped by the calling program once no further use of
this form of parallelism is needed. Typically, this will be via
statements of the form
\begin{verbatim}
      CALL MPI_INITIALIZED( flag, ierr )
      IF ( flag ) CALL MPI_FINALIZE( ierr )
\end{verbatim}

\noindent
The code may be compiled and run in serial mode.


%%%%%%%%%%%%%%%%%%%%%% derived types %%%%%%%%%%%%%%%%%%%%%%%%

\galtypes
Six derived data types are accessible from the package.

%%%%%%%%%%% matrix data type %%%%%%%%%%%

\subsubsection{The derived data type for holding matrices}\label{typesmt}
The derived data type {\tt SMT\_TYPE} is used to hold the matrix $\bmA$.
The components of {\tt SMT\_TYPE} used here are:

\begin{description}

\ittf{m} is a scalar component of type default \integer, 
that holds the number of rows in the matrix. 
 
\ittf{n} is a scalar component of type default \integer, 
that holds the number of columns in the matrix. 
 
\ittf{ne} is a scalar variable of type default \integer, that
holds the number of matrix entries

\ittf{type} is a rank-one allocatable array of type default \character, that
is used to indicate the matrix storage scheme used. Its precise length and
content depends on the type of matrix to be stored (see \S\ref{typeprob}).

\ittf{val} is a rank-one allocatable array of type default \realdp\, 
and dimension at least {\tt ne}, that holds the values of the entries. 
Each pair of off-diagonal entries $a_{ij} = a_{ji}$ of a matrix $\bmA$ 
is represented as a single entry 
(see \S\ref{dense}--\ref{rowwise}).
Any duplicated entries that appear in the sparse 
co-ordinate or row-wise schemes will be summed. 

\ittf{row} is a rank-one allocatable array of type default \integer, 
and dimension at least {\tt ne}, that may hold the row indices of the entries. 
(see \S\ref{coordinate}).

\ittf{col} is a rank-one allocatable array of type default \integer, 
and dimension at least {\tt ne}, that may hold the column indices of the entries
(see \S\ref{coordinate}--\ref{rowwise}).

\ittf{ptr} is a rank-one allocatable array of type default \integer, 
and dimension at least {\tt m + 1}, that may hold the pointers to
the first entry in each row (see \S\ref{rowwise}).

\end{description}

%%%%%%%%%%% problem type %%%%%%%%%%%

\subsubsection{The derived data type for holding the problem}\label{typeprob}
The derived data type {\tt QPT\_problem\_type} is used to hold 
the problem. The components of 
{\tt QPT\_problem\_type} 
are:

\begin{description}

\itt{new\_problem\_structure} is a scalar variable of type default \logical, 
 that is \true\ if this is the first (or only) problem in a sequence of 
 problems with identical "structure" to be attempted, and \false\ if 
 a previous problem with the same "structure" (but different
 numerical data) has been solved. Here, the term "structure" refers both to 
 the sparsity patterns of the Jacobian matrices $\bmA$ involved 
 (but not their numerical values), to the zero/nonzero/infinity patterns 
 (a bound is either zero, $\pm$ infinity, or a finite but arbitrary 
 nonzero) of each of the constraint bounds, and to the variables and constraints
 that are fixed (both bounds are the same) or free (the lower and upper
 bounds are $\pm$ infinity, respectively).

\ittf{n} is a scalar variable of type default \integer, 
 that holds the number of optimization variables, $n$.  
              
\ittf{m} is a scalar variable of type default \integer, 
 that holds the number of general linear constraints, $m$.
              
\itt{gradient\_kind} is a scalar variable of type default \integer, 
that is used to indicate whether the components of the target vector $\bmg$ 
have special or general values. Possible values for {\tt gradient\_kind} are:
\begin{description}
\itt{0}  In this case, $\bmg = 0$.

\itt{1} In this case, $g_{i} = 1$ for $i = 1, \ldots ,n$.

\itt{$\neq$ 0,1} In this case, general values of $\bmg$ will be used,
     and will be provided by the user in the component {\tt G}.
\end{description}

\ittf{G} is a rank-one allocatable array of dimension {\tt n} and type 
default \realdp, that holds the target vector $\bmg$.
The $j$-th component of 
{\tt G}, $j = 1,  \ldots ,  n$, contains $\bmg_{j}$.
If {\tt gradient\_kind} {= 0, 1}, {\tt G} need not be allocated.

\ittf{A} is scalar variable of type {\tt SMT\_TYPE} 
that holds the Jacobian matrix $\bmA$. The following components are used:

\begin{description}

\itt{A\%type} is an allocatable array of rank one and type default \character, that
is used to indicate the storage scheme used. If the dense storage scheme 
(see Section~\ref{dense}) is used, 
the first five components of {\tt A\%type} must contain the
string {\tt DENSE}.
For the sparse co-ordinate scheme (see Section~\ref{coordinate}), 
the first ten components of {\tt A\%type} must contain the
string {\tt COORDINATE}, while 
for the sparse row-wise storage scheme (see Section~\ref{rowwise}),
the first fourteen components of {\tt A\%type} must contain the
string {\tt SPARSE\_BY\_ROWS}.

For convenience, the procedure {\tt SMT\_put} 
may be used to allocate sufficient space and insert the required keyword
into {\tt A\%type}.
For example, if {\tt prob} is of derived type {\tt \packagename\_problem\_type}
and involves a Jacobian we wish to store using the sparse row-wise 
storage scheme, we may simply
%\vspace*{-2mm}
{\tt 
\begin{verbatim}
        CALL SMT_put( prob%A%type, 'SPARSE_BY_ROWS' )
\end{verbatim}
}
%\vspace*{-4mm}
\noindent
See the documentation for the \galahad\ package {\tt SMT} 
for further details on the use of {\tt SMT\_put}.

\itt{A\%ne} is a scalar variable of type default \integer, that 
holds the number of entries in $\bmA$
in the sparse co-ordinate storage scheme (see Section~\ref{coordinate}). 
It need not be set for either of the other two schemes.

\itt{A\%val} is a rank-one allocatable array of type default \realdp, that holds
the values of the entries of the Jacobian matrix $\bmA$ in any of the 
storage schemes discussed in Section~\ref{galmatrix}.

\itt{A\%row} is a rank-one allocatable array of type default \integer,
that holds the row indices of $\bmA$ in the sparse co-ordinate storage
scheme (see Section~\ref{coordinate}). 
It need not be allocated for either of the other two schemes.

\itt{A\%col} is a rank-one allocatable array variable of type default \integer,
that holds the column indices of $\bmA$ in either the sparse co-ordinate 
(see Section~\ref{coordinate}), or the sparse row-wise 
(see Section~\ref{rowwise}) storage scheme.
It need not be allocated when the dense storage scheme is used.

\itt{A\%ptr} is a rank-one allocatable array of dimension {\tt m+1} and type 
default \integer, that holds the 
starting position of each row of $\bmA$, as well
as the total number of entries plus one, in the sparse row-wise storage
scheme (see Section~\ref{rowwise}). It need not be allocated when the
other schemes are used.

\end{description}

\ittf{C\_l} is a rank-one allocatable array of dimension {\tt m} and type 
default \realdp, that holds the vector of lower bounds $\bmc^{l}$ 
on the general constraints. The $i$-th component of 
{\tt C\_l}, $i = 1, \ldots , m$, contains $\bmc_{i}^{l}$.
Infinite bounds are allowed by setting the corresponding 
components of {\tt C\_l} to any value smaller than {\tt -infinity}, 
where {\tt infinity} is a component of the control array {\tt control} 
(see Section~\ref{typecontrol}).

\ittf{C\_u} is a rank-one allocatable array of dimension {\tt m} and type 
default \realdp, that holds the vector of upper bounds $\bmc^{u}$ 
on the general constraints. The $i$-th component of 
{\tt C\_u}, $i = 1,  \ldots ,  m$, contains $\bmc_{i}^{u}$.
Infinite bounds are allowed by setting the corresponding 
components of {\tt C\_u} to any value larger than {\tt infinity}, 
where {\tt infinity} is a component of the control array {\tt control} 
(see Section~\ref{typecontrol}).

\ittf{X\_l} is a rank-one allocatable array of dimension {\tt n} and type 
default \realdp, that holds
the vector of lower bounds $\bmx^{l}$ on the the variables.
The $j$-th component of {\tt X\_l}, $j = 1, \ldots , n$, 
contains $\bmx_{j}^{l}$.
Infinite bounds are allowed by setting the corresponding 
components of {\tt X\_l} to any value smaller than {\tt -infinity}, 
where {\tt infinity} is a component of the control array {\tt control} 
(see Section~\ref{typecontrol}).

\ittf{X\_u} is a rank-one allocatable array of dimension {\tt n} and type 
default \realdp, that holds
the vector of upper bounds $\bmx^{u}$ on the variables.
The $j$-th component of {\tt X\_u}, $j = 1, \ldots , n$, 
contains $\bmx_{j}^{u}$.
Infinite bounds are allowed by setting the corresponding 
components of {\tt X\_u} to any value larger than that {\tt infinity}, 
where {\tt infinity} is a component of the control array {\tt control} 
(see Section~\ref{typecontrol}).

\ittf{X} is a rank-one allocatable array of dimension {\tt n} and type 
default \realdp, 
that holds the values $\bmx$ of the optimization variables.
The $j$-th component of {\tt X}, $j = 1,  \ldots , n$, contains $x_{j}$.  

\ittf{Z\_l} is a rank-one allocatable array of dimension {\tt n} and type default 
\realdp, that holds
the values $\bmz^{l}$ of estimates  of the dual variables 
corresponding to the lower simple bound constraints $\bmx^{l} \leq \bmx$
(see equation \req{dual}).
The $j$-th component of {\tt Z\_l}, $j = 1,  \ldots ,  n$, 
contains $z_{j}^{l}$.  

\ittf{Z\_u} is a rank-one allocatable array of dimension {\tt n} and type default 
\realdp, that holds
the values $\bmz^{u}$ of estimates  of the dual variables 
corresponding to the upper simple bound constraints $\bmx \leq \bmx^{u}$
(see equation \req{dual}).
The $j$-th component of {\tt Z\_l}, $j = 1,  \ldots ,  n$, 
contains $z_{j}^{l}$.  

\ittf{C} is a rank-one allocatable array of dimension {\tt m} and type default 
\realdp, that holds
the values $\bmA \bmx$ of the constraints.
The $i$-th component of {\tt C}, $i = 1,  \ldots ,  m$, contains 
$\bma_{i}^{T}\bmx \equiv (\bmA \bmx)_{i}$.  

\ittf{Y\_l} is a rank-one allocatable array of dimension {\tt m} and type default 
\realdp, that holds
the values $\bmy^{l}$ of estimates  of the Lagrange multipliers
corresponding to the lower general constraints $\bmc^{l} \leq \bmA \bmx$
(see equation \req{dual}).
The $i$-th component of {\tt Y\_l}, $i = 1,  \ldots ,  m$, 
contains $y_{i}^{l}$.  

\ittf{Y\_u} is a rank-one allocatable array of dimension {\tt m} and type default 
\realdp, that holds
the values $\bmy^{u}$ of estimates  of the Lagrange multipliers
corresponding to the upper general constraints $\bmA \bmx \leq \bmc^{u}$
(see equation \req{dual}).
The $i$-th component of {\tt Y\_u}, $i = 1,  \ldots ,  m$, 
contains $y_{i}^{u}$.  

\end{description}

%%%%%%%%%%% control type %%%%%%%%%%%

\subsubsection{The derived data type for holding control 
 parameters}\label{typecontrol}
The derived data type 
{\tt \packagename\_control\_type} 
is used to hold controlling data. Default values may be obtained by calling 
{\tt \packagename\_initialize}
(see Section~\ref{subinit}),
while components may also be changed by calling 
{\tt \fullpackagename\_read\-\_spec}
(see Section~\ref{readspec}). 
The components of 
{\tt \packagename\_control\_type} 
are:

\begin{description}

\itt{error} is a scalar variable of type default \integer, that holds the
stream number for error messages. Printing of error messages in 
{\tt \packagename\_solve} and {\tt \packagename\_terminate} is suppressed if 
{\tt error} $\leq 0$.
The default is {\tt error = 6}.

\ittf{out} is a scalar variable of type default \integer, that holds the
stream number for informational messages. Printing of informational messages in 
{\tt \packagename\_solve} is suppressed if {\tt out} $< 0$.
The default is {\tt out = 6}.

\itt{print\_level} is a scalar variable of type default \integer, that is used
to control the amount of informational output which is required. No 
informational output will occur if {\tt print\_level} $\leq 0$. If 
{\tt print\_level} $= 1$, a single line of output will be produced for each
iteration of the process. If {\tt print\_level} $\geq 2$, this output will be
increased to provide significant detail of each iteration.
The default is {\tt print\_level = 0}.

\itt{maxit} is a scalar variable of type default \integer, that holds the
maximum number of iterations which will be allowed in {\tt \packagename\_solve}.
The default is {\tt maxit = 1000}.

\itt{start\_print} is a scalar variable of type default \integer, that specifies
the first iteration for which printing will occur in {\tt \packagename\_solve}.
If {\tt start\_print} is negative, printing will occur from the outset.
The default is {\tt start\_print = -1}.

\itt{stop\_print} is a scalar variable of type default \integer, that specifies
the last iteration for which printing will occur in  {\tt \packagename\_solve}.
If {\tt stop\_print} is negative, printing will occur once it has been 
started by {\tt start\_print}.
The default is {\tt stop\_print = -1}.

\itt{initial\_point} is a scalar variable of type default \integer, 
that indicates how the initial point is chosen.
Possible values are:

\begin{description}
\itt{0} the input point $\bmx$ may be perturbed to ensure that its $j$-th
component is feasible with respect to its
bounds $x_{j}^{l}  \leq  x_{j}^{ } \leq  x_{j}^{u}$, and
if possible at least {\tt prfeas} from either bound. If possible, 
the slack variable $\bmc$ will be feasible and at least {\tt prfeas} from 
its bounds $\bmc^{l} \leq \bmc \leq \bmc^{u}$.
The input dual variables/Lagrange multipliers will be perturbed so that they 
are feasible and at least 
{\tt dufeas} from their bounds  $\bmy^{l} \geq 0$, $\bmy^{u} \leq 0$,
$\bmz^{l} \geq 0$ and  $\bmz^{u} \leq 0$. The remaining constraints
\req{prim}--\req{slknessx} will most likely not be satisfied. The feasible region
is that enlarged by perturbing all inequality constraints by 
{\tt perturb\_start} (see below and \S\ref{galmethod}).
\itt{1} a point satisfying \req{prim}--\req{dual} that also tries to satisfy
\req{interior} will be computed. The feasible region is enlarged by perturbing
each inequality constraints so that its residual is at least
{\tt prfeas} (primal and slack variables) or
{\tt dufeas} (dual variables and Lagrange multipliers).
\itt{2} as in {\tt 1}, but the perturbations are equal and chosen so that
the smallest residual is at least {\tt prfeas} (primal and slack variables) or
{\tt dufeas} (dual variables and Lagrange multipliers).
\end{description}
The default is {\tt initial\_point = 0}.

%\itt{factor} is a scalar variable of type default \integer, that indicates
%the type of factorization of the Newton-equation matrix to be used.  
%Possible values are:

%\begin{description}
%\itt{0} the type is chosen automatically on the basis of which option looks
%        likely to be the most efficient.
%\itt{1} a Schur-complement factorization will be used.
%\itt{2} an augmented-system factorization will be used.
%\end{description}
%The default is {\tt factor = 0}.

%\itt{max\_col} is a scalar variable of type default \integer, that specifies
%the maximum number of nonzeros in a column of $\bmA$ which is permitted
%by the Schur-complement factorization.
%The default is {\tt max\_col = 35}.

%\itt{indmin} is a scalar variable of type default \integer, that specifies
%an initial estimate as to the amount of integer workspace required by 
%the factorization package {\tt SBLS}.
%The default is {\tt indmin = 1000}.

%\itt{valmin} is a scalar variable of type default \integer, that specifies
%an initial estimate as to the amount of real workspace required by 
%the factorization package {\tt SBLS}.
%The default is {\tt valmin = 1000}.

\itt{perturbation\_strategy} is a scalar variable of type default \integer, 
that specifies the strategy used for perturbing and then tightening
the inequality constraints. The perturbations are chosen so that the
iterates lie interior to the perturbed constraints (see \S\ref{galmethod}).
Possible values are:

\begin{description}
\itt{$\leq$ 0} no perturbation is allowed.
\itt{1} the initial perturbation is determined by the initial-point
strategy (see {\tt initial\_point} above). The perturbation $\theta$, say,
on a generic constraint $w \geq w^l$ is subsequently 
reduced to $\bar{\theta}$ so that $w - w^l + \bar{\theta}
= ( 1 - \eta ) ( w - w^l + \theta )$
where $\eta$ has the value {\tt frac\_perturb}.
\itt{2} the initial perturbation is determined by the initial-point
strategy (see {\tt initial\_point} above). The perturbation $\theta$, say,
on a generic constraint $w \geq w^l$ is subsequently 
reduced to $\bar{\theta}$ so that $w - w^l + \bar{\theta}
= ( 1 - \eta ) ( w - w^l + \theta )$,
where $\eta$ has the value {\tt frac\_perturb}, whenever $w \leq w^l$. If 
$w > w^l$, the perturbation  $\bar{\theta}$ is set to zero.
\itt{3} the same as {\tt 1}, except that $\eta \leq$ {\tt frac\_perturb}
is reduced gradually to zero.
\itt{4} the same as {\tt 2}, except that $\eta \leq$ {\tt frac\_perturb}
is reduced gradually to zero.
\end{description}
The default is {perturbation\_strategy = 2}.

%\itt{itref\_max} is a scalar variable of type default \integer, that specifies 
%the maximum number of iterative refinements allowed for each solution of
%the Newton equations.
%The default is {\tt itref\_max = 1}.

\itt{infeas\_max} is a scalar variable of type default \integer, that specifies
the number of iterations for which the overall infeasibility
of the problem is not reduced by at least a factor 
{\tt required\_infeas\_reduction
before} the problem is flagged as infeasible 
(see {\tt required\_infeas\_reduction}).
The default is {\tt infeas\_max = 200}.
%The default is {\tt infeas\_max = 10}.

\itt{restore\_problem} is a scalar variable of type default \integer, that 
specifies how much of the input problem is to be retored on output. 
Possible values are:
\begin{description}
\itt{0} nothing is restored. 
\itt{1} the vector data $\bmw$, $\bmg$, 
   $\bmc^{l}$, $\bmc^{u}$, $\bmx^{l}$, and $\bmx^{u}$ 
   will be restored to their input values.
\itt{2} the entire problem, that is the above vector data along with 
the Jacobian matrix $\bmA$, will be restored.
\end{description}
The default is {\tt restore\_problem = 2}.

\itt{infinity} is a scalar variable of type default \realdp, that is used to
specify which constraint bounds are infinite.
Any bound larger than {\tt infinity} in modulus will be regarded as infinite.
The default is {\tt infinity =} $10^{19}$.

\itt{stop\_p} is a scalar variable of type default \realdp, that holds the
required accuracy for the primal infeasibility \req{prim}.
The default is {\tt stop\_p =} $u^{1/3}$,
where $u$ is {\tt EPSILON(1.0)} ({\tt EPSILON(1.0D0)} in 
{\tt \fullpackagename\_double}).

\itt{stop\_d} is a scalar variable of type default 
\realdp, that holds the
required accuracy for the dual infeasibility \req{dual}
The default is {\tt stop\_d =} $u^{1/3}$,
where $u$ is {\tt EPSILON(1.0)} ({\tt EPSILON(1.0D0)} in 
{\tt \fullpackagename\_double}).

\itt{stop\_c} is a scalar variable of type default 
\realdp, that holds the
required accuracy for the violation of perturbed complementarity slackness 
\req{slknessc}--\req{slknessx}.
The default is {\tt stop\_c =} $u^{1/3}$,
where $u$ is {\tt EPSILON(1.0)} ({\tt EPSILON(1.0D0)} in 
{\tt \fullpackagename\_double}).

\itt{perturb\_start} is a scalar variable of type default \realdp, 
that is used to assign the initial value of the perturbations
to each bound constraint ((see \S\ref{galmethod}) 
whenever the initial point strategy {\tt initial\_point = 0} is used (see above).
If {\tt perturb\_start}$\leq$ {\tt 0}, a suitable value will be
computed by the package. The default is {\tt perturb\_start = - 1.0}.

\itt{reduce\_perturb\_factor} is a scalar variable of type default \realdp, 
that is used to adjust the perturbations. This corresponds to the
value $\xi$ in \S\ref{galmethod}, and should lie strictly between zero and
one. Any value outside $(0,1)$ will be reset to a suitable value by
the package. The default is {\tt reduce\_perturb\_factor = 0.25}.

\itt{reduce\_perturb\_multiplier} is a scalar variable of type default \realdp, 
that is used to adjust the perturbations. This corresponds to the
value $\rho$ in \S\ref{galmethod}, and should lie strictly between zero and
one. Any value outside $(0,1)$ will be reset to a suitable value by
The default is {\tt reduce\_perturb\_factor = 0.01}.

\itt{insufficiently\_feasible} is a scalar variable of type default \realdp, 
that is used to adjust the perturbations. This corresponds to the
value $\epsilon$ in \S\ref{galmethod}, and should be strictly positive.
Any value non-negative value will be reset to a suitable value by the package.
The default is {\tt insufficiently\_fe\-asible =} $u^{1/4}$,
where $u$ is {\tt EPSILON(1.0)} ({\tt EPSILON(1.0D0)} in 
{\tt \fullpackagename\_double}).

\itt{implicit\_tol} is a scalar variable of type default \realdp, 
that is used to assess whether a constraint defines an implicit equality
(see \S\ref{galmethod}). Any of the constraints \req{boundary}
that is less feasible than {\tt implicit\_tol} will be regarded as defining 
an implicit equality.
The default is {\tt implicit\_tol =} $u^{1/3}$,
where $u$ is {\tt EPSILON(1.0)} ({\tt EPSILON(1.0D0)} in 
{\tt \fullpackagename\_double}).

\itt{perturbation\_small} is a scalar variable of type default \realdp, 
that is used to determine whether the problem is feasible, but not
an interior, point. If the maximum constraint perturbation is strictly 
positive but smaller than {\tt perturbation\_small} and the maximum violation 
of \req{boundary} is smaller than {\tt implicit\_tol}, the 
method will deduce that there is a feasible point but no interior.
If {\tt perturbation\_sm\-all}$\leq$ {\tt 0}, a suitable value will be
computed by the package. The default is {\tt perturbation\_sm\-all = - 1.0}.

\itt{prfeas} is a scalar variable of type default \realdp, that aims to specify
the closest that any initial variable may be to infeasibility. Any variable
closer to infeasibility than {\tt prfeas} will be moved to {\tt prfeas} from 
the offending bound. However, if a variable is range bounded, and its bounds 
are closer than {\tt prfeas} apart, it will be moved to the mid-point of the 
two bounds.
The default is {\tt prfeas = 1.0}.

\itt{dufeas} is a scalar variable of type default \realdp, that aims to specify
the closest that any initial dual variable or Lagrange multiplier may be to 
infeasibility. Any variable closer to infeasibility than {\tt prfeas} will be
moved to {\tt dufeas} from the offending bound. However, if a dual variable
is range bounded, and its bounds are closer than {\tt dufeas} apart, it will
be moved to the mid-point of the two bounds.
The default is {\tt dufeas = 1.0}.

\itt{required\_infeas\_reduction}  is a scalar variable of type default 
\realdp, that specifies the
least factor by which the overall infeasibility of the problem must be reduced,
over {\tt infeas\_max} consecutive iterations, 
for it not be declared infeasible (see {\tt infeas\_max)}.
The default is {\tt required\_infeas\_re\-duction = 0.99}.

%\itt{pivot\_tol}  is a scalar variable of type default 
%\realdp, that holds the
%threshold pivot tolerance used by the matrix factorization.  See 
%the documentation for the package {\tt SBLS} for details.
%The default is {\tt pivot\_tol =} $u^{3/4}$,
%%The default is {\tt pivot\_tol =} $0.1 \sqrt{u}$,
%where $u$ is {\tt EPSILON(1.0)} ({\tt EPSILON(1.0D0)} in 
%{\tt \fullpackagename\_double}).
%%The default is {\tt pivot\_tol = 0.01}.

%\itt{pivot\_tol\_for\_dependencies} is a scalar variable of type default 
%\realdp, that holds the
%threshold pivot  tolerance used by the matrix factorization when 
%attempting to detect linearly dependent constraints. A value larger
%than  {\tt pivot\_tol} is appropriate. See 
%the documentation for the package {\tt SBLS} for details.
%The default is {\tt pivot\_tol\_for\_dependencies = 0.5}.

%\itt{zero\_pivot} is a scalar variable of type default \realdp.
%Any pivots smaller than  {\tt zero\_pivot} in absolute value will be regarded 
%to be zero when attempting to detect linearly dependent constraints. 
%The default is {\tt zero\_pivot =} $u^{3/4}$,
%where $u$ is {\tt EPSILON(1.0)} ({\tt EPSILON(1.0D0)} in 
%{\tt \fullpackagename\_double}).

\itt{alpha\_scale} is a scalar variable of type default \realdp, that is used to
specify the scaling factor $\alpha > 0$ that is used to assess dependent 
constraints (see \S\ref{galmethod}). Any non-positive value will be
reset by the package to the default. The default is {\tt alpha\_scale = 0.01}.

\itt{identical\_bounds\_tol}
is a scalar variable of type default \realdp.
Every pair of constraint bounds 
$(c_{i}^{l}, c_{i}^{u})$ or $(x_{j}^{l}, x_{j}^{u})$
that is closer than {\tt identical\_bounds\_tol} 
will be reset to the average of their values,
$\half (c_{i}^{l} + c_{i}^{u})$ or $\half (x_{j}^{l} + x_{j}^{u})$
respectively.
The default is {\tt identical\_bounds\_tol =} $u$,
where $u$ is {\tt EPSILON(1.0)} ({\tt EPSILON(1.0D0)} in 
{\tt \fullpackagename\_double}).

\itt{mu\_target} is a scalar variable of type default \realdp, 
that gives the initial value for each of the targets
$\bmmu_c^{l}$, $\bmmu_c^{u}$, $\bmmu_x^{l}$ and $\bmmu_x^{u}$.
If {\tt mu\_target\_start}$\leq$ {\tt 0}, a suitable value will be
computed by the package. The default is {\tt mu\_target = - 1.0}.

\itt{mu\_accept\_fraction} is a scalar variable of type default \realdp, 
is used to allow flexibility when attempting to satisfy the perturbed 
complementarity equations \req{slknessc}--\req{slknessx}. The value
corresponds to the parameter $\gamma > 0$ in the termination condition
\req{req-pert-slkness} (see \S\ref{galmethod}).
If {\tt mu\_accept\_fraction}$\leq$ {\tt 0}, a suitable positive value will be
computed by the package. The default is {\tt mu\_accept\_fraction = 1.0}.

\itt{mu\_increase\_factor} is a scalar variable of type default \realdp, 
that is used to specify the value $\beta \geq 1$ by which the targets
corresponding to nonzero bound perturbations will be increased
(see \S\ref{galmethod}). Any value smaller than one will be reset to one.
The default is {\tt mu\_increase\_factor = 2.0}.

\itt{cpu\_time\_limit} is a scalar variable of type default \realdp, 
that is used to specify the maximum permitted CPU time. Any negative
value indicates no limit will be imposed. The default is 
{\tt cpu\_time\_limit = - 1.0}.

\itt{clock\_time\_limit} is a scalar variable of type default \realdp, 
that is used to specify the maximum permitted elapsed system clock time. 
Any negative value indicates no limit will be imposed. The default is 
{\tt clock\_time\_limit = - 1.0}.

\itt{remove\_dependencies} is a scalar variable of type 
default \logical, that must be set \true\ if the algorithm
is to attempt to remove any linearly dependent constraints before
solving the problem, and \false\ otherwise. 
We recommend removing linearly dependencies.
The default is {\tt remove\_dependencies = .TRUE.}.

\itt{treat\_zero\_bounds\_as\_general} is a scalar variable of type 
default \logical.
If it is set to \false, variables which 
are only bounded on one side, and whose bound is zero,
will be recognised as non-negativities/non-positivities rather than simply as
lower- or upper-bounded variables.
If it is set to \true, any variable bound 
$x_{j}^{l}$ or $x_{j}^{u}$ which has the value 0.0 will be
treated as if it had a general value.
Setting {\tt treat\_zero\_bounds\_as\_general} to \true\ has the advantage
that if a sequence of problems are reordered, then bounds which are
``accidentally'' zero will be considered to have the same structure as
those which are nonzero. However, {\tt \fullpackagename} is
able to take special advantage of non-negativities/non-positivities, so
if a single problem, or if a sequence of problems whose 
bound structure is known not to change, is/are to be solved, 
it will pay to set the variable to \false.
The default is {\tt treat\_zero\_bounds\_as\_general = .FALSE.}.

\itt{just\_feasible} is a scalar variable of type default \logical, that 
must be set \true\ if the algorithm should stop as soon as a feasible 
interior point of the constraint set is found. Otherwise a a well-centered
interior point will be sought
The default is {\tt just\_feasible = .FALSE.}.

\itt{balance\_initial\_complementarity} is a scalar variable of type 
default \logical, that must be set \true\ if the initial dual variables
and Lagrange multipliers are to be reset to ensure that the
perturbed complementarity \req{pert-slknessc}--\req{pert-interior}
is satisfied (see \S\ref{galmethod}), and \false\ otherwise.
The default is {\tt balance\_initial\_complementarity = .FALSE.}.

\itt{use\_corrector} is a scalar variable of type 
default \logical, that must be set \true\ if a corrector step $\ddot{\bmv}$
is to be used (see \S\ref{galmethod}), and \false\ otherwise. 
The default is {\tt use\_corrector = .FALSE.}.

\itt{record\_x\_status} is a scalar variable of type 
default \logical, that must be set \true\ if the array 
{\tt inform\%X\_status} (see \S\ref{typeinform})
is to be allocated and the status of the bound 
constraints reported on exit. Otherwise, {\tt inform\%X\_status} will
not be allocated. The default is {\tt record\_x\_status = .TRUE.}.

\itt{record\_c\_status} is a scalar variable of type 
default \logical, that must be set \true\ if the array 
{\tt inform\%C\_status} (see \S\ref{typeinform})
is to be allocated and the status of the bound 
constraints reported on exit. Otherwise, {\tt inform\%C\_status} will
not be allocated. The default is {\tt record\_c\_status = .TRUE.}.

\itt{space\_critical} is a scalar variable of type default \logical, 
that must be set \true\ if space is critical when allocating arrays
and  \false\ otherwise. The package may run faster if 
{\tt space\_critical} is \false\ but at the possible expense of a larger
storage requirement. The default is {\tt space\_critical = .FALSE.}.

\itt{deallocate\_error\_fatal} is a scalar variable of type default \logical, 
that must be set \true\ if the user wishes to terminate execution if
a deallocation  fails, and \false\ if an attempt to continue
will be made. The default is {\tt deallocate\_error\_fatal = .FALSE.}.

\itt{feasol} is a scalar variable of type default \logical, that 
should be set \true\ 
if the final solution obtained will be perturbed 
so that variables close to their bounds are moved onto these bounds,
and \false\ otherwise. 
The default is {\tt feasol = .FALSE.}.
%The default is {\tt feasol = .TRUE.}.

\itt{prefix} is a scalar variable of type default \character\
and length 30, that may be used to provide a user-selected 
character string to preface every line of printed output. 
Specifically, each line of output will be prefaced by the string 
{\tt prefix(2:LEN(TRIM(prefix))-1)},
thus ignoreing the first and last non-null components of the
supplied string. If the user does not want to preface lines by such
a string, they may use the default {\tt prefix = ""}.

\itt{FDC\_control} is a scalar variable of type 
{\tt FDC\_control\_type}
whose components are used to control any detection of linear dependencies
performed by the package 
{\tt \libraryname\_FDC}. 
See the specification sheet for the package 
{\tt \libraryname\_FDC} 
for details, and appropriate default values.

\itt{SBLS\_control} is a scalar variable of type 
{\tt SBLS\_control\_type}
whose components are used to control factorizations
performed by the package 
{\tt \libraryname\_SBLS}. 
See the specification sheet for the package 
{\tt \libraryname\_SBLS} 
for details, and appropriate default values.

\end{description}

%%%%%%%%%%% time type %%%%%%%%%%%

\subsubsection{The derived data type for holding timing 
 information}\label{typetime}
The derived data type 
{\tt \packagename\_time\_type} 
is used to hold elapsed CPU and system clock times for the various parts of 
the calculation. The components of 
{\tt \packagename\_time\_type} 
are:
\begin{description}
\itt{total} is a scalar variable of type default \realdp, that gives
 the total CPU time spent in the package.

\itt{preprocess} is a scalar variable of type default \realdp, that gives
 the CPU time spent reordering the problem to standard form prior to solution.

\itt{find\_dependent} is a scalar variable of type default \realdp, that gives
 the CPU time spent detecting and removing linearly-dependent equality 
constraints

\itt{analyse} is a scalar variable of type default \realdp, that gives
 the CPU time spent analysing the required matrices prior to factorization.

\itt{factorize} is a scalar variable of type default \realdp, that gives
 the CPU time spent factorizing the required matrices.

\itt{solve} is a scalar variable of type default \realdp, that gives
 the CPU time spent computing the search direction.

\itt{clock\_total} is a scalar variable of type default \realdp, that gives
 the total elapsed system clock time spent in the package.

\itt{clock\_preprocess} is a scalar variable of type default \realdp, that gives
 the elapsed system clock time spent reordering the problem to standard form 
prior to solution.

\itt{clock\_find\_dependent} is a scalar variable of type default \realdp, that 
gives  the elapsed system clock time spent detecting and removing 
linearly-dependent equality 
 constraints

\itt{clock\_analyse} is a scalar variable of type default \realdp, that gives
 the elapsed system clock time spent analysing the required matrices prior to 
factorization.

\itt{clock\_factorize} is a scalar variable of type default \realdp, that gives
 the elapsed system clock time spent factorizing the required matrices.

\itt{clock\_solve} is a scalar variable of type default \realdp, that gives
 the elapsed system clock time spent computing the search direction.

\end{description}

%%%%%%%%%%% info type %%%%%%%%%%%

\subsubsection{The derived data type for holding informational
 parameters}\label{typeinform}
The derived data type 
{\tt \packagename\_inform\_type} 
is used to hold parameters that give information about the progress and needs 
of the algorithm. The components of 
{\tt \packagename\_inform\_type} 
are:

\begin{description}

\itt{status} is a scalar variable of type default \integer, that gives the
exit status of the algorithm. 
%See Sections~\ref{galerrors} and \ref{galinfo}
See Section~\ref{galerrors}
for details.

\itt{alloc\_status} is a scalar variable of type default \integer, that gives
the status of the last attempted array allocation or deallocation.
This will be 0 if {\tt status = 0}.

\itt{bad\_alloc} is a scalar variable of type default \character\
and length 80, that  gives the name of the last internal array 
for which there were allocation or deallocation errors.
This will be the null string if {\tt status = 0}. 

\ittf{iter} is a scalar variable of type default \integer, that gives the
total number of iterations required.

\itt{factorization\_status} is a scalar variable of type default \integer, that 
gives the return status from the matrix factorization.

\itt{factorization\_integer} is a scalar variable of type long
\integer, that gives the amount of integer storage used for the matrix 
factorization.

\itt{factorization\_real} is a scalar variable of type long \integer, 
that gives the amount of real storage used for the matrix factorization.

\itt{c\_implicit} is a scalar variable of type default \integer, that gives the
number of variables $\bmc$ that lie on (one) of their bounds for all
feasible solutions to \req{prim}--\req{boundary}. Each of these variables 
implies that the corresponding value of the constraint $\bmA \bmx$ may
be fixed at the appropriate bound, and the constraint subsequently treated as
an equality. See {\tt C\_status}.

\itt{x\_implicit} is a scalar variable of type default \integer, that gives the
number of variables $\bmx$ that lie on (one) of their bounds for all
feasible solutions to \req{prim}--\req{boundary}. Each of these variables 
may then be fixed at its appropriate bound, and the variable subsequently
removed from the problem.
See {\tt X\_status}.

\itt{y\_implicit} is a scalar variable of type default \integer, that gives the
number of Lagrange multipliers $\bmy^l$ and $\bmy^u$ that lie on their bounds 
for all feasible solutions to \req{prim}--\req{boundary}. Each of the
corresponding constraints $\bmA \bmx \geq \bmc^l$ and/or 
$\bmA \bmx \leq \bmc^u$ may subsequently be removed from the problem.
See {\tt C\_status}.

\itt{z\_implicit} is a scalar variable of type default \integer, that gives the
number of dual variables $\bmz^l$ and $\bmz^u$ that lie on their bounds 
for all feasible solutions to \req{prim}--\req{boundary}. Each of the
corresponding simple bounds $\bmx \geq \bmx^l$ and/or 
$\bmx \leq \bmx^u$ may subsequently be removed from the problem.
See {\tt X\_status}.

\ittf{obj} is a scalar variable of type default \realdp, that holds the
value of the (primal) objective function, $\bmg^T \bmx$, 
at the well-centered point found.

\itt{mu\_final\_target\_max} is a scalar variable of type default \realdp, 
that holds the largest component of all the targets
$\bmmu_c^{l}$, $\bmmu_c^{u}$, $\bmmu_x^{l}$ and $\bmmu_x^{u}$ used
at the well-centered point found.

\itt{non\_negligible\_pivot} is a scalar variable of type default \realdp, 
that holds the value of the smallest pivot larger than 
{\tt control\%zero\_pivot}
when searching for dependent linear constraints. If 
{\tt non\_negligible\_pivot} is close to  {\tt control\%zero\_pivot},
this may indicate that there are further dependent constraints, and
it may be worth increasing {\tt control\%zero\_pivot} above 
{\tt non\_negligible\_pivot} and solving again.

\itt{feasible} is a scalar variable of type default \logical, that has the
value \true\ if the output values of $\bmx$ and the Lagrange multipliers and
dual variables lie in the strict interior of the primal-dual feasible
region, and the value \false\ otherwise.

\ittf{time} is a scalar variable of type {\tt \packagename\_time\_type} 
whose components are used to hold elapsed CPU and system clock times for 
the various parts of the calculation (see Section~\ref{typetime}).

\itt{X\_status} is a rank-one allocatable array of type default \integer,
that will be allocated to be of length {\tt n} and filled with values
that give the status of each of the bound constraints on $x$ if 
{\tt record\_x\_status} is \true. In this case, the $j$-th component of 
{\tt X\_status} will have one of the following values:
\begin{description}
\itt{\,\,\,0} variable $x_j$ lies between its bounds.
\itt{-1} variable $x_j$ lies on its lower bound for all feasible $x$ (and thus 
may be fixed at the value {\tt X\_l}$(j)$ and removed from the problem).
\itt{\,\,\,1} variable $x_j$ lies on its upper bound for all feasible $x$ 
(and thus 
may be fixed at the value {\tt X\_u}$(j)$ and removed from the problem).
\itt{-2} variable $x_j$ never lies on its lower bound for any feasible $x$ 
(and thus the lower bound {\tt X\_l}$(j)$ may be set to minus infinity).
\itt{\,\,\,2} variable $x_j$ never lies on its upper bound for any feasible $x$ 
(and thus the upper bound {\tt X\_u}$(j)$ may be set to infinity).
\itt{\,\,\,3} the bounds on variable $x_j$ are equal
(and thus $x_j$
may be fixed at the value {\tt X\_l}$(j)$ and removed from the problem).
\itt{-3} variable $x_j$ never lies on its either bound for any feasible $x$ 
(and thus the lower bound {\tt X\_l}$(j)$ may be set to minus infinity
and the upper bound {\tt X\_u}$(j)$ may be set to infinity).
\end{description}
{\tt X\_status} will remain unallocated if
{\tt record\_x\_status} is \false.

\itt{C\_status} is a rank-one allocatable array of  
type default \integer, that will be allocated to be of length {\tt m}
and filled with values that give the status of each of the general constraints
on $x$ if {\tt record\_c\_status} is \true. In this case, the $i$-th
component of {\tt C\_status} will have one of the following values:
\begin{description}
\itt{\,\,\,0} constraint value $(\bmA \bmx)_i$ lies between its bounds.
\itt{-1} constraint $(\bmA \bmx)_i$ lies on its lower bound for all feasible 
$x$ (and thus 
may be fixed at the value {\tt C\_l}$(i)$ and treated as an equality constraint).
\itt{\,\,\,1} constraint value $(\bmA \bmx)_i$ lies on its upper bound for all 
feasible $x$ (and thus may be fixed at the value 
{\tt C\_u}$(i)$ and removed from the problem).
\itt{-2} constraint value $(\bmA \bmx)_i$ never lies on its lower bound for 
any feasible $x$ 
(and thus the lower bound {\tt C\_l}$(i)$ may be set to minus infinity).
\itt{\,\,\,2} constraint value $(\bmA \bmx)_i$  never lies on its upper 
bound for any feasible $x$ 
(and thus the upper bound {\tt C\_u}$(i)$ may be set to infinity).
\itt{\,\,\,3} the bounds on constraint value $(\bmA \bmx)_i$ are equal
(and thus $(\bmA \bmx)_i$ is equality constrained).
\itt{-3} constraint value $(\bmA \bmx)_i$ never lies on its either bound for any 
feasible $x$ (and thus the constraint may be removed from the problem).
\itt{\,\,\,4} constraint $i$ is implied by others 
(and thus may be removed from the problem).
\end{description}
{\tt C\_status} will remain unallocated if
{\tt record\_c\_status} is \false.

\itt{FDC\_inform} is a scalar variable of type 
{\tt FDC\_inform\_type}
whose components are used to provide information about 
any detection of linear dependencies
performed by the package 
{\tt \libraryname\_FDC}. 
See the specification sheet for the package 
{\tt \libraryname\_FDC} 
for details, and appropriate default values.

\itt{SBLS\_inform} is a scalar variable of type 
{\tt SBLS\_inform\_type}
whose components are used to provide information about factorizations
performed by the package 
{\tt \libraryname\_SBLS}. 
See the specification sheet for the package 
{\tt \libraryname\_SBLS} 
for details, and appropriate default values.

\end{description}

%%%%%%%%%%% data type %%%%%%%%%%%

\subsubsection{The derived data type for holding problem data}\label{typedata}
The derived data type 
{\tt \packagename\_data\_type} 
is used to hold all the data for a particular problem,
or sequences of problems with the same structure, between calls of 
{\tt \packagename} procedures. 
This data should be preserved, untouched, from the initial call to 
{\tt \packagename\_initialize}
to the final call to
{\tt \packagename\_terminate}.

%%%%%%%%%%%%%%%%%%%%%% argument lists %%%%%%%%%%%%%%%%%%%%%%%%

\galarguments
There are three procedures for user calls
(see Section~\ref{galfeatures} for further features): 

\begin{enumerate}
\item The subroutine 
      {\tt \packagename\_initialize} 
      is used to set default values, and initialize private data, 
      before solving one or more problems with the
      same sparsity and bound structure.
\item The subroutine 
      {\tt \packagename\_solve} 
      is called to solve the problem.
\item The subroutine 
      {\tt \packagename\_terminate} 
      is provided to allow the user to automatically deallocate array 
       components of the private data, allocated by 
       {\tt \packagename\_solve}, 
       at the end of the solution process. 
       It is important to do this if the data object is re-used for another 
       problem {\bf with a different structure}
       since {\tt \packagename\_initialize} cannot test for this situation, 
       and any existing associated targets will subsequently become unreachable.
\end{enumerate}
%We use square brackets {\tt [ ]} to indicate \optional arguments.

%%%%%% initialization subroutine %%%%%%

\subsubsection{The initialization subroutine}\label{subinit}
 Default values are provided as follows:
\vspace*{1mm}

\hspace{8mm}
{\tt CALL \packagename\_initialize( data, control, inform )}

\vspace*{-3mm}
\begin{description}

\itt{data} is a scalar \intentinout\ argument of type 
{\tt \packagename\_data\_type}
(see Section~\ref{typedata}). It is used to hold data about the problem being 
solved. 

\itt{control} is a scalar \intentout\ argument of type 
{\tt \packagename\_control\_type}
(see Section~\ref{typecontrol}). 
On exit, {\tt control} contains default values for the components as
described in Section~\ref{typecontrol}.
These values should only be changed after calling 
{\tt \packagename\_initialize}.

\itt{inform} is a scalar \intentinout\ argument of type 
{\tt \packagename\_inform\_type}
(see Section~\ref{typeinform}). A successful call to
{\tt \packagename\_initialize}
is indicated when the  component {\tt status} has the value 0. 
For other return values of {\tt status}, see Section~\ref{galerrors}.

\end{description}

%%%%%%%%% main solution subroutine %%%%%%

\subsubsection{The linear or separable convex quadratic programming problem 
solution subroutine}
The  constrained least-distance problem solution algorithm is called as follows:
\vspace*{1mm}

\hspace{8mm}
{\tt CALL \packagename\_solve( p, data, control, info )}

\vspace*{-3mm}
\begin{description}
\ittf{p} is a scalar \intentinout\ argument of type 
{\tt QPT\_problem\_type}
(see Section~\ref{typeprob}). 
It is used to hold data about the problem being solved.
For a new problem, the user must allocate all the array components,
and set values for all components except {\tt p\%C}.
{\tt p\%new\_problem\_structure} must be set \true, but will have been reset
to \false\ on exit from {\tt \packagename\_solve}.
Users are free to choose whichever
of the three matrix formats described in Section~\ref{galmatrix} 
is appropriate for $\bmA$ for their application.

For a problem with the same structure as one that has just been 
solved, the user may set {\tt p\%new\_problem\_str\-u\-cture} to \false,
so long as {\tt \packagename\_terminate} has not been called in the interim.
The \integer\ components must be unaltered since the
previous call to {\tt \packagename\_solve}, but the \realdp\ 
may be altered to reflect the new problem.

The components {\tt p\%X}, {\tt p\%Y\_l}, {\tt p\%Y\_u}, {\tt p\%Z\_l} 
and {\tt p\%Z\_u}
must be set to initial estimates, $\bmx^{0}$, of the primal variables, 
$\bmx$, Lagrange multipliers for the general constraints, $\bmy_l$ and $\bmy_u$, 
and dual variables for the bound constraints, $\bmz_l$ and $\bmz_u$, 
respectively.
Inappropriate initial values will be altered, so the user should
not be overly concerned if suitable values are not apparent, and may be
content with merely setting {\tt p\%X=0.0},
{\tt p\%Y\_l=0.0}, {\tt p\%Y\_u=0.0}, {\tt p\%Z\_l=0.0} 
and 
{\tt p\%Z\_u=0.0}.
The component {\tt p\%C} need not be set on entry.

On exit, the components {\tt p\%X},
{\tt p\%Y\_l}, {\tt p\%Y\_u}, {\tt p\%Z\_l}, {\tt p\%Z\_u} and {\tt p\%C}
will contain the best estimates of the primal variables $\bmx$, 
Lagrange multipliers for the general
constraints $\bmy_l$ and $\bmy_u$, 
dual variables for the bound constraints $\bmz_l$ and $\bmz_u$, 
and values of the constraints $\bmA \bmx$ respectively.
What of the remaining problem data has been restored depends upon
the input value of the control parameter {\tt control\%restore\_problem}.
The return format for a restored array component will be the same as its
input format.
\restrictions {\tt p\%n} $> 0$, {\tt p\%m} $\geq 0$ and {\tt p\%A\_ne} 
              $\geq -2$.

\itt{data} is a scalar \intentinout\ argument of type 
{\tt \packagename\_data\_type}
(see Section~\ref{typedata}). It is used to hold data about the problem being 
solved. It must not have been altered {\bf by the user} since the last call to 
{\tt \packagename\_initialize}.

\itt{control} is a scalar \intentin\ argument of type 
{\tt \packagename\_control\_type}
(see Section~\ref{typecontrol}). Default values may be assigned by calling 
{\tt \packagename\_initialize} prior to the first call to 
{\tt \packagename\_solve}.

\itt{inform} is a scalar \intentout\ argument of type 
{\tt \packagename\_inform\_type}
(see Section~\ref{typeinform}). A successful call to
{\tt \packagename\_solve}
is indicated when the  component {\tt status} has the value 0. 
For other return values of {\tt status}, see Section~\ref{galerrors}.

%\itt{v} is an \optional rank-one \intentin\ allocatable array of type 
% default \realdp,  that holds \ldots

\end{description}

%%%%%%% termination subroutine %%%%%%

\subsubsection{The  termination subroutine}
All previously allocated arrays are deallocated as follows:
\vspace*{1mm}

\hspace{8mm}
{\tt CALL \packagename\_terminate( data, control, info )}

\vspace*{-3mm}
\begin{description}

\itt{data} is a scalar \intentinout\ argument of type 
{\tt \packagename\_data\_type} 
exactly as for
{\tt \packagename\_solve},
which must not have been altered {\bf by the user} since the last call to 
{\tt \packagename\_initialize}.
On exit, array components will have been deallocated.

\itt{control} is a scalar \intentin\ argument of type 
{\tt \packagename\_control\_type}
exactly as for
{\tt \packagename\_solve}.

\itt{inform} is a scalar \intentout\ argument of type
{\tt \packagename\_inform\_type}
exactly as for
{\tt \packagename\_solve}.
Only the component {\tt status} will be set on exit, and a 
successful call to 
{\tt \packagename\_terminate}
is indicated when this  component {\tt status} has the value 0. 
For other return values of {\tt status}, see Section~\ref{galerrors}.

\end{description}

%%%%%%%%%%%%%%%%%%%%%% Warning and error messages %%%%%%%%%%%%%%%%%%%%%%%%

\galerrors
A negative value of {\tt inform\%status} on exit from 
{\tt \packagename\_solve}
or 
{\tt \packagename\_terminate}
indicates that an error has occurred. No further calls should be made
until the error has been corrected. Possible values are:

\begin{description}

\itt{-1.} An allocation error occurred. A message indicating the offending 
array is written on unit {\tt control\%error}, and the returned allocation 
status and a string containing the name of the offending array
are held in {\tt inform\%alloc\_\-status}
and {\tt inform\%bad\_alloc} respectively.

\itt{-2.} A deallocation error occurred. A message indicating the offending 
array is written on unit {\tt control\%error} and the returned allocation 
status and a string containing the name of the offending array
are held in {\tt inform\%alloc\_\-status}
and {\tt inform\%bad\_alloc} respectively.
status is given by the value {\tt inform\%alloc\_status}.

\itt{-3.} One of the restrictions {\tt prob\%n} $> 0$ or {\tt prob\%m} $\geq  0$
    or requirements that {\tt prob\%A\_type} contains its relevant string
    {\tt 'DENSE'}, {\tt 'COORDINATE'}  or {\tt 'SPARSE\_BY\_ROWS'} 
    has been violated.

\itt{-4.} The equality constraints are inconsistent.

\itt{-5.} The constraints appear to have no feasible point.

\itt{-10.} The factorization failed; the return status from the factorization
    package is given in the component {\tt inform\%fac\-t\-or\_status}.

\itt{-17.} The problem is so ill-conditioned that further progress is 
   impossible.

\itt{-18} Too many iterations have been performed. This may happen if
    {\tt control\%maxit} is too small, but may also be symptomatic of 
    a badly scaled problem.

\itt{\galerrcpulimit.} The elapsed CPU or system clock time limit has been 
    reached. This may happen if either {\tt control\%cpu\_time\_limit} or
    {\tt control\%clock\_time\_limit} is too small, but may also be symptomatic 
    of a badly scaled problem.

\end{description}

%%%%%%%%%%%%%%%%%%%%%% Further features %%%%%%%%%%%%%%%%%%%%%%%%

\galfeatures
\noindent In this section, we describe an alternative means of setting 
control parameters, that is components of the variable {\tt control} of type
{\tt \packagename\_control\_type}
(see Section~\ref{typecontrol}), 
by reading an appropriate data specification file using the
subroutine {\tt \packagename\_read\_specfile}. This facility
is useful as it allows a user to change  {\tt \packagename} control parameters 
without editing and recompiling programs that call {\tt \packagename}.

A specification file, or specfile, is a data file containing a number of 
"specification commands". Each command occurs on a separate line, 
and comprises a "keyword", 
which is a string (in a close-to-natural language) used to identify a 
control parameter, and 
an (optional) "value", which defines the value to be assigned to the given
control parameter. All keywords and values are case insensitive, 
keywords may be preceded by one or more blanks but
values must not contain blanks, and
each value must be separated from its keyword by at least one blank.
Values must not contain more than 30 characters, and 
each line of the specfile is limited to 80 characters,
including the blanks separating keyword and value.

The portion of the specification file used by 
{\tt \packagename\_read\_specfile}
must start
with a "{\tt BEGIN \packagename}" command and end with an 
"{\tt END}" command.  The syntax of the specfile is thus defined as follows:
\begin{verbatim}
  ( .. lines ignored by WCP_read_specfile .. )
    BEGIN WCP
       keyword    value
       .......    .....
       keyword    value
    END 
  ( .. lines ignored by WCP_read_specfile .. )
\end{verbatim}
where keyword and value are two strings separated by (at least) one blank.
The ``{\tt BEGIN \packagename}'' and ``{\tt END}'' delimiter command lines 
may contain additional (trailing) strings so long as such strings are 
separated by one or more blanks, so that lines such as
\begin{verbatim}
    BEGIN WCP SPECIFICATION
\end{verbatim}
and
\begin{verbatim}
    END WCP SPECIFICATION
\end{verbatim}
are acceptable. Furthermore, 
between the
``{\tt BEGIN \packagename}'' and ``{\tt END}'' delimiters,
specification commands may occur in any order.  Blank lines and
lines whose first non-blank character is {\tt !} or {\tt *} are ignored. 
The content 
of a line after a {\tt !} or {\tt *} character is also 
ignored (as is the {\tt !} or {\tt *}
character itself). This provides an easy manner to "comment out" some 
specification commands, or to comment specific values 
of certain control parameters.  

The value of a control parameters may be of three different types, namely
integer, logical or real.
Integer and real values may be expressed in any relevant Fortran integer and
floating-point formats (respectively). Permitted values for logical
parameters are "{\tt ON}", "{\tt TRUE}", "{\tt .TRUE.}", "{\tt T}", 
"{\tt YES}", "{\tt Y}", or "{\tt OFF}", "{\tt NO}",
"{\tt N}", "{\tt FALSE}", "{\tt .FALSE.}" and "{\tt F}". 
Empty values are also allowed for 
logical control parameters, and are interpreted as "{\tt TRUE}".  

The specification file must be open for 
input when {\tt \packagename\_read\_specfile}
is called, and the associated device number 
passed to the routine in device (see below). 
Note that the corresponding 
file is {\tt REWIND}ed, which makes it possible to combine the specifications 
for more than one program/routine.  For the same reason, the file is not
closed by {\tt \packagename\_read\_specfile}.

Control parameters corresponding to the components 
{\tt FDC\_control}
and
{\tt SBLS\_control} may be changed by including additional sections enclosed by
``{\tt BEGIN FDC}'' and 
``{\tt END FDC}'', and
``{\tt BEGIN SBLS}'' and 
``{\tt END SBLS}'', respectively. 
See the specification sheets for the packages 
{\tt \libraryname\_FDC} 
and
{\tt \libraryname\_SBLS}
for further details.

\subsubsection{To read control parameters from a specification file}
\label{readspec}

Control parameters may be read from a file as follows:
\hskip0.5in 
\def\baselinestretch{0.8} {\tt \begin{verbatim}
     CALL WCP_read_specfile( control, device )
\end{verbatim}}
\def\baselinestretch{1.0}

\begin{description}
\itt{control} is a scalar \intentinout argument of type 
{\tt \packagename\_control\_type}
(see Section~\ref{typecontrol}). 
Default values should have already been set, perhaps by calling 
{\tt \packagename\_initialize}.
On exit, individual components of {\tt control} may have been changed
according to the commands found in the specfile. Specfile commands and 
the component (see Section~\ref{typecontrol}) of {\tt control} 
that each affects are given in Table~\ref{specfile}.

\pctable{|l|l|l|} 
\hline
  command & component of {\tt control} & value type \\ 
\hline
  {\tt error-printout-device} & {\tt \%error} & integer \\
  {\tt printout-device} & {\tt \%out} & integer \\
  {\tt print-level} & {\tt \%print\_level} & integer \\
  {\tt maximum-number-of-iterations} & {\tt \%maxit} & integer \\
  {\tt start-print} & {\tt \%start\_print} & integer \\
  {\tt stop-print} & {\tt \%stop\_print} & integer \\
  {\tt initial-point-used} & {\tt \%initial\_point} & integer \\
%  {\tt factorization-used} & {\tt \%factor} & integer \\
%  {\tt maximum-column-nonzeros-in-schur-complement} & {\tt \%max\_col} & integer \\
%  {\tt initial-integer-workspace} & {\tt \%indmin} & integer \\
%  {\tt initial-real-workspace} & {\tt \%valmin} & integer \\
%  {\tt maximum-refinements} & {\tt \%itref\_max} & integer \\
  {\tt maximum-poor-iterations-before-infeasible} & {\tt \%infeas\_max} & integer \\
  {\tt perturbation-strategy} & {\tt \%perturbation\_strategy} & integer \\
  {\tt restore-problem-on-output} & {\tt \%restore\_problem} & integer \\
  {\tt infinity-value} & {\tt \%infinity} & real \\
  {\tt primal-accuracy-required} & {\tt \%stop\_p} & real \\
  {\tt dual-accuracy-required} & {\tt \%stop\_d} & real \\
  {\tt complementary-slackness-accuracy-required} & {\tt \%stop\_c} & real \\
  {\tt initial-bound-perturbation} & {\tt perturb\_start} & real \\
  {\tt perturbation-small} & {\tt \%perturbation\_small} & real \\
  {\tt reduce-perturbation-factor} & {\tt \%reduce\_perturb\_factor} & real \\
  {\tt reduce-perturbation-multiplier} & {\tt \%reduce\_perturb\_multiplier} & real \\
  {\tt insufficiently-feasible-tolerance} & {\tt \%insufficiently\_feasible} & real \\
  {\tt implicit-variable-tolerance} & {\tt \%implicit\_tol} & real \\
  {\tt mininum-initial-primal-feasibility} & {\tt \%prfeas} & real \\
  {\tt mininum-initial-dual-feasibility} & {\tt \%dufeas} & real \\
  {\tt target-barrier-parameter} & {\tt \%mu\_target} & real \\
  {\tt target-barrier-accept-fraction} & {\tt \%mu\_accept\_fraction} & real \\
  {\tt increase-barrier-parameter-by} & {\tt \%mu\_increase\_factor} & real \\
  {\tt required-infeasibility-reduction} & {\tt \%required\_infeas\_reduction} & real \\
%  {\tt pivot-tolerance-used} & {\tt \%pivot\_tol} & real \\
%  {\tt pivot-tolerance-used-for-dependencies} & {\tt \%pivot\_tol\_for\_dependencies} & real \\
%  {\tt zero-pivot-tolerance} & {\tt \%zero\_pivot} & real \\
  {\tt alpha-scaling-tolerance} & {\tt \%alpha\_scale} & real \\
  {\tt identical-bounds-tolerance} & {\tt \%identical\_bounds\_tol} & real \\
  {\tt maximum-cpu-time-limit} & {\tt \%cpu\_time\_limit} & real \\
  {\tt maximum-clock-time-limit} & {\tt \%clock\_time\_limit} & real \\
  {\tt remove-linear-dependencies} & {\tt \%remove\_dependencies} & logical \\
  {\tt treat-zero-bounds-as-general} & {\tt \%treat\_zero\_bounds\_as\_general} & logical \\
  {\tt just-find-feasible-point} & {\tt \%just\_feasible} & logical \\
  {\tt balance-initial-complementarity} & {\tt \%balance\_initial\_complementarity} & logical \\
  {\tt use-corrector-step} & {\tt \%use\_corrector} & logical \\
  {\tt space-critical}   & {\tt \%space\_critical} & logical \\
  {\tt deallocate-error-fatal}   & {\tt \%deallocate\_error\_fatal} & logical \\
  {\tt record-x-status} & {\tt \%record\_x\_status} & logical \\
  {\tt record-c-status} & {\tt \%record\_c\_status} & logical \\
\hline
\ectable{\label{specfile}Specfile commands and associated 
components of {\tt control}.}

\itt{device} is a scalar \intentin argument of type default \integer,
that must be set to the unit number on which the specfile
has been opened. If {\tt device} is not open, {\tt control} will
not be altered and execution will continue, but an error message
will be printed on unit {\tt control\%error}.

\end{description}

%%%%%%%%%%%%%%%%%%%%%% Information printed %%%%%%%%%%%%%%%%%%%%%%%%

\galinfo
If {\tt control\%print\_level} is positive, information about the progress 
of the algorithm will be printed on unit {\tt control\-\%out}.
If {\tt control\%print\_level} $= 1$, a single line of output will be produced 
for each iteration of the process. 
This will include values of the current primal and dual infeasibility, and
violation of complementary slackness, the feasibility-phase objective value,
the current steplength, the value of the barrier parameter, the
number of backtracks in the linesearch and the elapsed clock time in seconds.

If {\tt control\%print\_level} $\geq 2$ this 
output will be increased to provide significant detail of each iteration. 
This extra output includes residuals of the linear systems solved, and,
for larger values of {\tt control\%print\_level}, values of the primal and dual
variables and Lagrange multipliers. 

%%%%%%%%%%%%%%%%%%%%%% GENERAL INFORMATION %%%%%%%%%%%%%%%%%%%%%%%%

\galgeneral

\galcommon None.
\galworkspace Provided automatically by the module.
\galroutines None. 
\galmodules {\tt \packagename\_solve} calls the \galahad\ packages
{\tt GALAHAD\_\-CLOCK},
{\tt GALAHAD\_SY\-M\-BOLS}, 
{\tt GALAHAD\_SMT},
{\tt GALAHAD\_QPT}, {\tt GALAHAD\_QPD}, {\tt GALAHAD\_SPECFILE}, 
{\tt GALAHAD\_QPP}, {\tt GALAHAD\_SPACE}, {\tt GALAHAD\_SORT}, 
{\tt GALAHAD\_STRING}, {\tt GALAHAD\_\-ROOTS}, \linebreak {\tt GALAHAD\_\-SBLS},
{\tt GALAHAD\_FDC}.
\galio Output is under control of the arguments
 {\tt control\%error}, {\tt control\%out} and {\tt control\%print\_level}.
\galrestrictions {\tt prob\%n} $> 0$, {\tt prob\%m} $\geq  0$, 
{\tt prob\%A\_type} $\in \{${\tt 'DENSE'}, 
 {\tt 'COORDINATE'}, {\tt 'SPARSE\_BY\_ROWS'}$\}$. 
\galportability ISO Fortran~95 + TR 15581 or Fortran~2003. 
The package is thread-safe.

%%%%%%%%%%%%%%%%%%%%%% METHOD %%%%%%%%%%%%%%%%%%%%%%%%

\galmethod

The algorithm is iterative, and at each major iteration attempts
to find a solution to the perturbed system
\req{prim}, \req{dual},
\eqn{pert-slknessc}{
( c_i^{} - c^l_i + (\theta_c^l)_i^{} ) 
( y^{l}_i + (\theta_y^l)_i^{} )
= (\mu_c^{l})_i^{} \tim{and}
( c_i^{} - c_i^u - (\theta_c^u)_i^{} ) 
( y^{u}_i - (\theta_y^u)_i^{} )
= (\mu_c^{u})_i^{}, \;\;\;
 i = 1, \ldots , m,}
\eqn{pert-slknessx}{
( x_j^{} - x^l_j + (\theta_x^l)_j^{} ) 
( z^{l}_j + (\theta_z^l)_j^{} )
= (\mu_x^{l})_j^{} \tim{and}
( x_j^{} - x_j^u - (\theta_x^u)_j^{} ) 
( z^{u}_j - (\theta_z^u)_j^{} )
= (\mu_x^{u})_j^{}, \;\;\;
 j = 1, \ldots , n,}
and
\eqn{pert-interior}{
\bmc^{l} - \bmtheta_c^l < \bmc < \bmc^{u} + \bmtheta_c^u, \;\;
\bmx^{l} - \bmtheta_x^l < \bmx < \bmx^{u} + \bmtheta_x^u, \;\;
 \bmy^{l} > - \bmtheta_y^l , \;\;
 \bmy^{u} < \bmtheta_y^u , \;\;
 \bmz^{l} > - \bmtheta_z^l \;\; \mbox{and} \;\;
 \bmz^{u} < \bmtheta_z^u ,}
where the vectors of perturbations 
$\bmtheta^l_c$,
$\bmtheta^u_c$,
$\bmtheta^l_x$,
$\bmtheta^u_x$,
$\bmtheta^l_x$,
$\bmtheta^u_x$,
$\bmtheta^l_y$,
$\bmtheta^u_y$,
$\bmtheta^l_z$ and
$\bmtheta^u_z$,
are non-negative. Rather than solve 
\req{prim}--\req{dual} and \req{pert-slknessc}--\req{pert-interior}
exactly, we instead seek a feasible point for the easier relaxation
\req{prim}--\req{dual} and
\eqn{req-pert-slkness}{\arr{cccccl}{
\gamma (\mu_c^{l})_i^{} & \leq &
( c_i^{} - c^l_i + (\theta_c^l)_i^{} ) ( y^{l}_i + (\theta_y^l)_i^{} )
& \leq & (\mu_c^{l})_i^{} / \gamma & \tim{and} \\
\gamma (\mu_c^{u})_i^{} & \leq &
( c_i^{} - c_i^u - (\theta_c^u)_i^{} ) ( y^{u}_i - (\theta_y^u)_i^{} )
& \leq & (\mu_c^{u})_i^{}, /\gamma &
\;\;\;  i = 1, \ldots , m, \tim{and} \\
\gamma (\mu_x^{l})_j^{} & \leq &
( x_j^{} - x^l_j + (\theta_x^l)_j^{} ) ( z^{l}_j + (\theta_z^l)_j^{} )
& \leq & (\mu_x^{l})_j^{} /\gamma & \tim{and} \\
\gamma (\mu_x^{u})_j^{}  & \leq &
( x_j^{} - x_j^u - (\theta_x^u)_j^{} ) 
( z^{u}_j - (\theta_z^u)_j^{} ) 
& \leq & (\mu_x^{u})_j^{} /\gamma , & \;\;\; j = 1, \ldots , n,}}
for some $\gamma \in (0,1]$ which is allowed to be smaller than one 
if there is a nonzero perturbation.

Given any solution to \req{prim}--\req{dual} and \req{req-pert-slkness}
satisfying \req{pert-interior},,
the perturbations are reduced (sometimes to zero) so as to ensure that the 
current solution is feasible for the next perturbed problem. Specifically,
the perturbation $(\theta^l_c)_i^{}$ for the constraint $c_i^{} \geq c^l_i$
is set to zero if $c_i$ is larger than some given parameter $\epsilon > 0$.
If not, but $c_i$ is strictly positive, the perturbation will be
reduced by a multiplier $\rho \in (0,1)$. Otherwise, the new perturbation 
will be set to $\xi (\theta^l_c)_i^{} + ( 1 - \xi ) ( c_i^l - c_i^{} )$
for some factor $\xi \in (0,1)$. Identical rules are used to reduce the
remaining primal and dual perturbations. 
The targets $\bmmu_c^{l}$, $\bmmu_c^{u}$, $\bmmu_x^{l}$ and $\bmmu_x^{u}$
will also be increased by the factor $\beta \geq 1$ for those
(primal and/or dual) variables with strictly 
positive perturbations so as to try to accelerate the convergence. 

Ultimately the intention is to drive all the perturbations to zero. 
It can be shown that if the original problem \req{prim}--\req{slknessx}
and \req{interior}
has a solution, the perturbations will be zero after a finite number of major 
iterations. Equally, if there is no interior solution  \req{interior}, 
the sets of (primal and dual) variables that are necessarily at (one of) their
bounds for all feasible points---we refer to these as {\em implicit} 
equalities---will be identified, as will the possibility that there is
no point (interior or otherwise) in the primal and/or dual feasible regions.

Each major iteration requires the solution 
$\bmv = (\bmx,\bmc,\bmz^l,\bmz^u,\bmy^l,\bmy^u)$
of the nonlinear system
\req{prim}, \req{dual} and \req{pert-slknessc}--\req{pert-interior}
for fixed perturbations, using a minor iteration. The minor iteration
uses a stabilized (predictor-corrector) Newton method, 
in which the arc 
$\bmv(\alpha) = \bmv + \alpha \dot{\bmv} + \alpha^2 \ddot{\bmv}$, $\alpha \in 
[0,1]$,
involving the standard Newton step $\dot{\bmv}$
for the equations \req{prim}, \req{dual}, \req{pert-slknessc}
and \req{pert-slknessx}, optionally augmented by a corrector $\ddot{\bmv}$
to account for the nonlinearity in \req{pert-slknessc}--\req{pert-slknessx},
is truncated so as to ensure that
\disp{
( c_i^{}(\alpha) - c^l_i + (\theta_c^l)_i^{} ) 
( y^{l}_i(\alpha) + (\theta_y^l)_i^{} )
\geq \tau (\mu_c^{l})_i^{} \tim{and}
( c_i^{}(\alpha) - c_i^u - (\theta_c^u)_i^{} ) 
( y^{u}_i(\alpha) - (\theta_y^u)_i^{} )
\geq \tau (\mu_c^{u})_i^{}, \;\;\;
 i = 1, \ldots , m,}
and 
\disp{
( x_j^{}(\alpha) - x^l_j + (\theta_x^l)_j^{} ) 
( z^{l}_j(\alpha) + (\theta_z^l)_j^{} )
\geq \tau (\mu_x^{l})_j^{} \tim{and}
( x_j^{}(\alpha) - x_j^u - (\theta_x^u)_j^{} ) 
( z^{u}_j(\alpha) - (\theta_z^u)_j^{} )
\geq \tau (\mu_x^{u})_j^{}, \;\;\;
 j = 1, \ldots , n,}
for some $\tau \in (0,1)$, always holds, and also so that the norm
of the residuals to 
\req{prim}, \req{dual}, \req{pert-slknessc} and \req{pert-slknessx}
is reduced as much as possible. 
The Newton and corrector systems are solved using a factorization of
the Jacobian of its defining functions (the so-called ``augmented system''
approach) or of a reduced system in which some of the trivial equations are
eliminated (the ``Schur-complement'' approach).
The factors are obtained using the \galahad\ package {\tt GALAHAD\_SBLS}.

In order to make the solution as efficient as possible, the 
variables and constraints are reordered internally
by the \galahad\ package {\tt GALAHAD\_QPP} prior to solution. 
In particular, fixed variables, and 
free (unbounded on both sides) constraints are temporarily removed.
In addition, an attempt to identify and remove linearly dependent
equality constraints may be made by factorizing
\disp{ \mat{cc}{\alpha \bmI & \bmA^T_{\cal E} \\ \bmA^{}_{\cal E} & \bmzero},}
where $\bmA_{\cal E}$  denotes the gradients of the equality constraints and
$\alpha > 0$ is a given scaling factor,
using {\tt GALAHAD\_SBLS}, and examining small pivot blocks.

\vspace*{1mm}

\galreferences
\vspace*{1mm}

\noindent
The basic algorithm, its convergence analysis and results of
numerical experiments are given in
\vspace*{1mm}

\noindent
C. Cartis and N. I. M. Gould (2006).
Finding a point n the relative interior of a polyhedron.
Technical Report TR-2006-016, Rutherford Appleton Laboratory.

%%%%%%%%%%%%%%%%%%%%%% EXAMPLE %%%%%%%%%%%%%%%%%%%%%%%%

\galexample
Suppose we wish to find a well-centered interior point
that satisfies the general linear constraints
$1 \leq  2 x_{1}  +  x_{2}  \leq  2$, 
$x_{2}  +  x_{3}  =  2$, and simple bounds
$-1  \leq  x_{1}  \leq  1$ and $x_{3}  \leq  2$.
starting from $\bmx^{0}  =  ( -2, 1, 3 )^{T}$.
Then, on writing the data for this problem as
\disp{\bmA = \mat{ccc}{ 2 & 1 & \\ & 1 & 1},\;\;
 \bmc^{l} = \vect{ 1 \\ 2 }, \;\;
 \bmc^{u} = \vect{ 2 \\ 2 }, \;\;
\bmx^{l} = \vect{ -1 \\ - \infty \\ - \infty } \tim{and}
\bmx^{u} = \vect{ 1 \\ \infty \\ 2 }, }
we may use the following code.

{\tt \small
\VerbatimInput{\packageexample}
}
\noindent
This produces the following output:
{\tt \small
\VerbatimInput{\packageresults}
}

\noindent
The same problem may be solved holding the data in 
a sparse row-wise storage format by replacing the lines
{\tt \small
\begin{verbatim}
! sparse co-ordinate storage format: integer components
...
! integer components complete 
\end{verbatim}
}
\noindent
by
{\tt \small
\begin{verbatim}
! sparse row-wise storage format: integer components
   CALL SMT_put( p%A%type, 'SPARSE_BY_ROWS' )  ! Specify sparse-by-row storage
   ALLOCATE( p%A%val( a_ne ), p%A%col( a_ne ), p%A%ptr( m + 1 )  )
   p%A%col = (/ 1, 2, 2, 3 /)                 ! Jacobian A
   p%A%ptr = (/ 1, 3, 5 /)                    ! Set row pointers  
! integer components complete
\end{verbatim}
}
\noindent
and
{\tt \small
\begin{verbatim}
! sparse co-ordinate storage format: real components
...
! real components complete   
\end{verbatim}
}
\noindent
by
{\tt \small
\begin{verbatim}
! sparse row-wise storage format: real components
     p%A%val = (/ 2.0_wp, 1.0_wp, 1.0_wp, 1.0_wp /) ! Jacobian A
! real components complete   
\end{verbatim}
}
\noindent
or using a dense storage format with the replacement lines
{\tt \small
\begin{verbatim}
! dense storage format: integer components
   CALL SMT_put( p%A%type, 'DENSE' )  ! Specify dense storage for A
   ALLOCATE( p%A%val( n * m ) )
! integer components complete
\end{verbatim}
}
\noindent
and 
{\tt \small
\begin{verbatim}
! dense storage format: real components
     p%A%val = (/ 2.0_wp, 1.0_wp, 0.0_wp, 0.0_wp, 1.0_wp, 1.0_wp /) ! Jacobian
! real components complete   
\end{verbatim}
}
\noindent
respectively.

\end{document}







