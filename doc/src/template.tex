\documentclass{galahad}

% set the release and package names

\newcommand{\package}{demo}
\newcommand{\packagename}{DEMO}
\newcommand{\fullpackagename}{\libraryname\_\packagename}

\begin{document}

\galheader

%%%%%%%%%%%%%%%%%%%%%% SUMMARY %%%%%%%%%%%%%%%%%%%%%%%%

\galsummary

{\tt \fullpackagename} is a suite of Fortran~90 procedures for \ldots

%%%%%%%%%%%%%%%%%%%%%% attributes %%%%%%%%%%%%%%%%%%%%%%%%

\galattributes
\galversions{\tt  \fullpackagename\_single, \fullpackagename\_double},
\galuses {\tt packages.}
\galdate April 2001.
\galorigin N. I. M. Gould, Rutherford Appleton Laboratory, and
Ph. L. Toint, University of Namur, Belgium.
\gallanguage Fortran~90. 

%%%%%%%%%%%%%%%%%%%%%% HOW TO USE %%%%%%%%%%%%%%%%%%%%%%%%

\galhowto

The package is available using both single and double precision reals, 
and either 32-bit or 64-bit integers. Access to the 32-bit integer,
single precision version requires the {\tt USE} statement
\medskip

\hspace{8mm} {\tt USE \fullpackagename\_single}

\medskip
\noindent
with the obvious substitution {\tt \fullpackagename\_double},
{\tt \fullpackagename\_single\_64} and 
{\tt \fullpackagename\_double\_64} for the other variants.

\noindent
If it is required to use more than one of the modules at the same time, 
the derived types
(Section~\ref{galtypes})
and the subroutines
(Section~\ref{galarguments})
must be renamed on one of the {\tt USE} statements.

%%%%%%%%%%%%%%%%%%%%%%%%%%% kinds %%%%%%%%%%%%%%%%%%%%%%%%%%%

%%%%%%%%%%%%%%%%%%%%%% kinds %%%%%%%%%%%%%%%%%%%%%%

\galkinds
We use the terms integer and real to refer to the fortran
keywords {\tt REAL(rp\_)} and {\tt INTEGER(ip\_)}, where 
{\tt rp\_} and {\tt ip\_} are the relevant kind values for the
real and integer types employed by the particular module in use. 
The former are equivalent to
default {\tt REAL} for the single precision versions,
{\tt DOUBLE PRECISION} for the double precision cases
and quadruple-precision if 128-bit reals are available, and 
correspond to {\tt rp\_ = real32}, {\tt rp\_ = real64} and
{\tt rp\_ = real128} respectively as defined by the 
fortran {\tt iso\_fortran\_env} module.
The latter are default (32-bit) and long (64-bit) integers, and
correspond to {\tt ip\_ = int32} and {\tt ip\_ = int64},
respectively, again from the {\tt iso\_fortran\_env} module.


%%%%%%%%%%%%%%%%%%%%%% derived types %%%%%%%%%%%%%%%%%%%%%%%%

\galtypes
Four derived data types are accessible from the package.

%%%%%%%%%%% problem type %%%%%%%%%%%

\subsubsection{The derived data type for holding the problem}\label{typeprob}
The derived data type {\tt \packagename\_problem\_type} is used to hold 
the problem. The components of 
{\tt \packagename\_problem\_type} 
are:

\begin{description}

\itt{n} is a scalar variable of type \integer, that holds \ldots

\itt{on} is a scalar variable of type default \logical, that holds \ldots

\itt{time} is a scalar variable of type default \real, that holds \ldots

\itt{f} is a scalar variable of type \realdp, that holds \ldots

\itt{A} is a rank-one pointer array of type \realdp, that holds \ldots

\itt{X} is a rank-one array of dimension {\tt n} and type 
\realdp, that holds \ldots

\end{description}

%%%%%%%%%%% control type %%%%%%%%%%%

\subsubsection{The derived data type for holding control 
 parameters}\label{typecontrol}
The derived data type 
{\tt \packagename\_control\_type} 
is used to hold controlling data. Default values may be obtained by calling 
{\tt \packagename\_initialize}
(see Section~\ref{subinit}). The components of 
{\tt \packagename\_control\_type} 
are:

\begin{description}

\itt{print\_level} is a scalar variable of type \integer, that is used
to control the amount of informational output which is required. No 
informational output will occur if {tt print\_level} $\leq$ 0. If 
{\tt print\_level} $=$ 1, \ldots .
The default is {\tt print\_level} $=$ 0.

\itt{out} is a scalar variable of type \integer, that holds the
stream number for informational messages.
Printing of informational messages in 
{\tt \packagename\_solve}
is suppressed if {\tt out} $<$ 0.
The default is {\tt out} $=$ 6.

\itt{error} is a scalar variable of type \integer, that holds the
stream number for error messages.
Printing of error messages in 
{\tt \packagename\_solve}
or 
{\tt \packagename\_terminate}
is suppressed if {\tt error} $\leq$ 0.
The default is {\tt error} $=$ 6.

\end{description}

%%%%%%%%%%% inform type %%%%%%%%%%%

\subsubsection{The derived data type for holding informational
 parameters}\label{typeinform}
The derived data type 
{\tt \packagename\_inform\_type} 
is used to hold parameters that give information about the progress and needs 
of the algorithm. The components of
{\tt \packagename\_inform\_type} 
are:

\begin{description}
\itt{status} is a scalar variable of type \integer, that gives the
exit status of the algorithm. See Sections~\ref{galerrors} and \ref{galinfo}
for details.
\end{description}

%%%%%%%%%%% data type %%%%%%%%%%%

\subsubsection{The derived data type for holding problem data}\label{typedata}
The derived data type 
{\tt \packagename\_data\_type} 
is used to hold all the data for a particular problem between calls of 
{\tt \packagename} procedures. 
This data should be preserved, untouched, from the initial call to 
{\tt \packagename\_initialize}
to the final call to
{\tt \packagename\_terminate}.

%%%%%%%%%%%%%%%%%%%%%% argument lists %%%%%%%%%%%%%%%%%%%%%%%%

\galarguments
There are three procedures for user calls:

\begin{enumerate}
\item The subroutine 
      {\tt \packagename\_initialize} 
      is used to set default values, and initialize private data, 
      before solving \ldots
\item The subroutine 
      {\tt \packagename\_solve} 
      is called to solve \ldots
\item The subroutine 
      {\tt \packagename\_terminate} 
      is provided to allow the user to automatically deallocate array 
       components of the private data, allocated by 
       {\tt \packagename\_solve}, 
       at the end of the solution process. 
\end{enumerate}
We use square brackets {\tt [ ]} to indicate \optional\ arguments.

%%%%%% initialization subroutine %%%%%%

\subsubsection{The initialization subroutine}\label{subinit}
 Default values are provided as follows:
\vspace*{1mm}

\hspace{8mm}
{\tt CALL \packagename\_initialize( data, control )}

\vspace*{-3mm}
\begin{description}

\itt{data} is a scalar \intentinout\ argument of type 
{\tt \packagename\_data\_type}
(see Section~\ref{typedata}). It is used to hold data about the problem being 
solved.

\itt{control} is a scalar \intentout\ argument of type 
{\tt \packagename\_control\_type}
(see Section~\ref{typecontrol}). 
On exit, {\tt control} contains default values for the components as
described in Section~\ref{typecontrol}.
These values should only be changed after calling 
{\tt \packagename\_initialize}.

\end{description}

%%%%%%%%% main solution subroutine %%%%%%

\subsubsection{The \ldots solution subroutine}
The \ldots solution algorithm is called as follows:
\vspace*{1mm}

\hspace{8mm}
{\tt CALL \packagename\_solve( p, data, control, info, \ldots, [v] )}

\vspace*{-3mm}
\begin{description}
\itt{p} is a scalar \intentinout\ argument of type 
{\tt \packagename\_problem\_type}
(see Section~\ref{typeprob}). 
It is used to hold data about the problem being solved.

\itt{data} is a scalar \intentinout\ argument of type 
{\tt \packagename\_data\_type}
(see Section~\ref{typedata}). It is used to hold data about the problem being 
solved. It must not have been altered {\bf by the user} since the last call to 
{\tt \packagename\_initialize}.

\itt{control} is a scalar \intentin\ argument of type 
{\tt \packagename\_control\_type}
(see Section~\ref{typecontrol}). Default values may be assigned by calling 
{\tt \packagename\_initialize} 
prior to the first call to 
{\tt \packagename\_solve}.

\itt{info} is a scalar \intentout\ argument of type 
{\tt \packagename\_inform\_type}
(see Section~\ref{typeinform}). A successful call to
{\tt \packagename\_solve}
is indicated when the  component {\tt status} has the value 0. 
For other return values of {\tt status}, see Section~\ref{galerrors}.

\itt{v} is an \optional\ rank-one \intentin\ pointer array of type 
 \realdp,  that holds \ldots

\end{description}

%%%%%%% termination subroutine %%%%%%

\subsubsection{The  termination subroutine}
All previously allocated arrays are deallocated as follows:
\vspace*{1mm}

\hspace{8mm}
{\tt CALL \packagename\_terminate( data, control, info )}

\vspace*{-3mm}
\begin{description}

\itt{data} is a scalar \intentinout\ argument of type 
{\tt \packagename\_data\_type} 
exactly as for
{\tt \packagename\_solve}
that must not have been altered {\bf by the user} since the last call to 
{\tt \packagename\_initialize}.
On exit, array components will have been deallocated.

\itt{control} is a scalar \intentin\ argument of type 
{\tt \packagename\_control\_type}
exactly as for
{\tt \packagename\_solve}.

\itt{info} is a scalar \intentout\ argument of type 
{\tt \packagename\_inform\_type}
exactly as for
{\tt \packagename\_solve}.
Only the component {\tt status} will be set on exit, and a 
successful call to 
{\tt \packagename\_terminate}
is indicated when this  component {\tt status} has the value 0. 
For other return values of {\tt status}, see Section~\ref{galerrors}.

\end{description}

%%%%%%%%%%%%%%%%%%%%%% Warning and error messages %%%%%%%%%%%%%%%%%%%%%%%%

\galerrors
A negative value of {\tt info@status} on exit from 
{\tt \packagename\_solve}
or 
{\tt \packagename\_terminate}
indicates that an error has occurred. No further calls should be made
until the error has been corrected. Possible values are:

\begin{description}
\item{- 1 } One of the restrictions \ldots
          has been violated.
\end{description}
A positive value of {\tt info@status} on exit from 
{\tt \packagename\_solve}
is used to warn the user that the data may be faulty or that 
the subroutine cannot guarantee the solution obtained.
Possible values are:


%%%%%%%%%%%%%%%%%%%%%% Information printed %%%%%%%%%%%%%%%%%%%%%%%%

\galinfo
If {\tt control@print\_level} is positive, information about the progress 
of the algorithm will be printed on unit {\tt control@out}.

%%%%%%%%%%%%%%%%%%%%%% GENERAL INFORMATION %%%%%%%%%%%%%%%%%%%%%%%%

\galgeneral

\galcommon None.
\galworkspace Provided automatically by the module.
\galroutines None. 
\galmodules \ldots
\galio Output is provided under the control of {\tt control@print\_level}.
     If the user supplies a unit number in {\tt control@out}, the message
     is printed to the user supplied unit. However if this unit
     number is negative, printing is suppressed.
\galrestrictions None.
\galportability ISO Fortran 90.

%%%%%%%%%%%%%%%%%%%%%% METHOD %%%%%%%%%%%%%%%%%%%%%%%%

\galmethod
\ldots

\galreference
The method is described in detail in \ldots

%%%%%%%%%%%%%%%%%%%%%% EXAMPLE %%%%%%%%%%%%%%%%%%%%%%%%

\galexample
Suppose we wish to solve the problem \ldots

\noindent
Then we may use the following code
{\tt \small
\VerbatimInput{\packageexample}
}
\noindent
with the following data
{\tt \small
\VerbatimInput{\packagedata}
}
\noindent
This produces the following output:
This produces the following output:
{\tt \small
\VerbatimInput{\packageresults}
}
\noindent

\end{document}
